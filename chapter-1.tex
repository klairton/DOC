\chapter{Introdução}\label{chapter:XDSEJBWV}

O estudo da evolução dos organismos é uma tarefa de fundamental importância no campo da biologia. Ao decorrer do tempo mudanças podem ocorrer nos organismos, que refletem adaptações desenvolvidas para melhor se adequar e prosperar no ambiente que estão inseridos. Em particular, mudanças genéticas são uma das características utilizadas no campo da \emph{genômica comparativa} para estimar a proximidade de dois organismos com base na similaridade de seus materiais genéticos. O genoma pode sofrer modificações a partir de mutações que podem ser pontuais ou afetar grandes trechos do genoma, que são chamadas de eventos de rearrajos de genomas. Tais eventos podem afetar o genoma modificando, inserindo ou removendo material genético~\cite{2009-fertin-etal}. Uma das formas bem aceita de estimar a proximidade de dois organismos é determinando uma sequência de eventos de rearranjos de genomas com tamanho mínimo e capaz de transformar o genoma de um organismo em outro. O tamanho de tal sequência é chamada de \emph{distância de rearranjo}.

Reversão e transposição são os eventos de rearranjo mais estudados na literatura~\cite{1999-hannenhalli-pevzner,1999-caprara,2012-bulteau-etal,2019b-oliveira-etal}. Uma reversão atua em um segmento do genoma invertendo a posição e a orientação dos genes contidos no segmento, enquanto uma transposição troca dois segmentos consecutivos do genoma, mas sem afetar a posição e a orientação dos genes nos segmentos. Os eventos de reversão e transposição são chamados de conservativos, pois não alteram a quantidade de material genético do genoma. Existem também eventos não conservativos, como é o caso dos eventos de inserção, deleção e duplicação~\cite{2013-willing-etal,2012-elmabrouk-sankoff,2008-kahn-raphael,2020-mane-etal,2009-bader}, que inserem, removem e duplicam material genético de uma região específica do genoma, respectivamente. Um modelo de rearranjo é caracterizado pelo conjunto de eventos de rearranjo permitidos para transformar um genoma em outro.

Um genoma pode ser representado computacionalmente de diferentes maneiras. Quan\-do o genoma é tratado como uma sequência ordenada de genes, podemos encontrar casos em que determinados genes apresentam múltiplas cópias, sendo comum utilizarmos uma representação na forma de uma cadeia de caracteres (strings), onde cada caractere é associado a um gene. Por outro lado, se existir apenas uma cópia de cada gene, podemos associar um número inteiro para cada gene e a representação é dada na forma de uma permutação. Em ambos os casos, quando a orientação dos genes é conhecida, um sinal de positivo ou negativo é atribuído para cada elemento e a representação é chamada com sinais (string com sinais e permutação com sinais). Caso contrário, o sinal é omitido e a representação é chamada sem sinais (string sem sinais e permutação sem sinais). Chamamos a representação de um genoma através de uma permutação de representação clássica e a representação de um genoma por meio de uma string por representção por strings.

Ao utilizar a representação clássica, podemos simplificar o problema como sendo um problema de ordenação. Nesse caso, o objetivo consiste em transformar uma permutação $\pi$ qualquer em uma permutação específica na qual os elementos encontram-se ordenados de maneira crescente e com sinal positivo para o caso com sinais, essa permutação é chamada de identidade.

Quando consideramos um modelo de rearranjo composto apenas pelo evento de reversão e utilizando uma representação clássica com sinais, temos a variação com sinais do problema de Ordenação de Permutações por Reversões (\SbR). Hannenhalli e Pevzner~\cite{1999-hannenhalli-pevzner} apresentaram o primeiro algoritmo exato polinomial para o problema, sendo posteriormente simplificado por Bergeron~\cite{2005-bergeron}. Atualmente, temos um algoritmo com complexidade subquadrática para determinar a sequência de reversões capaz de ordenar uma permutação com sinais~\cite{2007-tannier-etal}. Entretanto, se estivermos interessados somente na distância de reversão, existe um algoritmo que executa em tempo linear~\cite{2001-bader-etal}. Entretanto, quando consideramos uma representação clássica sem sinais, temos a variação sem sinais do problema \SbR. Caprara~\cite{1999-caprara} provou que o problema faz parte da classe de problemas NP-Difícil. Um dos primeiros algoritmos para o problema apresentou um fator de aproximação $1.75$~\cite{1996-bafna-pevzner}. Em seguida, Christie~\cite{1998a-christie} apresentou um algoritmo com fator de aproximação $1.5$. Atualmente, o melhor algoritmo para o problema apresenta um fator de aproximação $1.375$~\cite{2002-berman-etal}.

Quando consideramos um modelo de rearranjo composto apenas pelo evento de transposição, a orientação dos genes não é considerada, tendo em vista que o evento de transposição não altera a orientação dos genes. Dessa forma, ao adotar uma representação clássica sem sinais, temos o problema de Ordenação de Permutações por Transposições. O problema também pertence à classe de problemas NP-Difícil, sendo a prova apresentada por Bulteau \textit{et al.}~\cite{2012-bulteau-etal}. O primeiro algoritmo para o problema foi proposto por Bafna e Pevzner~\cite{1998-bafna-pevzner} com um fator de aproximação $1.5$. Atualmente, o melhor algoritmo para o problema apresenta um fator de aproximação $1.375$~\cite{2006-elias-hartman,2022-silva-etal} e heurísticas foram apresentadas por Dias e Dias~\cite{2010c-dias-dias} visando a obtenção de resultados práticos melhores. 

Outro evento de rearranjo que podemos mencionar é o block-interchange, que troca a posição de dois segmentos do genoma (não necessariamente consecutivos). Note que com o evento de block-interchange é possível simular uma transposição. Considerando um modelo de rearranjo composto exclusivamente pelo evento de block-interchange, temos o problema de Ordenação de Permutações por Block-Interchanges. Em 1996, foi apresentado um algoritmo exato polinomial para o problema~\cite{1996-christie}.

Ao considerar um modelo de rearranjo composto pelos eventos de reversão e transposição adotando uma representação clássica, obtemos o problema de Ordenação de Permutações por Reversões e Transposições (\SbRT). O problema possui a variação com e sem sinais e ambas pertencem à classe de problemas NP-Difícil~\cite{2019b-oliveira-etal}. Os melhores algoritmos para o problema apresentam fatores de aproximação $2$~\cite{1998-walter-etal} e $2k$~\cite{2008-rahman-etal} (onde $k$ é o fator de aproximação para a decomposição de ciclos~\cite{2013-chen}) para as variações com e sem sinais, respectivamente. Diversas heurísticas considerando esses problemas foram apresentadas na literatura~\cite{2014a-dias-etal,2018-brito-etal}.

Além dos eventos de reversão e transposição, podemos mencionar ainda o evento de rearranjo \emph{double cut and join}(DJC)~\cite{2005-yancopoulos-etal}, que atua fragmentando o genoma em dois pontos e, em seguida, as extremidades dos segmentos resultantes são unidas obedecendo certas restrições. A Tabela~\ref{table:XZPFGPAM} sumariza os resultados considerando uma representação clássica de um genoma e adotando um modelo de rearranjo composto pela combinação dos eventos de reversão, transposição e DCJ. 

\begin{table}[!htb]\label{table:XZPFGPAM}
  \caption[Sumarização dos resultados conhecidos considerando a representação clássica de um genoma.]{Resultados conhecidos considerando a representação clássica e adotando um modelo de rearranjo composto pela combinação dos eventos de reversão, transposição e DCJ.}
  \centering
  \begin{tabular}{|p{8cm}|p{3cm}|p{3cm}|}
    \hline
    \multicolumn{3}{|c|}{Representação Clássica com Sinais}                                                                      \\ \hline
    Modelo                  & Complexidade                                 & Aproximação                                         \\ \hline
    DCJ                     & P~\cite{2005-yancopoulos-etal}               & Exato~\cite{2005-yancopoulos-etal}                  \\ \hline
    Reversão                & P~\cite{1999-hannenhalli-pevzner}            & Exato~\cite{1999-hannenhalli-pevzner}               \\ \hline
    Reversão e Transposição & NP-difícil~\cite{2019b-oliveira-etal}        & $2$~\cite{1998-walter-etal}                         \\ \hline
  \end{tabular}

  \hfill \break

  \begin{tabular}{|p{8cm}|p{3cm}|p{3cm}|}
    \hline
    \multicolumn{3}{|c|}{Representação Clássica sem Sinais}                                                                      \\ \hline
    Modelo                  & Complexidade                                 & Aproximação                                         \\ \hline
    DCJ                     & NP-difícil~\cite{2013-chen}                  & $\frac{17}{12}+\epsilon$~\cite{2013-chen}           \\ \hline
    Reversão                & NP-difícil~\cite{1999-caprara}               & $1.375$~\cite{2002-berman-etal}                     \\ \hline
    Transposição            & NP-difícil~\cite{2012-bulteau-etal}          & $1.375$~\cite{2006-elias-hartman,2022-silva-etal}   \\ \hline
    Reversão e Transposição & NP-difícil~\cite{2019b-oliveira-etal}        & $2k$~\cite{2008-rahman-etal,2013-chen}              \\ \hline
  \end{tabular}
\end{table}

Quando passamos a considerar uma representação por strings, em 2001, Christie e Irving~\cite{2001-christie-irving} mostraram que a variação sem sinais do problema de Distância de Strings por Reversões pertence à classe de problemas NP-Difícil, mesmo se considerarmos um alfabeto binário (os caracteres das strings comparadas pertencem ao conjunto $\{0,1\})$. Para isso, os autores apresentaram uma redução do problema 3-partition~\cite{1990-garey-johnson}. Em 2005, Radcliffe \textit{et al.}~\cite{2005-radcliffe-etal} mostraram que a variação com sinais do problema de Distância de Strings por Reversões e o problema de Distância de Strings por Transposições também pertencem à classe de problemas NP-Difícil, mesmo se considerarmos um alfabeto binário. Outra contribuição importante do trabalho foi que os autores caracterizaram um conjunto de instâncias em que é possível obter uma solução ótima em tempo polinomial.

Uma relação entre o problema de Distância de Strings por Reversões e o problema de Partição Mínima em Strings foi apresentada por Chen \textit{et al.}~\cite{2005-chen-etal}. Com essa relação entre os problemas, foi apresentado por Kolman e Wale{\'n}~\cite{2006-kolman-walen} um algoritmo de aproximação com fator $\Theta(k)$ para ambas as variações do problema de Distância de Strings por Reversões, onde $k$ representa o número máximo de cópias de um caractere nas strings consideradas.

A representação do genoma como uma sequência ordenada de genes é uma abordagem simples e prática, mas acarreta na perda de informação referente às estruturas genéticas que não fazem parte da sequência de genes. Estudos apontaram que considerar informações adicionais contidas no genoma, além da sequência de genes, pode tornar a comparação entre genomas mais realista~\cite{2016a-biller-etal, 2016b-biller-etal}. Em particular, os pesquisadores abordaram a importância de considerar o tamanho das regiões presentes entre cada par de genes consecutivos e nas extremidades do genoma, chamadas de regiões intergênicas. 

Trabalhos que levam em conta a sequência de genes e também consideram os tamanhos das regiões intergênicas começaram a ser apresentados. Assumindo que em um genoma existe uma cópia de cada gene, então sua representação computacional pode ser dada por uma permutação (com ou sem sinais), que repesenta a ordem e a orientação dos genes, e uma lista ordenada de números naturais representando o tamanho de cada região intergência do genoma. Essa representação é chamada de intergênica.

Fertin \textit{et al.}~\cite{2017-fertin-etal}, adotando uma representação intergênica com sinais, apresentaram um modelo composto pelo evento de rearranjo DCJ, mostraram que o problema pertence à classe de problemas NP-Difícil e desenvolveram um algoritmo de aproximação com fator $4/3$. Bulteau \textit{et al.}~\cite{2016b-bulteau-etal} apresentaram um modelo que permite o uso do evento DCJ juntamente com os eventos não conservativos de inserção e deleção restritos a atuarem apenas sobre as regiões intergênicas. Para esse problema, os autores apresentaram um algoritmo exato polinomial. 

Considerando um modelo composto exclusivamente pelo evento de reversão e adotando uma representação intergênica, temos o problema de Ordenação de Permutações por Reversões Intergênicas (\SbIR). Aa variações com e sem sinais do problema pertencem à classe de problemas NP-Difícil~\cite{2021b-oliveira-etal,2020a-brito-etal}, sendo que os melhores algoritmos conhecidos para o problema possuem fatores de aproximação de $2$~\cite{2021b-oliveira-etal} e $4$~\cite{2020a-brito-etal} para as variações com e sem sinais, respectivamente. Considerando um modelo composto exclusivamente pelo evento de transposição, temos o problema de Ordenação de Permutações por Transposições Intergênicas (\SbIT). O problema também pertence à classe de problemas NP-Difícil~\cite{2021a-oliveira-etal}, sendo que o melhor algoritmo conhecido para o problema possui um fator de aproximação de $3.5$~\cite{2021a-oliveira-etal}. Considerando um modelo composto exclusivamente pelo evento de block-interchange, temos o problema de Ordenação de Permutações por Block-Interchanges Intergênicos (\SbIBI). A complexidade do problema ainda permanece desconhecida e o melhor algoritmo para o problema possui um fator de aproximação $2$~\cite{2019-dias-etal}.

Quando utilizamos um modelo composto pelos eventos de reversão e transposição e adotando uma representação intergênica, temos o problema de Ordenação de Permutações por Operações Intergênicas de Reversão e Transposição (\SbIRT). As variações com e sem sinais do problema pertencem à classe de problemas NP-Difícil~\cite{2021a-oliveira-etal,2020a-brito-etal}, sendo que os melhores algoritmos conhecidos para o problema possuem fatores de aproximação de $3$~\cite{2021a-oliveira-etal} e $4$~\cite{2021b-brito-etal} para as variações com e sem sinais, respectivamente.

No contexto de um cenário em que é adotado a representação intergênica, Oliveira \textit{et al.}~\cite{2021a-oliveira-etal} introduziram o evento de rearranjo chamado de \emph{move}. Esse evento é similar ao evento de transposição, mas um dos segmentos afetado é composto exclusivamente por uma região intergênica. Além disso, os autores apresentaram os problemas de Ordenação de Permutações por Operações Intergênicas de Transposição e Move (\SbITM) e Ordenação de Permutações por Operações Intergênicas de Reversão, Transposição e Move (\SbIRTM). Para o problema \SbITM{} os autores apresentaram um algoritmo de aproximação com fator $2.5$~\cite{2021a-oliveira-etal}. Para as variações com e sem sinais do problema \SbIRTM{} existem algoritmos de aproximação com fatores $2.5$~\cite{2021a-oliveira-etal} e $3$~\cite{2021b-brito-etal}, respectivamente.  

Permitindo o uso dos eventos de reversão e move, temos o problema de Ordenação de Permutações por Operações Intergênicas de Reversão e Move (\SbIRM). A variação com sinais desse problema pertencem à classe de problemas NP-Difícil e o melhor algoritmo para o problema possui um fator de aproximação $2$~\cite{2022b-brito-etal}.

Considerando uma restrição adicional no número máximo de genes afetados pelos eventos de reversão e transposição, Oliveira \textit{et al.}~\cite{2019c-oliveira-etal}, apresentaram modelos compostos pela combinação dos eventos super curtos de reversão e transposição. Um evento de rearranjo super curto pode afetar segmentos do genoma com no máximo dois genes. Adotando uma representação intergênica sem sinais, os autores mostraram algoritmos de aproximação com fator $3$, enquanto para uma representação intergênica com sinais apresentaram algoritmos de aproximação com fator $5$.

A Tabela~\ref{table:GNCKDPJY} sumariza os resultados conhecidos considerando uma representação intergênica de um genoma.

\begin{table}[!htb]
  \caption{Resultados de complexidade e fator de aproximação dos modelos considerando uma representação intergênica.}
  \label{table:GNCKDPJY}
  \centering
  \begin{tabular}{|p{8cm}|p{3cm}|p{3cm}|}
    \hline
    \multicolumn{3}{|c|}{Representação Intergênica com Sinais}                                                                                   \\ \hline
    Modelo                                  & Complexidade                                 & Aproximação                                         \\ \hline
    DCJ                                     & NP-difícil~\cite{2017-fertin-etal}           & $\frac{4}{3}$~\cite{2017-fertin-etal}               \\ \hline
    DCJ, Inserção e Deleção                 & P~\cite{2016b-bulteau-etal}                  & Exato~\cite{2016b-bulteau-etal}                     \\ \hline
    Reversão                                & NP-difícil~\cite{2021b-oliveira-etal}        & $2$~\cite{2021b-oliveira-etal}                      \\ \hline
    Reversão (Super Curta)                  & Desconhecida                                 & $5$~\cite{2019c-oliveira-etal}                      \\ \hline
    Reversão e Move                         & NP-difícil~\cite{2022b-brito-etal}           & $2$~\cite{2022b-brito-etal}                         \\ \hline
    Reversão e Transposição                 & NP-difícil~\cite{2021a-oliveira-etal}        & $3$~\cite{2021b-oliveira-etal}                      \\ \hline
    Reversão e Transposição (Super Curta)   & Desconhecida                                 & $5$~\cite{2019c-oliveira-etal}                      \\ \hline
    Reversão, Transposição e Move           & NP-difícil~\cite{2021a-oliveira-etal}        & $\frac{5}{2}$~\cite{2021a-oliveira-etal}            \\ \hline
  \end{tabular}

  \hfill \break

  \begin{tabular}{|p{8cm}|p{3cm}|p{3cm}|}
    \hline
    \multicolumn{3}{|c|}{Representação Intergênica sem Sinais}                                                                                   \\ \hline
    Modelo                  & Complexidade                                 & Aproximação                                                         \\ \hline
    Reversão                                & NP-difícil~\cite{2020a-brito-etal}           & $4$~\cite{2020a-brito-etal}                         \\ \hline
    Reversão (Super Curta)                  & Desconhecida                                 & $3$~\cite{2019c-oliveira-etal}                      \\ \hline
    Transposição                            & NP-difícil~\cite{2021a-oliveira-etal}        & $\frac{7}{2}$~\cite{2021a-oliveira-etal}            \\ \hline
    Transposição (Super Curta)              & Desconhecida                                 & $3$~\cite{2019c-oliveira-etal}                      \\ \hline
    Transposição e Move                     & NP-difícil~\cite{2021a-oliveira-etal}        & $\frac{5}{2}$~\cite{2021a-oliveira-etal}            \\ \hline
    Block-Interchange                       & Desconhecida                                 & $2$~\cite{2019-dias-etal}                           \\ \hline
    Reversão e Transposição                 & NP-difícil~\cite{2020a-brito-etal}           & $4$~\cite{2021b-brito-etal}                         \\ \hline
    Reversão e Transposição (Super Curta)   & Desconhecida                                 & $3$~\cite{2019c-oliveira-etal}                      \\ \hline
    Reversão, Transposição e Move           & NP-difícil~\cite{2021b-brito-etal}           & $3$~\cite{2021b-brito-etal}                         \\ \hline
  \end{tabular}
\end{table}

Esta tese está dividida da seguinte forma: O Capítulo~\ref{chapter:CNDSVAJR} apresenta definições, conceitos e estruturas que serão utilizadas em três capítulos desta tese, e que são fundamentais para obtenção de resultados nos problemas que trabalhamos. O Capítulo~\ref{chapter:JWIGFELF} introduz um novo problema de rearranjo onde uma restrição de proporção entre operações é adicionada ao modelo. Os capítulos~\ref{chapter:DOVAEMLI} e ~\ref{chapter:GMJBMTWF} investigam problemas em que a informação referente ao tamanho das regiões intergênicas é levada em consideração. Por fim, o Capítulo~\ref{chapter:IXYEIWKC} apresenta as conclusões finais desta tese. 


