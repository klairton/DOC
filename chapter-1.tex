\chapter{Introdução}

O estudo da evolução dos organismos é uma tarefa de fundamental importância no campo da biologia. Ao decorrer do tempo mudanças podem ocorrer nos organismos, que refletem adaptações desenvolvidas para melhor se adequar e prosperar no ambiente que estão inseridos. Em particular, mudanças genéticas são uma das características utilizadas no campo da \emph{genômica comparativa} para estimar a proximidade de dois organismos com base na similaridade de seus materiais genéticos. O genoma pode sofrer modificações a partir de mutações que podem ser pontuais ou afetar grandes trechos do genoma, que são chamadas de eventos de rearrajos de genomas. Tais eventos podem afetar o genoma modificando, inserindo ou removendo material genético~\cite{2009-fertin-etal}. Uma das formas bem aceita de estimar a proximidade de dois organismos é determinando uma sequência de eventos de rearranjos de genomas com tamanho mínimo e capaz de transformar o genoma de um organismo em outro. O tamanho de tal sequência é chamada de \emph{distância de rearranjo}.

Reversão e transposição são os eventos de rearranjo mais estudados na literatura~\cite{2005-bergeron,1998-bafna-pevzner}. Uma reversão atua em um segmento do genoma invertendo a posição e a orientação dos genes contidos no segmento, enquanto uma transposição troca dois segmentos consecutivos do genoma, mas sem afetar a posição e a orientação dos genes nos segmentos. Os eventos de reversão e transposição são chamados de conservativos, pois não alteram a quantidade de material genético do genoma. Existem também eventos não conservativos, como é o caso dos eventos de inserção, deleção e duplicação~\cite{2013-willing-etal,2012-elmabrouk-sankoff,2008-kahn-raphael,2020-lafond-etal,2010-lin-etal,2009-bader}, que inserem, removem e duplicam material genético de uma região específica do genoma, respectivamente. Um modelo de rearranjo é caracterizado pelo conjunto de eventos de rearranjo permitidos para transformar um genoma em outro e a representação do genoma utilizada.

Um genoma pode ser representado computacionalmente de diferentes maneiras. Quando o genoma é tratado como uma sequência ordenada de genes, podemos encontrar casos em que determinados genes apresentam múltiplas cópias, sendo comum utilizarmos uma representação na forma de uma cadeia de caracteres, onde cada caractere é associado a um gene. Por outro lado, se existir apenas uma cópia de cada gene, podemos associar um número inteiro para cada gene e a representação é dada na forma de uma permutação. Em ambos os casos, quando a orientação dos genes é conhecida, um sinal de positivo ou negativo é atribuído para cada elemento e a representação é chamada com sinais (string com sinais e permutação com sinais). Caso contrário, o sinal é omitido e a representação é chamada sem sinais (string sem sinais e permutação sem sinais).

Ao utilizar a representação de um genoma como uma permutação, podemos simplificar o problema como sendo um problema de ordenação. Nesse caso, o objetivo consiste em transformar uma permutação $\pi$ qualquer em uma permutação específica na qual os elementos encontram-se ordenados de maneira crescente e com sinal positivo para o caso com sinais, essa permutação é chamada de identidade.

Quando consideramos um modelo de rearranjo composto apenas pelo evento de reversão e utilizando uma representação do genoma na forma de permutações com sinais, temos o problema de Ordenação de Permutações com Sinais por Reversões. Hannenhalli e Pevzner~\cite{1999-hannenhalli-pevzner} apresentaram o primeiro algoritmo exato em tempo polinomial para o problema, sendo posteriormente simplificado por Bergeron~\cite{2005-bergeron}. Atualmente, temos um algoritmo com complexidade subquadrática para determinar a sequência de reversões capaz de ordenar uma permutação com sinais~\cite{2007-tannier-etal}. Entretanto, se estivermos interessados somente na distância de reversão, existe um algoritmo que executa em tempo linear~\cite{2001-bader-etal}. Entretanto, quando consideramos uma representação utilizando permutações sem sinais, temos o problema de Ordenação de Permutações sem Sinais por Reversões. Caprara~\cite{1999-caprara} provou que o problema faz parte da classe de problemas NP-Difícil. Um dos primeiros algoritmos para o problema apresentou um fator de aproximação 1.75~\cite{1996-bafna-pevzner}. Em seguida, Christie~\cite{1998a-christie} apresentou um algoritmo com fator de aproximação 1.5. Atualmente, o melhor algoritmo para o problema apresenta um fator de aproximação 1.375~\cite{2002-berman-etal}.

Quando consideramos um modelo de rearranjo composto apenas pelo evento de transposição, a orientação dos genes não é considerada, tendo em vista que o evento de transposição não altera a orientação dos genes. Dessa forma, ao adotar permutações sem sinais, temos o problema de Ordenação de Permutações sem Sinais por Transposições. O problema também pertence à classe de problemas NP-Difícil, sendo a prova apresentada por Bulteau e coautores~\cite{2012-bulteau-etal}. O primeiro algoritmo para o problema foi proposto por Bafna e Pevzner~\cite{1998-bafna-pevzner} com um fator de aproximação 1.5. Posteriormente, Christie~\cite{1998b-christie} apresentou uma implementação mais simples para esse algoritmo. Atualmente, o melhor algoritmo para o problema apresenta um fator de aproximação 1.375~\cite{2006-elias-hartman} e heurísticas foram apresentadas por Dias e Dias~\cite{2010c-dias-dias} visando a obtenção de resultados práticos melhores.

Ao considerar um modelo de rearranjo composto pelos eventos de reversão e transposição em permutações com e sem sinais, obtemos os problemas de Ordenação de Permutações com Sinais por Reversões e Transposições, e Ordenação de Permutações sem Sinais por Reversões e Transposições, respectivamente. Ambos os problemas pertencem à classe de problemas NP-Difícil~\cite{2019b-oliveira-etal}. Os melhores algoritmos para os problemas apresentam fatores de aproximação $2$~\cite{1998-walter-etal} e $2k$~\cite{2008-rahman-etal} (onde $k$ é o fator de aproximação para a decomposição de ciclos~\cite{2013-chen}) para os casos com e sem sinais, respectivamente. Diversas heurísticas considerando esses problemas foram apresentadas na literatura~\cite{2014a-dias-etal,2018-brito-etal}.

Quando passamos a considerar que o genoma pode apresentar genes repetidos, em 2001, Christie e Irving~\cite{2001-christie-irving} mostraram que o problema de Distância de Strings sem Sinais por Reversões pertence à classe de problemas NP-Difícil, mesmo se considerarmos um alfabeto binário (os caracteres das strings comparadas pertencem ao conjunto $\{0,1\})$. Para isso, os autores apresentaram uma redução do problema 3-partition~\cite{1990-garey-johnson}. Em 2005, Radcliffe e coautores~\cite{2005-radcliffe-etal} mostraram que a Distância de Strings com Sinais por Reversões e Distância de Strings sem Sinais por Transposições também pertencem à classe de problemas NP-Difícil, mesmo se considerarmos um alfabeto binário. Outra contribuição importante do trabalho foi que os autores caracterizaram um conjunto de instâncias em que é possível obter uma solução ótima em tempo polinomial.

Uma relação entre o problema de Distância de Strings por Reversões e o problema de Partição Mínima em Strings foi apresentada por Chen \textit{et al.}~\cite{2005-chen-etal}. Com essa relação entre os problemas, foi apresentado por Kolman e Wale{\'n}~\cite{2006-kolman-walen} um algoritmo de aproximação com fator $\Theta(k)$ para o problema de Distância de Strings com e sem Sinais por Reversões, onde $k$ representa o número máximo de cópias de um caractere nas strings consideradas.

A representação do genoma como uma sequência ordenada de genes é uma abordagem simples e prática, mas acarreta na perda de informação referente às estruturas genéticas que não fazem parte da sequência de genes. Estudos apontaram que considerar informações adicionais contidas no genoma, além da sequência de genes, pode tornar a comparação entre genomas mais realista~\cite{2016a-biller-etal, 2016b-biller-etal} . Em particular, os pesquisadores abordaram a importância de considerar o tamanho das regiões presentes entre cada par de genes consecutivos e nas extremidades do genoma, chamadas de regiões intergênicas.

Trabalhos que levam em conta a sequência de genes e também consideram os tamanhos das regiões intergênicas começaram a ser apresentados recentemente. Fertin \textit{et al.}~\cite{2017-fertin-etal} apresentaram um modelo que permite o uso do evento de rearranjo Double-Cut and Join (DCJ), mostraram que o problema pertence à classe de problemas NP-Difícil e desenvolveram um algoritmo de aproximação com fator $4/3$. O evento de rearranjo DJC~\cite{2005-yancopoulos-etal} atua fragmentando o genoma em dois pontos e, em seguida, as extremidades dos segmentos resultantes são unidas obedecendo certas restrições. Bulteau \textit{et al.}~\cite{2016b-bulteau-etal} apresentaram um modelo que permite o uso do evento DCJ juntamente com os eventos não conservativos de inserção e deleção restritos a atuarem apenas sobre as regiões intergênicas. Para esse problema, os autores apresentaram um algoritmo exato em tempo polinomial. Oliveira \textit{et al.}~\cite{2018b-oliveira-etal} apresentaram um modelo que permite o uso apenas de reversões super curtas (esse evento de rearranjo possui uma restrição adicional que faz com que todo evento de reversão afete um segmento com no máximo dois genes). Juntamente com o modelo, os autores desenvolveram algoritmos de aproximação para o problema de forma geral e para instâncias do problema com características específicas.

Trabalhos considerando a ordem dos genes e o tamanho das regiões intergênicas são recentes, sendo uma promissora linha de pesquisa, tendo em vista as melhorias que podem ser obtidas nas estimativas para a distância evolutiva entre os organismos.