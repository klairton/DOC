\chapter{Modelos com Porporção entre Operações}

Os problemas de distância entre genomas podem utilizar uma abordagem \emph{não ponderada}, ou seja, cada evento de rearranjo utilizado para transformar o genoma de origem no genoma alvo contribui em uma unidade para a distância. Essa abordagem tem como característica que cada tipo de evento de rearranjo, pertencente ao modelo de rearranjo adotado, possui a mesma probabilidade de ocorrer em um cenário evolutivo. Outra abordagem que surgiu para possibilitar uma contribuição diferente para cada evento de rearranjo é chamada de \emph{ponderada}. Nesse abordagem, cada tipo de evento de rearranjo possui um peso associado que é contabilizado na distância evolutiva entre os genomas. A abordagem ponderada geralmente é utilizada para mapear um cenário em que queremos que determinado eventos de rearrajo tenham uma possibilidade maior de ocorrer do que outros. Para isso, basta atribuir um peso menor nos eventos de rearranjo que esperados que ocorram mais. Esses pesos podem ser atribuídos com base em observações empíricas de determinados organismos ou através de análises realizadas especificamente para esse objetivo~\cite{2008-bader-etal,2001-eriksen}. 

Os eventos de rearranjo de reversão e transposição são dois dos eventos mais estudados na literatura~\cite{2002-berman-etal,2006-elias-hartman,2022-silva-etal}. Considerando uma representação clássica e uma abordagem não ponderada, temos o problema de Ordenção de Permutações por Reversões e Transposições (\SbRT), sendo que o problema possui a variação com e sem sinais. Ambas as variações pertencem à classe NP-difícil de problemas~\cite{2019b-oliveira-etal}, para a variação com sinais do problema existe um algoritmo de aproximação com fator 2~\cite{1998-walter-etal}. Para a variação sem sinais, existe um algoritmo de aproximação com fator $2k$~\cite{2008-rahman-etal}, onde $k$~\cite{2013-chen} é o fator de aproximação do algoritmo utilizado para a decomposição de ciclos do Grafo de Ciclos~\cite{1999-caprara}.

Considerando um abordagem poderada, temos o problema de Ordenção de Permutações por Reversões e Transposições Ponderadas (\SbWRT) na variação com e sem sinais. In 2002, Eriksen~\cite{2002-eriksen} apresentou um algoritmo com factor de aproximação $7/6$ para a variação com sinais do problema utilizando os pesos $1$ e $2$ para os eventos de reversão e transposição, respectivamente. Oliveira \textit{et al.}\cite{2019a-oliveira-etal} desenvolveram um algoritmo de aproximação com fator $1.5$ para a variação com sinais do problema \SbWRT{} utilizando os pesos $2$ e $3$ para os eventos de reversão e transposição, respectivamente. Além disso, os autores mostraram que as variações com e sem sinais do problema \SbWRT{} pertencem à classe NP-difícil quando a razão entre os pesos dos eventos de transposição e reversão é maior ou igual a $1.5$.  

Em 2007, Bader e Ohlebusch~\cite{2007-bader-ohlebusch} apresentaram o problema de Ordenção de Permutações por Reversões, Transposições e Transposições Inversa Ponderadas (\SbWRTIT). A transposição inversa é um evento similar ao evento de transposição, mas com um dos segmentos adjacentes afetados sendo invertido. Para a variação com sinais do problema os autores apresentaram um algoritmo de aproximação com fator $1.5$ utilizando o peso $1$ para o evento de reversão e o mesmo peso, no intervalo $[1..2]$, para os eventos de transposição e transposição inversa. Em 2020, Alexandrino \textit{et al.}\cite{2020b-alexandrino-etal} mostraram que as variações com e sem sinais do problema \SbWRTIT{} pertencem à classe NP-difícil quando os eventos de transposição e transposição inversa possuem o mesmo peso e a razão entre os pesos dos eventos de transposição e reversão é maior ou igual a $1.5$.

A abordagem ponderada possui vantagens em comparação com a abordagem não ponderada quando queremos mapear um cenário evolutivo dando mais prioridade para determinados tipos de eventos de rearranjo. Entretanto, ela não garantem que os rearranjos de menor custo, que são supostamente os mais frequentes em um cenário evolutivo, serão os mais utilizados pelos algoritmos. Para contornar esse ponto, propomos e investigamos o problema de Ordenção de Permutações por Reversões e Transposições com Restrição de Proporção (\SbPRT) em instâncias clássicas com e sem sinais. Neste cenário, buscamos uma sequência de reversões e transposições $S$ capaz de transformar o genoma de origem no genoma alvo com uma restrição adicional na qual a relação entre o número de reversões e o tamanho da sequência $S$ deve ser maior ou igual a um determinado parâmetro $k \in [0..1]$. 

Observe que tanto as abordagens ponderada e proporcional tentam incorporar no modelo a frequência na qual os eventos de rearranjo afetam o genoma de um determinado organismo. É importante notar que, do ponto de vista biológico, a frequência e o conjunto de eventos de rearranjo podem variar dependendo do organismo considerado. De um ponto de vista teórico, as abordagens possuem objetivos diferentes, apesar de compartilharem características comuns. Uma característica que difere da abordagem de proporção é que uma vez conhecida a frequência na qual os eventos afetam o genoma, a proporção pode ser facilmente derivada dessa informação, enquanto que na abordagem ponderada o peso associado a cada tipo de evento precisa ser ajustado e validado através de testes experimentais.

O Exemplo~\ref{example:DFGJNHTP} mostra uma solução ótimo $S$ para a instância clássica com sinais $(({+0}~{-1}~{+4}~{-8}~{+3}~{+5}~{+2}~{-7}~{-6}~{+9}),({+0}~{+1}~{+2}~{+3}~{+4}~{+5}~{+6}~{+7}~{+8}~{+9}))$ considerando os problemas \SbRT{} e \SbWRT{} (utilizando os pesos $2$ e $3$ para os eventos de reversão e transposição, respectivamente). Note que metade dos eventos de rearranjo de $S$ são reversões e a outra metade transposições, mesmo utilizando um custo maior para o evento de transposição.

\begin{example}\label{example:DFGJNHTP}
  \hfill \break
  % \scriptsize
  \begin{tabular}{lllll}
    $\pi    $ & $=$ &                                                             &     & $({+0}~{-1}~{+4}~{-8}~{+3}~{+5}~{+2}~{-7}~{-6}~{+9})                         $  \\
    $\pi^{1}$ & $=$ & $\pi \cdot \rho^{(1,5)}$                                    & $=$ & $({+0}~\underline{{-5}~{-3}~{+8}~{-4}~{+1}}~{+2}~{-7}~{-6}~{+9})             $  \\
    $\pi^{2}$ & $=$ & $\pi^{1} \cdot \tau^{(2,4,9)}$                              & $=$ & $({+0}~{-5}~\underline{{-4}~{+1}~{+2}~{-7}~{-6}}~\underline{{-3}~{+8}}~{+9}) $  \\
    $\pi^{3}$ & $=$ & $\pi^{2}\cdot \tau^{(1,3,7)}$                               & $=$ & $({+0}~\underline{{+1}~{+2}~{-7}~{-6}}~\underline{{-5}~{-4}}~{-3}~{+8}~{+9}) $  \\
    $\pi^{4}$ & $=$ & $\pi^{3} \cdot \rho^{(3,7)}$                                & $=$ & $({+0}~{+1}~{+2}~\underline{{+3}~{+4}~{+5}~{+6}~{+7}}~{+8}~{+9})             $  \\
    $S      $ & $=$ & $(\rho^{(1,5)},\tau^{(2,4,9)},\tau^{(1,3,7)},\rho^{(3,7)})$ &     &                                                                                 
  \end{tabular}
\end{example}


O Exemplo~\ref{example:MODRXOJQ} mostra uma solução ótima $S'$ para a mesma instância clássica com sinais apresentada no Exemplo~\ref{example:DFGJNHTP} considerando o problem \SbPRT{} e adotando um valor de $k = 0.6$, ou seja, pelo menos 60\% dos eventos de rearranjo em $S'$ devem ser reversões. Quando comparamos com o Exemplo~\ref{example:DFGJNHTP}, podemos perceber que $S'$ possui apenas um evento a mais que $S$, mas a proporção mínima de reversões em relação ao tamanho da sequência $S'$ é garantida.

\begin{example}\label{example:MODRXOJQ}
  \hfill \break
  % \scriptsize
  \begin{tabular}{lllll}
    $\pi    $ & $=$ &                                                                         &     & $({+0}~{-1}~{+4}~{-8}~{+3}~{+5}~{+2}~{-7}~{-6}~{+9})                         $ \\
    $\pi^{1}$ & $=$ & $\pi \cdot \rho^{(2,8)}$                                                & $=$ & $({+0}~{-1}~\underline{{+6}~{+7}~{-2}~{-5}~{-3}~{+8}~{-4}}~{+9})             $ \\
    $\pi^{2}$ & $=$ & $\pi^{1}\cdot \rho^{(2,4)}$                                             & $=$ & $({+0}~{-1}~\underline{{+2}~{-7}~{-6}}~{-5}~{-3}~{+8}~{-4}~{+9})             $ \\
    $\pi^{3}$ & $=$ & $\pi^{2} \cdot \tau^{(6,8,9)}$                                          & $=$ & $({+0}~{-1}~{+2}~{-7}~{-6}~{-5}~\underline{{-4}}~\underline{{-3}~{+8}}~{+9}) $ \\
    $\pi^{4}$ & $=$ & $\pi^{3} \cdot \rho^{(1,1)}$                                            & $=$ & $({+0}~\underline{{+1}}~{+2}~{-7}~{-6}~{-5}~{-4}~{-3}~{+8}~{+9})             $ \\
    $\pi^{5}$ & $=$ & $\pi^{4} \cdot \rho^{(3,7)}$                                            & $=$ & $({+0}~{+1}~{+2}~\underline{{+3}~{+4}~{+5}~{+6}~{+7}}~{+8}~{+9})             $ \\
    $S'     $ & $=$ & $(\rho^{(2,8)},\rho^{(2,4)},\tau^{(6,8,9)},\rho^{(1,1)},\rho^{(3,7)})$  &     &                                                                               
  \end{tabular}
\end{example}

Dada uma sequência de eventos de rearranjo $S$, denotamos por $|S|$ o tamanho da sequência $S$, ou seja, a quantidade de eventos em $S$. Além disso, denotamos por $|S_{\rho}|$ a quantidade de eventos de reversão em $S$. A seguir, descrevemos formalmente o problema de Ordenção de Permutações por Reversões e Transposições com Restrição de Proporção.

\begin{task}
  \problemtitle{Ordenção de Permutações por Reversões e Transposições com Restrição de Proporção (\SbPRT)}
  \probleminput{Uma instância clássica com ou sem sinais $\mathcal{I}=(\pi,\iota)$ e um número racional $k \in [0..1]$.}
  \problemtask{Com base no modelo de rearranjo $\mathcal{M}=\{\rho,\tau\}$, determinar uma sequência de eventos de rearranjo $S$ de tamanho mínimo capaz de transformar $\pi$ em $\iota$, tal que $\frac{|S_{\rho}|}{|S|} \ge k$.}
\end{task}

Dada uma instância clássica com ou sem sinais $\mathcal{I}=(\pi,\iota)$ e um número racional $k \in [0..1]$, a \emph{distância de propoção} entre $\pi$ e $\iota$, denotada por $dp_{k}(\mathcal{I})$, é o tamanho da menor sequências de eventos de rearranjo $S$, tal que todo evento de $S$ pertence ao modelo $\mathcal{M}=\{\rho,\tau\}$, $\pi \cdot S = \iota$ e $\frac{|S_{\rho}|}{|S|} \ge k$.

Nesse capítulo, provamos que o problema \SbPRT{} pertence à classe NP-difícil em instâncias clássicas sem sinais para qualquer valor de $k$. Em instâncias clássicas com sinais mostramos que existe um algoritmo exato polinomial para o problema quando $k=1$ e provamos que o problema pertence à classe NP-difícil quando $k < 1$. Para as variações com e sem sinais do problema \SbPRT{} apresentamos algoritmos de aproximação com fatores $3 - \frac{3k}{2}$ e $3-k$, respectivamente. Além disso, apresentamos um algoritmo de aproximação assintótico com um fator teórico melhor para instâncias clássicas com sinais. Por fim, realizamos experimentos comparando o desempenho práticos dos algoritmo propostos.  


% ------------------------------------------------------------------ %
\section{Limitantes Inferiores}
% ------------------------------------------------------------------ %

Nessa seção, apresentamos limitantes inferiores para as variações com e sem sinais do problema \SbPRT{}.

\begin{lemma}[Kececioglu e Sankoff~\cite{1995-kececioglu-sankoff}]\label{lemma:QIRAVPQT}
Dada uma instância clássica sem sinais $\mathcal{I} = (\pi,\iota)$, para qualquer reversão $\rho$ temos que $\Delta b_1(\mathcal{I}, S = (\rho)) \ge -2$.
\end{lemma}

\begin{lemma}[Walter \textit{et al.}~\cite{1998-walter-etal}]\label{lemma:NJATEDCC}
Dada uma instância clássica sem sinais $\mathcal{I} = (\pi,\iota)$, para qualquer transposição $\tau$ temos que $\Delta b_1(\mathcal{I}, S = (\tau)) \ge -3$.
\end{lemma}

\begin{lemma}\label{lemma:JYYZBREC}
Dada uma instância clássica sem sinais $\mathcal{I} = (\pi,\iota)$ para o problema $\SbPRT{}$ considerando a proporção $k \in [0..1]$ e seja $S$ uma sequência ótima de eventos de rearranjo para o problema. O número de breakpoints tipo um removidos por cada evento de $S$, em média, é menor ou igual a $3-k$.
\end{lemma}
\begin{proof}
Como $S$ é uma sequência ótima para a instância $\mathcal{I}$ com base na proporção $k$, temos que pelo menos $|S|k$ eventos presentes em $S$ são reversões. Pelos lemas~\ref{lemma:QIRAVPQT} e \ref{lemma:NJATEDCC}, temos que uma reversão pode remover até dois breakpoints tipo um enquanto uma transposição pode remover até três. Seja $\phi b(S)$ o número médio de breakpoints tipo um removidos por um evento de $S$, temos que:
$$\phi b(S) \le \frac{(2 |S| k) + (3 |S| (1 - k))}{|S|} = 2k + 3(1 - k) = 3 - k.$$ 
\end{proof}

\begin{lemma}[Hannenhalli e Pevzner~\cite{1999-hannenhalli-pevzner}]\label{lemma:MYFTFTWE}
Dada uma instância clássica com sinais $\mathcal{I} = (\pi,\iota)$, para qualquer reversão $\rho$ temos que $\Delta b_2(\mathcal{I}, S = (\rho)) \ge -2$.
\end{lemma}

\begin{lemma}\label{lemma:AKEQRMOY}
Dada uma instância clássica com sinais $\mathcal{I} = (\pi,\iota)$, para qualquer transposição $\tau$ temos que $\Delta b_2(\mathcal{I}, S = (\tau)) \ge -3$.
\end{lemma}
\begin{proof}
Note que uma transposição pode afetar no máximo três breakpoints tipo dois de $\mathcal{I}$. Logo, no melhor cenário, os três breakpoints são removidos e o lema segue.
\end{proof}

\begin{lemma}\label{lemma:ZZLNPRWJ}
Dada uma instância clássica com sinais $\mathcal{I} = (\pi,\iota)$ para o problema $\SbPRT{}$ considerando a proporção $k \in [0..1]$ e seja $S$ uma sequência ótima de eventos de rearranjo para o problema. O número de breakpoints tipo dois removidos por cada evento de $S$, em média, é menor ou igual a $3-k$.
\end{lemma}
\begin{proof}
A prova é similar a descrita no Lema~\ref{lemma:JYYZBREC}, mas considerando os lemas~\ref{lemma:MYFTFTWE} e \ref{lemma:AKEQRMOY}.
\end{proof}

\begin{lemma}[Hannenhalli e Pevzner~\cite{1999-hannenhalli-pevzner}]\label{lemma:WMUBEYJS}
Dada uma instância clássica com sinais $\mathcal{I} = (\pi,\iota)$, para qualquer reversão $\rho$ temos que $\Delta c(G(\mathcal{I}), S = (\rho)) \le 1$.
\end{lemma}

\begin{lemma}[Bafna e Pevzner~\cite{1995b-bafna-pevzner}; Walter \textit{et al.}~\cite{1998-walter-etal}]\label{lemma:WITSEXYZ}
Dada uma instância clássica com ou sem sinais $\mathcal{I} = (\pi,\iota)$, para qualquer transposição $\tau$ temos que $\Delta c(G(\mathcal{I}), S = (\tau)) \le 2$.
\end{lemma}

\begin{lemma}\label{lemma:IMSCPWKN}
Dada uma instância clássica com sinais $\mathcal{I} = (\pi,\iota)$ para o problema $\SbPRT{}$ considerando a proporção $k \in [0..1]$ e seja $S$ uma sequência ótima de eventos de rearranjo para o problema. A variação no número de ciclos para cada evento de $S$, em média, é menor ou igual a $2-k$.
\end{lemma}
\begin{proof}
Como $S$ é uma sequência ótima para a instância $\mathcal{I}$ com base na proporção $k$, temos que pelo menos $|S|k$ eventos presentes em $S$ são reversões. Pelos lemas~\ref{lemma:WMUBEYJS} e \ref{lemma:WITSEXYZ}, temos que uma reversão pode criar até um novo ciclo enquanto uma transposição pode criar até dois. Seja $\phi c(S)$ o número médio de ciclos criados por um evento de $S$, temos que:
$$\phi c(S) \leq \frac{(1 |S| k) + (2 |S| (1 - k))}{|S|} = 1k + 2(1 - k) = 2 - k.$$
\end{proof}

\begin{theorem}\label{theorem:KJXGKJIP}
Dada uma instância clássica sem sinais $\mathcal{I} = (\pi,\iota)$ para o problema $\SbPRT{}$ e uma proporção $k \in [0..1]$, temos que $dp_{k}(\mathcal{I}) \ge \frac{b_1(\mathcal{I})}{3-k}$.
\end{theorem}
\begin{proof}
Como $b_1(\mathcal{I})$ breakpoints tipo um devem ser removidos para transformar $\pi$ em $\iota$ e, pelo Lema~\ref{lemma:JYYZBREC}, até $3-k$ breakpoints tipo um são removidos, em média, por cada operação de uma sequência ótima para o problema. Logo, o teorema segue.
\end{proof}

\begin{theorem}\label{theorem:ABGZQHIL}
Dada uma instância clássica com sinais $\mathcal{I} = (\pi,\iota)$ para o problema $\SbPRT{}$ e uma proporção $k \in [0..1]$, temos que $dp_{k}(\mathcal{I}) \ge \frac{b_2(\mathcal{I})}{3-k}$.
\end{theorem}
\begin{proof}
A prova é similar a descrita no Teorema~\ref{theorem:KJXGKJIP}, mas considerando o número de breakpoints tipo dois em $\mathcal{I}$ e o Lema~\ref{lemma:ZZLNPRWJ}.
\end{proof}

\begin{theorem}\label{theorem:ABGZQHIL}
Dada uma instância clássica com sinais $\mathcal{I} = (\pi,\iota)$ para o problema $\SbPRT{}$ e uma proporção $k \in [0..1]$, temos que $dp_{k}(\mathcal{I}) \ge \frac{n + 1 - c(G(\mathcal{I}))}{2-k}$.
\end{theorem}
\begin{proof}
Note que, pela Obervação~\ref{remark:OYRVGHTB}, $n+1 - c(G(\mathcal{I}))$ novos ciclos precisam ser criados para transformar $\pi$ em $\iota$. Pelo Lema~\ref{lemma:IMSCPWKN}, até $2-k$ novos ciclos são criados, em média, por cada operação de uma sequência ótima para o problema. Logo, o teorema segue.
\end{proof}

% ------------------------------------------------------------------ %
\section{Análise de Complexidade}
% ------------------------------------------------------------------ %

Nessa seção, apresentamos uma análise de complexidade do problema \SbPRT{} em instâncias clássicas com e sem sinais para todos os possível valores de $k$. A seguir descrevemos formalmente a versão de decisão do problema \SbPRT{}.

\begin{decision}
  \problemtitle{\SbPRT (Versão de Decisão)}
  \probleminput{Uma instância clássica com ou sem sinais $\mathcal{I}=(\pi,\iota)$, um número racional $k \in [0..1]$ e um número natural $t$.}
  \problemquestion{Existe uma sequência de eventos de rearranjo $S$, com base no modelo de rearranjo $\mathcal{M}=\{\rho,\tau\}$, capaz de transformar $\pi$ em $\iota$, tal que $\frac{|S_{\rho}|}{|S|} \ge k$ e $|S| = t$?}
\end{decision}

Note que para o problema \SbPRT{} é possível fornecer como entrada diferentes valores para $k$. Entretanto, quando utilizamos o valor de $k=0$ obtemos o problema \SbRT{}, uma vez que estipulamos que em uma solução não é necessário obter uma porcentagem mínima de eventos de reversão em comparação ao tamanho da sequência de eventos de rearranjo. Por outro lado, quando adotamos o valor de $k=1$, obtemos o problema \SbR{}. Note que nesse caso toda solução pra o problema deve ser composta exclusivamente por eventos de reversão. Com base nessa característica do problema obtemos os seguintes lemmas.

\begin{lemma}
O problema \SbPRT{} em instâncias clássicas com sinais pertence à classe NP-difícil quando $k=1$ e existe um algoritmo exato polinomial quando $k=0$.
\end{lemma}
\begin{proof}
Quando $k=0$ o problema \SbPRT{} em instâncias clássicas com sinais torna-se a variação com sinais do problema \SbRT{}, que é NP-difícil~\cite{2019b-oliveira-etal}. Por outro lado, quando $k=1$ o problema \SbPRT{} em instâncias clássicas com sinais torna-se a variação com sinais do problema \SbR{}, que possui um algoritmo exato polinomial~\cite{1999-hannenhalli-pevzner}.
\end{proof}

\begin{lemma}
O problema \SbPRT{} em instâncias clássicas sem sinais pertence à classe NP-difícil quando $k \in \{0,1\}$.
\end{lemma}
\begin{proof}
Quando $k=0$ e $k=1$ o problema \SbPRT{} em instâncias clássicas sem sinais torna-se a variação sem sinais dos problemas \SbRT{} e \SbR{}, respectivamente. Ambos os problemas pertencem à classe NP-difícil~\cite{2019b-oliveira-etal,1999-caprara}.
\end{proof}

A seguir investigamos a complexidade do problema \SbPRT{} quando $k$ pertence ao intervalo $(0..1)$. Para isso, apresentamos definições que serão utilizadas para provar a complexidade do problema para esse intervalo de valores de $k$. As transformações de \emph{duplicação}, \emph{orientação}, \emph{extensão bridge} e \emph{extensão gadget} descritas a seguir utilizam uma representação clássica de um genoma na sua forma não estendida. Caso a representação esteja na forma estendida, os elementos $\pi_0$ e $\pi_{n+1}$ são ignorados, a transformação é aplicada e a nova representação clássica resultante é então estendida.

\begin{definition}
Dada uma representação clássica sem sinais $\pi$ de tamanho $n$, a \emph{duplicação} cria uma representação clássica sem sinais $\pi'$ de tamanho $2n$ de forma que cada elemento $\pi_i \in \pi$ é mapeado em dois novos valores, com $\pi'_{2i-1} = 2\pi_i-1$ e $\pi'_{2i} = 2\pi_i$, para $i \in [1..n]$.
\end{definition}

O Exemplo~\ref{example:QJCKPQSS} mostra o transformação de duplicação sendo aplicado na representação clássica sem sinais $\pi=(4~1~5~3~2)$.

\begin{example}\label{example:QJCKPQSS}
  \hfill
  \begin{\position}
    \begin{tabular}{lll}
      $\pi$  & $=$ & $(4~1~5~3~2)$ \\
      $\pi'$ & $=$ & $(7~8~1~2~9~10~5~6~3~4)$ \\
    \end{tabular}
  \end{\position}
\end{example}

\begin{definition}
Dada uma representação clássica sem sinais $\pi$ de tamanho $n$, a \emph{orientação} cria uma representação clássica com sinais $\pi'$ também de tamanho $n$ de forma que $\pi'_{i} = +\pi_i$, para $i \in [1..n]$.
\end{definition}

O Exemplo~\ref{example:GUELXUJE} mostra a transformação de orientação sendo aplicado na representação clássica sem sinais $\pi=(4~1~5~3~2)$.

\begin{example}\label{example:GUELXUJE}
  \hfill \break
  % \centering
  \begin{tabular}{lll}
    $\pi$  & $=$ & $(4~1~5~3~2)$ \\
    $\pi'$ & $=$ & $({+4}~{+1}~{+5}~{+3}~{+2})$ \\
  \end{tabular}
\end{example}

\begin{definition}
Dada uma representação clássica com ou sem sinais $\pi$ de tamanho $n$, a \emph{extensão bridge} cria uma representação clássica $\pi'$ de tamanho $n + 3$. Caso $\pi$ seja uma representação com sinais, $\pi'$ é gerado da seguinte forma: (i) $\pi'_{i} = \pi_i$ e (ii) $\pi'_{n+j} = +(n{+j})$, para $i \in [1..n]$ e $j \in [1..3]$. Caso contrário, $\pi'$ é gerado da seguinte forma: (i) $\pi'_{i} = \pi_i$ e (ii) $\pi'_{n+j} = n{+j}$, para $i \in [1..n]$ e $j \in [1..3]$.
\end{definition}

O Exemplo~\ref{example:AWNIOTEZ} mostra a transformação de extensão bridge sendo aplicada na representação clássica com sinais $\pi=({+4}~{+1}~{+5}~{-3}~{-2})$.

\begin{example}\label{example:AWNIOTEZ}
  \hfill \break
  % \centering
  \begin{tabular}{lll}
    $\pi$  & $=$ & $({+4}~{+1}~{+5}~{-3}~{-2})$ \\
    $\pi'$ & $=$ & $({+4}~{+1}~{+5}~{-3}~{-2}~{+6}~{+7}~{+8})$ \\
  \end{tabular}
\end{example}

O Exemplo~\ref{example:BYQQHUAS} mostra a transformação de extensão bridge sendo aplicada na representação clássica sem sinais $\pi=(4~1~5~3~2)$.

\begin{example}\label{example:BYQQHUAS}
  \hfill \break
  % \centering
  \begin{tabular}{lll}
    $\pi$  & $=$ & $(4~1~5~3~2)$ \\
    $\pi'$ & $=$ & $(4~1~5~3~2~6~7~8)$ \\
  \end{tabular}
\end{example}

\begin{definition}
Dada uma representação clássica com ou sem sinais $\pi$ de tamanho $n$, a \emph{extensão gadget} cria uma representação clássica $\pi'$ de tamanho $n + 6$. Caso $\pi$ seja uma representação com sinais, $\pi'$ é gerado da seguinte forma: (i) $\pi'_{i} = \pi_i$; (ii) $\pi'_j = -(n+4-j)$; (iii) $\pi'_{n+k} = +(n{+k})$, para $i \in [1..n]$, $j \in [1..3]$ e $k \in [4..6]$. Caso contrário, $\pi'$ é gerado da seguinte forma: (i) $\pi'_{i} = \pi_i$; (ii) $\pi'_j = n+4-j$; (iii) $\pi'_{n+k} = n{+k}$, para $i \in [1..n]$, $j \in [1..3]$ e $k \in [4..6]$.
\end{definition}

O Exemplo~\ref{example:TCTQPMWV} mostra a transformação de extensão gadget sendo aplicada na representação clássica com sinais $\pi=({+4}~{+1}~{+5}~{-3}~{-2})$.

\begin{example}\label{example:TCTQPMWV}
  \hfill
  \begin{\position}
    \begin{tabular}{lll}
      $\pi$  & $=$ & $({+4}~{+1}~{+5}~{-3}~{-2})$ \\
      $\pi'$ & $=$ & $({+4}~{+1}~{+5}~{-3}~{-2}~{-8}~{-7}~{-6}~{+9}~{+10}~{+11})$ \\
    \end{tabular}
  \end{\position}
\end{example}

O Exemplo~\ref{example:ZMGTJRFE} mostra a transformação de extensão gadget sendo aplicada na representação clássica sem sinais $\pi=(4~1~5~3~2)$.

\begin{example}\label{example:ZMGTJRFE}
  \hfill \break
  % \centering
  \begin{tabular}{lll}
    $\pi$  & $=$ & $(4~1~5~3~2)$ \\
    $\pi'$ & $=$ & $(4~1~5~3~2~8~7~6~9~10~11)$ \\
  \end{tabular}
\end{example}

A seguir descrevemos formalmente a versão de decisão do problema de Ordenação de Permutações por 3-Transposições (\textbf{B3T}).

\begin{decision}
  \problemtitle{\textbf{B3T} (Versão de Decisão)}
  \probleminput{Uma instância clássica sem sinais $\mathcal{I}=(\pi,\iota)$, tal que $b_2(\mathcal{I}) = 3s$ e $s$ é um número natural não nulo.}
  \problemquestion{Existe uma sequência de eventos de rearranjo $S$, com base no modelo de rearranjo $\mathcal{M}=\{\tau\}$, capaz de transformar $\pi$ em $\iota$, tal que $|S| = \frac{b_2(\mathcal{I})}{3}$?}
\end{decision}

Bulteau e coautores~\cite{2012-bulteau-etal} provaram que o problem \textbf{B3T} pertence à classe NP-difícil. Utilizaremos uma redução do problema \textbf{B3T} para provar que o problema \SbPRT{} é NP-difícil quando $k$ pertence ao intervalo $(0..1)$.

\begin{lemma}[Oliveira~\textit{et al.}~\cite{2019b-oliveira-etal}]\label{lemma:CWNRJAPM}
Se uma instância clássica com sinais $\mathcal{I}=(\pi,\iota)$ possui apenas strips positivas, para qualquer reversão $\rho$ temos que $\Delta b_2(\mathcal{I}, S=(\rho)) \ge 0$.
\end{lemma}

\begin{theorem}\label{theorem:NSWQYFLG}
O problema \SbPRT{} em instâncias clássicas com sinais pertence à classe NP-difícil quando $k \in (0..1)$.
\end{theorem}
\begin{proof}
Dada uma instância clássica sem sinais $\mathcal{I}=(\pi,\iota)$ para o problema \textbf{B3T}, definimos $\ell = \frac{b_2(\mathcal{I})}{3} \ge 1$. Criamos uma instância clássica com sinais $\mathcal{I'}=(\pi',\iota')$ para o problema \SbPRT{} da seguinte maneira:

\begin{enumerate}
  \item Seja $\sigma$ uma representação clássica com sinais de tamanho $n+3$ obtida através do processo de orientação aplicado em $\pi$ e seguido da extenção bridge.
  \item Seja $k$ um número racional no intervalo $(0..1)$, definimos $p = \lceil\frac{\ell k}{1-k}\rceil \ge 1$, ou seja, $p$ é o menor número inteiro tal que $\frac{p}{p+\ell} \ge k$.
  \item Seja $\pi'$ uma representação clássica com sinais de tamanho $n+3+6p$ obtida através da aplicação consecutiva de $p$ extensões gadget em $\sigma$.
  \item Seja $\iota'$ uma representação clássica com sinais de tamanho $n+3+6p$. Caso $\pi$ esteja na sua forma estendida, $\iota'_i = +i$ para $i \in [1..(n+3+6p)]$. Caso contrário,  $\iota'_i = +i$ para $i \in [0..(n+3+6p-1)]$.
\end{enumerate}

O Exemplo~\ref{example:NDFPEMFC} mostra o processo de criação de uma instância clássica com sinais $\mathcal{I'}=(\pi',\iota')$ para o problema \SbPRT{} a partir de uma instância clássica sem sinais $\mathcal{I}=(\pi,\iota)$ para o problema \textbf{B3T}. Note que em ambas as instâncias os genomas de origem e alvo são representados na forma estendida. Além disso, é importante lembrar que o problema \textbf{B3T} e a variação com sinais do problema \SbPRT{} utilizam breakpoints tipo dois. Note que a transformação de orientação preserva o número de breakpoints tipo dois, já que adicionamos apenas um sinal positivo aos elementos da permutação. A extensão bridge também preserva o número breakpoints tipo dois, já que adiciona apenas três elementos consecutivos ao final da permutação. Por outro lado, cada extensão gadget adiciona dois novos breakpoints tipo dois (ou seja, as extremidades de cada strip negativa), então $b_2(\mathcal{I'}) = b_2(\mathcal{I}) +2p$.

Agora mostramos que a instância $\mathcal{I}$ do problema \textbf{B3T} é satisfeita se e somente se $dp_k(\mathcal{I'}) \le \ell+p$.

($\Rightarrow$) Suponha que existe uma sequência $S$ com $\ell$ transposições, tal que $\pi \cdot S = \iota$. Isso significa que cada transposição de $S$ remove exatamente três breakpoints tipo dois de $\mathcal{I}$. Considere a sequência $S'$ como sendo uma cópia de $S$ e incluindo $p$ reversões, de forma que cada reversão é aplicada sobre uma strip negativa de $\mathcal{I'}$. Como $\pi'_i = +\pi_i$ para $i \in [1..n]$, cada transposição de $S'$ também remove exatamente três breakpoints tipo dois, restando apenas $2p$ breakpoints tipo dois para serem removidos. Contúdo, cada reversão $\rho \in S'$ remove dois breakpoints tipo dois (criados pela extensão gadget). Logo, $|S'| = \ell+p$, $\pi' \cdot S' = \iota'$ e $\frac{|S_{\rho}'|}{|S'|}$.

($\Leftarrow$) Pelo Teorema~\ref{theorem:ABGZQHIL}, temos que $dp_k(\mathcal{I'}) \ge \frac{b_2(\mathcal{I'})}{3-k} = \frac{b_2(\mathcal{I})+2p}{3-k}$. Como temos por construção que $b_2(\mathcal{I}) = 3\ell$ e $\frac{p-1}{\ell+(p-1)} < k \leq \frac{p}{\ell+p}$, segue que $dp_k(\mathcal{I'}) > \frac{(\ell+p-1)(3\ell+2p)}{3\ell+2p-2}$. Além disso, $\ell \geq 1$ e $p \geq 1$, então $\frac{3\ell+2p}{3\ell+2p-2} > 1$ e $dp_k(\mathcal{I'}) > \ell+p-1$, o que resulta em $d_k(\mathcal{I'}) \ge \ell + p$. Suponha que existe uma sequência de eventos de rearranjo $S'$ de tamanho $\ell + p$, tal que $\pi' \cdot S' = \iota'$ e $\frac{|S_{\rho}'|}{|S'|} \ge k$.

Como $b_2(\mathcal{I'}) = 3\ell+2p$, então deve existir pelo menos $\ell$ transposições em $S'$ com cada uma removendo três breakpoints tipo dois. Caso contrário, $S'$ não seria capaz de transformar $\pi'$ em $\iota'$. Além disso, deve exitir no máximo $\ell$ transpositions in $S'$. Caso contrário, a proporção $\frac{|S_{\rho}'|}{|S'|}$ não seria satisfeita. Dessa forma, temos que existe $\ell$ transposições em $S'$ com cada uma removendo três breakpoints tipo dois. Logo, restam $|S'| - \ell = \ell+p - \ell = p$ reversões em $S'$, e cada reversão deve remover dois breakpoints tipo dois. Caso contrário, $S'$ não seria capaz de transformar $\pi'$ em $\iota'$.

Vamos definir três tipos de elementos em $\pi'$. Dizemos que um dado elemento $\pi'_i$ é (i) original se $i \in [1..n]$; (ii) transitório se $i \in [n{+1}..n{+3}]$; e (iii) estendido se $i > n{+3}$. Como os elementos originais e transitórios são todos positivos, as strips nas primeiras $n+3$ posições são todas positivas. Pelo Lemma~\ref{lemma:CWNRJAPM}, nenhuma reversão $\rho$ aplicada nesses elementos remove breakpoints tipo dois, e isto permanece verdadeiro enquanto as transposições afetam apenas os elementos originais.

Como não é aplicada nenhuma reversão aos elementos originais, os $3\ell$ breakpoints tipo dois $(\pi'_i,\pi'_{i+1})$, tal que pelo menos $\pi'_i$ é um elemento original, devem ser removidos por transposições. Dessa forma, $S'$ possui $\ell$ transposições $\tau^{(i,j,k)}$ de tal maneira que $1 \le i < j < k \le n+1$ (ou seja, as transposições afetam apenas os elementos originais). 

Os restantes eventos de rearranjo de $S'$, ou seja, as $p$ reversões, devem remover $2p$ breakpoints tipo dois $(\pi'_i,\pi'_{i+1})$, de tal forma que pelo menos $\pi'_{i+1}$ seja um elemento estendido (ou seja, $i \ge n+3$). A cada iteração, as únicas reversões que removem dois breakpoints tipo dois são aquelas aplicadas nas duas extremidades de uma strip negativa, implicando que cada reversão de $S'$ é aplicada em uma das $p$ strips negativas adicionadas pelas extensões gadget.

Perceba que $S'$ possui $\ell$ transposições que removem $3\ell$ breakpoints tipo dois $(\pi'_i,\pi'_{i+1})$, tal que $i \le n$. Seja $S$ uma sequência de transposições criada a partir das transposições de $S'$ mantendo a mesma ordem relativa. Como $\pi'_i = +\pi_i$ para $i \in [1..n]$, $\pi \cdot S = \iota$, e o teorema segue.
\end{proof}

\begin{example}\label{example:NDFPEMFC}
  Dada a instância clássica sem sinais $\mathcal{I} = ((0 \; 3 \; 5 \; 1 \; 4 \; 2 \; 6),(0 \; 1 \; 2 \; 3 \; 4 \; 5 \; 6))$ para o problema \textbf{B3T}, temos que $b_2(\mathcal{I}) = 6$. Para a criação da instância clássica com sinais $\mathcal{I'}=(\pi',\iota')$ para o problema \SbPRT{} temos no passo 1 a obteção da representação clássica com sinais $\sigma = ({+0} \; {+3} \; {+5} \; {+1} \; {+4} \; {+2} \; {+6} \; {+7} \; {+8} \; {+9})$. Usando $k = 0.3$, temos que $p = \lceil\frac{2\times 0.3}{1 - 0.3}\rceil = \lceil\frac{0.6}{0.7}\rceil = 1$ no passo 2. No passo 3, obtemos a representação clássica com sinais $\pi' = ({+0} \; {+3} \; {+5} \; {+1} \; {+4} \; {+2} \; {+6} \; {+7} \; {+8} \; {-11} \; {-10} \; {-9} \; {+12} \; {+13} \; {+14} \; {+15})$ após aplicar $p = 1$ extenções gadget em $\sigma$. No passo 4, obtemos a representação clássica com sinais $\iota' = ({+0} \; {+1} \; {+2} \; {+3} \; {+4} \; {+5} \; {+6} \; {+7} \; {+8} \; {+9} \; {+10} \; {+11} \; {+12} \; {+13} \; {+14} \; {+15})$. Note que $b_2(\mathcal{I'}) = b_2(\mathcal{I}) + 2p = 6 + 2 = 8$. A sequência $S = (\tau^{(1,3,6)},\tau^{(2,3,5)})$ é tal que $\pi \cdot S = \iota$ e $|S| = 2 = \frac{b_2(\mathcal{I})}{3} = \ell$, e a sequência $S' = (\tau^{(1,3,6)},\tau^{(2,3,5)},\rho^{(9,11)})$ que possui a mesma sequência de transposições de $S$ é tal que (i) $\pi' \cdot S' = \iota'$; (ii) $\frac{|S'_\rho|}{|S'|} = 0.333 \ge 0.3 = k$; e (iii) $|S'| = 3 = \frac{b_2(\mathcal{I})}{3} + 1 = \ell+p$.
\end{example}

\begin{lemma}[Oliveira~\textit{et al.}~\cite{2019b-oliveira-etal}]\label{lemma:PXXMRMWO}
Se uma instância clássica sem sinais $\mathcal{I}=(\pi,\iota)$ possui apenas strips crescentes, para qualquer reversão $\rho$ temos que $\Delta b_1(\mathcal{I}, S=(\rho)) \ge 0$.
\end{lemma}

\begin{theorem}\label{theorem:QMHEKDLW}
O problema \SbPRT{} em instâncias clássicas sem sinais pertence à classe NP-difícil quando $k \in (0..1)$.
\end{theorem}
\begin{proof}
Dada uma instância clássica sem sinais $\mathcal{I}=(\pi,\iota)$ para o problema \textbf{B3T}, definimos $\ell = \frac{b_2(\mathcal{I})}{3} \ge 1$. Criamos uma instância clássica com sinais $\mathcal{I'}=(\pi',\iota')$ para o problema \SbPRT{} da seguinte maneira:

\begin{enumerate}
  \item Seja $\sigma$ uma representação clássica sem sinais de tamanho $2n+3$ obtida através do processo de duplicação aplicado em $\pi$ e seguido da extenção bridge.
  \item Seja $k$ um número racional no intervalo $(0..1)$, definimos $p = \lceil\frac{\ell k}{1-k}\rceil \ge 1$, ou seja, $p$ é o menor número inteiro tal que $\frac{p}{p+\ell} \ge k$.
  \item Seja $\pi'$ uma representação clássica sem sinais de tamanho $2n+3+6p$ obtida através da aplicação consecutiva de $p$ extensões gadget em $\sigma$.
  \item Seja $\iota'$ uma representação clássica com sinais de tamanho $2n+3+6p$. Caso $\pi$ esteja na sua forma estendida, $\iota'_i = i$ para $i \in [1..(n+3+6p)]$. Caso contrário,  $\iota'_i = i$ para $i \in [0..(n+3+6p-1)]$.
\end{enumerate}

O Exemplo~\ref{example:QBULCCOI} mostra o processo de criação de uma instância clássica sem sinais $\mathcal{I'}=(\pi',\iota')$ para o problema \SbPRT{} a partir de uma instância clássica sem sinais $\mathcal{I}=(\pi,\iota)$ para o problema \textbf{B3T}. Note que em ambas as instâncias os genomas de origem e alvo são representados na forma estendida. Note que, exceto por $\sigma_0$, cada elemento em posições pares de $\sigma$ é igual ao elemento à sua esquerda mais um. Isto significa que (i) exceto para a primeira e última strip, qualquer outra strip em $\sigma$ deve ter pelo menos dois elementos, ou seja, não existem singletons, e (ii) cada strip de $\sigma$ é crescente. Estas observações também são válidas para as primeiras $2n+3$ posições de $\pi'$. Além disso, é importante lembrar que o problema \textbf{B3T} utiliza breakpoints tipo dois enquanto a variação sem sinais do problema \SbPRT{} utiliza breakpoints tipo um. Note que (i) para cada breakpoint tipo dois $(\pi_i,\pi_{i+1})$ de $\mathcal{I}$ existe um breakpoint tipo um $(\pi'_{2i},\pi'_{2i+1})$ em $\mathcal{I'}$ (criado durante a transformação de duplicação), (ii) os pares $(\pi'_{2i-1},\pi'_{2i})$ não são breakpoints tipo um, para $i \in [1..n]$ e (iii) os pares $(\pi'_{2n+j},\pi'_{2n+j+1})$ não são breakpoints tipo um, para $j \in [1..3]$. Por outro lado, cada extensão gadget adiciona dois novos breakpoints tipo um (ou seja, as extremidades de cada strip decrescente), então $b_2(\mathcal{I'}) = b_1(\mathcal{I}) +2p$.

Agora mostramos que a instância $\mathcal{I}$ do problema \textbf{B3T} é satisfeita se e somente se $dp_k(\mathcal{I'}) \le \ell+p$.

($\Rightarrow$) Suponha que existe uma sequência $S$ com $\ell$ transposições, tal que $\pi \cdot S = \iota$. Isso significa que cada transposição de $S$ remove exatamente três breakpoints tipo dois de $\mathcal{I}$. Considere a sequência $S'$ criada da seguinte forma: (i) para cada transposição $\tau^{(i,j,k)}$ de $S$, seguindo a ordem relativa, adicione em $S'$ a transposição $\tau^{(2i-1,2j-1,2k-1)}$; (ii) Em seguida, adicione $p$ reversões em $S'$, de forma que cada reversão é aplicada sobre uma strip decrescente de $\mathcal{I'}$. Note que cada transposição de $S$ remove três breakpoints tipo dois de $\mathcal{I}$. Como temos que para cada breakpoint tipo dois $(\pi_i,\pi_{i+1})$ em $\mathcal{I}$ temos um breakpoint tipo um $(\pi'_{2i},\pi'_{2i+1})$ em $\mathcal{I'}$, isso significa que cada transposição de $S'$ remove três breakpoints tipo um de $\mathcal{I'}$. Com isso, restam apenas $2p$ breakpoints tipo um para serem removidos em $\mathcal{I'}$. Contúdo, cada reversão $\rho \in S'$ remove dois breakpoints tipo dois (criados pela extensão gadget). Logo, $|S'| = \ell+p$, $\pi' \cdot S' = \iota'$ e $\frac{|S_{\rho}'|}{|S'|}$.

($\Leftarrow$) Pelo Teorema~\ref{theorem:KJXGKJIP}, temos que $dp_k(\mathcal{I'}) \ge \frac{b_1(\mathcal{I'})}{3-k} = \frac{b_2(\mathcal{I})+2p}{3-k}$. Como temos por construção que $b_2(\mathcal{I}) = 3\ell$ e $\frac{p-1}{\ell+(p-1)} < k \leq \frac{p}{\ell+p}$, segue que $dp_k(\mathcal{I'}) > \frac{(\ell+p-1)(3\ell+2p)}{3\ell+2p-2}$. Além disso, $\ell \geq 1$ e $p \geq 1$, então $\frac{3\ell+2p}{3\ell+2p-2} > 1$ e $dp_k(\mathcal{I'}) > \ell+p-1$, o que resulta em $d_k(\mathcal{I'}) \ge \ell + p$. Suponha que existe uma sequência de eventos de rearranjo $S'$ de tamanho $\ell + p$, tal que $\pi' \cdot S' = \iota'$ e $\frac{|S_{\rho}'|}{|S'|} \ge k$.

Como $b_1(\mathcal{I'}) = 3\ell+2p$, então deve existir pelo menos $\ell$ transposições em $S'$ com cada uma removendo três breakpoints tipo um. Caso contrário, $S'$ não seria capaz de transformar $\pi'$ em $\iota'$. Além disso, deve exitir no máximo $\ell$ transpositions in $S'$. Caso contrário, a proporção $\frac{|S_{\rho}'|}{|S'|}$ não seria satisfeita. Dessa forma, temos que existe $\ell$ transposições em $S'$ com cada uma removendo três breakpoints tipo um. Logo, restam $|S'| - \ell = \ell+p - \ell = p$ reversões em $S'$, e cada reversão deve remover dois breakpoints tipo um. Caso contrário, $S'$ não seria capaz de transformar $\pi'$ em $\iota'$.

Vamos definir três tipos de elementos em $\pi'$. Dizemos que um dado elemento $\pi'_i$ é (i) original se $i \in [1..2n]$; (ii) transitório se $i \in [2n{+1}..2n{+3}]$; e (iii) estendido se $i > 2n{+3}$. Como todos elementos originais e transitórios fazem parte de uma strip crescente, pelo Lemma~\ref{lemma:PXXMRMWO}, nenhuma reversão $\rho$ aplicada nesses elementos remove breakpoints tipo um, e isto permanece verdadeiro enquanto as transposições afetam breakpoints tipo um entre os elementos originais.

Como não é aplicada nenhuma reversão aos elementos originais, os $3\ell$ breakpoints tipo um $(\pi'_i,\pi'_{i+1})$, tal que pelo menos $\pi'_i$ é um elemento original, devem ser removidos por transposições. Dessa forma, $S'$ possui $\ell$ transposições $\tau^{(i,j,k)}$ de tal maneira que $1 \le i < j < k \le 2n+1$ (ou seja, as transposições afetam apenas os elementos originais). 

Os restantes eventos de rearranjo de $S'$, ou seja, as $p$ reversões, devem remover $2p$ breakpoints tipo um $(\pi'_i,\pi'_{i+1})$, de tal forma que pelo menos $\pi'_{i+1}$ seja um elemento estendido (ou seja, $i \ge 2n+3$). A cada iteração, as únicas reversões que removem dois breakpoints tipo um são aquelas aplicadas nas duas extremidades de uma strip decrescente, implicando que cada reversão de $S'$ é aplicada em uma das $p$ strips decrescentes adicionadas pelas extensões gadget.

Perceba que $S'$ possui $\ell$ transposições que removem $3\ell$ breakpoints tipo um $(\pi'_i,\pi'_{i+1})$, tal que $i \le 2n$. Seja $S$ uma sequência de transposições criada a partir das transposições de $S'$ da seguinte forma: (i) mantendo a mesma ordem relativa, para cada transposição $\tau^{(i,j,k)}$ de $S'$ adicione em $S$ a transposição $\tau^{(\frac{i+1}{2},\frac{j+1}{2},\frac{k+1}{2})}$. Como mapeamento feito reflete que cada transposição em $S$ remove três breakpoints tipo dois de $\mathcal{I}$, temos que $\pi \cdot S = \iota$, e o teorema segue.
\end{proof}

\begin{example}\label{example:QBULCCOI}
  Dada a instância clássica sem sinais $\mathcal{I} = ((0 \;1 \; 3 \; 2 \; 4 \; 5),(0 \; 1 \; 2 \; 3 \; 4 \; 5))$ para o problema \textbf{B3T}, temos que $b_2(\mathcal{I}) = 3$. Para a criação da instância clássica sem sinais $\mathcal{I'}=(\pi',\iota')$ para o problema \SbPRT{} temos, no passo 1, a obtenção da representação clássica sem sinais $\sigma = ({0} \; {1} \; {2} \; {5} \; {6} \; {3} \; {4} \; {7} \; {8} \; {9} \; {10} \; {11} \; {12})$. Usando $k = 0.6$, temos que $p = \lceil\frac{1\times 0.6}{1 - 0.6}\rceil = \lceil\frac{0.6}{0.4}\rceil = 2$ no passo 2. No passo 3, obtemos a representação clássica sem sinais $\pi' = ({0} \; {1} \; {2} \; {5} \; {6} \; {3} \; {4} \; {7} \; {8} \; {9} \; {10} \; {11} \; {14} \; {13} \; {12} \; {15} \; {16} \; {17} \; {20} \; {19} \; {18} \; {21} \; {22} \; {23} \;{24})$ após aplicar $p = 2$ extenções gadget em $\sigma$. No passo 4, obtemos a representação clássica sem sinais $\iota' = ({0} \; {1} \; {2} \; {3} \; {4} \; {5} \; {6} \; {7} \; {8} \; {9} \; {10} \; {11} \; {12} \; {13} \; {14} \; {15} \; {16} \; {17} \; {18} \; {19} \; {20} \; {21} \; {22} \; {23} \; {24})$. Note que $b_1(\mathcal{I'}) = b_2(\mathcal{I}) + 2p = 3 + 4 = 7$. A sequência $S = (\tau^{(2,3,4)})$ é tal que $\pi \cdot S = \iota$ e $|S| = 1 = \frac{b_2(\mathcal{I})}{3} = \ell$, e a sequência $S' = (\tau^{(3,5,7)},\rho^{(12,14)},\rho^{(18,20)})$, que possui a mesma quantidade de transposições de $S$, é tal que (i) $\pi' \cdot S' = \iota'$; (ii) $\frac{|S'_\rho|}{|S'|} = 0.666 \ge 0.6 = k$; e (iii) $|S'| = 3 = \frac{b_2(\mathcal{I})}{3} + 2 = \ell+p$.
\end{example}


% ------------------------------------------------------------------ %
\section{Algoritmos de Aproximação}
% ------------------------------------------------------------------ %

% ------------------------------------------------------------------ %
\subsection{Instâncias sem Sinais}
% ------------------------------------------------------------------ %

% ------------------------------------------------------------------ %
\subsection{Instâncias com Sinais}
% ------------------------------------------------------------------ %

% ------------------------------------------------------------------ %
\section{Resultados Práticos}
% ------------------------------------------------------------------ %

% ------------------------------------------------------------------ %
\subsection{Algoritmos Comparados}
% ------------------------------------------------------------------ %

% ------------------------------------------------------------------ %
\subsection{Base de Dados}
% ------------------------------------------------------------------ %

% ------------------------------------------------------------------ %
\subsection{Resultados}
% ------------------------------------------------------------------ %
