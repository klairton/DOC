\chapter{Modelos Intergênicos Flexíveis}\label{chapter:GMJBMTWF}

Neste capítulo, investigaremos problemas que levam em conta tanto a ordem dos genes como o tamanho das regiões intergênicas, mas considerando um grau de flexibilidade em relação ao tamanho das regiões intergênicas no genoma alvo que é desejado. Nesse contexto, nós consideramos os eventos de reversão intergênica, transposição intergênica, move intergênico e indel intergênico, e investigaremos as variações com e sem sinais dos seguintes problemas.

\begin{itemize}
  \item Ordenação de Permutações por Reversões Intergênicas com Regiões Intergênicas Flexíveis (\SbFIR)
  \item Ordenação de Permutações por Operações Intergênicas de Reversão e Indel com Regiões Intergênicas Flexíveis (\SbFIRI)
  \item Ordenação de Permutações por Operações Intergênicas de Reversão e Move com Regiões Intergênicas Flexíveis (\SbFIRM)
  \item Ordenação de Permutações por Operações Intergênicas de Reversão, Move e Indel com Regiões Intergênicas Flexíveis (\SbFIRMI)
  \item Ordenação de Permutações por Operações Intergênicas de Reversão e Transposição com Regiões Intergênicas Flexíveis (\SbFIRT)
  \item Ordenação de Permutações por Operações Intergênicas de Reversão, Transposição e Indel com Regiões Intergênicas Flexíveis (\SbFIRTI)
  \item Ordenação de Permutações por Operações Intergênicas de Reversão, Transposição e Move com Regiões Intergênicas Flexíveis (\SbFIRTM)
  \item Ordenação de Permutações por Operações Intergênicas de Reversão, Transposição, Move e Indel com Regiões Intergênicas Flexíveis (\SbFIRTMI)
\end{itemize}

Além disso, investigaremos as variações sem sinais dos seguintes problemas.

\begin{itemize}
  \item Ordenação de Permutações por Transposições Intergênicas com Regiões Intergênicas Flexíveis (\SbFIT)
  \item Ordenação de Permutações por Operações Intergênicas Transposição e Move com Regiões Intergênicas Flexíveis (\SbFITM)
\end{itemize}

Note que nos dois problemas apresentados anteriormente tanto o evento de transposição intergênica como o evento de move intergênico não alteram a orientação dos genes. Por esse motivo, apenas a variação sem sinais será investigada.

Neste capítulo, iremos nos referenciar aos eventos de rearranjo de reversão intergênica, transposição intergênica, move intergênico e indel intergênico simplesmente por reversão, transposição, move e indel, respectivamente. Além disso, iremos nos referir a um breakpoint intergênico simplesmente como um breakpoint. Dada uma sequência de eventos de rearranjo $S$, denotamos por $|S|$ o tamanho da sequência $S$, ou seja, a quantidade de eventos em $S$.

Dada uma instância intergênica flexível com ou sem sinais $\mathcal{I} = ((\pi,\breve\pi),(\iota,\breve\iota^{\min},\breve\iota^{\max}))$, a \emph{distância flexível} entre $(\pi,\breve\pi)$ e $(\iota,\breve\iota^{\min},\breve\iota^{\max})$, denotada por $df_{\mathcal{M}}(\mathcal{I})$, é o tamanho da menor sequência de eventos de rearranjo $S$, tal que todo evento de $S$ pertence ao modelo $\mathcal{M}$ e $(\pi,\breve\pi) \cdot S = (\iota,\breve\pi^{\prime})$, onde $\breve\iota^{\min}_i \le \breve\pi^{\prime}_i \le \breve\iota^{\max}_i$ para $i \in [1..n+1]$. Os modelos de rearranjo considerados neste capítulo são identificados por siglas apresentadas na Tabela~\ref{table:CGOLSOYF}.

\begin{table}[!htb]
  \caption{Siglas dos modelos de rearranjo considerados para instâncias intergênicas flexíveis.}
  \label{table:CGOLSOYF}
  \centering
  \begin{tabular}{|p{2.5cm}|p{3.5cm}|p{8cm}|}
    \hline
    \textbf{Problema}     & \textbf{Sigla do Modelo} & \textbf{Conjunto de Eventos de Rearranjo}          \\ \hline
    \SbFIR                & \R                       & $\{\rho\}                              $           \\ \hline
    \SbFIRI               & \RI                      & $\{\rho,\delta\}                       $           \\ \hline
    \SbFIRM               & \RM                      & $\{\rho,\mu\}                          $           \\ \hline
    \SbFIRMI              & \RMI                     & $\{\rho,\mu,\delta\}                   $           \\ \hline
    \SbFIT                & \T                       & $\{\tau\}                              $           \\ \hline
    \SbFITM               & \TM                      & $\{\tau,\mu\}                          $           \\ \hline
    \SbFIRT               & \RT                      & $\{\rho,\tau\}                         $           \\ \hline
    \SbFIRTI              & \RTI                     & $\{\rho,\tau,\delta\}                  $           \\ \hline
    \SbFIRTM              & \RTM                     & $\{\rho,\tau,\mu\}                     $           \\ \hline
    \SbFIRTMI             & \RTMI                    & $\{\rho,\tau,\mu,\delta\}              $           \\ \hline
  \end{tabular}
\end{table}

Dada uma instância intergênica flexível com ou sem sinais $\mathcal{I} = ((\pi,\breve\pi),(\iota,\breve\iota^{\min},\breve\iota^{\max}))$, nós utilizaremos a expressão \emph{atingir o genoma alvo} quanto $\pi = \iota$ e $\breve\iota^{\min}_i \le \breve\pi_i \le \breve\iota^{\max}_i$ para $i \in [1..n+1]$.

Parte dos resultados que serão apresentados neste capítulo foram aceitos para publicação na revista \emph{IEEE/ACM Transactions on Computational Biology and Bioinformatics}~\cite{2022a-brito-etal} em 2022.

% ------------------------------------------------------------------ %
\section{Análise de Complexidade}
% ------------------------------------------------------------------ %

Nesta seção, realizaremos uma análise de complexidade dos problemas que consideram um grau de flexibilidade em relação ao tamanho das regiões intergênicas no genoma alvo que é desejado. Note que todos os problemas investigados neste capítulo generalizam suas respectivas versões considerando um tamanho estrito (rígido) para o tamanho das regiões intergênicas desejadas no genoma alvo.

Isso pode ser facilmente constatado por meio de uma redução. Sejam $\mathcal{P}_f$ e $\mathcal{P}_r$ problemas com base no mesmo modelo de rearranjo $\mathcal{M}$, mas $\mathcal{P}_f$ e $\mathcal{P}_r$ possuem, respectivamente, uma característica de flexibilidade e rígidez em relação ao tamanho das regiões intergênicas no genoma alvo. Seja $\mathcal{I}=((\pi,\breve\pi),(\iota,\breve\iota))$ uma instância intergênica rígida para o problema $\mathcal{P}_r$, então podemos criar uma instância intergênica flexível $\mathcal{I'} = ((\pi',\breve\pi'),(\iota',\breve\iota'^{\min},\breve\iota'^{\max}))$ para o problema $\mathcal{P}_f$ da seguinte forma:

\begin{itemize}
  \item $(\pi',\breve\pi') = (\pi,\breve\pi)$
  \item $\iota' = \iota$
  \item $\breve\iota'^{\min} = \breve\iota'^{\max} = \breve\iota$
\end{itemize}

Note que pela construção da instância $\mathcal{I'}$ e pelo fato de que $\mathcal{P}_f$ e $\mathcal{P}_r$ adotam o mesmo modelo de rearranjo $\mathcal{M}$, temos que $df_{\mathcal{M}}(\mathcal{I'}) = d_{\mathcal{M}}(\mathcal{I})$.

No Capítulo~\ref{chapter:DOVAEMLI} foi mostrado que a variação sem sinais dos problemas \SbIR{}, \SbIRI{}, \SbIRM{}, \SbIRMI{}, \SbIRT{}, \SbIRTI{}, \SbIRTM{} e \SbIRTMI{} pertencem à classe NP-difícil. Adicionalmente, temos que a variação sem sinais dos problemas \SbIT{} e \SbITM{} também pertencem à classe NP-difícil~\cite{2021a-oliveira-etal}. Dessa forma, temos o seguinte lema.

\begin{lemma}\label{lemma:BEBGUYUB}
Os problemas \SbFIR{}, \SbFIRI{}, \SbFIRM{}, \SbFIRMI{}, \SbFIRT{}, \SbFIRTI{}, \SbFIRTM{}, \SbFIRTMI{}, \SbFIT{} e \SbFITM{} em instâncias intergênicas flexíveis sem sinais pertencem à classe NP-difícil.
\end{lemma}

Além disso, o Capítulo~\ref{chapter:DOVAEMLI} apresenta a informação de que a variação com sinais dos problemas \SbIR{}, \SbIRM{}, \SbIRMI{}, \SbIRT{}, \SbIRTI{}, \SbIRTM{} e \SbIRTMI{} também pertencem à classe NP-difícil. Com isso, obtemos os seguinte lema.

\begin{lemma}\label{lemma:XPRZJZES}
Os problemas \SbFIR{},\SbFIRM{}, \SbFIRMI{}, \SbFIRT{}, \SbFIRTI{}, \SbFIRTM{} e \SbFIRTMI{} em instâncias intergênicas flexíveis com sinais pertencem à classe NP-difícil.
\end{lemma}

% ------------------------------------------------------------------ %
\section{Limitantes Inferiores}
% ------------------------------------------------------------------ %

Nesta seção, apresentaremos limitantes inferiores para as variações com e sem sinais dos problemas investigados neste capítulo.

Para a variação sem sinais dos problemas \SbFIR{}, \SbFIRI{}, \SbFIRM{}, \SbFIRMI{}, \SbFIRT{}, \SbFIRTI{}, \SbFIRTM{} e  \SbFIRTMI{} utilizaremos o conceito de região intergênica tipo um, enquanto para a variação com sinais utilizaremos o conceito de região intergênica tipo dois.

Note que os eventos de rearranjo de reversão, transposição, move e indel afetam, respectivamente, a seguinte quantidade de regiões intergênicas: duas, três, duas e uma. No melhor cenário, cada uma das regiões intergênicas afetadas pode ser instável ou auxiliar, que é removida após o evento de rearranjo ser aplicado. Com isso, obtemos os seguintes lemas.

\begin{lemma}\label{lemma:VJKGLBQG}
Dada uma instância intergênica flexível sem sinais $\mathcal{I} = ((\pi,\breve\pi),(\iota,\breve\iota^{\min},\break\breve\iota^{\max}))$, para qualquer reversão $\rho$ temos que $\Delta i_1(\mathcal{I}, S = (\rho)) \ge -2$.
\end{lemma}

\begin{lemma}\label{lemma:HJTCKDGV}
Dada uma instância intergênica flexível com sinais $\mathcal{I} = ((\pi,\breve\pi),(\iota,\breve\iota^{\min},\break\breve\iota^{\max}))$, para qualquer reversão $\rho$ temos que $\Delta i_2(\mathcal{I}, S = (\rho)) \ge -2$.
\end{lemma}

\begin{lemma}\label{lemma:XLUTQDGV}
Dada uma instância intergênica flexível $\mathcal{I} = ((\pi,\breve\pi),(\iota,\breve\iota^{\min},\breve\iota^{\max}))$, para qualquer tranposição $\tau$ temos que $\Delta i_1(\mathcal{I}, S = (\tau)) \ge -3$ e $\Delta i_2(\mathcal{I}, S = (\tau)) \ge -3$.
\end{lemma}

\begin{lemma}\label{lemma:ZOCGWWGV}
Dada uma instância intergênica flexível $\mathcal{I} = ((\pi,\breve\pi),(\iota,\breve\iota^{\min},\breve\iota^{\max}))$, para qualquer move $\mu$ temos que $\Delta i_1(\mathcal{I}, S = (\mu)) \ge -2$ e $\Delta i_2(\mathcal{I}, S = (\mu)) \ge -2$.
\end{lemma}

\begin{lemma}\label{lemma:HQJMMZCU}
Dada uma instância intergênica flexível $\mathcal{I} = ((\pi,\breve\pi),(\iota,\breve\iota^{\min},\breve\iota^{\max}))$, para qualquer indel $\delta$ temos que $\Delta i_1(\mathcal{I}, S = (\delta)) \ge -1$ e $\Delta i_2(\mathcal{I}, S = (\delta)) \ge -1$.
\end{lemma}

Além disso, considerando uma instância intergênica flexível balanceada e com base em um modelo composto exclusivamente por eventos conservativos, temos os seguintes lemas.

\begin{lemma}\label{lemma:IERALSKC}
Dada uma instância intergênica flexível balanceada sem sinais $\mathcal{I} = ((\pi,\breve\pi),\break(\iota,\breve\iota^{\min},\breve\iota^{\max}))$, para qualquer reversão $\rho$ temos que $\Delta i_1(\mathcal{I}, S = (\rho)) + \Delta a_1(\mathcal{I}, S = (\rho)) \ge -2$.
\end{lemma}

\begin{lemma}\label{lemma:LMCJCLQU}
Dada uma instância intergênica flexível balanceada com sinais $\mathcal{I} = ((\pi,\breve\pi),\break(\iota,\breve\iota^{\min},\breve\iota^{\max}))$, para qualquer reversão $\rho$ temos que $\Delta i_2(\mathcal{I}, S = (\rho)) + \Delta a_2(\mathcal{I}, S = (\rho)) \ge -2$.
\end{lemma}

\begin{lemma}\label{lemma:FOXQSODF}
Dada uma instância intergênica flexível balanceada $\mathcal{I} = ((\pi,\breve\pi),\break(\iota,\breve\iota^{\min},\breve\iota^{\max}))$, para qualquer tranposição $\tau$ temos que $\Delta i_1(\mathcal{I}, S = (\tau)) + \Delta a_1(\mathcal{I}, S = (\tau)) \ge -3$ e $\Delta i_2(\mathcal{I}, S = (\tau)) + \Delta a_2(\mathcal{I}, S = (\tau)) \ge -3$.
\end{lemma}

\begin{lemma}\label{lemma:AXMNYRLB}
Dada uma instância intergênica flexível balanceada $\mathcal{I} = ((\pi,\breve\pi),\break(\iota,\breve\iota^{\min},\breve\iota^{\max}))$, para qualquer move $\mu$ temos que $\Delta i_1(\mathcal{I}, S = (\mu)) + \Delta a_1(\mathcal{I}, S = (\mu)) \ge -2$ e $\Delta i_2(\mathcal{I}, S = (\mu)) + \Delta a_2(\mathcal{I}, S = (\mu)) \ge -2$.
\end{lemma}

Com base na quantidade máxima de regiões intergênicas instáveis e auxiliares que cada evento pode remover de uma instância intergênica flexível, obtemos os seguintes limitantes inferiores.

\begin{theorem}\label{theorem:BOTBXFZQ}
Dada uma instância intergênica flexível sem sinais $\mathcal{I} = ((\pi,\breve\pi),(\iota,\breve\iota^{\min},\break\breve\iota^{\max}))$, temos que:

\begin{tabular}{lll}
  $df_{\SbFIRI}(\mathcal{I})$     & $ \ge $ & $\frac{i_1(\mathcal{I})}{2}$, \\
  $df_{\SbFIRMI}(\mathcal{I})$    & $ \ge $ & $\frac{i_1(\mathcal{I})}{2}$, \\
  $df_{\SbFIRTI}(\mathcal{I})$    & $ \ge $ & $\frac{i_1(\mathcal{I})}{3}$, \\
  e $df_{\SbFIRTMI}(\mathcal{I})$ & $ \ge $ & $\frac{i_1(\mathcal{I})}{3}$. \\
\end{tabular}
\end{theorem}
\begin{proof}
Pela Observação~\ref{remark:EUSNDMWS}, temos que todas as regiões intergênicas instáveis tipo um devem ser removidas para que o genoma alvo seja alcançado. Pelos lemas~\ref{lemma:VJKGLBQG}, \ref{lemma:XLUTQDGV}, \ref{lemma:ZOCGWWGV} e \ref{lemma:HQJMMZCU}, temos que os eventos de reversão, transposição, move e indel podem remover, no máximo, $2$, $3$, $2$ e $1$ região intergênica instável tipo um, respectivamente. Como a instância $\mathcal{I}$ possui $i_1(\mathcal{I})$ regiões intergênicas instáveis tipo um e considerando o máximo de regiões intergênicas instávéis tipo um que podem ser removidas por cada evento nos modelos de rearranjo \SbFIRI{}, \SbFIRMI{}, \SbFIRTI{} e \SbFIRTMI{}, o teorema segue.
\end{proof}

\begin{theorem}\label{theorem:KKKUCDHN}
Dada uma instância intergênica flexível balanceada sem sinais $\mathcal{I} = ((\pi,\breve\pi),(\iota,\breve\iota^{\min},\breve\iota^{\max}))$, temos que:

\begin{tabular}{lll}
  $df_{\SbFIR}(\mathcal{I})$      & $ \ge $ & $\frac{i_1(\mathcal{I}) + a_1(\mathcal{I})}{2}$, \\ 
  $df_{\SbFIRM}(\mathcal{I})$     & $ \ge $ & $\frac{i_1(\mathcal{I}) + a_1(\mathcal{I})}{2}$, \\
  $df_{\SbFIRT}(\mathcal{I})$     & $ \ge $ & $\frac{i_1(\mathcal{I}) + a_1(\mathcal{I})}{3}$, \\
  e $df_{\SbFIRTM}(\mathcal{I})$  & $ \ge $ & $\frac{i_1(\mathcal{I}) + a_1(\mathcal{I})}{3}$. \\
\end{tabular}
\end{theorem}
\begin{proof}
Note que em todos os problemas o modelo de rearranjo é composto exclusivamente por eventos conservativos. Pela Observação~\ref{remark:PGEYZJME}, temos que todas as regiões intergênicas instáveis tipo um e auxiliares tipo um devem ser removidas para que o genoma alvo seja alcançado. Pelos lemas~\ref{lemma:IERALSKC}, \ref{lemma:FOXQSODF} e \ref{lemma:AXMNYRLB}, temos que variação no número de regiões intergênicas instáveis tipo um mais a variação no número de regiões intergênicas auxiliares tipo um após aplicar um evento de reversão, transposição e move é maior ou igual que $-2$, $-3$ e $-2$, respectivamente. Como a instância $\mathcal{I}$ possui $i_1(\mathcal{I}) + a_1(\mathcal{I})$ regiões intergênicas instáveis tipo um e auxiliares tipo um, considerando o máximo de regiões intergênicas instávéis tipo um e auxiliares tipo um que podem ser removidas por cada evento nos modelos de rearranjo \SbFIR{}, \SbFIRM{}, \SbFIRT{} e \SbFIRTM{}, o teorema segue.
\end{proof}

\begin{theorem}\label{theorem:ROVRTGBJ}
Dada uma instância intergênica flexível com sinais $\mathcal{I} = ((\pi,\breve\pi),(\iota,\breve\iota^{\min},\break\breve\iota^{\max}))$, temos que:

\begin{tabular}{lll}
  $df_{\SbFIRI}(\mathcal{I})$     & $ \ge $ & $\frac{i_2(\mathcal{I})}{2}$, \\
  $df_{\SbFIRMI}(\mathcal{I})$    & $ \ge $ & $\frac{i_2(\mathcal{I})}{2}$, \\
  $df_{\SbFIRTI}(\mathcal{I})$    & $ \ge $ & $\frac{i_2(\mathcal{I})}{3}$, \\
  e $df_{\SbFIRTMI}(\mathcal{I})$ & $ \ge $ & $\frac{i_2(\mathcal{I})}{3}$. \\
\end{tabular}
\end{theorem}
\begin{proof}
A prova é similar a descrita no Teorema~\ref{theorem:BOTBXFZQ}, mas considerando regiões intergênicas instáveis tipo dois e os lemas~\ref{lemma:HJTCKDGV}, \ref{lemma:XLUTQDGV}, \ref{lemma:ZOCGWWGV} e \ref{lemma:HQJMMZCU}.
\end{proof}

\begin{theorem}\label{theorem:ZJKGKRKE}
Dada uma instância intergênica flexível balanceada com sinais $\mathcal{I} = ((\pi,\breve\pi),(\iota,\breve\iota^{\min},\breve\iota^{\max}))$, temos que:

\begin{tabular}{lll}
  $df_{\SbFIR}(\mathcal{I})$      & $ \ge $ & $\frac{i_2(\mathcal{I}) + a_2(\mathcal{I})}{2}$, \\ 
  $df_{\SbFIRM}(\mathcal{I})$     & $ \ge $ & $\frac{i_2(\mathcal{I}) + a_2(\mathcal{I})}{2}$, \\
  $df_{\SbFIRT}(\mathcal{I})$     & $ \ge $ & $\frac{i_2(\mathcal{I}) + a_2(\mathcal{I})}{3}$, \\
  e $df_{\SbFIRTM}(\mathcal{I})$  & $ \ge $ & $\frac{i_2(\mathcal{I}) + a_2(\mathcal{I})}{3}$. \\
\end{tabular}
\end{theorem}
\begin{proof}
A prova é similar a descrita no Teorema~\ref{theorem:KKKUCDHN}, mas considerando regiões intergênicas instáveis tipo dois e auxiliares tipo dois, e os lemas~\ref{lemma:LMCJCLQU}, \ref{lemma:FOXQSODF} e \ref{lemma:AXMNYRLB}.
\end{proof}

A seguir, com base na estrutura de grafo de ciclos ponderado flexível, apresentamos limitantes inferiores para a variação sem sinais dos problemas \SbFIT{} e \SbFITM{}, e para a variação com sinais dos problemas \SbFIR{}, \SbFIRI{}, \SbFIRM{}, \SbFIRMI{}, \SbFIRT{} e \SbFIRTM{}.

Note que os evento de reversão e tranposição afetam, respectivamente, duas e três arestas pretas do grafo de ciclos ponderado flexível e podem aumentar tanto o número de ciclos como também o número de ciclos estáveis e definitivos. O evento de move afeta duas arestas pretas do grafo, mas pode aumentar somente o número de ciclos estáveis e definitivos. Já o evento de indel afeta apenas uma aresta preta do grafo e pode aumentar somente o número de ciclos estáveis.

Dessa forma, dada uma instância intergênica flexível $\mathcal{I} = ((\pi,\breve\pi),(\iota,\breve\iota^{\min},\breve\iota^{\max}))$, temos que $\Delta c(G(\mathcal{I}), S=(\rho)) \in \{1,0,-1\}$, $\Delta c_e(G(\mathcal{I}), S=(\rho)) \in \{1,0,-1\}$ e $\Delta c_d(G(\mathcal{I}), S=(\rho)) \in \{1,0,-1\}$ para qualquer reversão $\rho$. Para qualquer transposição $\tau$, temos que $\Delta c(G(\mathcal{I}), S=(\tau)) \in \{2,0,-2\}$, $\Delta c_e(G(\mathcal{I}), S=(\tau)) \in \{2,1,0,-1,-2\}$ e $\Delta c_d(G(\mathcal{I}), S=(\tau)) \in \{2,1,0,-1,-2\}$. Para qualquer move $\mu$, temos que $\Delta c(G(\mathcal{I}), S=(\mu)) = 0$, $\Delta c_e(G(\mathcal{I}), S=(\mu)) \in \{2,1,0,-1,-2\}$ e $\Delta c_d(G(\mathcal{I}), S=(\mu)) \in \{2,1,0,-1,-2\}$. Por fim, para qualquer indel $\delta$, temos que $\Delta c(G(\mathcal{I}), S=(\delta)) = 0$ e $\Delta c_e(G(\mathcal{I}), S=(\delta)) \in \{1,0,{-1}\}$. Com isso, obtemos os seguintes limitantes inferiores.

\begin{theorem}\label{theorem:PQQUYBMS}
Dada uma instância intergênica flexível balanceada sem sinais $\mathcal{I} = ((\pi,\breve\pi),(\iota,\breve\iota^{\min},\breve\iota^{\max}))$, temos que

\begin{tabular}{lll}
  $df_{\SbFIT}(\mathcal{I})$      & $ \ge $ & $\frac{{n+1} - c_d(G(\mathcal{I} ))}{2}$, \\
  e $df_{\SbFITM}(\mathcal{I})$   & $ \ge $ & $\frac{{n+1} - c_d(G(\mathcal{I} ))}{2}$. \\
\end{tabular}
\end{theorem}
\begin{proof}
Pela Observação~\ref{remark:HLVDQLCE}, temos que para atingir o genoma alvo é necessário que $c(G(\mathcal{I})) = c_d(G(\mathcal{I})) = n+1$. Temos por definição que $c_d(G(\mathcal{I})) \le c(G(\mathcal{I}))$, então se fizermos que $G(\mathcal{I})$ possua $n+1$ ciclos definitivos temos garantidamente que $c(G(\mathcal{I})) = c_d(G(\mathcal{I})) = n+1$. Tanto o evento de transposição como o evento de move aumentam o número de ciclos definitivos, no máximo, em duas unidades. Logo, são necessárias pelo menos $\frac{{n+1} - c_d(G(\mathcal{I} ))}{2}$ operações de transposição ou move para atingir o genoma alvo, e o teorema segue. 
\end{proof}


\begin{theorem}\label{theorem:EUNBEQEX}
Dada uma instância intergênica flexível balanceada com sinais $\mathcal{I} = ((\pi,\breve\pi),(\iota,\breve\iota^{\min},\breve\iota^{\max}))$, temos que $df_{\SbFIR}(\mathcal{I}) \ge {n+1} - c_d(G(\mathcal{I} ))$.
\end{theorem}
\begin{proof}
A prova é similar a apresentada no Teorema~\ref{theorem:PQQUYBMS} considerando que o evento de reversão pode aumentar o número de ciclos definitivos, no máximo, em uma unidade.
\end{proof}

\begin{theorem}\label{theorem:SZNBDWOM}
Dada uma instância intergênica flexível com sinais $\mathcal{I} = ((\pi,\breve\pi),(\iota,\breve\iota^{\min},\break\breve\iota^{\max}))$, temos que $df_{\SbFIRI}(\mathcal{I}) \ge {n+1} - c_e(G(\mathcal{I} ))$.
\end{theorem}
\begin{proof}
Pela Observação~\ref{remark:IRNWKUZA}, temos que para atingir o genoma alvo é necessário que $c(G(\mathcal{I})) = c_e(G(\mathcal{I})) = n+1$. Temos por definição que $c_e(G(\mathcal{I})) \le c(G(\mathcal{I}))$, então se fizermos que $G(\mathcal{I})$ possua $n+1$ ciclos estáveis temos garantidamente que $c(G(\mathcal{I})) = c_e(G(\mathcal{I})) = n+1$. Tanto o evento de reversão como o evento de indel aumentam o número de ciclos estáveis, no máximo, em uma unidade. Logo, são necessárias pelo menos ${n+1} - c_e(G(\mathcal{I} ))$ operações de reversão ou indel para atingir o genoma alvo, e o teorema segue. 
\end{proof}

\begin{theorem}\label{theorem:CNMFNKPK}
Dada uma instância intergênica flexível balanceada com sinais $\mathcal{I} = ((\pi,\breve\pi),(\iota,\breve\iota^{\min},\breve\iota^{\max}))$, temos que $df_{\SbFIRM}(\mathcal{I}) \ge {n+1} - \frac{c(G(\mathcal{I} )) + c_d(G(\mathcal{I} ))}{2}$.
\end{theorem}
\begin{proof}
Pela Observação~\ref{remark:HLVDQLCE}, temos que para atingir o genoma alvo é necessário que $c(G(\mathcal{I})) = c_d(G(\mathcal{I})) = n+1$. Logo, temos que aumentar a quantidade de ciclos e ciclos definitivos em ${n+1} - c(G(\mathcal{I}))$ e ${n+1} - c_d(G(\mathcal{I}))$ unidades, respectivamente. Totalizando a quantidade de ciclos e ciclos definitivos que precisam ser criados temos o seguinte valor: $2(n+1) - (c(G(\mathcal{I})) + c_b(G(\mathcal{I})))$. Considerando os eventos de reversão e move, temos que para qualquer evento $\gamma \in \{\rho, \mu\}$ é verdade que $\Delta c(G(\mathcal{I}), S=(\gamma)) + \Delta c_d(G(\mathcal{I}), S=(\gamma)) \le 2$. Dessa forma, são necessárias pelo menos $\frac{2({n+1}) - (c(G(\mathcal{I})) + c_d(G(\mathcal{I})))}{2} = {n+1} - \frac{c(G(\mathcal{I} )) + c_d(G(\mathcal{I} ))}{2}$ operações de reversão ou move para atingir o genoma alvo, e o teorema segue. 
\end{proof}

\begin{theorem}\label{theorem:XQPRYMFX}
Dada uma instância intergênica flexível com sinais $\mathcal{I} = ((\pi,\breve\pi),(\iota,\breve\iota^{\min},\break\breve\iota^{\max}))$, temos que $df_{\SbFIRMI}(\mathcal{I}) \ge {n+1} - \frac{c(G(\mathcal{I} )) + c_e(G(\mathcal{I} ))}{2}$.
\end{theorem}
\begin{proof}
A prova é similar a apresentada no Teorema~\ref{theorem:CNMFNKPK} considerando que para qualquer evento $\gamma \in \{\rho, \mu\,\delta\}$ é verdade que $\Delta c(G(\mathcal{I}), S=(\gamma)) + \Delta c_d(G(\mathcal{I}), S=(\gamma)) \le 2$.
\end{proof}

\begin{theorem}\label{theorem:HELIIGVZ}
Dada uma instância intergênica flexível balanceada com sinais $\mathcal{I} = ((\pi,\breve\pi),(\iota,\breve\iota^{\min},\breve\iota^{\max}))$, temos que

\begin{tabular}{lll}
  $df_{\SbFIRT}(\mathcal{I})$     & $ \ge $ & $\frac{{n+1} - c_d(G(\mathcal{I} ))}{2}$, \\
  e $df_{\SbFIRTM}(\mathcal{I})$  & $ \ge $ & $\frac{{n+1} - c_d(G(\mathcal{I} ))}{2}$. \\
\end{tabular}
\end{theorem}
\begin{proof}
A prova é similar a apresentada no Teorema~\ref{theorem:PQQUYBMS} incluindo a consideração de que o evento de reversão pode aumentar o número de ciclos definitivos, no máximo, em uma unidade.
\end{proof}


% ------------------------------------------------------------------ %
\section{Algoritmos de Aproximação}
% ------------------------------------------------------------------ %

Nesta seção apresentaremos algoritmos de aproximação para as variações dos problemas investigados neste capítulo. Inicialmente apresentaremos algumas funções de redução que criam uma instância intergênica rígida a partir de uma instância intergênica flexível.

Dada uma instância intergênica flexível $\mathcal{I} = ((\pi,\breve\pi),(\iota,\breve\iota^{\min},\breve\iota^{\max}))$ a função $\mathcal{F}_{1}^{'}$ cria uma instância intergênica rígida $\mathcal{I'} = ((\pi',\breve\pi'),(\iota',\breve\iota'))$ da seguinte forma:

\begin{itemize}
  \item $(\pi',\breve\pi') = (\pi,\breve\pi)$
  \item $\iota' = \iota$
  \item Inicialmente, atribua em $\breve\iota_{i}'$ o valor $\breve\iota^{\min}_i$, para $i \in [1..({n+1})]$. Em seguida, para cada região intergênica estável tipo um $\breve\pi_i \in \mathcal{S}_{e_{1}}(\mathcal{I})$ atribua o valor $\breve\pi_i$ em $\breve\iota_{k}'$, onde $k = \max(\pi_{i-1},\pi_i)$.
\end{itemize}

Denotamos por $\mathcal{F}_{1}^{'}(\mathcal{I})$ a instância intergênica rígida criada pela função $\mathcal{F}_{1}^{'}$ a partir de uma instância intergênica flexível $\mathcal{I}$. Perceba que a função $\mathcal{F}_{1}^{'}$ apenas define valores estritos para os tamanhos das regiões intergênicas no genoma alvo $\breve\iota'$ da instância intergênica rígida $\mathcal{I'}$, uma vez que $(\pi',\breve\pi') = (\pi,\breve\pi)$ e $\iota' = \iota$. Além disso, a função $\mathcal{F}_{1}^{'}$ pode ser executada em tempo linear.

\begin{lemma}\label{lemma:UFTVNRSX}
Seja $\mathcal{I'} = ((\pi',\breve\pi'),(\iota',\breve\iota'))$ uma instância intergênica rígida tal que $\mathcal{I'} = \mathcal{F}_{1}^{'}(\mathcal{I})$, onde $\mathcal{I} = ((\pi,\breve\pi),(\iota,\breve\iota^{\min},\breve\iota^{\max}))$ é uma instância intergênica flexível, então temos que $i_1(\mathcal{I}) = ib_1(\mathcal{I'})$.
\end{lemma}
\begin{proof}
Pela construção da função $\mathcal{F}_{1}^{'}$ temos que cada região intergênica estável tipo um em $\mathcal{I}$ é mapeada em uma adjacência intergênica em $\mathcal{I'}$. Além disso, cada região intergênica instável tipo um acaba tornando-se um breakpoint tipo um em $\mathcal{I'}$, seja porque os elementos antes e depois da região intergênica não são adjacentes no genoma alvo ou porque o tamanho da região intergênica é menor do que o mínimo ou maior do que o máximo permitido no genoma alvo. Logo, $i_1(\mathcal{I}) = ib_1(\mathcal{I'})$ e o lema segue.
\end{proof}

\begin{lemma}\label{lemma:SVKOAOXA}
Seja $\mathcal{I'} = ((\pi',\breve\pi'),(\iota',\breve\iota'))$ uma instância intergênica rígida tal que $\mathcal{I'} = \mathcal{F}_{1}^{'}(\mathcal{I})$, onde $\mathcal{I} = ((\pi,\breve\pi),(\iota,\breve\iota^{\min},\breve\iota^{\max}))$ é uma instância intergênica flexível, e seja $S'$ uma sequência de eventos de rearranjo tal que $(\pi',\breve\pi') \cdot S' = (\iota',\breve\iota')$, então $S'$ é uma sequência que faz com que o genoma alvo em $\mathcal{I}$ seja atingido.
\end{lemma}
\begin{proof}
Diretamente pela construção da função $\mathcal{F}_{1}^{'}$. Note que $(\pi',\breve\pi') = (\pi,\breve\pi)$, $\iota' = \iota$ e $\breve\iota^{\min}_i \le \breve\iota'_i \le \breve\iota^{\max}_i$, para $i \in [0..({n+1})]$.
\end{proof}


% ------------------------------------------------------------------ %
\section{Conclusões}
% ------------------------------------------------------------------ %

Neste capítulo, estudamos uma generalização de problemas que consideram tanto a ordem dos genes como o tamanho estrito das regiões intergênicas. Nessa versão generalizada adicionamos um grau de flexibilidade em relação ao tamanho das regiões regiões intergênicas desejadas no genoma alvo. Para isso, nos modelos que propomos, chamados de modelos intergênicos flexíveis, é possível especificar um intervalo de valores permitidos para o tamanho de cada região intergênica no genoma alvo. 

Esse grau de flexibilidade tenta trazer uma importância maior para a ordem e orientação dos genes em comparação com o tamanho das regiões intergênicas, uma vez que considerando um intervalo de valores permitidos para o tamanho de cada região intergênica no genoma alvo acabos ampliando as possibilidades de configuração para os tamanhos das regiões intergênicas de modo que todas as restrições de um modelo sejam atendidas.

Para as variações com (quando aplicável) e sem sinais dos dez problemas investigados nós apresentamos algoritmos de aproximação com um fator constante com base em um processo de redução. Por fim, realizamos testes experimentais para verificar o desempenho prático dos algoritmos que foram apresentados.