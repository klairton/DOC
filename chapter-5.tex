\chapter{Modelos Intergênicos Flexíveis}\label{chapter:GMJBMTWF}

Neste capítulo, investigaremos problemas que levam em conta tanto a ordem dos genes como o tamanho das regiões intergênicas, mas considerando um grau de flexibilidade em relação ao tamanho das regiões intergênicas no genoma alvo que é desejado. Nesse contexto, nós consideramos os eventos de reversão intergênica, transposição intergênica, move intergênico e indel intergênico, e investigaremos as variações com e sem sinais dos seguintes problemas.

\begin{itemize}
  \item Ordenação de Permutações por Reversões Intergênicas com Regiões Intergênicas Flexíveis (\SbFIR)
  \item Ordenação de Permutações por Operações Intergênicas de Reversão e Indel com Regiões Intergênicas Flexíveis (\SbFIRI)
  \item Ordenação de Permutações por Operações Intergênicas de Reversão e Move com Regiões Intergênicas Flexíveis (\SbFIRM)
  \item Ordenação de Permutações por Operações Intergênicas de Reversão, Move e Indel com Regiões Intergênicas Flexíveis (\SbFIRMI)
  \item Ordenação de Permutações por Operações Intergênicas de Reversão e Transposição com Regiões Intergênicas Flexíveis (\SbFIRT)
  \item Ordenação de Permutações por Operações Intergênicas de Reversão, Transposição e Indel com Regiões Intergênicas Flexíveis (\SbFIRTI)
  \item Ordenação de Permutações por Operações Intergênicas de Reversão, Transposição e Move com Regiões Intergênicas Flexíveis (\SbFIRTM)
  \item Ordenação de Permutações por Operações Intergênicas de Reversão, Transposição, Move e Indel com Regiões Intergênicas Flexíveis (\SbFIRTMI)
\end{itemize}

Além disso, investigaremos as variações sem sinais dos seguintes problemas.

\begin{itemize}
  \item Ordenação de Permutações por Transposições Intergênicas com Regiões Intergênicas Flexíveis (\SbFIT)
  \item Ordenação de Permutações por Operações Intergênicas Transposição e Move com Regiões Intergênicas Flexíveis (\SbFITM)
\end{itemize}

Neste capítulo, iremos nos referenciar aos eventos de rearranjo de reversão intergênica, transposição intergênica, move intergênico e indel intergênico simplesmente por reversão, transposição, move e indel, respectivamente. Além disso, iremos nos referir a um breakpoint intergênico simplesmente como um breakpoint. Dada uma sequência de eventos de rearranjo $S$, denotamos por $|S|$ o tamanho da sequência $S$, ou seja, a quantidade de eventos em $S$.

Dada uma instância intergênica flexível com ou sem sinais $\mathcal{I} = ((\pi,\breve\pi),(\iota,\breve\iota^{\min},\breve\iota^{\max}))$, a \emph{distância flexível} entre $(\pi,\breve\pi)$ e $(\iota,\breve\iota^{\min},\breve\iota^{\max})$, denotada por $df_{\mathcal{M}}(\mathcal{I})$, é o tamanho da menor sequência de eventos de rearranjo $S$, tal que todo evento de $S$ pertence ao modelo $\mathcal{M}$ e $(\pi,\breve\pi) \cdot S = (\iota,\breve\pi^{\prime})$, onde $\breve\iota^{\min}_i \le \breve\pi^{\prime}_i \le \breve\iota^{\max}_i$ para $i \in [1..n+1]$. Os modelos de rearranjo considerados neste capítulo são identificados por siglas apresentadas na Tabela~\ref{table:CGOLSOYF}.

\begin{table}[!htb]
  \caption{Siglas dos modelos de rearranjo considerados para instâncias intergênicas flexíveis.}
  \label{table:CGOLSOYF}
  \centering
  \begin{tabular}{|p{2.5cm}|p{3.5cm}|p{8cm}|}
    \hline
    \textbf{Problema}     & \textbf{Sigla do Modelo} & \textbf{Conjunto de Eventos de Rearranjo}          \\ \hline
    \SbFIR                & \R                       & $\{\rho\}                              $           \\ \hline
    \SbFIRI               & \RI                      & $\{\rho,\delta\}                       $           \\ \hline
    \SbFIRM               & \RM                      & $\{\rho,\mu\}                          $           \\ \hline
    \SbFIRMI              & \RMI                     & $\{\rho,\mu,\delta\}                   $           \\ \hline
    \SbFIT                & \T                       & $\{\tau\}                              $           \\ \hline
    \SbFITM               & \TM                      & $\{\tau,\mu\}                          $           \\ \hline
    \SbFIRT               & \RT                      & $\{\rho,\tau\}                         $           \\ \hline
    \SbFIRTI              & \RTI                     & $\{\rho,\tau,\delta\}                  $           \\ \hline
    \SbFIRTM              & \RTM                     & $\{\rho,\tau,\mu\}                     $           \\ \hline
    \SbFIRTMI             & \RTMI                    & $\{\rho,\tau,\mu,\delta\}              $           \\ \hline
  \end{tabular}
\end{table}

Parte dos resultados que serão apresentados neste capítulo foram aceitos para publicação na revista \emph{IEEE/ACM Transactions on Computational Biology and Bioinformatics}~\cite{2022a-brito-etal} em 2022.