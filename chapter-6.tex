
Nesta tese apresentamos os resultados obtidos durante o período do doutorado. De maneira geral, os resultados aqui apresentados incorporam novas características aos problemas de rearranjo de genomas que vão desde a adição de uma nova restrição de proporção entre a quantidade de um tipo de evento em relação ao tamanho da sequência de eventos rearranjo em uma solução, até a inclusão de novas estruturas genéticas na representação computacional que é adota nos modelos. Além disso, foram investigados problemas em que é considerado um grau de flexibilidade nas características de um genoma alvo que é desejado. Para a grande maioria dos problemas que investigamos nós apresentamos a prova de NP-dificuldade e desenvolvemos algoritmos de aproximação. Além disso, sempre que possível, criamos mecanismos que visam melhorar os resultados práticos dos algoritmos que foram propostos. Os resultados apresentados nesta tese geraram os seguintes artigos:

\begin{itemize}
  \item ``\textit{A New Approach for the Reversal Distance with Indels and Moves in Intergenic Regions}'', em coautoria com Andre Rodrigues Oliveira, Alexsandro Oliveira Alexandrino, Ulisses Dias e Zanoni Dias, foi apresentado no 19th Annual Satellite Conference of RECOMB on Comparative Genomics (RECOMB-CG), realizado em La Jolla, USA, no mês de Maio de 2022~\cite{2022b-brito-etal}.

  \item ``\textit{Genome Rearrangement Distance with a Flexible Intergenic Regions Aspect}'', aceito para publicação na revista IEEE-ACM Transactions on Computational Biology and Bioinformatics, e em coautoria com Alexsandro Oliveira Alexandrino, Andre Rodrigues Oliveira, Ulisses Dias e Zanoni Dias~\cite{2022a-brito-etal}. Uma versão preliminar deste artigo foi apresentada na 8th International Conference on Algorithms for Computational Biology (AlCoB), realizado de forma virtual, no mês de Novembro de 2021~\cite{2021c-brito-etal}. Uma versão abordando instâncias sem sinais foi apresentada no XI Latin and American Algorithms, Graphs and Optimization Symposium (LAGOS), realizado de forma virtual, no mês de Maio de 2021~\cite{2021d-brito-etal}.

  \item ``\textit{An Improved Approximation Algorithm for the Reversal and Transposition Distance Considering Gene order and Intergenic Sizes}'', publicado na revista Algorithms for Molecular Biology, e em coautoria com Andre Rodrigues Oliveira, Alexsandro Oliveira Alexandrino, Ulisses Dias e Zanoni Dias~\cite{2021b-brito-etal}.

  \item ``\textit{Reversals and Transpositions Distance with Proportion Restriction}'', publicado na revista Journal of Bioinformatics and Computational Biology, e em coautoria com Alexsandro Oliveira Alexandrino, Andre Rodrigues Oliveira, Ulisses Dias e Zanoni Dias~\cite{2021a-brito-etal}. Uma versão preliminar deste artigo foi apresentada no 13th Brazilian Symposium on Bioinformatics (BSB), realizado de forma virtual, no mês de Novembro de 2020~\cite{2020c-brito-etal}.

  \item ``\textit{Sorting by Genome Rearrangements on Both Gene Order and Intergenic Sizes}'', publicado na revista Journal of Computational Biology, e em coautoria com Géraldine Jean, Guillaume Fertin, Andre Rodrigues Oliveira, Ulisses Dias e Zanoni Dias~\cite{2020a-brito-etal}. Uma versão preliminar deste artigo foi apresentada no 15th International Symposium on Bioinformatics Research and Applications (ISBRA), realizado em Barcelona, Espanha, no mês de Junho de 2019~\cite{2019-brito-etal}.
\end{itemize}

Além dos artigos listados acima, que estão diretamente relacionados com esta tese, outras contribuições foram apresentadas nos seguintes artigos:

\begin{itemize}
  \item ``\textit{Reversal Distance on Genomes with Different Gene Content and Intergenic Regions Information}'', em coautoria com Alexsandro Oliveira Alexandrino, Andre Rodrigues Oliveira, Ulisses Dias e Zanoni Dias, foi apresentado na 8th International Conference on Algorithms for Computational Biology (AlCoB), realizado de forma virtual, no mês de Novembro de 2021~\cite{2021b-alexandrino-etal}

  \item ``\textit{Sorting Permutations by Intergenic Operations}'', publicado na revista IEEE-ACM Transactions on Computational Biology and Bioinformatics, e em coautoria com Andre Rodrigues Oliveira, Géraldine Jean, Guillaume Fertin, Ulisses Dias e Zanoni Dias~\cite{2021a-oliveira-etal}. Uma versão preliminar deste artigo foi apresentada na 7th International Conference on Algorithms for Computational Biology (AlCoB), realizado de forma virtual, no mês de Novembro de 2021~\cite{2020-oliveira-etal}.

  \item ``\textit{Sorting Signed Permutations by Intergenic Reversals}'', publicado na revista IEEE-ACM Transactions on Computational Biology and Bioinformatics, e em coautoria com Andre Rodrigues Oliveira, Géraldine Jean, Guillaume Fertin, Laurent Bulteau, Ulisses Dias e Zanoni Dias~\cite{2021b-oliveira-etal}.

  \item ``\textit{Heuristics for Genome Rearrangement Distance with Replicated Genes}'', publicado na revista IEEE-ACM Transactions on Computational Biology and Bioinformatics, e em coautoria com Gabriel Siqueira, Ulisses Dias e Zanoni Dias~\cite{2021a-siqueira-etal}. Uma versão preliminar deste artigo foi apresentada na 7th International Conference on Algorithms for Computational Biology (AlCoB), realizado de forma virtual, no mês de Novembro de 2021~\cite{2020-siqueira-etal}.

  \item ``\textit{On the Complexity of Sorting by Reversals and Transpositions Problem}'', publicado na revista Journal of Computational Biology, e em coautoria com Andre Rodrigues Oliveira, Ulisses Dias e Zanoni Dias~\cite{2019b-oliveira-etal}.

  \item ``\textit{Block-Interchange Distance Considering Intergenic Regions}'', em coautoria com Ulisses Dias, Andre Rodrigues Oliveira e Zanoni Dias, foi apresentado no 12th Brazilian Symposium on Bioinformatics (BSB), realizado em Fortaleza, Brasil, no mês de Outubro de 2019~\cite{2019-dias-etal}.
\end{itemize}

A partir das contribuições que foram apresentadas é possível mencionar algumas possibilidades de trabalhos futuros: i) Considerando problemas com restrição de proporção entre operações é possível incorporar uma representação intergênica aos modelos. Além disso, é possível investigar outras restrições de proporção considerando mais eventos de rearranjo. ii) Para os modelos que consideram a informação referente aos genes e ao tamanho das regiões intergênicas, é possível adicionar eventos de rearranjo não conservativos que afetam tanto os genes como as regiões intergênicas. iii) Outra opção é considerar uma representação intergênica que permita múltiplas cópias de um gene. iv) Por fim, uma possível linha de investigação seria o estudo de novos algoritmos visando obter resultados práticos ou teóricos melhores para os problemas que aqui foram investigados.  Estas são algumas das possibilidades de investigação de trabalhos futuros que podem permitir uma aplicação dos resultados de maneira mais direta em genomas reais. 