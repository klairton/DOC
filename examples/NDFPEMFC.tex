\begin{example}\label{example:NDFPEMFC}
  Dada a instância clássica sem sinais $\mathcal{I} = ((0 \; 3 \; 5 \; 1 \; 4 \; 2 \; 6),(0 \; 1 \; 2 \; 3 \; 4 \; 5 \; 6))$ para o problema \textbf{B3T}, temos que $b_2(\mathcal{I}) = 6$. Para a criação da instância clássica com sinais $\mathcal{I'}=(\pi',\iota')$ para o problema \SbPRT{} temos no passo 1 a obteção da representação clássica com sinais $\sigma = ({+0} \; {+3} \; {+5} \; {+1} \; {+4} \; {+2} \; {+6} \; {+7} \; {+8} \; {+9})$. Usando $k = 0.3$, temos que $p = \lceil\frac{2\times 0.3}{1 - 0.3}\rceil = \lceil\frac{0.6}{0.7}\rceil = 1$ no passo 2. No passo 3 obtemos a representação clássica com sinais $\pi' = ({+0} \; {+3} \; {+5} \; {+1} \; {+4} \; {+2} \; {+6} \; {+7} \; {+8} \; {-11} \; {-10} \; {-9} \; {+12} \; {+13} \; {+14} \; {+15})$ após aplicar $p = 1$ extenções gadget em $\sigma$. No passo 4 obtemos a representação clássica com sinais $\iota' = ({+0} \; {+1} \; {+2} \; {+3} \; {+4} \; {+5} \; {+6} \; {+7} \; {+8} \; {+9} \; {+10} \; {+11} \; {+12} \; {+13} \; {+14} \; {+15})$. Note que $b_2(\mathcal{I'}) = b_2(\mathcal{I}) + 2p = 6 + 2 = 8$. A sequência $S = (\tau^{(1,3,6)},\tau^{(2,3,5)})$ é tal que $\pi \cdot S = \iota$ e $|S| = 2 = \frac{b_2(\pi)}{3} = \ell$, e a sequência $S' = (\tau^{(1,3,6)},\tau^{(2,3,5)},\rho^{(9,11)})$ que possui a mesma sequência de transposições de $S$ é tal que (i) $\pi' \cdot S' = \iota'$; (ii) $\frac{|S'_\rho|}{|S'|} = 0.333 \ge 0.3 = k$; e (iii) $|S'| = 3 = \frac{b_2(\pi)}{3} + 1 = \ell+p$.
\end{example}