\begin{example}\label{example:QBULCCOI}
  Dada a instância clássica sem sinais $\mathcal{I} = ((0 \;1 \; 3 \; 2 \; 4 \; 5),(0 \; 1 \; 2 \; 3 \; 4 \; 5))$ para o problema \textbf{B3T}, temos que $b_2(\mathcal{I}) = 3$. Para a criação da instância clássica sem sinais $\mathcal{I'}=(\pi',\iota')$ para o problema \SbPRT{} temos no passo 1 a obteção da representação clássica sem sinais $\sigma = ({0} \; {1} \; {2} \; {5} \; {6} \; {3} \; {4} \; {7} \; {8} \; {9} \; {10} \; {11} \; {12})$. Usando $k = 0.6$, temos que $p = \lceil\frac{1\times 0.6}{1 - 0.6}\rceil = \lceil\frac{0.6}{0.4}\rceil = 2$ no passo 2. No passo 3, obtemos a representação clássica sem sinais $\pi' = ({0} \; {1} \; {2} \; {5} \; {6} \; {3} \; {4} \; {7} \; {8} \; {9} \; {10} \; {11} \; {14} \; {13} \; {12} \; {15} \; {16} \; {17} \; {20} \; {19} \; {18} \; {21} \; {22} \; {23} \;{24})$ após aplicar $p = 2$ extenções gadget em $\sigma$. No passo 4, obtemos a representação clássica sem sinais $\iota' = ({0} \; {1} \; {2} \; {3} \; {4} \; {5} \; {6} \; {7} \; {8} \; {9} \; {10} \; {11} \; {12} \; {13} \; {14} \; {15} \; {16} \; {17} \; {18} \; {19} \; {20} \; {21} \; {22} \; {23} \; {24})$. Note que $b_1(\mathcal{I'}) = b_2(\mathcal{I}) + 2p = 3 + 4 = 7$. A sequência $S = (\tau^{(2,3,4)})$ é tal que $\pi \cdot S = \iota$ e $|S| = 1 = \frac{b_2(\mathcal{I})}{3} = \ell$, e a sequência $S' = (\tau^{(3,5,7)},\rho^{(12,14)},\rho^{(18,20)})$ que possui a mesma quantidade de transposições de $S$ é tal que (i) $\pi' \cdot S' = \iota'$; (ii) $\frac{|S'_\rho|}{|S'|} = 0.666 \ge 0.6 = k$; e (iii) $|S'| = 3 = \frac{b_2(\mathcal{I})}{3} + 2 = \ell+p$.
\end{example}
