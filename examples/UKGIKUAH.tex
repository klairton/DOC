\begin{example}\label{example:UKGIKUAH}
  \scriptsize
  \hfill
  \begin{\position}
    \begin{tikzpicture}[scale=0.7]
      \begin{scope}[every node/.style={inner sep=1.5pt, minimum size = 0pt}]
        \node[circle, draw] (p0) at (0,0) {$+0$};
        \node[circle, draw] (m4) at (1.5,0) {$-4$};
        \node[circle, draw] (p4) at (3,0) {$+4$};
        \node[circle, draw] (m3) at (4.5,0) {$-3$};
        \node[circle, draw] (p3) at (6,0) {$+3$};
        \node[circle, draw] (p1) at (7.5,0) {$+1$};
        \node[circle, draw] (m1) at (9,0) {$-1$};
        \node[circle, draw] (m2) at (10.5,0) {$-2$};
        \node[circle, draw] (p2) at (12,0) {$+2$};
        \node[circle, draw] (m5) at (13.5,0) {$-5$};
        \node[circle, draw] (p5) at (15,0) {$+5$};
        \node[circle, draw] (m6) at (16.5,0) {$-6$};
      \end{scope}
      \begin{scope}[>={Stealth[black]},
                    every edge/.style={draw=black}]
        \path [-] (p0) edge (m4);
        \node[draw=none, fill=none, align=center, minimum width=1cm, text width=1cm] at (0.75, -1.0) {$\ell = {-1}$};
        \path [-] (p4) edge (m3);
        \node[draw=none, fill=none, align=center, minimum width=1cm, text width=1cm] at (3.75, -1.0) {$\ell = 2$};
        \path [-] (p3) edge (p1);
        \node[draw=none, fill=none, align=center, minimum width=1cm, text width=1cm] at (6.75, -1.0) {$\ell = {-3}$};
        \path [-] (m1) edge (m2);
        \node[draw=none, fill=none, align=center, minimum width=1cm, text width=1cm] at (9.75, -1.0) {$\ell = 4$};
        \path [-] (p2) edge (m5);
        \node[draw=none, fill=none, align=center, minimum width=1cm, text width=1cm] at (12.75, -1.0) {$\ell = 5$};
        \path [-] (p5) edge (m6);
        \node[draw=none, fill=none, align=center, minimum width=1cm, text width=1cm] at (15.75, -1.0) {$\ell = 6$};
      \end{scope}
      \begin{scope}[>={Stealth[black]},
                    every edge/.style={draw=black}]
        \path [-] (p0) edge [bend left=70, dashed] (m1);
        \path [-] (p1) edge [bend left=70, dashed] (m2);
        \path [-] (p2) edge [bend right=60, dashed] (m3);
        \path [-] (p3) edge [bend right=60, dashed] (m4);
        \path [-] (p4) edge [bend left=70, dashed] (m5);
        \path [-] (p5) edge [bend left=70, dashed] (m6);
      \end{scope}
    \end{tikzpicture}
  \end{\position}
\end{example}