% Escolha: Portugues ou Ingles ou Espanhol.
% Para a versão final do texto, após a defesa, acrescente Final:

\documentclass[Portugues]{ic-tese-v3}
%\documentclass[Portugues,Final]{ic-tese-v3}

\usepackage[latin1,utf8]{inputenc}
\usepackage[brazil]{babel}

% Para acrescentar comentários ao PDF descomente:
\usepackage
 % [pdfauthor={nome do autor},
  % pdftitle={titulo},
%   pdfkeywords={palavra-chave, palavra-chave},
%   pdfproducer={Latex with hyperref},
%   pdfcreator={pdflatex}]
{hyperref}

\usepackage{graphicx}

\usepackage{xcolor}
\newcommand{\todo}[1]{{\color{red}{\large TODO:} #1}}

\usepackage[portugues, vlined, noend, ruled, linesnumbered]{algorithm2e}

\usepackage{multirow}
\usepackage{enumerate}

\usepackage{amsfonts}
\usepackage{amssymb}
\usepackage{amsthm}
\usepackage{amsmath}

\newtheorem{theorem}{Teorema}[section]
\newtheorem{corollary}{Corolário}[theorem]
\newtheorem{lemma}[theorem]{Lema}

\theoremstyle{definition}
\newtheorem{definition}{Definição}[section]

\theoremstyle{remark}
\newtheorem{remark}{Observação}[section]

\theoremstyle{definition}
\newtheorem{example}{Exemplo}[section]

% Classic
\newcommand{\SbT}{\textbf{SbT}}
\newcommand{\SbR}{\textbf{SbR}}
\newcommand{\SbRT}{\textbf{SbRT}}
% Weighted
\newcommand{\SbWRT}{\textbf{Sb\textsubscript{W}RT}}
\newcommand{\SbWRTIT}{\textbf{Sb\textsubscript{W}RTIT}}
\newcommand{\SbWIRI}{\textbf{Sb\textsubscript{WI}RI}}
\newcommand{\SbWIRTI}{\textbf{Sb\textsubscript{WI}RTI}}
\newcommand{\SbWIRT}{\textbf{Sb\textsubscript{WI}RT}}
% Proportion
\newcommand{\SbPRT}{\textbf{Sb\textsubscript{P}RT}}
% Intergenic
\newcommand{\SbIBI}{\textbf{Sb\textsubscript{I}BI}}
\newcommand{\SbIR}{\textbf{Sb\textsubscript{I}R}}
\newcommand{\SbIRI}{\textbf{Sb\textsubscript{I}RI}}
\newcommand{\SbIRM}{\textbf{Sb\textsubscript{I}RM}}
\newcommand{\SbIRMI}{\textbf{Sb\textsubscript{I}RMI}}
\newcommand{\SbIT}{\textbf{Sb\textsubscript{I}T}}
\newcommand{\SbITM}{\textbf{Sb\textsubscript{I}TM}}
\newcommand{\SbIRT}{\textbf{Sb\textsubscript{I}RT}}
\newcommand{\SbIRTI}{\textbf{Sb\textsubscript{I}RTI}}
\newcommand{\SbIRTM}{\textbf{Sb\textsubscript{I}RTM}}
\newcommand{\SbIRTMI}{\textbf{Sb\textsubscript{I}RTMI}}
% Flexible
\newcommand{\SbFIR}{\textbf{Sb\textsubscript{FI}R}}
\newcommand{\SbFIRI}{\textbf{Sb\textsubscript{FI}RI}}
\newcommand{\SbFIRM}{\textbf{Sb\textsubscript{FI}RM}}
\newcommand{\SbFIRMI}{\textbf{Sb\textsubscript{FI}RMI}}
\newcommand{\SbFIT}{\textbf{Sb\textsubscript{FI}T}}
\newcommand{\SbFITM}{\textbf{Sb\textsubscript{FI}TM}}
\newcommand{\SbFIRT}{\textbf{Sb\textsubscript{FI}RT}}
\newcommand{\SbFIRTI}{\textbf{Sb\textsubscript{FI}RTI}}
\newcommand{\SbFIRTM}{\textbf{Sb\textsubscript{FI}RTM}}
\newcommand{\SbFIRTMI}{\textbf{Sb\textsubscript{FI}RTMI}}

\usepackage{environ}

\makeatletter
\newcommand{\problemtitle}[1]{\gdef\@problemtitle{#1}}% Store problem title
\newcommand{\probleminput}[1]{\gdef\@probleminput{#1}}% Store problem input
\newcommand{\problemtask}[1]{\gdef\@problemtask{#1}}% Store problem task
\newcommand{\problemquestion}[1]{\gdef\@problemquestion{#1}}% Store problem question

% Problem task enviroment
\NewEnviron{task}{
  \problemtitle{}\probleminput{}\problemtask{}% Default input is empty
  \BODY
  \begin{center}
  \fbox{
    \begin{minipage}{0.9\linewidth}
      \raggedright
      \parskip=1ex
      \textbf{\@problemtitle}          \\% Title
      \textbf{Entrada:} \@probleminput \\% Input
      \textbf{Objetivo:} \@problemtask \\% Goal
    \end{minipage}
  } 
  \end{center}
}
\NewEnviron{decision}{
  \problemtitle{}\probleminput{}\problemquestion{}% Default input is empty
  \BODY
  \begin{center}
  \fbox{
    \begin{minipage}{0.9\linewidth}
      \raggedright
      \parskip=1ex
      \@problemtitle                   \\% Title
      \textbf{Entrada:} \@probleminput \\% Input
      \textbf{Pergunta:} \@problemquestion \\% Goal
    \end{minipage}
  } 
  \end{center}
}
\makeatother

\usepackage{tikz}
\usetikzlibrary{arrows}
\usetikzlibrary{arrows.meta}
\usetikzlibrary{positioning, shapes.arrows, shapes.geometric, shapes.misc, backgrounds}
\usetikzlibrary{fit}

\tikzset{ 
signed gene/.style = { single arrow, draw = black, fill = white, minimum size = 10mm, single arrow head extend = 1mm },
unsigned gene/.style = { circle, draw = black, fill = white, minimum size = 10mm },
ir/.style = { rectangle, draw = black, fill = white, minimum size = 10mm, rounded corners = 5pt},
}


\tikzset{
pics/right gene/.style args={#1, #2}{
        code = {
        % \path[draw = black, fill = #2] (0.5, 0.0) -- (0.3, -0.5) -- (-0.5, -0.5) -- (-0.3, 0.0) -- (-0.5, 0.5) -- (0.3, 0.5) -- cycle;
        \path[draw = black, fill = #2] (0.5, 0.0) -- (0.3, -0.5) -- (-0.5, -0.5) -- (-0.5, 0.5) -- (0.3, 0.5) -- cycle;
        \node[black, draw = none, fill = none, minimum size = 10mm] at (0.0, 0.0) {#1};
        }
    },
pics/left gene/.style args={#1, #2}{
        code = {
        % \path[draw = black, fill = #2] (-0.5, 0.0) -- (-0.3, 0.5) -- (0.5, 0.5) -- (0.3, 0.0) -- (0.5, -0.5) -- (-0.3, -0.5) -- cycle;
        \path[draw = black, fill = #2] (-0.5, 0.0) -- (-0.3, 0.5) -- (0.5, 0.5) -- (0.5, -0.5) -- (-0.3, -0.5) -- cycle;
        \node[black, draw = none, fill = none, minimum size = 10mm] at (0.0, 0.0) {#1};
        }
    },
pics/gene/.style args={#1, #2}{
        code = {
        \node[black, circle, draw = black, fill = #2, minimum size = 10mm] at (0.0, 0.0) {#1};
        }
    },
pics/ir/.style args={#1, #2}{
        code = {
        \node[rectangle, draw = black, fill = #2, minimum size = 10mm, rounded corners = 5pt] at (0.0, 0.0) {#1};
        }
    },
pics/flex ir/.style args={#1, #2, #3}{
        code = {
        \node[rectangle, draw = black, fill = #3, minimum size = 10mm, rounded corners = 5pt] at (0.0, 0.0) {};
        \node [black] at (0.0, 0.25) {#2};
        \node [black] at (0.0, -0.25) {#1};
        }
    }
}




\begin{document}

% Escolha entre autor ou autora:
\autor{Klairton de Lima Brito}
%\autora{Nome da Autora}

% Sempre deve haver um título em português:
\titulo{Modelos com Proporção entre Operações e Regiões Intergênicas Rígidas e Flexíveis}

% Se a língua for o inglês ou o espanhol defina:
\title{Models with a Proportion Ratio between Operations and Rigid and Flexible Intergenic Regions}

% Escolha entre orientador ou orientadora. Inclua os títulos acadêmicos:
\orientador{Prof. Dr. Zanoni Dias}
%\orientadora{Profa. Dra. Nome da Orientadora}

% Escolha entre coorientador ou coorientadora, se houver.  Inclua os títulos acadêmicos:
\coorientador{Prof. Dr. Ulisses Martins Dias}
%\coorientadora{Profa. Dra. Eng. Lic. Nome da Co-Orientadora}

% Escolha entre mestrado ou doutorado:
% \mestrado
\doutorado

% Se houve cotutela, defina:
%\cotutela{Universidade Nova de Plutão}

\datadadefesa{15}{11}{2022}

% Para a versão final defina:
%\avaliadorA{Prof. Dr. Primeiro Avaliador}{Instituição do primeiro avaliador}
%\avaliadorB{Profa. Dra. Segunda Avaliadora}{Instituição da segunda avaliadora}
%\avaliadorC{Dr. Terceiro Avaliador}{Instituição do terceiro avaliador}
%\avaliadorD{Prof. Dr. Quarto Avaliador}{Instituição do quarto avaliador}
%\avaliadorE{Prof. Dr. Quinto Avaliador}{Instituição do quinto avaliador}
%\avaliadorF{Prof. Dr. Sexto Avaliador}{Instituição do sexto avaliador}
%\avaliadorG{Prof. Dr. Sétimo Avaliador}{Instituição do sétimo avaliador}
%\avaliadorH{Prof. Dr. Oitavo Avaliador}{Instituição do oitavo avaliador}


% Para incluir a ficha catalográfica em PDF na versão final, descomente e ajuste:
%\fichacatalografica{arquivo.pdf}


% Este comando deve ficar aqui:
\paginasiniciais


% Se houver dedicatória, descomente:
%\prefacesection{Dedicatória}
%A dedicatória deve ocupar uma única página.


% Se houver epígrafe, descomente e edite:
% \begin{epigrafe}
% {\it
% Vita brevis,\\
% ars longa,\\
% occasio praeceps,\\
% experimentum periculosum,\\
% iudicium difficile.}
%
% \hfill (Hippocrates)
% \end{epigrafe}


% Agradecimentos ou Acknowledgements ou Agradecimientos
\prefacesection{Agradecimentos}
Os agradecimentos devem ocupar uma única página.


% Sempre deve haver um resumo em português:
\begin{resumo}
O resumo deve ter no máximo 500 palavras e deve ocupar uma única página.
\end{resumo}


% Sempre deve haver um abstract:
\begin{abstract}
The abstract must have at most 500 words and must fit in a single page.
\end{abstract}


% Se houver um resumo em espanhol, descomente:
%\begin{resumen}
% A mesma regra aplica-se.
%\end{resumen}


% A lista de figuras é opcional:
\listoffigures

% A lista de tabelas é opcional:
\listoftables

% A lista de abreviações e siglas é opcional:
% \prefacesection{Lista de Abreviações e Siglas}

% A lista de símbolos é opcional:
% \prefacesection{Lista de Símbolos}

% Quem usa o pacote nomencl pode incluir:
%\renewcommand{\nomname}{Lista de Abreviações e Siglas}
%\printnomenclature[3cm]


% O sumário vem aqui:
\setcounter{tocdepth}{3}
\tableofcontents


% E esta linha deve ficar bem aqui:
\fimdaspaginasiniciais


% O corpo da dissertação ou tese começa aqui:
\chapter{Introdução}

O estudo da evolução dos organismos é uma tarefa de fundamental importância no campo da biologia. Ao decorrer do tempo mudanças podem ocorrer nos organismos, que refletem adaptações desenvolvidas para melhor se adequar e prosperar no ambiente que estão inseridos. Em particular, mudanças genéticas são uma das características utilizadas no campo da \emph{genômica comparativa} para estimar a proximidade de dois organismos com base na similaridade de seus materiais genéticos. O genoma pode sofrer modificações a partir de mutações que podem ser pontuais ou afetar grandes trechos do genoma, que são chamadas de eventos de rearrajos de genomas. Tais eventos podem afetar o genoma modificando, inserindo ou removendo material genético~\cite{2009-fertin-etal}. Uma das formas bem aceita de estimar a proximidade de dois organismos é determinando uma sequência de eventos de rearranjos de genomas com tamanho mínimo e capaz de transformar o genoma de um organismo em outro. O tamanho de tal sequência é chamada de \emph{distância de rearranjo}.

Reversão e transposição são os eventos de rearranjo mais estudados na literatura~\cite{2005-bergeron,1998-bafna-pevzner}. Uma reversão atua em um segmento do genoma invertendo a posição e a orientação dos genes contidos no segmento, enquanto uma transposição troca dois segmentos consecutivos do genoma, mas sem afetar a posição e a orientação dos genes nos segmentos. Os eventos de reversão e transposição são chamados de conservativos, pois não alteram a quantidade de material genético do genoma. Existem também eventos não conservativos, como é o caso dos eventos de inserção, deleção e duplicação~\cite{2013-willing-etal,2012-elmabrouk-sankoff,2008-kahn-raphael,2020-lafond-etal,2010-lin-etal,2009-bader}, que inserem, removem e duplicam material genético de uma região específica do genoma, respectivamente. Um modelo de rearranjo é caracterizado pelo conjunto de eventos de rearranjo permitidos para transformar um genoma em outro e a representação do genoma utilizada.

Um genoma pode ser representado computacionalmente de diferentes maneiras. Quando o genoma é tratado como uma sequência ordenada de genes, podemos encontrar casos em que determinados genes apresentam múltiplas cópias, sendo comum utilizarmos uma representação na forma de uma cadeia de caracteres, onde cada caractere é associado a um gene. Por outro lado, se existir apenas uma cópia de cada gene, podemos associar um número inteiro para cada gene e a representação é dada na forma de uma permutação. Em ambos os casos, quando a orientação dos genes é conhecida, um sinal de positivo ou negativo é atribuído para cada elemento e a representação é chamada com sinais (string com sinais e permutação com sinais). Caso contrário, o sinal é omitido e a representação é chamada sem sinais (string sem sinais e permutação sem sinais).

Ao utilizar a representação de um genoma como uma permutação, podemos simplificar o problema como sendo um problema de ordenação. Nesse caso, o objetivo consiste em transformar uma permutação $\pi$ qualquer em uma permutação específica na qual os elementos encontram-se ordenados de maneira crescente e com sinal positivo para o caso com sinais, essa permutação é chamada de identidade.

Quando consideramos um modelo de rearranjo composto apenas pelo evento de reversão e utilizando uma representação do genoma na forma de permutações com sinais, temos o problema de Ordenação de Permutações com Sinais por Reversões. Hannenhalli e Pevzner~\cite{1999-hannenhalli-pevzner} apresentaram o primeiro algoritmo exato em tempo polinomial para o problema, sendo posteriormente simplificado por Bergeron~\cite{2005-bergeron}. Atualmente, temos um algoritmo com complexidade subquadrática para determinar a sequência de reversões capaz de ordenar uma permutação com sinais~\cite{2007-tannier-etal}. Entretanto, se estivermos interessados somente na distância de reversão, existe um algoritmo que executa em tempo linear~\cite{2001-bader-etal}. Entretanto, quando consideramos uma representação utilizando permutações sem sinais, temos o problema de Ordenação de Permutações sem Sinais por Reversões. Caprara~\cite{1999-caprara} provou que o problema faz parte da classe de problemas NP-Difícil. Um dos primeiros algoritmos para o problema apresentou um fator de aproximação 1.75~\cite{1996-bafna-pevzner}. Em seguida, Christie~\cite{1998a-christie} apresentou um algoritmo com fator de aproximação 1.5. Atualmente, o melhor algoritmo para o problema apresenta um fator de aproximação 1.375~\cite{2002-berman-etal}.

Quando consideramos um modelo de rearranjo composto apenas pelo evento de transposição, a orientação dos genes não é considerada, tendo em vista que o evento de transposição não altera a orientação dos genes. Dessa forma, ao adotar permutações sem sinais, temos o problema de Ordenação de Permutações sem Sinais por Transposições. O problema também pertence à classe de problemas NP-Difícil, sendo a prova apresentada por Bulteau e coautores~\cite{2012-bulteau-etal}. O primeiro algoritmo para o problema foi proposto por Bafna e Pevzner~\cite{1998-bafna-pevzner} com um fator de aproximação 1.5. Posteriormente, Christie~\cite{1998b-christie} apresentou uma implementação mais simples para esse algoritmo. Atualmente, o melhor algoritmo para o problema apresenta um fator de aproximação 1.375~\cite{2006-elias-hartman} e heurísticas foram apresentadas por Dias e Dias~\cite{2010c-dias-dias} visando a obtenção de resultados práticos melhores.

Ao considerar um modelo de rearranjo composto pelos eventos de reversão e transposição em permutações com e sem sinais, obtemos os problemas de Ordenação de Permutações com Sinais por Reversões e Transposições, e Ordenação de Permutações sem Sinais por Reversões e Transposições, respectivamente. Ambos os problemas pertencem à classe de problemas NP-Difícil~\cite{2019b-oliveira-etal}. Os melhores algoritmos para os problemas apresentam fatores de aproximação $2$~\cite{1998-walter-etal} e $2k$~\cite{2008-rahman-etal} (onde $k$ é o fator de aproximação para a decomposição de ciclos~\cite{2013-chen}) para os casos com e sem sinais, respectivamente. Diversas heurísticas considerando esses problemas foram apresentadas na literatura~\cite{2014a-dias-etal,2018-brito-etal}.

Quando passamos a considerar que o genoma pode apresentar genes repetidos, em 2001, Christie e Irving~\cite{2001-christie-irving} mostraram que o problema de Distância de Strings sem Sinais por Reversões pertence à classe de problemas NP-Difícil, mesmo se considerarmos um alfabeto binário (os caracteres das strings comparadas pertencem ao conjunto $\{0,1\})$. Para isso, os autores apresentaram uma redução do problema 3-partition~\cite{1990-garey-johnson}. Em 2005, Radcliffe e coautores~\cite{2005-radcliffe-etal} mostraram que a Distância de Strings com Sinais por Reversões e Distância de Strings sem Sinais por Transposições também pertencem à classe de problemas NP-Difícil, mesmo se considerarmos um alfabeto binário. Outra contribuição importante do trabalho foi que os autores caracterizaram um conjunto de instâncias em que é possível obter uma solução ótima em tempo polinomial.

Uma relação entre o problema de Distância de Strings por Reversões e o problema de Partição Mínima em Strings foi apresentada por Chen \textit{et al.}~\cite{2005-chen-etal}. Com essa relação entre os problemas, foi apresentado por Kolman e Wale{\'n}~\cite{2006-kolman-walen} um algoritmo de aproximação com fator $\Theta(k)$ para o problema de Distância de Strings com e sem Sinais por Reversões, onde $k$ representa o número máximo de cópias de um caractere nas strings consideradas.

A representação do genoma como uma sequência ordenada de genes é uma abordagem simples e prática, mas acarreta na perda de informação referente às estruturas genéticas que não fazem parte da sequência de genes. Estudos apontaram que considerar informações adicionais contidas no genoma, além da sequência de genes, pode tornar a comparação entre genomas mais realista~\cite{2016a-biller-etal, 2016b-biller-etal} . Em particular, os pesquisadores abordaram a importância de considerar o tamanho das regiões presentes entre cada par de genes consecutivos e nas extremidades do genoma, chamadas de regiões intergênicas.

Trabalhos que levam em conta a sequência de genes e também consideram os tamanhos das regiões intergênicas começaram a ser apresentados recentemente. Fertin \textit{et al.}~\cite{2017-fertin-etal} apresentaram um modelo que permite o uso do evento de rearranjo Double-Cut and Join (DCJ), mostraram que o problema pertence à classe de problemas NP-Difícil e desenvolveram um algoritmo de aproximação com fator $4/3$. O evento de rearranjo DJC~\cite{2005-yancopoulos-etal} atua fragmentando o genoma em dois pontos e, em seguida, as extremidades dos segmentos resultantes são unidas obedecendo certas restrições. Bulteau \textit{et al.}~\cite{2016b-bulteau-etal} apresentaram um modelo que permite o uso do evento DCJ juntamente com os eventos não conservativos de inserção e deleção restritos a atuarem apenas sobre as regiões intergênicas. Para esse problema, os autores apresentaram um algoritmo exato em tempo polinomial. Oliveira \textit{et al.}~\cite{2018b-oliveira-etal} apresentaram um modelo que permite o uso apenas de reversões super curtas (esse evento de rearranjo possui uma restrição adicional que faz com que todo evento de reversão afete um segmento com no máximo dois genes). Juntamente com o modelo, os autores desenvolveram algoritmos de aproximação para o problema de forma geral e para instâncias do problema com características específicas.

Trabalhos considerando a ordem dos genes e o tamanho das regiões intergênicas são recentes, sendo uma promissora linha de pesquisa, tendo em vista as melhorias que podem ser obtidas nas estimativas para a distância evolutiva entre os organismos.

Neste capítulo, apresentaremos as formas como um genoma será representado e como os eventos de rearranjo de genomas podem afetá-lo. Além disso, definiremos o formato das instâncias utilizadas pelos problemas investigados nos capítulos seguintes e apresentaremos definições e conceitos utilizados para obtenção de resultados.

% ------------------------------------------------------------------ %
\section{Representação de Genomas}
% ------------------------------------------------------------------ %

Nesta seção, apresentamos três representações de genomas que diferem quanto às estruturas genéticas incorporadas na representação computacional.

Dado um genoma $\mathcal{G}=(\mathcal{G}_1,\:\mathcal{G}_2,\allowbreak\:\dots,\:\mathcal{G}_n)$ com $n$ genes não repetidos, utilizamos uma representação através de uma permutação $\pi=(\pi_1~\pi_2~\dots~\pi_n)$, de forma que cada elemento $\pi_i$, com $1 \le i \le n$, da permutação $\pi$ representa o gene $\mathcal{G}_i$ do genoma $\mathcal{G}$. Caso a orientação dos genes no genoma $\mathcal{G}$ seja conhecida, associamos um sinal ``$+$'' ou ``$-$'' a cada elemento $\pi_i \in \pi$ para representar a orientação de cada um dos genes de $\mathcal{G}$. Caso contrário, o sinal é omitido. Quando representamos um genoma utilizando apenas as informações obtidas com base no posicionamento dos genes, denominamos de \emph{representação clássica}. Além disso, denotamos por \emph{representação clássica com sinais} quando a orientação dos genes é conhecida e por \emph{representação clássica sem sinais} caso contrário. O Exemplo~\ref{example:AGNBBMYY} mostra representações clássicas com sinais e sem sinais de genomas fictícios. Os elementos coloridos com letras no interior representam os genes. Na parte superior os elementos possuem orientação e na parte inferior não possuem.

\begin{example}\label{example:AGNBBMYY}
  \hfill \break
  \scriptsize
  \begin{tikzpicture}
    \draw (0, 4) pic{left gene = {$e$, red!50}};
    \draw (1, 4) pic{right gene = {$b$, blue!50}};
    \draw (2, 4) pic{right gene = {$a$, orange!50}};
    \draw (3, 4) pic{left gene = {$d$, green!50}};
    \draw (4, 4) pic{right gene = {$c$, teal!50}};
    % \node[] at (2, 3) {$\pi = ({-5}~{+2}~{+1}~{-4}~{+3})$};
  \end{tikzpicture}
  \hfill \break
  \begin{tabular}{lll}
    $\pi$ & $=$ & $({-5}~{+2}~{+1}~{-4}~{+3})$\\
  \end{tabular}
  \hfill \break
  \begin{tikzpicture}
    \draw (0, 2) pic{gene = {$e$, red!50}};
    \draw (1, 2) pic{gene = {$a$, orange!50}};
    \draw (2, 2) pic{gene = {$b$, blue!50}};
    \draw (3, 2) pic{gene = {$c$, teal!50}};
    \draw (4, 2) pic{gene = {$d$, green!50}};
    % \node[] at (2, 1) {$\pi = ({5}~{1}~{2}~{3}~{4})$};
  \end{tikzpicture}
  \hfill \break
  \begin{tabular}{lll}
    $\pi$ & $=$ & $({5}~{1}~{2}~{3}~{4})$\\
  \end{tabular}
\end{example}

Dado um genoma $\mathcal{G}=(\mathfrak{R}_1,\mathcal{G}_1,\mathfrak{R}_2,\mathcal{G}_2\dots,\mathfrak{R}_n,\mathcal{G}_n,\mathfrak{R}_{n+1})$ com $n$ genes não repetidos $\{\mathcal{G}_1,\:\mathcal{G}_2,\:\dots,\:\mathcal{G}_n\}$ e $n+1$ regiões intergênicas $\{\mathfrak{R}_1,\:\mathfrak{R}_2,\:\dots,\:\mathfrak{R}_{n+1}$\}, utilizamos essas duas características para representar um genoma. As regiões intergênicas estão presentes nas extremidades do genoma e entre cada par de genes consecutivos. Denotamos o \emph{tamanho} de uma região intergênica pela quantidade de nucleotídeos contida nela. Representamos o genoma $\mathcal{G}$ utilizando dois elementos, sendo o primeiro elemento uma permutação $\pi=(\pi_1~\pi_2~\dots~\pi_n)$, de forma que cada elemento $\pi_i$, com $1 \le i \le n$, da permutação $\pi$ representa o gene $\mathcal{G}_i$ do genoma $\mathcal{G}$. Caso a orientação dos genes no genoma $\mathcal{G}$ seja conhecida, associamos um sinal ``$+$'' ou ``$-$'' a cada elemento $\pi_i \in \pi$ para representar a orientação de cada um dos genes de $\mathcal{G}$. Caso contrário, o sinal é omitido. O segundo elemento é uma lista de número inteiros não negativos $\breve\pi=(\breve\pi_1,\breve\pi_2,\dots,\breve\pi_{n+1})$, de forma que cada elemento $\breve\pi_i$, com $1 \le i \le {n+1}$, da lista $\breve\pi$ representa o tamanho da região intergênica $\mathfrak{R}_i$ do genoma $\mathcal{G}$. Quando representamos um genoma utilizando a informação do posicionamento dos genes e os tamanhos das regiões intergênicas, denominamos de \emph{representação intergênica rígida}. Além disso, denotamos por \emph{representação intergênica rígida com sinais} quando a orientação dos genes é conhecida e por \emph{representação intergênica rígida sem sinais} caso contrário. O Exemplo~\ref{example:ARGRKMMV} mostra representações intergênicas rígidas com sinais e sem sinais de genomas fictícios. Os elementos coloridos com letras no interior representam os genes, sendo que na parte superior eles possuem orientação e na parte inferior não possuem. Os retângulos com bordas arredondadas, localizados nas extremidades e entre cada par de genes, representam as regiões intergênicas, com o número no interior indicando seu tamanho. 

\begin{example}\label{example:ARGRKMMV}
  \hfill
  \begin{\position}
    \begin{tikzpicture}
      \draw (0, 6) pic{ir = {$5$, black!10}};
      \draw (2, 6) pic{ir = {$1$, black!10}};
      \draw (4, 6) pic{ir = {$0$, black!10}};
      \draw (6, 6) pic{ir = {$2$, black!10}};
      \draw (8, 6) pic{ir = {$1$, black!10}};
      \draw (10, 6) pic{ir = {$0$, black!10}};
      \draw (1, 6) pic{left gene = {$e$, red!50}};
      \draw (3, 6) pic{right gene = {$b$, blue!50}};
      \draw (5, 6) pic{right gene = {$a$, orange!50}};
      \draw (7, 6) pic{left gene = {$d$, green!50}};
      \draw (9, 6) pic{right gene = {$c$, teal!50}};
    \end{tikzpicture}
  \end{\position}
  \begin{\position}
    \vspace{-5mm}
    \begin{tabular}{lll}
      $\pi$ & $=$ & $({-5}~{+2}~{+1}~{-4}~{+3})$\\
      $\breve\pi$ & $=$ & $(5,1,0,2,1,0)$\\
    \end{tabular}
  \end{\position}
  \begin{\position}
    \begin{tikzpicture}
      \draw (0, 3) pic{ir = {$3$, black!10}};
      \draw (2, 3) pic{ir = {$2$, black!10}};
      \draw (4, 3) pic{ir = {$2$, black!10}};
      \draw (6, 3) pic{ir = {$0$, black!10}};
      \draw (8, 3) pic{ir = {$4$, black!10}};
      \draw (10, 3) pic{ir = {$1$, black!10}};
      \draw (1, 3) pic{gene = {$e$, red!50}};
      \draw (3, 3) pic{gene = {$a$, orange!50}};
      \draw (5, 3) pic{gene = {$b$, blue!50}};
      \draw (7, 3) pic{gene = {$c$, teal!50}};
      \draw (9, 3) pic{gene = {$d$, green!50}};
    \end{tikzpicture}
  \end{\position}
  \begin{\position}
    \vspace{-5mm}
    \begin{tabular}{lll}
      $\pi$ & $=$ & $({5}~{1}~{2}~{3}~{4})$\\
      $\breve\pi$ & $=$ & $(3,2,2,0,4,1)$\\
    \end{tabular}
  \end{\position}
\end{example}

Para tornar a especificação em relação ao tamanho de cada região intergênica menos rígida, criamos uma representação denominada de \emph{representação intergênica flexível}. Para isso, representamos um genoma $\mathcal{G}=(\mathfrak{R}_1,\mathcal{G}_1,\mathfrak{R}_2,\mathcal{G}_2\dots,\mathfrak{R}_n,\mathcal{G}_n,\mathfrak{R}_{n+1})$ com $n$ genes não repetidos $\{\mathcal{G}_1,\:\mathcal{G}_2,\:\dots,\:\mathcal{G}_n\}$ e $n+1$ regiões intergênicas $\{\mathfrak{R}_1,\:\mathfrak{R}_2,\:\dots,\:\mathfrak{R}_{n+1}$\} utilizando três elementos. O primeiro elemento é uma permutação $\pi=(\pi_1~\pi_2~\dots~\pi_n)$, de forma que cada elemento $\pi_i$, com $1 \le i \le n$, da permutação $\pi$ representa o gene $\mathcal{G}_i$ do genoma $\mathcal{G}$. Caso a orientação dos genes no genoma $\mathcal{G}$ seja conhecida, associamos um sinal ``$+$'' ou ``$-$'' a cada elemento $\pi_i \in \pi$ para representar a orientação de cada um dos genes de $\mathcal{G}$. Caso contrário, o sinal é omitido. Os demais elementos são duas listas de número inteiros não negativos $\breve\pi^{\min}=(\breve\pi^{\min}_1,\breve\pi^{\min}_2,\dots,\breve\pi^{\min}_{n+1})$ e $\breve\pi^{\max}=(\breve\pi^{\max}_1,\breve\pi^{\max}_2,\dots,\breve\pi^{\max}_{n+1})$, de forma que $\breve\pi^{\min}_i \le \breve\pi^{\max}_i$ e o tamanho da região intergênica $\mathfrak{R}_i$ pertence ao intervalo $[\breve\pi^{\min}_{i} .. \breve\pi^{\max}_{i}]$, com $1 \le i \le {n+1}$. Isso faz com que o tamanho de cada região intergênica seja flexível, tornando possível especificar um intervalo de valores aceitáveis para o tamanho de cada uma delas ao invés de apenas um único valor. Por fim, denotamos por \emph{representação intergênica flexível com sinais} quando a orientação dos genes é conhecida e por \emph{representação intergênica flexível sem sinais} caso contrário. O Exemplo~\ref{example:BIXCBOSI} mostra representações intergênicas flexíveis com sinais e sem sinais de genomas fictícios. Os elementos coloridos com letras no interior representam os genes, sendo que na parte superior eles possuem orientação e na parte inferior não possuem. Os retângulos com bordas arredondadas, localizados nas extremidades e entre cada par de genes, representam as regiões intergênicas. O número na parte superior de cada região intergênica indica o tamanho máximo permitido, enquanto o número na parte inferior indica o tamanho mínimo permitido.

\begin{example}\label{example:BIXCBOSI}
  \hfill \break
  \scriptsize
  \begin{tikzpicture}
    \draw (0, 6) pic{flex ir = {$3$, $6$, black!10}};
    \draw (2, 6) pic{flex ir = {$2$, $3$, black!10}};
    \draw (4, 6) pic{flex ir = {$2$, $4$, black!10}};
    \draw (6, 6) pic{flex ir = {$1$, $5$, black!10}};
    \draw (8, 6) pic{flex ir = {$3$, $6$, black!10}};
    \draw (10, 6) pic{flex ir = {$0$, $3$, black!10}};
    \draw (1, 6) pic{left gene = {$e$, red!50}};
    \draw (3, 6) pic{right gene = {$b$, blue!50}};
    \draw (5, 6) pic{right gene = {$a$, orange!50}};
    \draw (7, 6) pic{left gene = {$d$, green!50}};
    \draw (9, 6) pic{right gene = {$c$, teal!50}};
    % \node[] at (5, 5) {$\pi = ({-5}~{+2}~{+1}~{-4}~{+3})$};
    % \node[] at (5, 4.5) {$\breve\pi^{\min} = (3,2,2,1,3,0)$};
    % \node[] at (5, 4) {$\breve\pi^{\max} = (6,3,4,5,6,3)$};
  \end{tikzpicture}
  \hfill \break
  \begin{tabular}{lll}
    $\pi$ & $=$ & $({-5}~{+2}~{+1}~{-4}~{+3})$\\
    $\breve\pi^{\min}$ & $=$ & $(3,2,2,1,3,0)$\\
    $\breve\pi^{\max}$ & $=$ & $(6,3,4,5,6,3)$\\
  \end{tabular}
  \hfill \break
  \begin{tikzpicture}
    \draw (0, 3) pic{flex ir = {$2$, $5$, black!10}};
    \draw (2, 3) pic{flex ir = {$5$, $7$, black!10}};
    \draw (4, 3) pic{flex ir = {$2$, $2$, black!10}};
    \draw (6, 3) pic{flex ir = {$4$, $5$, black!10}};
    \draw (8, 3) pic{flex ir = {$1$, $9$, black!10}};
    \draw (10, 3) pic{flex ir = {$3$, $4$, black!10}};
    \draw (1, 3) pic{gene = {$e$, red!50}};
    \draw (3, 3) pic{gene = {$a$, orange!50}};
    \draw (5, 3) pic{gene = {$b$, blue!50}};
    \draw (7, 3) pic{gene = {$c$, teal!50}};
    \draw (9, 3) pic{gene = {$d$, green!50}};
    % \node[] at (5, 2) {$\pi = ({5}~{1}~{2}~{3}~{4})$};
    % \node[] at (5, 1.5) {$\breve\pi^{\min} = (2,5,2,4,1,3)$};
    % \node[] at (5, 1) {$\breve\pi^{\max} = (5,7,2,5,9,4)$};
  \end{tikzpicture}
  \hfill \break
  \begin{tabular}{lll}
    $\pi$ & $=$ & $({5}~{1}~{2}~{3}~{4})$\\
    $\breve\pi^{\min}$ & $=$ & $(2,5,2,4,1,3)$\\
    $\breve\pi^{\max}$ & $=$ & $(5,7,2,5,9,4)$\\
  \end{tabular}
\end{example}

Dada a representação $\mathcal{R}$ de um genoma $\mathcal{G}$, seja na forma clássica $\mathcal{R}=(\pi)$, intergênica rígida $\mathcal{R}=(\pi,\breve\pi)$ ou intergênica flexível $\mathcal{R}=(\pi,\breve\pi^{\min},\breve\pi^{\max})$, obtemos sua versão estendida adicionando dois novos elementos em $\pi$, com $\pi_0 = 0$ e $\pi_{n+1} = {n+1}$ inseridos no início e no fim da permutação $\pi$, respectivamente. Esses dois novos elementos adicionados em $\pi$ representam genes fictícios que não serão afetados por nenhum evento de rearranjo de genomas, e serão utilizados apenas para tornar algumas definições mais simples. De agora em diante, assumimos que qualquer representação de genoma estará na sua forma estendida, a não ser que seja dito expressamente o contrário.

% ------------------------------------------------------------------ %
\section{Eventos de Rearranjo}
% ------------------------------------------------------------------ %

Nesta seção, apresentamos os eventos de rearranjo considerados nesta tese e como eles podem afetar o genoma dependendo da representação utilizada. 

Os eventos de rearranjo de genomas são classificados em \emph{conservativos} ou \emph{não conservativos}. Os eventos de rearranjo conservativos não alteram a quantidade de material genético do genoma, enquanto os eventos de rearranjo não conservativos, sim. Dado um evento de rearranjo $\gamma$ e uma representação $\mathcal{R}$ de um genoma, denotamos por $\mathcal{R} \cdot  \gamma$ o genoma resultante após a aplicação do evento de rearranjo $\gamma$ em $\mathcal{R}$. De maneira similar, quando temos uma sequência de eventos de rearranjo $S=(\gamma_1,\gamma_2,\dots,\gamma_k)$ e uma representação $\mathcal{R}$ de um genoma, denotamos por $\mathcal{R} \cdot S = \mathcal{R} \cdot \gamma_1 \cdot \gamma_2 \cdot \ldots \cdot \gamma_k$ como sendo o genoma resultante após a aplicação da sequência $S$ em $\mathcal{R}$. A seguir, mostramos como os eventos de rearranjo conservativos de reversão e transposição afetam a representação clássica de um genoma.

\begin{definition}
Dada uma representação clássica de um genoma $\mathcal{R} = (\pi)$, sejam $i$ e $j$ números inteiros tais que $1 \le i \le j \le n$. Uma reversão $\rho^{(i,j)}$ inverte o segmento $(\pi_i~\pi_{i+1}~\dots~\pi_{j-1}~\pi_j)$ de $\pi$. Caso a representação $\mathcal{R}$ do genoma seja clássica com sinais, o sinal de cada elemento no segmento $(\pi_i~\pi_{i+1}~\dots~\pi_{j-1}~\pi_j)$ também é invertido.
\end{definition}

Os exemplos~\ref{example:BJBDRCHL} e~\ref{example:COJXWMAC} mostram uma reversão $\rho^{(i,j)}$ genérica sendo aplicada em uma representação clássica com e sem sinais de um genoma, respectivamente.

\begin{example}\label{example:BJBDRCHL}
  \hfill \break
  % \centering
  \begin{tabular}{lll}
    $\pi$ & $=$ & $(\pi_0~\pi_1~\dots~\pi_{i-1}~\underline{\pi_{i}~\pi_{i+1}~\dots~\pi_{j-1}~\pi_{j}}~\pi_{j+1}~\dots~\pi_{n}~\pi_{n+1})$ \\
    $\pi \cdot \rho^{(i,j)}$ & $=$ & $(\pi_0~\pi_1~\dots~\pi_{i-1}~\underline{{-\pi_{j}}~{-\pi_{j-1}}~\dots~{-\pi_{i+1}}~{-\pi_{i}}}~\pi_{j+1}~\dots~\pi_{n}~\pi_{n+1})$ \\
  \end{tabular}
\end{example}

\input{examples/COJXWMAC}

O Exemplo~\ref{example:DCRMAYUA} mostra uma reversão $\rho^{(2,4)}$ sendo aplicada na representação clássica com sinais $\mathcal{R} = (\pi) = ({+0}~{-3}~{+2}~{-4}~{+1}~{+5}~{+6})$ de um genoma, enquanto o Exemplo~\ref{example:DOFVIMRX} mostra uma reversão $\rho^{(1,5)}$ sendo aplicada na representação clássica sem sinais $\mathcal{R} = (\pi) = ({0}~{4}~{5}~{3}~{2}~{1}~{6})$ de um genoma.

\begin{example}\label{example:DCRMAYUA}
  \hfill \break
  % \centering
  \begin{tabular}{lll}
    $\pi$ & $=$ & $({+0}~{-3}~\underline{{+2}~{-4}~{+1}}~{+5}~{+6})$ \\
    $\pi \cdot \rho^{(2,4)}$ & $=$ & $({+0}~{-3}~\underline{{-1}~{+4}~{-2}}~{+5}~{+6})$ \\
  \end{tabular}
\end{example}

\begin{example}\label{example:DOFVIMRX}
  \hfill \break
  % \centering
  \begin{tabular}{lll}
    $\pi$ & $=$ & $({0}~\underline{{4}~{5}~{3}~{2}~{1}}~{6})$ \\
    $\pi \cdot \rho^{(1,5)}$ & $=$ & $({0}~\underline{{1}~{2}~{3}~{5}~{4}}~{6})$ \\
  \end{tabular}
\end{example}

\begin{definition}
Dada uma representação clássica de um genoma $\mathcal{R} = (\pi)$, sejam $i$, $j$ e $k$ números inteiros tais que $1 \le i < j < k \le n + 1$. Uma transposição $\tau^{(i,j,k)}$ troca a posição dos segmentos consecutivos $(\pi_i~\pi_{i+1}~\dots~\pi_{j-1})$ e $(\pi_j~\pi_{j+1}~\dots~\pi_{k-1})$ de $\pi$.
\end{definition}

O Exemplo~\ref{example:EZZBDRDA} mostra uma transposição $\tau^{(i,j,k)}$ genérica sendo aplicada em uma representação clássica de um genoma. Note que a transposição pode ser aplicada em ambas as representações clássicas, com e sem sinais.

\begin{example}\label{example:EZZBDRDA}
  \hfill
  \begin{\position}
    \begin{tabular}{lll}
      $\pi$ & $=$ & $(\pi_0~\pi_1~\dots~\pi_{i-1}~\underline{\pi_{i}~\pi_{i+1}~\dots~\pi_{j-1}}~\underline{\pi_{j}~\pi_{j+1}~\dots~\pi_{k-1}}~\pi_{k}~\dots~\pi_{n}~\pi_{n+1})$ \\
      $\pi \cdot \tau^{(i,j,k)}$ & $=$ & $(\pi_0~\pi_1~\dots~\pi_{i-1}~\underline{\pi_{j}~\pi_{j+1}~\dots~\pi_{k-1}}~\underline{\pi_{i}~\pi_{i+1}~\dots~\pi_{j-1}}~\pi_{k}~\dots~\pi_{n}~\pi_{n+1})$ \\
    \end{tabular}
  \end{\position}
\end{example}

O Exemplo~\ref{example:HFLGUNSS} mostra uma transposição $\tau^{(1,3,5)}$ sendo aplicada na representação clássica com sinais $\mathcal{R} = (\pi) = ({+0}~{-4}~{-3}~{+1}~{+2}~{+5}~{+6})$ de um genoma, enquanto o Exemplo~\ref{example:HGHICRCG} mostra uma transposição $\tau^{(4,5,6)}$ sendo aplicada na representação clássica sem sinais $\mathcal{R} = (\pi) = ({0}~{3}~{2}~{1}~{5}~{4}~{6})$ de um genoma.

\begin{example}\label{example:HFLGUNSS}
  \hfill \break
  % \centering
  \begin{tabular}{lll}
    $\pi$ & $=$ & $({+0}~\underline{{-4}~{-3}}~\underline{{+1}~{+2}}~{+5}~{+6})$ \\
    $\pi \cdot \tau^{(1,3,5)}$ & $=$ & $({+0}~\underline{{+1}~{+2}}~\underline{{-4}~{-3}}~{+5}~{+6})$ \\
  \end{tabular}
\end{example}

\begin{example}\label{example:HGHICRCG}
  \hfill
  \begin{\position}
    \begin{tabular}{lll}
      $\pi$ & $=$ & $({0}~{3}~{2}~{1}~\underline{{5}}~\underline{{4}}~{6})$ \\
      $\pi \cdot \tau^{(4,5,6)}$ & $=$ & $({0}~{3}~{2}~{1}~\underline{{4}}~\underline{{5}}~{6})$ \\
    \end{tabular}
  \end{\position}
\end{example}

A seguir, mostramos como os eventos de rearranjo conservativos de reversão intergênica, transposição intergênica e move intergênico afetam a representação intergênica rígida de um genoma. 

\begin{definition}
Dada uma representação intergênica rígida de um genoma $\mathcal{R} = (\pi,\breve\pi)$, sejam $i$, $j$, $x$ e $y$ números inteiros tais que $1 \le i \le j \le n$, $0 \le x \le \breve\pi_i$ e $0 \le y \le \breve\pi_{j+1}$. Uma reversão intergênica $\rho^{(i, j)}_{(x, y)}$ divide as regiões intergênicas $\breve\pi_i$ e $\breve\pi_{j+1}$ da seguinte forma: $\breve\pi_i$ em duas partes com tamanhos $x$ e $x^{\prime}$, onde $x^{\prime}=\breve\pi_i-x$; e $\breve\pi_{j+1}$ em duas partes com tamanhos $y$ e $y^{\prime}$, onde $y^{\prime}=\breve\pi_{j+1}-y$. Em seguida, o segmento $(x^{\prime},\pi_i,\breve\pi_{i+1}\dots\breve\pi_j,\pi_j,y)$ do genoma é invertido. Caso a representação seja com sinais, os sinais dos elementos de $\pi_i$ até $\pi_{j}$ também são invertidos. Por fim, os segmentos do genoma são remontados com os pares de partes $(x,y)$ e $(x^{\prime},y^{\prime})$ fundindo-se e formando as novas regiões intergênicas $\breve\pi_i$ e $\breve\pi_{j+1}$ com tamanhos $x + y$ e $x^{\prime}+y^{\prime}$, respectivamente.
\end{definition}

O Exemplo~\ref{example:HJOTQJWJ} mostra uma reversão intergênica $\rho^{(i, j)}_{(x, y)}$ genérica sendo aplicada em uma representação intergênica rígida com sinais de um genoma.

\input{examples/HJOTQJWJ}

O Exemplo~\ref{example:KQWRJKOC} mostra uma reversão intergênica $\rho^{(i, j)}_{(x, y)}$ genérica sendo aplicada em uma representação intergênica rígida sem sinais de um genoma.

\pagebreak

\begin{example}\label{example:KQWRJKOC}
  \scriptsize
  \hfill
  \begin{\position}
    \begin{tikzpicture}
      \draw pic at (1, 6) {ir = {$\cdots$, black!10}};
      \draw pic at (3, 6) {ir = {${\breve\pi_{i}}$, black!10}};
      \draw pic at (5, 6) {ir = {$\cdots$, black!10}};
      \draw pic at (7, 6) {ir = {${\breve\pi_{j+1}}$, black!10}};
      \draw pic at (9, 6) {ir = {$\cdots$, black!10}};
      \draw pic at (0, 6) {gene = {${\pi_{0}}$, red!50}};
      \draw pic at (2, 6) {gene = {${\pi_{i-1}}$, orange!50}};
      \draw pic at (4, 6) {gene = {${\pi_{i}}$, blue!50}};
      \draw pic at (6, 6) {gene = {${\pi_{j}}$, teal!50}};
      \draw pic at (8, 6) {gene = {${\pi_{j+1}}$, green!50}};
      \draw pic at (10, 6) {gene = {${\pi_{{n+1}}}$, brown!50}};
      \path[draw = black] (3, 6.7) -- (3, 7) -- (7,7) -- (7,6.7); 
      \draw pic at (1, 4) {ir = {$\cdots$, black!10}};
      \draw pic at (3, 4) {ir = {$x|x^{\prime}$, black!10}};
      \draw pic at (5, 4) {ir = {$\cdots$, black!10}};
      \draw pic at (7, 4) {ir = {$y|y^{\prime}$, black!10}};
      \draw pic at (9, 4) {ir = {$\cdots$, black!10}};
      \draw pic at (0, 4) {gene = {${\pi_{0}}$, red!50}};
      \draw pic at (2, 4) {gene = {${\pi_{i-1}}$, orange!50}};
      \draw pic at (4, 4) {gene = {${\pi_{i}}$, blue!50}};
      \draw pic at (6, 4) {gene = {${\pi_{j}}$, teal!50}};
      \draw pic at (8, 4) {gene = {${\pi_{j+1}}$, green!50}};
      \draw pic at (10, 4) {gene = {${\pi_{{n+1}}}$, brown!50}};
      \path[draw = black] (3, 4.7) -- (3, 5) -- (7,5) -- (7,4.7); 
      \draw pic at (1, 2) {ir = {$\cdots$, black!10}};
      \draw pic at (3, 2) {ir = {$x|y$, black!10}};
      \draw pic at (5, 2) {ir = {$\cdots$, black!10}};
      \draw pic at (7, 2) {ir = {$x^{\prime}|y^{\prime}$, black!10}};
      \draw pic at (9, 2) {ir = {$\cdots$, black!10}};
      \draw pic at (0, 2) {gene = {${\pi_{0}}$, red!50}};
      \draw pic at (2, 2) {gene = {${\pi_{i-1}}$, orange!50}};
      \draw pic at (4, 2) {gene = {${\pi_{j}}$, teal!50}};
      \draw pic at (6, 2) {gene = {${\pi_{i}}$, blue!50}};
      \draw pic at (8, 2) {gene = {${\pi_{j+1}}$, green!50}};
      \draw pic at (10, 2) {gene = {${\pi_{{n+1}}}$, brown!50}};
    \end{tikzpicture}
  \end{\position}
\end{example}

O Exemplo~\ref{example:KXVBWBTB} mostra uma reversão intergênica $\rho^{(2,4)}_{(2,0)}$ sendo aplicada na representação intergênica rígida com sinais $\mathcal{R} = (\pi,\breve\pi) = \allowbreak(({+0}~{-3}~{+2}~{-4}~{+1}~{+5}~{+6}),\allowbreak(1,4,4,2,0,3))$ de um genoma, enquanto o Exemplo~\ref{example:KXXIMDRH} mostra uma reversão intergênica $\rho^{(1,5)}_{(1,2)}$ sendo aplicada na representação intergênica rígida sem sinais $\mathcal{R} = (\pi,\breve\pi) = \allowbreak(({0}~{4}~{5}~{3}~{2}~{1}~{6}),\allowbreak(1,1,7,3,0,2))$ de um genoma. As regiões intergênicas marcadas com sobrescrito podem ter seus tamanhos alterados pelo evento, enquanto as regiões intergênicas marcadas com subscrito sofrem apenas uma troca de posição.

\begin{example}\label{example:KXVBWBTB}
  \hfill
  \begin{\position}
    \begin{tabular}{lll}
      $(\pi,\breve\pi)$ & $=$ & $(({+0}~{-3}~\underline{{+2}~{-4}~{+1}}~{+5}~{+6}),(1,\overline{4},\underline{4,2},\overline{0},3))$ \\
      $(\pi,\breve\pi) \cdot \rho^{(2,4)}_{(2,0)}$ & $=$ & $(({+0}~{-3}~\underline{{-1}~{+4}~{-2}}~{+5}~{+6}),(1,\overline{2},\underline{2,4},\overline{2},3))$ \\
    \end{tabular}
  \end{\position}
\end{example}

\begin{example}\label{example:KXXIMDRH}
  \hfill \break
  % \centering
  \begin{tabular}{lll}
    $(\pi,\breve\pi)$ & $=$ & $(({0}~\underline{{4}~{5}~{3}~{2}~{1}}~{6}),(\overline{1},\underline{1,7,3,0},\overline{2}))$ \\
    $(\pi,\breve\pi) \cdot \rho^{(1,5)}_{(1,2)}$ & $=$ & $(({0}~\underline{{1}~{2}~{3}~{5}~{4}}~{6}),(\overline{3},\underline{0,3,7,1},\overline{0}))$ \\
  \end{tabular}
\end{example}

\begin{definition}
Dada uma representação intergênica rígida de um genoma $\mathcal{R} = (\pi,\breve\pi)$, sejam $i$, $j$, $k$, $x$, $y$ e $z$ números inteiros tais que $1 \le i < j < k \le n+1$, $0 \le x \le \breve\pi_i$, $0 \le y \le \breve\pi_j$ e $0 \le z \le \breve\pi_k$. Uma transposição intergênica $\tau^{(i,j,k)}_{(x,y,z)}$ divide as regiões intergênicas $\breve\pi_i$, $\breve\pi_{j}$ e $\breve\pi_k$ da seguinte forma: $\breve\pi_i$ em duas partes com tamanhos $x$ e $x^{\prime}$, onde $x^{\prime}=\breve\pi_i-x$; $\breve\pi_{j}$ em duas partes com tamanhos $y$ e $y^{\prime}$, onde $y^{\prime}=\breve\pi_{j}-y$; e $\breve\pi_{k}$ em duas partes com tamanhos $z$ e $z^{\prime}$, onde $z^{\prime}=\breve\pi_{k}-z$. Em seguida, os segmentos consecutivos $(x^{\prime},\pi_i,\breve\pi_{i+1},\dots \breve \pi_{j-1},\pi_{j-1},y)$ e $(y^{\prime},\pi_j,\breve\pi_{j+1}\dots \breve\pi_{k-1},\pi_{k-1},z)$ trocam de posição sem alterar a orientação dos genes contidos nos segmentos. Por fim, os segmentos do genoma são remontados com os pares de partes $(x,y^{\prime})$, $(z,x^{\prime})$ e $(y,z^{\prime})$ fundindo-se e formando as novas regiões intergênicas $\breve\pi_{i}$, $\breve\pi_{k+i-j}$, e $\breve\pi_{k}$ com tamanhos $x + y^{\prime}$, $z + x^{\prime}$ e $y + z^{\prime}$, respectivamente.
\end{definition}

O Exemplo~\ref{example:LIZCKUBG} mostra uma transposição intergênica $\tau^{(i,j,k)}_{(x,y,z)}$ genérica sendo aplicada em uma representação intergênica rígida de um genoma. Note que, caso a representação utilizada seja com sinais, o evento não altera a orientação dos genes nos segmentos afetados.

\pagebreak

\begin{example}\label{example:LIZCKUBG}
  \scriptsize
  \hfill
  \begin{\position}
    \begin{tikzpicture}
      \draw pic at (1, 6) {ir = {$\cdots$, black!10}};
      \draw pic at (3, 6) {ir = {${\breve\pi_{i}}$, black!10}};
      \draw pic at (5, 6) {ir = {$\cdots$, black!10}};
      \draw pic at (7, 6) {ir = {${\breve\pi_{j}}$, black!10}};
      \draw pic at (9, 6) {ir = {$\cdots$, black!10}};
      \draw pic at (11, 6) {ir = {${\breve\pi_{k}}$, black!10}};
      \draw pic at (13, 6) {ir = {$\cdots$, black!10}};
      \draw pic at (0, 6) {gene = {${\pi_{0}}$, red!50}};
      \draw pic at (2, 6) {gene = {${\pi_{i-1}}$, orange!50}};
      \draw pic at (4, 6) {gene = {${\pi_{i}}$, blue!50}};
      \draw pic at (6, 6) {gene = {${\pi_{j-1}}$, teal!50}};
      \draw pic at (8, 6) {gene = {${\pi_{j}}$, green!50}};
      \draw pic at (10, 6) {gene = {${\pi_{k-1}}$, brown!50}};
      \draw pic at (12, 6) {gene = {${\pi_{k}}$, violet!50}};
      \draw pic at (14, 6) {gene = {${\pi_{{n+1}}}$, purple!50}};
      \path[draw = black] (3, 6.7) -- (3, 7) -- (7, 7) -- (7, 6.7) -- (7, 7) -- (11, 7) -- (11, 6.7); 
      \draw pic at (1, 4) {ir = {$\cdots$, black!10}};
      \draw pic at (3, 4) {ir = {$x|x^{\prime}$, black!10}};
      \draw pic at (5, 4) {ir = {$\cdots$, black!10}};
      \draw pic at (7, 4) {ir = {$y|y^{\prime}$, black!10}};
      \draw pic at (9, 4) {ir = {$\cdots$, black!10}};
      \draw pic at (11, 4) {ir = {$z|z^{\prime}$, black!10}};
      \draw pic at (13, 4) {ir = {$\cdots$, black!10}};
      \draw pic at (0, 4) {gene = {${\pi_{0}}$, red!50}};
      \draw pic at (2, 4) {gene = {${\pi_{i-1}}$, orange!50}};
      \draw pic at (4, 4) {gene = {${\pi_{i}}$, blue!50}};
      \draw pic at (6, 4) {gene = {${\pi_{j-1}}$, teal!50}};
      \draw pic at (8, 4) {gene = {${\pi_{j}}$, green!50}};
      \draw pic at (10, 4) {gene = {${\pi_{k-1}}$, brown!50}};
      \draw pic at (12, 4) {gene = {${\pi_{k}}$, violet!50}};
      \draw pic at (14, 4) {gene = {${\pi_{{n+1}}}$, purple!50}};
      \path[draw = black] (3, 4.7) -- (3, 5) -- (7, 5) -- (7, 4.7) -- (7, 5) -- (11, 5) -- (11, 4.7); 
      \draw pic at (1, 2) {ir = {$\cdots$, black!10}};
      \draw pic at (3, 2) {ir = {$x|y^{\prime}$, black!10}};
      \draw pic at (5, 2) {ir = {$\cdots$, black!10}};
      \draw pic at (7, 2) {ir = {$z|x^{\prime}$, black!10}};
      \draw pic at (9, 2) {ir = {$\cdots$, black!10}};
      \draw pic at (11, 2) {ir = {$y|z^{\prime}$, black!10}};
      \draw pic at (13, 2) {ir = {$\cdots$, black!10}};
      \draw pic at (0, 2) {gene = {${\pi_{0}}$, red!50}};
      \draw pic at (2, 2) {gene = {${\pi_{i-1}}$, orange!50}};
      \draw pic at (4, 2) {gene = {${\pi_{j}}$, green!50}};
      \draw pic at (6, 2) {gene = {${\pi_{k-1}}$, brown!50}};
      \draw pic at (8, 2) {gene = {${\pi_{i}}$, blue!50}};
      \draw pic at (10, 2) {gene = {${\pi_{j-1}}$, teal!50}};
      \draw pic at (12, 2) {gene = {${\pi_{k}}$, violet!50}};
      \draw pic at (14, 2) {gene = {${\pi_{{n+1}}}$, purple!50}};
    \end{tikzpicture}
  \end{\position}
\end{example}

O Exemplo~\ref{example:LLNCEMFB} mostra uma transposição intergênica $\tau^{(1,3,6)}_{(1,1,3)}$ sendo aplicada na representação intergênica rígida com sinais $\mathcal{R} = (\pi,\breve\pi) = \allowbreak(({+0}~{-4}~{-3}~{+1}~{+2}~{+5}~{+6}),\allowbreak(3,0,2,2,4,7))$ de um genoma, enquanto o Exemplo~\ref{example:LRYMWOFK} mostra uma transposição intergênica $\tau^{(4,5,6)}_{(0,0,1)}$ sendo aplicada na representação intergênica rígida sem sinais $\mathcal{R} = (\pi,\breve\pi) = \allowbreak(({0}~{3}~{2}~{1}~{5}~{4}~{6}),\allowbreak(3,2,4,1,0,2))$ de um genoma. As regiões intergênicas marcadas com sobrescrito podem ter seus tamanhos alterados pelo evento, enquanto as regiões intergênicas marcadas com subscrito sofrem apenas uma troca de posição.

\begin{example}\label{example:LLNCEMFB}
  \hfill
  \begin{\position}
    \begin{tabular}{lll}
      $(\pi,\breve\pi)$ & $=$ & $(({+0}~\underline{{-4}~{-3}}~\underline{{+1}~{+2}~{+5}}~{+6}),(\overline{3},\underline{0},\overline{2},\underline{2,4},\overline{7}))$ \\
      $(\pi,\breve\pi) \cdot \tau^{(1,3,6)}_{(1,1,3)}$ & $=$ & $(({+0}~\underline{{+1}~{+2}~{+5}}~\underline{{-4}~{-3}}~{+6}),(\overline{2},\underline{2,4},\overline{5},\underline{0},\overline{5}))$ \\
    \end{tabular}
  \end{\position}
\end{example}

\begin{example}\label{example:LRYMWOFK}
  \hfill
  \begin{\position}
    \begin{tabular}{lll}
      $(\pi,\breve\pi)$ & $=$ & $(({0}~{3}~{2}~{1}~\underline{{5}}~\underline{{4}}~{6}),(3,2,4,\overline{1},\overline{0},\overline{2}))$ \\
      $(\pi,\breve\pi) \cdot \tau^{(4,5,6)}_{(0,0,1)}$ & $=$ & $(({0}~{3}~{2}~{1}~\underline{{4}}~\underline{{5}}~{6}),(3,2,4,\overline{0},\overline{2},\overline{1}))$ \\
    \end{tabular}
  \end{\position}
\end{example}

\begin{definition}
Dada uma representação intergênica rígida de um genoma $\mathcal{R} = (\pi,\breve\pi)$, sejam $i$, $j$ e $x$ números inteiros tais que $1 \le i, j \le n$ e $0 \le x \le \breve\pi_i$. Um move intergênico $\mu^{(i,j)}_{(x)}$ transfere $x$ nucleotídeos da região intergênica $\breve\pi_i$ para a região intergênica $\breve\pi_{j}$.
\end{definition}

O Exemplo~\ref{example:LZHMUNDG} mostra um move intergênico $\mu^{(2,5)}_{(3)}$ sendo aplicado na representação intergênica rígida com sinais $\mathcal{R} = (\pi,\breve\pi) = \allowbreak(({+0}~{-3}~{+2}~{-4}~{+1}~{+5}~{+6}),\allowbreak(1,4,4,2,0,3))$ de um genoma, enquanto o Exemplo~\ref{example:NLNYPHRB} mostra um move intergênico $\mu^{(3,5)}_{(5)}$ sendo aplicado na representação intergênica rígida sem sinais $\mathcal{R} = (\pi,\breve\pi) = \allowbreak(({0}~{4}~{5}~{3}~{2}~{1}~{6}),\allowbreak(1,1,7,3,0,2))$ de um genoma. As regiões intergênicas marcadas com sobrescrito sofrem alteração no seu tamanho causada pelo evento.

\begin{example}\label{example:LZHMUNDG}
  \hfill
  \begin{\position}
    \begin{tabular}{lll}
      $(\pi,\breve\pi)$ & $=$ & $(({+0}~{-3}~{+2}~{-4}~{+1}~{+5}~{+6}),(1,\overline{4},4,2,\overline{0},3))$ \\
      $(\pi,\breve\pi) \cdot \mu^{(2,5)}_{(3)}$ & $=$ & $(({+0}~{-3}~{+2}~{-4}~{+1}~{+5}~{+6}),(1,\overline{1},4,2,\overline{3},3))$ \\
    \end{tabular}
  \end{\position}
\end{example}

\pagebreak

\begin{example}\label{example:NLNYPHRB}
  \hfill
  \begin{\position}
    \begin{tabular}{lll}
      $(\pi,\breve\pi)$ & $=$ & $(({0}~{4}~{5}~{3}~{2}~{1}~{6}),(1,1,\overline{7},3,\overline{0},2))$ \\
      $(\pi,\breve\pi) \cdot \mu^{(3,5)}_{(5)}$ & $=$ & $(({0}~{4}~{5}~{3}~{2}~{1}~{6}),(1,1,\overline{2},3,\overline{5},2))$ \\
    \end{tabular}
  \end{\position}
\end{example}

A seguir, mostramos como o evento de rearranjo não conservativo de indel intergênico afeta a representação intergênica rígida de um genoma.

\begin{definition}
Dada uma representação intergênica rígida de um genoma $\mathcal{R} = (\pi,\breve\pi)$, sejam $i$ e $x$ números inteiros tais que $1 \le i \le n$ e $x \ge -\breve\pi_i$. Um indel intergênico $\delta^{(i)}_{(x)}$ remove $x$ nucleotídeos da região intergênica $\breve\pi_i$ caso $x$ seja negativo. Caso contrário, um indel intergênico $\delta^{(i)}_{(x)}$ insere $x$ nucleotídeos na região intergênica $\breve\pi_i$.
\end{definition}

Note que o evento de rearranjo indel intergênico é uma forma compacta para definir os eventos de inserção e deleção utilizando a mesma notação. O Exemplo~\ref{example:ODSBTCQQ} mostra um indel intergênico $\delta^{(5)}_{(9)}$ sendo aplicado na representação intergênica rígida com sinais $\mathcal{R} = (\pi,\breve\pi) = \allowbreak(({+0}~{-3}~{+2}~{-4}~{+1}~{+5}~{+6}),\allowbreak(3,5,1,0,2,1))$ de um genoma, enquanto o Exemplo~\ref{example:OHVZBOMX} mostra um indel intergênico $\delta^{(6)}_{(-6)}$ sendo aplicado na representação intergênica rígida sem sinais $\mathcal{R} = (\pi,\breve\pi) = \allowbreak(({0}~{4}~{5}~{3}~{2}~{1}~{6}),\allowbreak(3,3,2,1,0,7))$ de um genoma. As regiões intergênicas marcadas com sobrescrito sofrem alteração no seu tamanho causada pelo evento.

\begin{example}\label{example:ODSBTCQQ}
  \hfill \break
  % \centering
  \begin{tabular}{lll}
    $(\pi,\breve\pi)$ & $=$ & $(({+0}~{-3}~{+2}~{-4}~{+1}~{+5}~{+6}),(3,5,1,0,\overline{2},1))$ \\
    $(\pi,\breve\pi) \cdot \delta^{(5)}_{(9)}$ & $=$ & $(({+0}~{-3}~{-1}~{+4}~{-2}~{+5}~{+6}),(3,5,1,0,\overline{11},1))$ \\
  \end{tabular}
\end{example}

\begin{example}\label{example:OHVZBOMX}
  \hfill \break
  % \centering
  \begin{tabular}{lll}
    $(\pi,\breve\pi)$ & $=$ & $(({0}~{4}~{5}~{3}~{2}~{1}~{6}),(3,3,2,1,0,\overline{7}))$ \\
    $(\pi,\breve\pi) \cdot \delta^{(6)}_{(-6)}$ & $=$ & $(({0}~{1}~{2}~{3}~{5}~{4}~{6}),(3,3,2,1,0,\overline{1}))$ \\
  \end{tabular}
\end{example}

% ------------------------------------------------------------------ %
\section{Caracterização das Instâncias}
% ------------------------------------------------------------------ %

Os problemas investigados nesta tese têm como principal objetivo transformar uma representação de um genoma de origem $\mathcal{R}_{o}$ em uma representação de um genoma alvo $\mathcal{R}_{a}$ utilizando eventos de rearranjo de genoma para essa finalidade. Um \emph{modelo de rearranjo} $\mathcal{M}$ é um conjunto de eventos de rearranjo que podem ser utilizados para transformar um genoma em outro. Os problemas de rearranjo de genomas diferenciam-se pela representação do genoma de origem e alvo utilizada e pelo modelo de rearranjo adotado. A seguir descrevemos os tipos de instâncias que os problemas investigados posteriormente recebem como entrada. 

\begin{itemize}
\item Uma \emph{instância clássica} é caracterizada por um par de representações clássicas de genomas $(\pi,\iota)$ que compartilham o mesmo conjunto de genes, sendo que ambas as representações podem ser com ou sem sinais. Por padrão, em uma instância clássica utilizaremos $\pi$ e $\iota$ como as representações dos genomas de origem e alvo, respectivamente. O objetivo principal dos problemas que utilizam esse tipo de instância consiste em transformar $\pi$ em $\iota$.
\item Uma \emph{instância intergênica rígida} é caracterizada por um par de representações intergênicas rígidas de genomas $((\pi,\breve\pi),(\iota,\breve\iota))$ que compartilham o mesmo conjunto de genes, sendo que ambas as representações podem ser com ou sem sinais. Por padrão, em uma instância intergênica rígida utilizaremos $(\pi,\breve\pi)$ e $(\iota,\breve\iota)$ como as representações dos genomas de origem e alvo, respectivamente. O objetivo principal dos problemas que utilizam esse tipo de instância consiste em transformar $(\pi,\breve\pi)$ em $(\iota,\breve\iota)$.
\item Uma \emph{instância intergênica flexível} é caracterizada por um par de representações de genomas $((\pi,\breve\pi),(\iota,\breve\iota^{\min},\breve\iota^{\max}))$ que compartilham o mesmo conjunto de genes, sendo a primeira representação intergênica rígida e a segunda intergênica flexível. Ambas as representações podem ser com ou sem sinais. Por padrão, em uma instância intergênica flexível utilizaremos $(\pi,\breve\pi)$ e $(\iota,\breve\iota^{\min},\breve\iota^{\max})$ como as representações dos genomas de origem e alvo, respectivamente. O objetivo principal dos problemas que utilizam esse tipo de instância consiste em transformar $(\pi,\breve\pi)$ em $(\iota,\breve\iota^{\prime})$, tal que $\forall i \in \{1,2,\dots,{n+1}\}: \breve\iota^{\min}_i \le \breve\iota^{\prime}_i \le \breve\iota^{\max}_i$.
\end{itemize}

Pelo fato dos genes serem representados por uma permutação e os genomas origem e alvo compartilharem o mesmo conjunto de genes, podemos determinar uma permutação padrão $\iota$ para os genes do genoma alvo e mapear a permutação do genoma de origem $\pi$ de acordo com os valores utilizados em $\iota$. A permutação padrão para os genes do genoma alvo é $\iota=({+1}~{+2}~\dots~{+n})$ para uma representação com sinais e $\iota=(1~2~\dots~n)$ para uma representação sem sinais. O Exemplo~\ref{example:OZBJAPOZ} mostra uma instância clássica com sinais.

\begin{example}\label{example:OZBJAPOZ}
  \scriptsize
  \hfill \break
  \begin{tikzpicture}
    \node[fill = white!10, align = left, text width = 25mm, minimum width = 25mm] at (0.0, 3) {$\pi = $};
    \draw (0, 3) pic{right gene = {${+0}$, red!50}};
    \draw (1, 3) pic{right gene = {${+2}$, blue!50}};
    \draw (2, 3) pic{left gene = {${-1}$, orange!50}};
    \draw (3, 3) pic{right gene = {${+4}$, green!50}};
    \draw (4, 3) pic{right gene = {${+3}$, teal!50}};
    \draw (5, 3) pic{left gene = {${-6}$, violet!50}};
    \draw (6, 3) pic{left gene = {${-5}$, brown!50}};
    \draw (7, 3) pic{right gene = {${+7}$, purple!50}};
    \node[fill = white!10, align = left, text width = 25mm, minimum width = 25mm] at (0.0, 1.5) {$\iota = $};
    \draw (0, 1.5) pic{right gene = {${+0}$, red!50}};
    \draw (1, 1.5) pic{right gene = {${+1}$, orange!50}};
    \draw (2, 1.5) pic{right gene = {${+2}$, blue!50}};
    \draw (3, 1.5) pic{right gene = {${+3}$, teal!50}};
    \draw (4, 1.5) pic{right gene = {${+4}$, green!50}};
    \draw (5, 1.5) pic{right gene = {${+5}$, brown!50}};
    \draw (6, 1.5) pic{right gene = {${+6}$, violet!50}};
    \draw (7, 1.5) pic{right gene = {${+7}$, purple!50}};
  \end{tikzpicture}
\end{example}

\begin{definition}
  Dada uma instância intergênica rígida $\mathcal{I} = ((\pi,\breve\pi),(\iota,\breve\iota))$, $\mathcal{I}$ é chamada de \emph{balanceada} se a seguinte igualdade é satisfeita: 
  $$\sum_{\breve\pi_i~\in~\breve\pi}\breve\pi_i = \sum_{\breve\iota_i~\in~\breve\iota}\breve\iota_i.$$
  Caso contrário, $\mathcal{I}$ é chamada de \emph{desbalanceada}.
\end{definition}

O Exemplo~\ref{example:PHXDSEMJ} mostra uma instância intergênica rígida balanceada sem sinais.

\begin{example}\label{example:PHXDSEMJ}
  \scriptsize
  \hfill \break
  \begin{tikzpicture}
    \node[fill = white!10, align = left, text width = 25mm, minimum width = 25mm] at (-0.5, 3) {$(\pi,\breve\pi) = $};
    \draw (1, 3) pic{ir = {$3$, black!10}};
    \draw (3, 3) pic{ir = {$2$, black!10}};
    \draw (5, 3) pic{ir = {$2$, black!10}};
    \draw (7, 3) pic{ir = {$0$, black!10}};
    \draw (9, 3) pic{ir = {$4$, black!10}};
    \draw (11, 3) pic{ir = {$1$, black!10}};
    \draw (0, 3) pic{gene = {$0$, red!50}};
    \draw (2, 3) pic{gene = {$2$, blue!50}};
    \draw (4, 3) pic{gene = {$1$, orange!50}};
    \draw (6, 3) pic{gene = {$3$, teal!50}};
    \draw (8, 3) pic{gene = {$5$, brown!50}};
    \draw (10, 3) pic{gene = {$4$, green!50}};
    \draw (12, 3) pic{gene = {$6$, violet!50}};
    \node[fill = white!10, align = left, text width = 25mm, minimum width = 25mm] at (-0.5, 1.5) {$(\iota,\breve\iota) = $};
    \draw (1, 1.5) pic{ir = {$2$, black!10}};
    \draw (3, 1.5) pic{ir = {$2$, black!10}};
    \draw (5, 1.5) pic{ir = {$1$, black!10}};
    \draw (7, 1.5) pic{ir = {$1$, black!10}};
    \draw (9, 1.5) pic{ir = {$4$, black!10}};
    \draw (11, 1.5) pic{ir = {$2$, black!10}};
    \draw (0, 1.5) pic{gene = {$0$, red!50}};
    \draw (2, 1.5) pic{gene = {$1$, orange!50}};
    \draw (4, 1.5) pic{gene = {$2$, blue!50}};
    \draw (6, 1.5) pic{gene = {$3$, teal!50}};
    \draw (8, 1.5) pic{gene = {$4$, green!50}};
    \draw (10, 1.5) pic{gene = {$5$, brown!50}};
    \draw (12, 1.5) pic{gene = {$6$, violet!50}};
  \end{tikzpicture}
\end{example}

\begin{definition}
  Dada uma instância intergênica flexível $\mathcal{I} = ((\pi,\breve\pi),(\iota,\breve\iota^{\min},\breve\iota^{\max}))$, $\mathcal{I}$ é chamada de \emph{balanceada} se a seguinte condição é satisfeita: 
  $$\sum_{\breve\iota^{\min}_i~\in~\breve\iota^{\min}} \breve\iota^{\min}_i \le \sum_{\breve\pi_i~\in~\breve\pi} \breve\pi_i \le \sum_{\breve\iota^{\max}_i~\in~\breve\iota^{\max}}{\breve\iota^{\max}_i}.$$
  Caso contrário, $\mathcal{I}$ é chamada de \emph{desbalanceada}.
\end{definition}

O Exemplo~\ref{example:PJKASBXB} mostra uma instância intergênica flexível balanceada com sinais.

\input{examples/PJKASBXB}

Note que instâncias intergênicas rígidas e flexíveis balanceadas possuem, no genoma de origem, um total de nucleotídeos em que é possível atender todas as restrições referentes aos tamanhos permitidos para cada região intergênica no genoma alvo. Por outro lado, em instâncias intergênicas rígidas e flexíveis desbalanceadas, é necessário inserir ou remover nucleotídeos nas regiões intergênicas do genoma de origem para ser possível transformá-lo no genoma alvo.

% ------------------------------------------------------------------ %
\section{Breakpoints}
% ------------------------------------------------------------------ %

Nesta seção, apresentamos os conceitos de breakpoints em instâncias clássicas e intergênicas rígidas. Esses conceitos são importantes para obtenção de limitantes inferiores e para o desenvolvimento de algoritmos.

% ------------------------------------------------------------------ %
\subsection{Breakpoint Clássico} \label{subsection:POEFMSHU}
% ------------------------------------------------------------------ %

Nesta seção, apresentamos o conceito de breakpoint para instâncias clássicas.

\begin{definition}
  Dada uma instância clássica $\mathcal{I} = (\pi,\iota)$, um par de elementos $(\pi_{i}, \pi_{i+1})$, de forma que $0 \le i \le n$, é um \emph{breakpoint clássico tipo um} se $|\pi_{i+1} - \pi_{i}| \ne 1$.
\end{definition}

\begin{definition}
  Dada uma instância clássica $\mathcal{I} = (\pi,\iota)$, um par de elementos $(\pi_{i}, \pi_{i+1})$, de forma que $0 \le i \le n$, é um \emph{breakpoint clássico tipo dois} se $\pi_{i+1} - \pi_{i} \ne 1$.
\end{definition}

Dada uma instância clássica $\mathcal{I} = (\pi,\iota)$, o número total de breakpoints clássicos tipo um é denotado por $b_{1}(\mathcal{I})$. A variação no número de breakpoints clássicos tipo um após a aplicação de uma sequência de eventos de rearranjo $S$ em $\pi$ é denotada por  $\Delta b_1(\mathcal{I},S) = b_1(\mathcal{I}^{\prime}) - b_1(\mathcal{I})$, onde $\mathcal{I}^{\prime} = (\pi^{\prime},\iota)$ e $\pi^{\prime} = \pi \cdot S$. O número total de breakpoints clássicos tipo dois é denotado por $b_{2}(\mathcal{I})$. A variação no número de breakpoints clássicos tipo dois após a aplicação de uma sequência de eventos de rearranjo $S$ em $\pi$ é denotada por $\Delta b_2(\mathcal{I},S) = b_2(\mathcal{I}^{\prime}) - b_2(\mathcal{I})$, onde $\mathcal{I}^{\prime} = (\pi^{\prime},\iota)$ e $\pi^{\prime} = \pi \cdot S$.

\begin{remark}\label{remark:AEYJTIDG}
  A única instância clássica $\mathcal{I}$ com $b_1(\mathcal{I}) = 0$ ou $b_2(\mathcal{I}) = 0$ é $\mathcal{I} = (\iota,\iota)$.
\end{remark}

\begin{definition}
  Dada uma instância clássica $\mathcal{I} = (\pi,\iota)$, \emph{strips} são sequências maximais de elementos de $\pi$ sem breakpoints clássicos.
\end{definition}

Uma strip obtida de uma instância clássica sem sinais $\mathcal{I} = (\pi,\iota)$ com apenas um elemento $\pi_i$ é chamada de \emph{singleton} e é definida como crescente caso  $i \in \{0,n\}$. Caso contrário, é definida como decrescente. Strips com mais de um elemento são chamadas de crescentes caso os elementos formem uma sequência crescente. Caso contrário, são chamadas de decrescentes. Uma strip obtida de uma instância clássica com sinais $\mathcal{I} = (\pi,\iota)$ é definida como positiva caso todos os elementos da strips tenham sinal positivo. Caso contrário, a strip é definida como negativa.

O Exemplo~\ref{example:PMRHWAPA} mostra uma instância clássica sem sinais $\mathcal{I} = ((0~1~2~5~4~3~6),\allowbreak(0~1~2~3~4~5~6))$. Note que a instância possui dois breakpoints clássicos tipo um ($b_{1}(\mathcal{I}) = 2$), sendo eles $(\pi_2,\pi_3)$ e $(\pi_5,\pi_6)$. Além disso, obtemos as seguintes strips da instância $\mathcal{I}$: $(0~1~2)$, $(5~4~3)$ e $(6)$, sendo que $(0~1~2)$ e $(6)$ são strips crescentes, enquanto $(5~4~3)$ é uma strip decrescente.

\input{examples/PMRHWAPA}

O Exemplo~\ref{example:POZIBUXA} mostra uma instância clássica com sinais $\mathcal{I} = \allowbreak(({+0}~{+1}~{+2}~{+5}~{-4}~{-3}\allowbreak~{+6}),\allowbreak({+0}~{+1}~{+2}~{+3}~{+4}~{+5}~{+6}))$. Note que a instância possui três breakpoints clássicos tipo dois ($b_{2}(\mathcal{I}) = 3$), sendo eles $(\pi_2,\pi_3)$, $(\pi_3,\pi_4)$ e $(\pi_5,\pi_6)$. As strips obtidas dessa instância com esses breakpoints clássicos tipo dois são: $({+0}~{+1}~{+2})$, $({+5})$, $({-4}~{-3})$ e $({+6})$. Sendo que $({+0}~{+1}~{+2})$, $({+5})$ e $({+6})$ são positivas enquanto a strip $({-4}~{-3})$ é negativa.

\begin{example}\label{example:POZIBUXA}
  \scriptsize
  \hfill
  \begin{\position}
    \begin{tikzpicture}
      \node[fill = white!10, align = left, text width = 25mm, minimum width = 25mm] at (0.0, 3) {$\pi = $};
      \path[draw = black] (2.1, 3.9) -- (2.1, 4.1) -- (2.1, 4.0) -- (2.9, 4.0) -- (2.9, 4.1) -- (2.9, 3.9);
      \path[draw = black] (3.1, 3.9) -- (3.1, 4.1) -- (3.1, 4.0) -- (3.9, 4.0) -- (3.9, 4.1) -- (3.9, 3.9);
      \path[draw = black] (5.1, 3.9) -- (5.1, 4.1) -- (5.1, 4.0) -- (5.9, 4.0) -- (5.9, 4.1) -- (5.9, 3.9);
      \node[minimum size = 10mm] at (0, 3.7) {$\pi_0$};
      \node[minimum size = 10mm] at (1, 3.7) {$\pi_1$};
      \node[minimum size = 10mm] at (2, 3.7) {$\pi_2$};
      \node[minimum size = 10mm] at (3, 3.7) {$\pi_3$};
      \node[minimum size = 10mm] at (4, 3.7) {$\pi_4$};
      \node[minimum size = 10mm] at (5, 3.7) {$\pi_5$};
      \node[minimum size = 10mm] at (6, 3.7) {$\pi_6$};
      \draw (0, 3) pic{right gene = {${+0}$, red!50}};
      \draw (1, 3) pic{right gene = {${+1}$, orange!50}};
      \draw (2, 3) pic{right gene = {${+2}$, blue!50}};
      \draw (3, 3) pic{right gene = {${+5}$, brown!50}};
      \draw (4, 3) pic{left gene = {${-4}$, green!50}};
      \draw (5, 3) pic{left gene = {${-3}$, teal!50}};
      \draw (6, 3) pic{right gene = {${+6}$, violet!50}};
      \node[fill = white!10, align = left, text width = 25mm, minimum width = 25mm] at (0.0, 1.5) {$\iota = $};
      \node[minimum size = 10mm] at (0, 2.2) {$\iota_0$};
      \node[minimum size = 10mm] at (1, 2.2) {$\iota_1$};
      \node[minimum size = 10mm] at (2, 2.2) {$\iota_2$};
      \node[minimum size = 10mm] at (3, 2.2) {$\iota_3$};
      \node[minimum size = 10mm] at (4, 2.2) {$\iota_4$};
      \node[minimum size = 10mm] at (5, 2.2) {$\iota_5$};
      \node[minimum size = 10mm] at (6, 2.2) {$\iota_6$};
      \draw (0, 1.5) pic{right gene = {${+0}$, red!50}};
      \draw (1, 1.5) pic{right gene = {${+1}$, orange!50}};
      \draw (2, 1.5) pic{right gene = {${+2}$, blue!50}};
      \draw (3, 1.5) pic{right gene = {${+3}$, teal!50}};
      \draw (4, 1.5) pic{right gene = {${+4}$, green!50}};
      \draw (5, 1.5) pic{right gene = {${+5}$, brown!50}};
      \draw (6, 1.5) pic{right gene = {${+6}$, violet!50}};
    \end{tikzpicture}
  \end{\position}
\end{example}

% ------------------------------------------------------------------ %
\subsection{Breakpoint Intergênico}\label{subsection:JHXBSDPQ}
% ------------------------------------------------------------------ %

Nesta seção, apresentamos o conceito de breakpoint para instâncias intergênicas.

\begin{definition}
  Dada uma instância intergênica rígida $\mathcal{I} = ((\pi,\breve\pi),(\iota,\breve\iota))$, um par de elementos $(\pi_{i}, \pi_{i+1})$, de forma que $0 \le i \le n$, é um \emph{breakpoint intergênico tipo um} se um dos seguintes casos ocorrer:
  \begin{itemize}
    \item $|\pi_{i+1} - \pi_{i}| \ne 1$
    \item $|\pi_{i+1} - \pi_{i}| = 1$ e $\breve\pi_{i+1} \ne \breve\iota_{x}$, tal que $x = \max(\pi_{i}, \pi_{i+1})$.
  \end{itemize}
\end{definition}

\begin{definition}
  Dada uma instância intergênica rígida $\mathcal{I} = ((\pi,\breve\pi),(\iota,\breve\iota))$, um par de elementos $(\pi_{a}, \pi_{b})$ é uma \emph{adjacência intergênica} se $|a-b|=1$ e o par $(\pi_{\min(a,b)}, \pi_{\max(a,b)})$ não é um breakpoint intergênico tipo um.
\end{definition}

Note que um breakpoint intergênico tipo um indica um ponto no genoma de origem que deve ser afetado por algum rearranjo de genoma com o objetivo de transformá-lo no genoma alvo. Por outro lado, uma adjacência intergênica mostra um ponto no genoma de origem em que o par de genes considerados também são consecutivos no genoma alvo. Além disso, a região intergênica entre os genes tem o mesmo tamanho no genoma de origem e alvo.

\begin{definition}
  Dada uma instância intergênica rígida $\mathcal{I} = ((\pi,\breve\pi),(\iota,\breve\iota))$, um breakpoint intergênico tipo um $(\pi_{i}, \pi_{i+1})$, tal que $|\pi_{i+1} - \pi_{i}| = 1$, é chamado de \emph{sobrecarregado} se $\breve\pi_{i+1} > \breve\iota_{x}$, com $x = \max(\pi_{i}, \pi_{i+1})$. Caso contrário, o breakpoint intergênico tipo um $(\pi_{i}, \pi_{i+1})$ é chamado de \emph{subcarregado}.
\end{definition}

Observe que um breakpoint intergênico sobrecarregado é formado por um par de genes que são consecutivos nos genomas de origem e alvo. Contudo, o tamanho da região intergênica entre o par de genes do genoma de origem é maior do que entre o mesmo par de genes no genoma alvo. Já um breakpoint intergênico subcarregado é justamente o cenário oposto: o par de genes são consecutivos nos genomas de origem e alvo, mas a região intergênica entre o par de genes do genoma de origem é menor do que entre o mesmo par de genes no genoma alvo.

\begin{definition}
  Um breakpoint intergênico tipo um $(\pi_{i}, \pi_{i+1})$ é chamado de \emph{forte} se $(\pi_{i}, \pi_{i+1})$ é um breakpoint intergênico sobrecarregado ou subcarregado. Caso contrário, o breakpoint intergênico tipo um $(\pi_{i}, \pi_{i+1})$ é chamado de \emph{suave}.
\end{definition}

\begin{definition}
  Um breakpoint intergênico forte $(\pi_{i}, \pi_{i+1})$ é chamado de \emph{super forte} se um dos seguintes casos ocorrer:
  \begin{itemize}
    \item $i \in \{0,n\}$
    \item $(\pi_{i-1}, \pi_{i})$ ou $(\pi_{i+1}, \pi_{i+2})$ é um breakpoint intergênico forte ou uma adjacência intergênica.
  \end{itemize}
\end{definition}

Note que um breakpoint intergênico super forte está em uma das extremidades do genoma de origem ou imediatamente antes ou depois existe um breakpoint intergênico forte ou uma adjacência intergênica.

\begin{definition}\label{definition:SUHDBNQI}
  Dada uma instância intergênica rígida $\mathcal{I} = ((\pi,\breve\pi),(\iota,\breve\iota))$, um par de breakpoints intergênicos tipo um $(\pi_{i},\pi_{i+1})$ e $(\pi_{j},\pi_{j+1})$ é chamado de \emph{conectado} se ambas as condições a seguir são satisfeitas:
  \begin{enumerate}
    \item Existe um par de elementos dentre $(\pi_{i},\pi_{i+1})$, $(\pi_{j},\pi_{j+1})$, $(\pi_{i},\pi_{j})$, $(\pi_{i},\pi_{j+1})$, $(\pi_{i+1},\pi_{j})$ e $(\pi_{i+1},\pi_{j+1})$ que são consecutivos em $\iota$ e tal par não forma uma adjacência intergênica.
    \item $\breve\pi_{i+1} + \breve\pi_{j+1} \ge \breve\iota_{k}$, tal que $\breve\iota_{k}$ é o tamanho da região intergênica entre o par de elementos consecutivos (que satisfaz a condição 1) em $\iota$.
  \end{enumerate}
\end{definition}

Um par de breakpoints intergênicos conectados indica a possibilidade de criar uma adjacência intergênica utilizando apenas o material de ambos os breakpoints intergênicos tipo um (genes e nucleotídeos das regiões intergênicas).

\begin{definition}
Dada uma instância intergênica rígida $\mathcal{I} = ((\pi,\breve\pi),(\iota,\breve\iota))$, um par de breakpoints intergênicos conectados $(\pi_{i},\pi_{i+1})$ e $(\pi_{j},\pi_{j+1})$ é chamado de \emph{suavemente conectado} se ambos os breakpoints intergênicos são suaves.
\end{definition}

\begin{definition}
  Dada uma instância intergênica rígida $\mathcal{I} = ((\pi,\breve\pi),(\iota,\breve\iota))$, \emph{strips suaves} são sequências maximais de elementos de $\pi$ sem breakpoints intergênicos suaves.
\end{definition}

Uma strip suave com apenas um elemento $\pi_i$ é chamada de \emph{singleton} e é definida como crescente caso  $i \in \{0,n\}$. Caso contrário, é definida como decrescente. Strips suaves com mais de um elemento são chamadas de crescentes caso os elementos formem uma sequência crescente. Caso contrário, são chamadas de decrescentes.

\begin{definition}
  Dada uma instância intergênica rígida $\mathcal{I} = ((\pi,\breve\pi),(\iota,\breve\iota))$, um par de elementos $(\pi_{i}, \pi_{i+1})$, de forma que $0 \le i \le n$, é um \emph{breakpoint intergênico tipo dois} se um dos seguintes casos ocorrer:
  \begin{itemize}
    \item $\pi_{i+1} - \pi_{i} \ne 1$
    \item $\pi_{i+1} - \pi_{i} = 1$ e $\breve\pi_{i+1} \ne \breve\iota_{x}$, tal que $x = \max(|\pi_{i}|, |\pi_{i+1}|)$.
  \end{itemize}
\end{definition}

Os breakpoints intergênicos tipo um e dois são utilizados dependendo do tipo da instância intergênica rígida (com ou sem sinais) e do modelo de rearranjo que é considerado, mas ambos os conceitos indicam a mesma informação: os pontos que devem ser afetados no genoma de origem para transformá-lo no genoma alvo.

Dada uma instância intergênica rígida $\mathcal{I} = ((\pi,\breve\pi),(\iota,\breve\iota))$, o número total de breakpoints fortes e suaves são denotados por $ib_f(\mathcal{I})$ e $ib_s(\mathcal{I})$, respectivamente. O número total de breakpoints intergênicos tipo um é denotado por $ib_{1}(\mathcal{I}) = ib_f(\mathcal{I}) + ib_s(\mathcal{I})$. A variação no número de breakpoints intergênicos tipo um após a aplicação de uma sequência de eventos de rearranjo $S$ em $(\pi,\breve\pi)$ é denotada por  $\Delta ib_1(\mathcal{I},S) = ib_1(\mathcal{I}^{\prime}) - ib_1(\mathcal{I})$, onde $\mathcal{I}^{\prime} = ((\pi^{\prime}, \breve\pi^{\prime}),(\iota,\breve\iota))$ e $(\pi^{\prime}, \breve\pi^{\prime}) = (\pi, \breve\pi) \cdot S$. O número total de breakpoints intergênicos tipo dois é denotado por $ib_{2}(\mathcal{I})$. A variação no número de breakpoints intergênicos tipo dois após a aplicação de uma sequência de eventos de rearranjo $S$ em $(\pi,\breve\pi)$ é denotada por  $\Delta ib_2(\mathcal{I},S) = ib_2(\mathcal{I}^{\prime}) - ib_2(\mathcal{I})$, onde $\mathcal{I}^{\prime} = ((\pi^{\prime}, \breve\pi^{\prime}),(\iota,\breve\iota))$ e $(\pi^{\prime}, \breve\pi^{\prime}) = (\pi, \breve\pi) \cdot S$.

\begin{remark}\label{remark:UDYJTHAH}
  A única instância intergênica rígida $\mathcal{I}$ com $ib_1(\mathcal{I}) = 0$ ou $ib_2(\mathcal{I}) = 0$ é $\mathcal{I} = ((\iota,\breve\iota),(\iota,\breve\iota))$.
\end{remark}

O Exemplo~\ref{example:RADQRLJI} mostra uma instância intergênica rígida sem sinais $\mathcal{I} = (((0~1~2~5~4~3\allowbreak~6),\allowbreak(5,5,3,1,1,2)),\allowbreak((0~1~2~3~4~5~6),\allowbreak(5,0,6,4,1,1)))$. Note que a instância possui quatro breakpoints intergênicos tipo um ($ib_{1}(\mathcal{I}) = 4$), sendo que $ib_f(\mathcal{I}) = 2$ e $ib_s(\mathcal{I}) = 2$. Os breakpoints intergênicos tipo um $(\pi_1,\pi_2)$ e $(\pi_4,\pi_5)$ são fortes, sendo que $(\pi_1,\pi_2)$ é super forte e sobrecarregado, enquanto $(\pi_4,\pi_5)$ é subcarregado. Os breakpoints intergênicos tipo um $(\pi_2,\pi_3)$ e $(\pi_5,\pi_6)$ são suaves. Entre os pares de breakpoints intergênicos que estão conectados na instância, podemos citar o par de breakpoints intergênicos tipo um $((\pi_1,\pi_2),(\pi_2,\pi_3))$, que está conectado, e o par de breakpoints intergênicos tipo um $((\pi_1,\pi_2),(\pi_4,\pi_5))$, que está suavemente conectado. Além disso, obtemos as seguintes strips suaves da instância $\mathcal{I}$: $(0~1~2)$, $(5~4~3)$ e $(6)$, sendo que $(0~1~2)$ e $(6)$ são strips suaves crescentes, enquanto $(5~4~3)$ é uma strip suave decrescente.

\begin{example}\label{example:RADQRLJI}
  \scriptsize
  \hfill
  \begin{\position}
    \begin{tikzpicture}
      \node[fill = white!10, align = left, text width = 25mm, minimum width = 25mm] at (-0.5, 3) {$(\pi,\breve\pi) = $};
      \path[draw = black] (2.1, 3.9) -- (2.1, 4.1) -- (2.1, 4.0) -- (3.9, 4.0) -- (3.9, 4.1) -- (3.9, 3.9);
      \path[draw = black] (4.1, 3.9) -- (4.1, 4.1) -- (4.1, 4.0) -- (5.9, 4.0) -- (5.9, 4.1) -- (5.9, 3.9);
      \path[draw = black] (8.1, 3.9) -- (8.1, 4.1) -- (8.1, 4.0) -- (9.9, 4.0) -- (9.9, 4.1) -- (9.9, 3.9);
      \path[draw = black] (10.1, 3.9) -- (10.1, 4.1) -- (10.1, 4.0) -- (11.9, 4.0) -- (11.9, 4.1) -- (11.9, 3.9);
      \node[minimum size = 10mm] at (0, 3.7) {$\pi_0$};
      \node[minimum size = 10mm] at (2, 3.7) {$\pi_1$};
      \node[minimum size = 10mm] at (4, 3.7) {$\pi_2$};
      \node[minimum size = 10mm] at (6, 3.7) {$\pi_3$};
      \node[minimum size = 10mm] at (8, 3.7) {$\pi_4$};
      \node[minimum size = 10mm] at (10, 3.7) {$\pi_5$};
      \node[minimum size = 10mm] at (12, 3.7) {$\pi_6$};
      \draw (1, 3) pic{ir = {$5$, black!10}};
      \draw (3, 3) pic{ir = {$5$, black!10}};
      \draw (5, 3) pic{ir = {$3$, black!10}};
      \draw (7, 3) pic{ir = {$1$, black!10}};
      \draw (9, 3) pic{ir = {$1$, black!10}};
      \draw (11, 3) pic{ir = {$2$, black!10}};
      \draw (0, 3) pic{gene = {$0$, red!50}};
      \draw (2, 3) pic{gene = {$1$, orange!50}};
      \draw (4, 3) pic{gene = {$2$, blue!50}};
      \draw (6, 3) pic{gene = {$5$, brown!50}};
      \draw (8, 3) pic{gene = {$4$, green!50}};
      \draw (10, 3) pic{gene = {$3$, teal!50}};
      \draw (12, 3) pic{gene = {$6$, violet!50}};
      \node[fill = white!10, align = left, text width = 25mm, minimum width = 25mm] at (-0.5, 1.5) {$(\iota,\breve\iota) = $};
      \node[minimum size = 10mm] at (0, 2.2) {$\iota_0$};
      \node[minimum size = 10mm] at (2, 2.2) {$\iota_1$};
      \node[minimum size = 10mm] at (4, 2.2) {$\iota_2$};
      \node[minimum size = 10mm] at (6, 2.2) {$\iota_3$};
      \node[minimum size = 10mm] at (8, 2.2) {$\iota_4$};
      \node[minimum size = 10mm] at (10, 2.2) {$\iota_5$};
      \node[minimum size = 10mm] at (12, 2.2) {$\iota_6$};
      \draw (1, 1.5) pic{ir = {$5$, black!10}};
      \draw (3, 1.5) pic{ir = {$0$, black!10}};
      \draw (5, 1.5) pic{ir = {$6$, black!10}};
      \draw (7, 1.5) pic{ir = {$4$, black!10}};
      \draw (9, 1.5) pic{ir = {$1$, black!10}};
      \draw (11, 1.5) pic{ir = {$1$, black!10}};
      \draw (0, 1.5) pic{gene = {$0$, red!50}};
      \draw (2, 1.5) pic{gene = {$1$, orange!50}};
      \draw (4, 1.5) pic{gene = {$2$, blue!50}};
      \draw (6, 1.5) pic{gene = {$3$, teal!50}};
      \draw (8, 1.5) pic{gene = {$4$, green!50}};
      \draw (10, 1.5) pic{gene = {$5$, brown!50}};
      \draw (12, 1.5) pic{gene = {$6$, violet!50}};
    \end{tikzpicture}
  \end{\position}
\end{example}

O Exemplo~\ref{example:RCRLXZXC} mostra uma instância intergênica rígida com sinais $\mathcal{I} = \allowbreak((({+0}~{+1}~{+2}~\allowbreak{+5}~{-4}~{-3}~{+6}),\allowbreak(5,5,3,1,4,2)),\allowbreak(({+0}~{+1}~{+2}~{+3}~{+4}~{+5}~{+6}),\allowbreak(2,5,6,4,1,2)))$. Note que a instância possui quatro breakpoints intergênicos tipo dois ($ib_{2}(\mathcal{I}) = 4$), sendo eles $(\pi_0,\pi_1)$, $(\pi_2,\pi_3)$, $(\pi_3,\pi_4)$ e $(\pi_5,\pi_6)$.

\begin{example}\label{example:RCRLXZXC}
  \scriptsize
  \hfill \break
  \begin{tikzpicture}
    \node[fill = white!10, align = left, text width = 25mm, minimum width = 25mm] at (-0.5, 3) {$(\pi,\breve\pi) = $};
    \path[draw = black] (0.1, 3.9) -- (0.1, 4.1) -- (0.1, 4.0) -- (1.9, 4.0) -- (1.9, 4.1) -- (1.9, 3.9);
    \path[draw = black] (4.1, 3.9) -- (4.1, 4.1) -- (4.1, 4.0) -- (5.9, 4.0) -- (5.9, 4.1) -- (5.9, 3.9);
    \path[draw = black] (6.1, 3.9) -- (6.1, 4.1) -- (6.1, 4.0) -- (7.9, 4.0) -- (7.9, 4.1) -- (7.9, 3.9);
    \path[draw = black] (10.1, 3.9) -- (10.1, 4.1) -- (10.1, 4.0) -- (11.9, 4.0) -- (11.9, 4.1) -- (11.9, 3.9);
    \node[minimum size = 10mm] at (0, 3.7) {$\pi_0$};
    \node[minimum size = 10mm] at (2, 3.7) {$\pi_1$};
    \node[minimum size = 10mm] at (4, 3.7) {$\pi_2$};
    \node[minimum size = 10mm] at (6, 3.7) {$\pi_3$};
    \node[minimum size = 10mm] at (8, 3.7) {$\pi_4$};
    \node[minimum size = 10mm] at (10, 3.7) {$\pi_5$};
    \node[minimum size = 10mm] at (12, 3.7) {$\pi_6$};
    \draw (1, 3) pic{ir = {$5$, black!10}};
    \draw (3, 3) pic{ir = {$5$, black!10}};
    \draw (5, 3) pic{ir = {$3$, black!10}};
    \draw (7, 3) pic{ir = {$1$, black!10}};
    \draw (9, 3) pic{ir = {$4$, black!10}};
    \draw (11, 3) pic{ir = {$2$, black!10}};
    \draw (0, 3) pic{right gene = {${+0}$, red!50}};
    \draw (2, 3) pic{right gene = {${+1}$, orange!50}};
    \draw (4, 3) pic{right gene = {${+2}$, blue!50}};
    \draw (6, 3) pic{right gene = {${+5}$, brown!50}};
    \draw (8, 3) pic{left gene = {${-4}$, green!50}};
    \draw (10, 3) pic{left gene = {${-3}$, teal!50}};
    \draw (12, 3) pic{right gene = {${+6}$, violet!50}};
    \node[fill = white!10, align = left, text width = 25mm, minimum width = 25mm] at (-0.5, 1.5) {$(\iota,\breve\iota) = $};
    \node[minimum size = 10mm] at (0, 2.2) {$\iota_0$};
    \node[minimum size = 10mm] at (2, 2.2) {$\iota_1$};
    \node[minimum size = 10mm] at (4, 2.2) {$\iota_2$};
    \node[minimum size = 10mm] at (6, 2.2) {$\iota_3$};
    \node[minimum size = 10mm] at (8, 2.2) {$\iota_4$};
    \node[minimum size = 10mm] at (10, 2.2) {$\iota_5$};
    \node[minimum size = 10mm] at (12, 2.2) {$\iota_6$};
    \draw (1, 1.5) pic{ir = {$2$, black!10}};
    \draw (3, 1.5) pic{ir = {$5$, black!10}};
    \draw (5, 1.5) pic{ir = {$6$, black!10}};
    \draw (7, 1.5) pic{ir = {$4$, black!10}};
    \draw (9, 1.5) pic{ir = {$1$, black!10}};
    \draw (11, 1.5) pic{ir = {$2$, black!10}};
    \draw (0, 1.5) pic{right gene = {${+0}$, red!50}};
    \draw (2, 1.5) pic{right gene = {${+1}$, orange!50}};
    \draw (4, 1.5) pic{right gene = {${+2}$, blue!50}};
    \draw (6, 1.5) pic{right gene = {${+3}$, teal!50}};
    \draw (8, 1.5) pic{right gene = {${+4}$, green!50}};
    \draw (10, 1.5) pic{right gene = {${+5}$, brown!50}};
    \draw (12, 1.5) pic{right gene = {${+6}$, violet!50}};
  \end{tikzpicture}
\end{example}

% ------------------------------------------------------------------ %
\section{Regiões Intergênicas Flexíveis}\label{section:JISAQWNF}
% ------------------------------------------------------------------ %

Nesta seção, apresentamos alguns conceitos relacionados às regiões intergênicas em instâncias intergênicas flexíveis sem sinais. Esses conceitos são importantes para o desenvolvimento de algoritmos e limitantes inferiores para os problemas investigados nos capítulos seguintes.

\begin{definition}
  Dada uma instância intergênica flexível sem sinais $\mathcal{I} = ((\pi,\breve\pi),\break(\iota,\breve\iota^{\min},\breve\iota^{\max}))$, uma região intergênica $\breve\pi_i$ é chamada de \emph{estável} se $|\pi_{i} - \pi_{i - 1}| = 1$ e $\breve\iota^{\min}_k \le \breve\pi_i \le \breve\iota^{\max}_k$, tal que $k = \max(\pi_{i-1}, \pi_i)$. Caso contrário, a região intergênica $\breve\pi_i$ é chamada de \emph{instável}. 
\end{definition}

Uma região intergênica instável deve necessariamente ser afetada por um evento de rearranjo, seja para unir genes consecutivos no genoma alvo ou para alterar a quantidade de nucleotídeos na região intergênica. Dada uma instância intergênica flexível sem sinais $\mathcal{I} = ((\pi,\breve\pi),(\iota,\breve\iota^{\min},\breve\iota^{\max}))$, os conjuntos de regiões intergênicas estáveis e instáveis em $\mathcal{I}$ são definidos como $\mathcal{S}_e(\mathcal{I})$ e $\mathcal{S}_i(\mathcal{I})$, respectivamente. O número de regiões intergênicas estáveis e instáveis em $\mathcal{I}$ é denotado por $ir_e(\mathcal{I})$ e $ir_i(\mathcal{I})$, respectivamente. A variação no número de regiões intergênicas instáveis após a aplicação de uma sequência de eventos de rearranjo $S$ em $(\pi,\breve\pi)$ é denotada por  $\Delta ir_i(\mathcal{I},S) = ir_i(\mathcal{I}^{\prime}) - ir_i(\mathcal{I})$, onde $\mathcal{I}^{\prime} = ((\pi^{\prime}, \breve\pi^{\prime}),(\iota,\breve\iota^{\min},\breve\iota^{\max}))$ e $(\pi^{\prime}, \breve\pi^{\prime}) = (\pi, \breve\pi) \cdot S$.

Dada uma instância intergênica flexível sem sinais $\mathcal{I} = ((\pi,\breve\pi),(\iota,\breve\iota^{\min},\breve\iota^{\max}))$ e seja $\breve\pi_i$ uma região intergênica estável. Denotamos por $gap_{\min}(\breve\pi_i) = \breve\pi_i - \breve\iota^{\min}_k$ e $gap_{\max}(\breve\pi_i) = \breve\iota^{\max}_k - \breve\pi_i$, tal que $k = \max(\pi_{i-1}, \pi_i)$, a quantidade de nucleotídeos que podem ser, respectivamente, removidos e adicionados mantendo $\breve\pi_i$ estável.

\begin{remark}\label{remark:EUSNDMWS}
Dada uma instância intergênica flexível sem sinais $\mathcal{I} = ((\pi,\breve\pi),\break(\iota,\breve\iota^{\min},\breve\iota^{\max}))$ tal que $ir_i(\mathcal{I}) = 0$, então $\pi = \iota$ e $\forall \breve\pi_i \in \breve\pi: \breve\iota^{\min}_i \le \breve\pi_i \le \breve\iota^{\max}_i$.
\end{remark}

De agora em diante, as definições e conceitos apresentados referem-se às instâncias intergênicas flexíveis balanceadas sem sinais, com a adoção de modelos compostos exclusivamente por eventos de rearranjo conservativos. Note que, dada uma instância intergênica flexível balanceada sem sinais $\mathcal{I} = ((\pi,\breve\pi),(\iota,\breve\iota^{\min},\breve\iota^{\max}))$, todas as regiões intergênicas instáveis precisam ser removidas para transformar $(\pi,\breve\pi)$ em $(\iota,\breve\pi^{\prime})$, tal que $\forall \breve\pi^{\prime}_i \in \breve\pi^{\prime} : \breve\iota^{\min}_i \le \breve\pi^{\prime}_i \le \breve\iota^{\max}_i$. Entretanto, nesse caso em particular, temos que algumas regiões intergênicas estáveis também podem ser afetadas com esse objetivo, dependendo do total de nucleotídeos nas regiões intergênicas instáveis. Regiões intergênicas estáveis devem obrigatoriamente ser afetadas por algum evento de rearranjo conservativo se algum dos seguintes cenários ocorrer:

$$\texttt{(i)}~\sum_{\breve\pi_i \in \mathcal{S}_{i}(\mathcal{I})} \breve\pi_i < \sum_{\breve\iota_{i}^{\min}  \in \breve\iota^{\min}} \breve\iota_{i}^{\min} - \sum_{\breve\pi_i \in \mathcal{S}_{e}(\mathcal{I})} (\breve\pi_i - gap_{\min}(\breve\pi_i))$$
$$\texttt{(ii)}\sum_{\breve\pi_i \in \mathcal{S}_{i}(\mathcal{I})} \breve\pi_i > \sum_{\breve\iota_{i}^{\max}  \in \breve\iota^{\max}} \breve\iota_{i}^{\max} - \sum_{\breve\pi_i \in \mathcal{S}_{e}(\mathcal{I})} (\breve\pi_i + gap_{\max}(\breve\pi_i))$$

No cenário \texttt{(i)}, chamado de \emph{fonte}, a quantidade de nucleotídeos nas regiões intergênicas instáveis não é suficiente para torná-las estáveis. Dessa forma, nucleotídeos das regiões intergênicas estáveis devem ser transferidos para as regiões intergênicas instáveis. No cenário \texttt{(ii)}, chamado de \emph{sorvedouro}, a quantidade de nucleotídeos nas regiões intergênicas instáveis excede o limite total permitido para essas regiões intergênicas. Dessa forma, nucleotídeos das regiões intergênicas instáveis devem ser transferidos para as regiões intergênicas estáveis. Uma instância intergênica flexível balanceada sem sinais que não pertence ao cenário fonte ou sorvedouro, pertence ao cenário de \emph{equilíbrio}. Nesse caso, a quantidade de nucleotídeos nas regiões intergênicas instáveis é suficiente para torná-las estáveis. Com a possível necessidade de afetar algumas regiões intergênicas estáveis, temos a seguinte definição. 

\begin{definition}
  Dada uma instância intergênica flexível balanceada sem sinais $\mathcal{I} = ((\pi,\breve\pi),(\iota,\breve\iota^{\min},\breve\iota^{\max}))$, uma região intergênica estável $\breve\pi_i$ é chamada de \emph{auxiliar} se $\breve\pi_i$ deve receber nucleotídeos de regiões intergênicas instáveis ou transferir nucleotídeos para regiões intergênicas instáveis. Caso contrário, $\breve\pi_i$ é chamada de \emph{definitiva}.
\end{definition}

O número total de regiões intergênicas auxiliares depende do cenário da instância $\mathcal{I}$. No caso do cenário fonte, o conjunto de regiões intergênicas auxiliares $\mathcal{S}_{a}(\mathcal{I})$ é tal que seu tamanho é mínimo e a seguinte restrição é satisfeita:

$$\sum_{\breve\pi_i \in \mathcal{S}_{a}(\mathcal{I})} gap_{\min}(\breve\pi_i) \ge \sum_{\breve\iota_{i}^{\min}  \in \breve\iota^{\min}} \breve\iota_{i}^{\min} - \sum_{\breve\pi_i \in \mathcal{S}_{e}(\mathcal{I})} (\breve\pi_i - gap_{\min}(\breve\pi_i)) - \sum_{\breve\pi_i \in \mathcal{S}_{i}(\mathcal{I})} \breve\pi_i$$

Note que o conjunto $\mathcal{S}_{a}(\mathcal{I})$ com tamanho mínimo pode ser facilmente obtido ordenando as regiões intergênicas estáveis em ordem decrescente pelo valor de $gap_{\min}$. Em seguida, cada região intergênica é rotulada como auxiliar até que a restrição seja satisfeita. No caso do cenário sorvedouro, o conjunto de regiões intergênicas auxiliares $\mathcal{S}_{a}(\mathcal{I})$ é tal que seu tamanho é mínimo e a seguinte restrição é satisfeita:

$$\sum_{\breve\pi_i \in \mathcal{S}_{a}(\mathcal{I})} gap_{\max}(\breve\pi_i) \ge \sum_{\breve\pi_i \in \mathcal{S}_{i}(\mathcal{I})} \breve\pi_i - \sum_{\breve\iota_{i}^{\max}  \in \breve\iota^{\max}} \breve\iota_{i}^{\max} - \sum_{\breve\pi_i \in \mathcal{S}_{e}(\mathcal{I})} (\breve\pi_i + gap_{\max}(\breve\pi_i))$$

Semelhante ao cenário anterior, o conjunto $\mathcal{S}_{a}(\mathcal{I})$ com tamanho mínimo também pode ser facilmente obtido ordenando as regiões intergênicas estáveis em ordem decrescente pelo valor de $gap_{\max}$ e efetuando a rotulação das regiões intergênicas como auxiliares até que a restrição seja satisfeita. Obtendo o conjunto $\mathcal{S}_{a}(\mathcal{I})$, temos que o conjunto das regiões intergênicas definitivas $\mathcal{S}_{d}(\mathcal{I})$ pode ser obtido pela operação $\mathcal{S}_{e}(\mathcal{I}) - \mathcal{S}_{a}(\mathcal{I})$. Note que $\mathcal{S}_{a}(\mathcal{I}) \cup \mathcal{S}_{d}(\mathcal{I}) = \mathcal{S}_{e}$.

Caso o cenário de equilíbrio ocorra, então temos que $\mathcal{S}_{a}(\mathcal{I})=\emptyset$ e $ \mathcal{S}_{d}(\mathcal{I}) = \mathcal{S}_{e}$, ou seja, o total de nucleotídeos nas regiões intergênicas instáveis é suficiente para torná-las estáveis sem ser preciso afetar as regiões intergênicas estáveis. Dada uma instância intergênica flexível balanceada sem sinais $\mathcal{I} = ((\pi,\breve\pi),(\iota,\breve\iota^{\min},\breve\iota^{\max}))$, o número de regiões intergênicas auxiliares em $\mathcal{I}$ é denotado por $ir_a(\mathcal{I})$. A variação no número de regiões intergênicas auxiliares após aplicar uma sequência de eventos de rearranjo $S$ em $(\pi,\breve\pi)$ é denotada por $\Delta ir_a(\mathcal{I},S) = ir_a(\mathcal{I}^{\prime}) - ir_a(\mathcal{I})$, onde $\mathcal{I}^{\prime} = ((\pi^{\prime}, \breve\pi^{\prime}),(\iota,\breve\iota^{\min},\breve\iota^{\max}))$ e $(\pi^{\prime}, \breve\pi^{\prime}) = (\pi, \breve\pi) \cdot S$.

\begin{remark}\label{remark:PGEYZJME}
Dada uma instância intergênica flexível balanceada sem sinais $\mathcal{I} = ((\pi,\breve\pi),(\iota,\breve\iota^{\min},\breve\iota^{\max}))$ tal que $ir_i(\mathcal{I}) + ir_a(\mathcal{I}) = 0$, então $\pi = \iota$ e $\forall \breve\pi_i \in \breve\pi: \breve\iota^{\min}_i \le \breve\pi_i \le \breve\iota^{\max}_i$.
\end{remark}

O Exemplo~\ref{example:RDGJHSWZ} mostra uma instância intergênica flexível balanceada sem sinais $\mathcal{I} = (((0~1~2~5~4~3~6),(5,0,3,1,6,2)),((0~1~2~3~4~5~6),(4,3,3,2,2,1),(6,4,8,7,3,3)))$ que pertence ao cenário fonte. Note que a instância $\mathcal{I}$ possui quatro regiões intergênicas instáveis ($ir_i(\mathcal{I}) = 4$, com $\mathcal{S}_{i}=\{\breve\pi_2,\breve\pi_3,\breve\pi_4,\breve\pi_6\}$) e duas regiões intergênicas estáveis ($\mathcal{S}_{e}=\{\breve\pi_1,\breve\pi_5\}$). No exemplo, temos apenas uma região intergênica auxiliar ($ir_a(\mathcal{I}) = 1$ e $\mathcal{S}_{a}=\{\breve\pi_5\}$). Note que $gap_{\min}(\breve\pi_1) = 1$ e $gap_{\min}(\breve\pi_5) = 4$.

\pagebreak

\begin{example}\label{example:RDGJHSWZ}
  \scriptsize
  \hfill \break
  \begin{tikzpicture}
    \node[fill = white!10, align = left, text width = 25mm, minimum width = 25mm] at (-1.5, 3) {$(\pi,\breve\pi) = $};
    \node[minimum size = 10mm] at (0, 3.7) {$\pi_0$};
    \node[minimum size = 10mm] at (2, 3.7) {$\pi_1$};
    \node[minimum size = 10mm] at (4, 3.7) {$\pi_2$};
    \node[minimum size = 10mm] at (6, 3.7) {$\pi_3$};
    \node[minimum size = 10mm] at (8, 3.7) {$\pi_4$};
    \node[minimum size = 10mm] at (10, 3.7) {$\pi_5$};
    \node[minimum size = 10mm] at (12, 3.7) {$\pi_6$};
    \node[minimum size = 10mm] at (1, 3.7) {$\breve\pi_1$};
    \node[minimum size = 10mm] at (3, 3.7) {$\breve\pi_2$};
    \node[minimum size = 10mm] at (5, 3.7) {$\breve\pi_3$};
    \node[minimum size = 10mm] at (7, 3.7) {$\breve\pi_4$};
    \node[minimum size = 10mm] at (9, 3.7) {$\breve\pi_5$};
    \node[minimum size = 10mm] at (11, 3.7) {$\breve\pi_6$};
    \draw (1, 3) pic{ir = {$5$, black!10}};
    \draw (3, 3) pic{ir = {$0$, black!10}};
    \draw (5, 3) pic{ir = {$3$, black!10}};
    \draw (7, 3) pic{ir = {$1$, black!10}};
    \draw (9, 3) pic{ir = {$6$, black!10}};
    \draw (11, 3) pic{ir = {$2$, black!10}};
    \draw (0, 3) pic{gene = {$0$, red!50}};
    \draw (2, 3) pic{gene = {$1$, orange!50}};
    \draw (4, 3) pic{gene = {$2$, blue!50}};
    \draw (6, 3) pic{gene = {$5$, brown!50}};
    \draw (8, 3) pic{gene = {$4$, green!50}};
    \draw (10, 3) pic{gene = {$3$, teal!50}};
    \draw (12, 3) pic{gene = {$6$, violet!50}};
    \node[fill = white!10, align = left, text width = 25mm, minimum width = 25mm] at (-1.5, 1.5) {$(\iota,\breve\iota^{\min},\breve\iota^{\max}) = $};
    \node[minimum size = 10mm] at (0, 2.2) {$\iota_0$};
    \node[minimum size = 10mm] at (2, 2.2) {$\iota_1$};
    \node[minimum size = 10mm] at (4, 2.2) {$\iota_2$};
    \node[minimum size = 10mm] at (6, 2.2) {$\iota_3$};
    \node[minimum size = 10mm] at (8, 2.2) {$\iota_4$};
    \node[minimum size = 10mm] at (10, 2.2) {$\iota_5$};
    \node[minimum size = 10mm] at (12, 2.2) {$\iota_6$};
    \draw (1, 1.5) pic{flex ir = {$4$, $6$, black!10}};
    \draw (3, 1.5) pic{flex ir = {$3$, $4$, black!10}};
    \draw (5, 1.5) pic{flex ir = {$3$, $8$, black!10}};
    \draw (7, 1.5) pic{flex ir = {$2$, $7$, black!10}};
    \draw (9, 1.5) pic{flex ir = {$2$, $3$, black!10}};
    \draw (11, 1.5) pic{flex ir = {$1$, $3$, black!10}};
    \draw (0, 1.5) pic{gene = {$0$, red!50}};
    \draw (2, 1.5) pic{gene = {$1$, orange!50}};
    \draw (4, 1.5) pic{gene = {$2$, blue!50}};
    \draw (6, 1.5) pic{gene = {$3$, teal!50}};
    \draw (8, 1.5) pic{gene = {$4$, green!50}};
    \draw (10, 1.5) pic{gene = {$5$, brown!50}};
    \draw (12, 1.5) pic{gene = {$6$, violet!50}};
  \end{tikzpicture}
\end{example}

O Exemplo~\ref{example:SCUCIMVA} mostra uma instância intergênica flexível balanceada sem sinais $\mathcal{I} = (((0~1~2~5~4~3~6),(4,3,4,1,2,8)),((0~1~2~3~4~5~6),(4,2,1,2,0,1),(6,4,3,7,1,3)))$ que pertence ao cenário sorvedouro. Note que a instância $\mathcal{I}$ possui duas regiões intergênicas instáveis ($ir_i(\mathcal{I}) = 2$ e $\mathcal{S}_{i}=\{\breve\pi_3,\breve\pi_6\}$) e quatro regiões intergênicas estáveis ($\mathcal{S}_{e}=\{\breve\pi_1,\breve\pi_2,\breve\pi_4,\breve\pi_5\}$). No exemplo, temos duas regiões intergênica auxiliares ($ir_a(\mathcal{I}) = 2$ e $\mathcal{S}_{a}=\{\breve\pi_1,\breve\pi_5\}$). Note que $gap_{\max}(\breve\pi_1) = 2$, $gap_{\max}(\breve\pi_2) = 1$, $gap_{\max}(\breve\pi_4) = 0$ e $gap_{\max}(\breve\pi_5) = 5$.

\input{examples/SCUCIMVA}

O Exemplo~\ref{example:TUDHODFC} mostra uma instância intergênica flexível balanceada sem sinais $\mathcal{I} = (((0~1~2~5~4~3~6),(5,5,3,1,2,3)),((0~1~2~3~4~5~6),(2,4,1,5,1,4),(4,6,3,7,3,6)))$ que não pertence ao cenário fonte ou sorvedouro. Note que por esse motivo a instância não possui regiões intergênicas auxiliares, ou seja, $ir_a(\mathcal{I}) = 0$ e $\mathcal{S}_{a}=\emptyset$. A instância $\mathcal{I}$ possui cinco regiões intergênicas instáveis ($ir_i(\mathcal{I}) = 5$ e $\mathcal{S}_{i}=\{\breve\pi_1,\breve\pi_3,\breve\pi_4,\breve\pi_5,\breve\pi_6\}$) e uma região intergênica estável ($\mathcal{S}_{e}=\{\breve\pi_2\}$).

\input{examples/TUDHODFC}

% ------------------------------------------------------------------ %
\section{Grafo de Ciclos}
% ------------------------------------------------------------------ %

Grafos são estruturas amplamente utilizadas em problemas de rearranjo de genomas para obtenção de limitantes inferiores e algoritmos. Nesta seção, apresentamos os grafos de ciclos clássico, ponderado e ponderado flexível.

% ------------------------------------------------------------------ %
\subsection{Grafo de Ciclos Clássico}\label{subsection:LSNDQWPI}
% ------------------------------------------------------------------ %

O grafo de ciclos clássico, também chamado de grafo de breakpoints, tem seu uso bastante difundido em problemas de rearranjo de genomas que utilizam instâncias clássicas. Esse grafo evidencia em uma mesma estrutura as adjacências presentes no genoma de origem e as adjacências desejadas no genoma alvo. A seguir definimos formalmente o grafo de ciclos clássico.

Dada uma instância clássica $\mathcal{I} = (\pi,\iota)$, definimos o grafo de ciclos clássico por $G(\mathcal{I}) = (V, E, \ell)$, tal que $V$, $E$ e $\ell$ representam o conjunto de vértices, o conjunto de arestas e uma função de rotulação de arestas, respectivamente. O conjunto de vértices $V$ é dado por $\{{+\pi_0}, {-\pi_1}, {+\pi_1}, {-\pi_2}, {+\pi_2}, \dots, {-\pi_n}, {+\pi_n}, {-\pi_{n+1}}\}$. Note que para cada elemento $\pi_i$, com $1 \le i \le n$, adicionamos em $V$ os vértices ${-\pi_i}$ e ${+\pi_i}$. Por fim, adicionamos em $V$ os vértices ${+\pi_0}$ e ${-\pi_{n+1}}$. O conjunto de arestas $E = E_p \cup E_c$ é dividido nos conjuntos de arestas pretas ($E_p$) e arestas cinzas ($E_c$), onde $E_p = \{(-\pi_i, +\pi_{i-1}) \,|\, 1 \leq i \leq n+1\}$ e $E_c = \{(+(i-1), -i) \,|\, 1 \leq i \leq n + 1\}$. Perceba que as arestas pretas representam os elementos adjacentes na permutação $\pi$, enquanto as arestas cinzas representam os elementos  adjacentes em $\iota$.

Existem diferentes formas de desenhar o grafo de ciclos clássico. Entretanto, utilizaremos a forma que chamamos de \emph{padrão}. Para essa forma de desenhar o grafo, os vértices são posicionados horizontalmente da esquerda para direita, seguindo a ordem ${+\pi_0}, {-\pi_1}, {+\pi_1}, {-\pi_2}, {+\pi_2}, \dots, {-\pi_n}, {+\pi_n}, {-\pi_{n+1}}$. As arestas pretas são desenhadas formando uma linha horizontal contínua, enquanto as arestas cinzas formam arcos com linhas tracejadas sobre os vértices. O Exemplo~\ref{example:UKGIKUAH} mostra o grafo de ciclos clássico construído a partir da instância clássica $\mathcal{I} = (({+0}~{+4}~{+3}~{-1}~{+2}~{+5}~{+6}),({+0}~{+1}~{+2}~{+3}~{+4}~{+5}~{+6}))$.

\begin{example}\label{example:UKGIKUAH}
  \scriptsize
  \hfill \break
  \begin{tikzpicture}[scale=0.7]
    \begin{scope}[every node/.style={inner sep=1.5pt, minimum size = 0pt}]
      \node[circle, draw] (p0) at (0,0) {$+0$};
      \node[circle, draw] (m4) at (1.5,0) {$-4$};
      \node[circle, draw] (p4) at (3,0) {$+4$};
      \node[circle, draw] (m3) at (4.5,0) {$-3$};
      \node[circle, draw] (p3) at (6,0) {$+3$};
      \node[circle, draw] (p1) at (7.5,0) {$+1$};
      \node[circle, draw] (m1) at (9,0) {$-1$};
      \node[circle, draw] (m2) at (10.5,0) {$-2$};
      \node[circle, draw] (p2) at (12,0) {$+2$};
      \node[circle, draw] (m5) at (13.5,0) {$-5$};
      \node[circle, draw] (p5) at (15,0) {$+5$};
      \node[circle, draw] (m6) at (16.5,0) {$-6$};
    \end{scope}
    \begin{scope}[>={Stealth[black]},
                  every edge/.style={draw=black}]
      \path [-] (p0) edge (m4);
      \node[draw=none, fill=none, align=center, minimum width=1cm, text width=1cm] at (0.75, -1.0) {$\ell = {-1}$};
      \path [-] (p4) edge (m3);
      \node[draw=none, fill=none, align=center, minimum width=1cm, text width=1cm] at (3.75, -1.0) {$\ell = 2$};
      \path [-] (p3) edge (p1);
      \node[draw=none, fill=none, align=center, minimum width=1cm, text width=1cm] at (6.75, -1.0) {$\ell = {-3}$};
      \path [-] (m1) edge (m2);
      \node[draw=none, fill=none, align=center, minimum width=1cm, text width=1cm] at (9.75, -1.0) {$\ell = 4$};
      \path [-] (p2) edge (m5);
      \node[draw=none, fill=none, align=center, minimum width=1cm, text width=1cm] at (12.75, -1.0) {$\ell = 5$};
      \path [-] (p5) edge (m6);
      \node[draw=none, fill=none, align=center, minimum width=1cm, text width=1cm] at (15.75, -1.0) {$\ell = 6$};
    \end{scope}
    \begin{scope}[>={Stealth[black]},
                  every edge/.style={draw=black}]
      \path [-] (p0) edge [bend left=70, dashed] (m1);
      \path [-] (p1) edge [bend left=70, dashed] (m2);
      \path [-] (p2) edge [bend right=60, dashed] (m3);
      \path [-] (p3) edge [bend right=60, dashed] (m4);
      \path [-] (p4) edge [bend left=70, dashed] (m5);
      \path [-] (p5) edge [bend left=70, dashed] (m6);
    \end{scope}
  \end{tikzpicture}
\end{example}

Pelo Exemplo~\ref{example:UKGIKUAH}, podemos perceber que o grafo de ciclos clássico possui $2n+2$ vértices e $2n+2$ arestas ($n+1$ pretas e $n+1$ cinzas), sendo que em cada vértice duas arestas são incidentes, uma preta e uma cinza. Por esse motivo, há uma decomposição única de $G(\mathcal{I})$ em ciclos com arestas de cores alternadas. 

A função de rotulação $\ell : E_p \rightarrow \{-(n+1),-n,\dots,-2,-1,1,2,\dots,n,(n+1)\}$ atribui um rótulo para cada aresta preta no grafo em função da direção em que a aresta é percorrida. Dada uma aresta preta $e_p = (-\pi_i, +\pi_{i-1}) \in E_p$, a função $\ell$ atribui o rótulo $i$ em $e_p$ caso ela seja percorrida de $-\pi_i$ até $+\pi_{i-1}$. Caso contrário, $e_p$ é rotulada com $-i$. Por padrão, cada ciclo de $G(\mathcal{I})$ é representado pela sequência de rótulos de suas arestas pretas na ordem em que elas são percorridas, sendo que a primeira aresta preta de um ciclo é aquela que encontra-se mais a direita no grafo e é percorrida da direita para esquerda, ou seja, de $-\pi_i$ até $+\pi_{i-1}$. Essa representação utilizada para os ciclos faz com que eles sejam representados unicamente. No Exemplo~\ref{example:UKGIKUAH}, $G(\mathcal{I})$ possui três ciclos: $C_1=(4,-1,-3)$, $C_2 = (5,2)$ e $C_3 = (6)$.

O tamanho de um ciclo $C\in G(\mathcal{I})$ é dado pela quantidade de arestas pretas do ciclo. Um ciclo de tamanho um é chamado de \emph{trivial}. Um ciclo com tamanho menor que três é chamado de \emph{curto}. Caso contrário, é chamado de \emph{longo}. 

\begin{definition}
Duas arestas pretas de um ciclo $C\in G(\mathcal{I})$ são chamadas de \emph{divergentes} se elas são percorridas em direções opostas. Caso contrário, são chamadas de \emph{convergentes}.
\end{definition}
\begin{definition}
Um ciclo $C\in G(\mathcal{I})$ é chamado de \emph{divergente} se pelo menos um par de arestas pretas de $C$ são divergentes. Caso contrário, $C$ é chamado de \emph{convergente}.
\end{definition}

Podemos ainda classificar ciclos convergentes como \emph{orientados} ou \emph{não orientados}. 

\begin{definition}
Um ciclo convergente $C = (c_1,c_2,\dots,c_k) \in G(\mathcal{I})$ é classificado como \emph{não orientado} se $c_i > c_{i+1}$, para todo $i$ com $1 \le i < k$. Caso contrário, $C$ é classificado como \emph{orientado}.
\end{definition}

Dois ciclos $C = (c_1, c_2, \ldots, c_k)$ e $D = (d_1, d_2, \ldots, d_k)$, ambos pertencentes ao grafo $G(\mathcal{I})$, são \emph{entrelaçados} se $|c_1| > |d_1| > |c_2|  > |d_2| > \ldots > |c_k| > |d_k|$ ou $|d_1| > |c_1| > |d_2|  > |c_2| > \ldots > |d_k| > |c_k|$. Seja $g_1$ uma aresta cinza adjacente às arestas pretas com rótulos $x_1$ e $y_1$, tal que $|x_1| < |y_1|$ e que $g_2$ seja uma aresta cinza adjacente às arestas pretas com rótulos $x_2$ e $y_2$, tal que $|x_2| < |y_2|$. Dizemos que duas arestas cinzas $g_1$ e $g_2$ cruzam-se caso $|x_1| < |x_2| \le |y_1| < |y_2|$. Dois ciclos $C$ e $D$ cruzam-se caso uma aresta cinza de $C$ cruza-se com uma aresta cinza de $D$. Um \emph{open gate} é uma aresta cinza de um ciclo não trivial $C \in G(\mathcal{I})$ que não se cruza com nenhuma outra aresta cinza de $C$. Um open gate $g_1$ de $C$ é fechado se outra aresta cinza (de um ciclo diferente de $C$) cruza com $g_1$.

\begin{remark}\label{remark:JBJWNCKF}
Todos os open gates em $G(\mathcal{I})$ são fechados~\cite{1996-bafna-pevzner}.
\end{remark}

No Exemplo~\ref{example:UKGIKUAH}, os ciclos $C_1=(4,-1,-3)$, $C_2 = (5,2)$ e $C_3 = (6)$ são, respectivamente, longo divergente, curto convergente orientado e trivial. Note que o ciclo $C_1$ possui o open gate $({+3},{-4})$, enquanto o ciclo $C_2$ possui os seguintes open gates: $({+2},{-3})$ e $({+4},{-5})$.

Dada uma instância clássica $\mathcal{I} = (\pi,\iota)$, denotamos por $c(G(\mathcal{I}))$ o número de ciclos em $G(\mathcal{I})$. Dada uma sequência de eventos de rearranjo $S$, denotamos por $\Delta c(G(\mathcal{I}), S) = c(G(\mathcal{I^{\prime}})) - c(G(\mathcal{I}))$, tal que $\mathcal{I^{\prime}} = (\pi \cdot S,\iota)$, a variação no número de ciclos após a aplicação da sequência $S$ no genoma de origem $\pi$ de $\mathcal{I}$.

\begin{remark}\label{remark:OYRVGHTB}
  A única instância clássica $\mathcal{I}$ com $c(G(\mathcal{I})) = n + 1$ é $\mathcal{I} = (\iota,\iota)$.
\end{remark}

% ------------------------------------------------------------------ %
\subsection{Grafo de Ciclos Ponderado Rígido}\label{subsection:SDWELPAZ}
% ------------------------------------------------------------------ %

O grafo de ciclos ponderado rígido é uma extensão do grafo de ciclos clássico, incorporando na sua estrutura, através de pesos nas arestas, informações referentes ao tamanho das regiões intergênicas do genoma de origem e do genoma alvo. A seguir, definimos formalmente o grafo de ciclos ponderado rígido.

Dada uma instância intergênica rígida $\mathcal{I} = ((\pi,\breve\pi),(\iota,\breve\iota))$, definimos o grafo de ciclos ponderado rígido por $G(\mathcal{I}) = (V, E=E_p \cup E_c, \ell, w_p, w_c)$, tal que $V$, $E$ e $\ell$ representam, respectivamente, o conjunto de vértices, o conjunto de arestas e uma função de rotulação de arestas, enquanto $w_p$ e $w_c$ são funções de peso. Pelo fato do grafo de ciclos ponderado rígido tratar-se de uma extensão do grafo de ciclos clássico, $V$, $E$ e $\ell$ comportam-se exatamente como anteriormente descrito. Além disso, todos os conceitos, definições e representações que foram apresentados no contexto de grafo de ciclos clássico também são válidos e utilizados no grafo de ciclos ponderado rígido.

A função de peso $w_p : E_p \rightarrow \mathbb{N}_0$ associa os tamanhos das regiões intergênicas no genoma de origem com pesos nas arestas pretas do grafo. A função de peso $w_c : E_c \rightarrow \mathbb{N}_0$ funciona de maneira similar, mas associando os tamanhos das regiões intergênicas no genoma alvo com pesos nas arestas cinzas do grafo. Para cada aresta preta $e_i = (-\pi_i, +\pi_{i-1}) \in E_p$, temos que $w_p(e_i) = \breve\pi_i$. Para cada aresta cinza $e^{\prime}_i = (+(i-1), -i) \in E_c$, temos que $w_c(e^{\prime}_i) = \breve\iota_i$. Dado um ciclo $C \in G(\mathcal{I})$, denotamos por $E_p(C)$ e $E_c(C)$, respectivamente, os conjuntos de arestas pretas e cinzas que pertencem ao ciclo $C$. 

\begin{definition}
Um ciclo $C \in G(\mathcal{I})$ é chamado de \emph{balanceado} caso $\sum_{e^{\prime}_i \in E_c(C)} [w_c(e^{\prime}_i)] - \sum_{e_i \in E_p(C)} [w_p(e_i)] = 0$. Caso contrário, o ciclo $C$ é chamado de \emph{desbalanceado}.
\end{definition}

Em outras palavras, num ciclo balanceado a soma dos pesos em suas arestas pretas é a mesma que a soma dos pesos em suas arestas cinzas. 

\begin{definition}
Um ciclo desbalanceado $C \in G(\mathcal{I})$ é chamado de \emph{negativo} quando $\sum_{e^{\prime}_i \in E_c(C)} [w_c(e^{\prime}_i)] - \sum_{e_i \in E_p(C)} [w_p(e_i)] < 0$. Caso contrário, o ciclo $C$ é chamado de \emph{positivo}.
\end{definition}

Note que os eventos de rearranjo afetam apenas as arestas pretas no grafo de ciclos e suas extensões. Considerando um contexto composto por eventos conservativos, um ciclo negativo após ser afetado por evento de rearranjo apresenta um potencial maior de gerar novos ciclos balanceados, uma vez que existe peso suficiente em suas arestas pretas para atender o peso exigido em cada uma de suas arestas cinzas. Dada uma instância intergênica rígida $\mathcal{I} = ((\pi,\breve\pi),(\iota,\breve\iota))$, denotamos por $c(G(\mathcal{I}))$ e $c_b(G(\mathcal{I}))$ o número de ciclos e o número ciclos balanceados em $G(\mathcal{I})$, respectivamente. Dada uma sequência de eventos de rearranjo $S$, denotamos por $\Delta c(G(\mathcal{I}), S) = c(G(\mathcal{I^{\prime}})) - c(G(\mathcal{I}))$ e $\Delta c_b(G(\mathcal{I}), S) = c_b(G(\mathcal{I^{\prime}})) - c_b(G(\mathcal{I}))$, tal que $\mathcal{I^{\prime}} = ((\pi,\breve\pi) \cdot S,(\iota,\breve\iota))$, a variação no número de ciclos e no número ciclos balanceados, respectivamente, após aplicar a sequência $S$ no genoma de origem $(\pi,\breve\pi)$ de $\mathcal{I}$.

O Exemplo~\ref{example:UMWQHOBI} mostra o grafo de ciclos ponderado rígido construído a partir da instância intergênica rígida $\mathcal{I} = ((({+0}~{+4}~{+3}~{-1}~{+2}~{+5}~{+6}),(0,6,2,5,1,3)),(({+0}~{+1}~{+2}\break{+3}~{+4}~{+5}~{+6}),(3,3,4,2,3,2)))$.

\input{examples/UMWQHOBI}

No Exemplo~\ref{example:UMWQHOBI}, os ciclos $C_1=(4,-1,-3)$, $C_2 = (5,2)$ e $C_3 = (6)$ são, respectivamente, longo positivo, curto balanceado e trivial negativo.

\begin{remark}\label{remark:WVLFPRDL}
  A única instância intergênica rígida $\mathcal{I}$ com $c(G(\mathcal{I})) = n + 1$ e $c_b(G(\mathcal{I})) = n + 1$ é $\mathcal{I} = ((\iota,\breve\iota),(\iota,\breve\iota))$.
\end{remark}

% ------------------------------------------------------------------ %
\subsection{Grafo de Ciclos Ponderado Flexível}\label{subsection:HUWEVNDS}
% ------------------------------------------------------------------ %

O grafo de ciclos ponderado flexível também é uma extensão do grafo de ciclos clássico, incorporando na sua estrutura, através de pesos nas arestas, informações referentes ao tamanho das regiões intergênicas do genoma de origem e os tamanhos mínimos e máximos permitidos para cada região intergênica no genoma alvo. A seguir, definimos formalmente o grafo de ciclos ponderado flexível.

Dada uma instância intergênica flexível $\mathcal{I} = ((\pi,\breve\pi),(\iota,\breve\iota^{\min},\breve\iota^{\max}))$, definimos o grafo de ciclos ponderado flexível por $G(\mathcal{I}) = (V, E=E_p \cup E_c, \ell, w_p, w^{\min}_c, w^{\max}_c)$, tal que $V$, $E$ e $\ell$ representam, respectivamente, o conjunto de vértices, o conjunto de arestas e uma função de rotulação de arestas, enquanto $w_p$, $w^{\min}_c$ e $w^{\max}_c$ são funções de peso. Pelo fato do grafo de ciclos ponderado flexível também tratar-se de uma extensão do grafo de ciclos clássico, $V$, $E$ e $\ell$ comportam-se exatamente como já descrito. Além disso, todos os conceitos, definições e representações que foram apresentados no contexto de grafo de ciclos clássico também são válidos e utilizados no grafo de ciclos ponderado flexível.

A função de peso $w_p : E_p \rightarrow \mathbb{N}_0$ associa os tamanhos das regiões intergênicas no genoma de origem com pesos nas arestas pretas do grafo. As funções de peso $w^{\min}_c : E_c \rightarrow \mathbb{N}_0$ e $w^{\max}_c : E_c \rightarrow \mathbb{N}_0$ associam, respectivamente, os tamanhos mínimos e máximos permitidos para as regiões intergênicas no genoma alvo com pesos nas arestas cinzas do grafo. Para cada aresta preta $e_i = (-\pi_i, +\pi_{i-1}) \in E_p$, temos que $w_p(e_i) = \breve\pi_i$. Para cada aresta cinza $e^{\prime}_i = (+(i-1), -i) \in E_c$, temos que $w^{\min}_c(e^{\prime}_i) = \breve\iota^{\min}_i$ e $w^{\max}_c(e^{\prime}_i) = \breve\iota^{\max}_i$. Dado um ciclo $C \in G(\mathcal{I})$, denotamos por $E_p(C)$ e $E_c(C)$, respectivamente, os conjuntos de arestas pretas e cinzas que pertencem ao ciclo $C$. Dado um ciclo $C \in G(\mathcal{I})$, denotamos por $W_p(C)=\sum_{e_i \in E_p(C)} w_p(e_i)$, $W^{\min}_c(C)=\sum_{e^{\prime}_i \in E_c(C)} w^{\min}_c(e^{\prime}_i)$ e $W^{\max}_c(C)=\sum_{e^{\prime}_i \in E_c(C)} w^{\max}_c(e^{\prime}_i)$ o \emph{peso total}, \emph{peso mínimo total} e \emph{peso máximo total} de $C$, respectivamente. Note que o peso total de um ciclo é a soma dos pesos em suas arestas pretas. Já os pesos mínimo total e máximo total são, respectivamente, a soma dos pesos mínimos e máximos em suas arestas cinzas. 

\begin{definition}
Dada uma instância intergênica flexível $\mathcal{I} = ((\pi,\breve\pi),(\iota,\breve\iota^{\min},\breve\iota^{\max}))$, um ciclo $C \in G(\mathcal{I})$ é chamado de \emph{estável} caso $W^{\min}_c(C) \le W_p(C) \le W^{\max}_c(C)$. Caso contrário, o ciclo $C$ é chamado de \emph{instável}.
\end{definition}

Em outras palavras, um ciclo estável indica que o peso total é suficiente para satisfazer as restrições relativas aos pesos mínimos e máximos em cada uma de suas arestas cinzas. Definimos os conjuntos de ciclos estáveis e instáveis em $G(\mathcal{I})$ como $\mathcal{S}_e(G(\mathcal{I}))$ e $\mathcal{S}_i(G(\mathcal{I}))$, respectivamente. Dado um ciclo $C \in G(\mathcal{I})$, denotamos por $gap_{\min}(C) = W_p(C) - W^{\min}_c(C)$ e $gap_{\max}(C) = W^{\max}_c(C) - W_p(C)$ os valores que se subtraídos e adicionados do peso total de $C$ resultam, respectivamente, nos pesos mínimo total e máximo total de $C$.

O Exemplo~\ref{example:VBQSYHZS} mostra o grafo de ciclos ponderado flexível construído a partir da instância intergênica flexível $\mathcal{I} = ((({+0}~{+4}~{+3}~{-1}~{+2}~{+5}~{+6}),(0,6,2,5,1,3)),(({+0}~{+1}~{+2}\break{+3}~{+4}~{+5}~{+6}),(5,4,2,0,1,2),(6,6,2,2,2,4)))$.

\begin{example}\label{example:VBQSYHZS}
  \scriptsize
  \hfill
  \begin{\position}
    \begin{tikzpicture}[scale=0.7]
      \begin{scope}[every node/.style={inner sep=1.5pt, minimum size = 0pt}]
        \node[circle, draw] (p0) at (0,0) {$+0$};
        \node[circle, draw] (m4) at (1.5,0) {$-4$};
        \node[circle, draw] (p4) at (3,0) {$+4$};
        \node[circle, draw] (m3) at (4.5,0) {$-3$};
        \node[circle, draw] (p3) at (6,0) {$+3$};
        \node[circle, draw] (p1) at (7.5,0) {$+1$};
        \node[circle, draw] (m1) at (9,0) {$-1$};
        \node[circle, draw] (m2) at (10.5,0) {$-2$};
        \node[circle, draw] (p2) at (12,0) {$+2$};
        \node[circle, draw] (m5) at (13.5,0) {$-5$};
        \node[circle, draw] (p5) at (15,0) {$+5$};
        \node[circle, draw] (m6) at (16.5,0) {$-6$};
      \end{scope}
      \begin{scope}[>={Stealth[black]},
                    every edge/.style={draw=black}]
        \path [-] (p0) edge node [black, pos=0.5, sloped, below, yshift=-0.15cm] {$0$} (m4);
        \node[draw=none, fill=none, align=center, minimum width=1cm, text width=1cm] at (0.75, -1.0) {$\ell = {-1}$};
        \path [-] (p4) edge node [black, pos=0.5, sloped, below, yshift=-0.15cm] {$6$} (m3);
        \node[draw=none, fill=none, align=center, minimum width=1cm, text width=1cm] at (3.75, -1.0) {$\ell = 2$};
        \path [-] (p3) edge node [black, pos=0.5, sloped, below, yshift=-0.15cm] {$2$} (p1);
        \node[draw=none, fill=none, align=center, minimum width=1cm, text width=1cm] at (6.75, -1.0) {$\ell = {-3}$};
        \path [-] (m1) edge node [black, pos=0.5, sloped, below, yshift=-0.15cm] {$5$} (m2);
        \node[draw=none, fill=none, align=center, minimum width=1cm, text width=1cm] at (9.75, -1.0) {$\ell = 4$};
        \path [-] (p2) edge node [black, pos=0.5, sloped, below, yshift=-0.15cm] {$1$} (m5);
        \node[draw=none, fill=none, align=center, minimum width=1cm, text width=1cm] at (12.75, -1.0) {$\ell = 5$};
        \path [-] (p5) edge node [black, pos=0.5, sloped, below, yshift=-0.15cm] {$3$} (m6);
        \node[draw=none, fill=none, align=center, minimum width=1cm, text width=1cm] at (15.75, -1.0) {$\ell = 6$};
      \end{scope}
      \begin{scope}[>={Stealth[black]},
                    every edge/.style={draw=black}]
        \path [-] (p0) edge [bend left=70, dashed] node [black, pos=0.5, sloped, below, yshift=-0.05cm] {$5$} node [black, pos=0.5, sloped, above, yshift=+0.05cm] {$6$} (m1);
        \path [-] (p1) edge [bend left=70, dashed] node [black, pos=0.5, sloped, below, yshift=-0.05cm] {$4$} node [black, pos=0.5, sloped, above, yshift=+0.05cm] {$6$} (m2);
        \path [-] (p2) edge [bend right=60, dashed] node [black, pos=0.5, sloped, below, yshift=-0.05cm] {$2$} node [black, pos=0.5, sloped, above, yshift=+0.05cm] {$2$} (m3);
        \path [-] (p3) edge [bend right=60, dashed] node [black, pos=0.5, sloped, below, yshift=-0.05cm] {$0$} node [black, pos=0.5, sloped, above, yshift=+0.05cm] {$2$} (m4);
        \path [-] (p4) edge [bend left=70, dashed] node [black, pos=0.5, sloped, below, yshift=-0.05cm] {$1$} node [black, pos=0.5, sloped, above, yshift=+0.05cm] {$2$} (m5);
        \path [-] (p5) edge [bend left=70, dashed] node [black, pos=0.5, sloped, below, yshift=-0.05cm] {$2$} node [black, pos=0.5, sloped, above, yshift=+0.05cm] {$4$} (m6);
      \end{scope}
    \end{tikzpicture}
  \end{\position}
\end{example}

No Exemplo~\ref{example:VBQSYHZS}, os ciclos $C_1=(4,-1,-3)$, $C_2 = (5,2)$ e $C_3 = (6)$ são, respectivamente, longo instável, curto instável e trivial estável.

Dada uma instância intergênica flexível $\mathcal{I} = ((\pi,\breve\pi),(\iota,\breve\iota^{\min},\breve\iota^{\max}))$, denotamos por $c(G(\mathcal{I}))$ e $c_e(G(\mathcal{I}))$ o número de ciclos e o número ciclos estáveis em $G(\mathcal{I})$, respectivamente. Dada uma sequência de eventos de rearranjo $S$, denotamos por $\Delta c(G(\mathcal{I}), S) = c(G(\mathcal{I^{\prime}})) - c(G(\mathcal{I}))$ e $\Delta c_e(G(\mathcal{I}), S) = c_e(G(\mathcal{I^{\prime}})) - c_e(G(\mathcal{I}))$, tal que $\mathcal{I^{\prime}} = ((\pi,\breve\pi) \cdot S,(\iota,\breve\iota))$, a variação no número de ciclos e no número ciclos estáveis, respectivamente, após a aplicação da sequência $S$ no genoma de origem $(\pi,\breve\pi)$ de $\mathcal{I}$.

\begin{remark}\label{remark:IRNWKUZA}
Dada uma instância intergênica flexível $\mathcal{I} = ((\pi,\breve\pi),(\iota,\breve\iota^{\min},\breve\iota^{\max}))$, tal que $c(G(\mathcal{I})) = c_e(G(\mathcal{I})) = n+1$, temos que $\pi = \iota$ e $\breve\iota^{\min}_i \le \breve\pi_i \le \breve\iota^{\max}_i$ para todo $\breve\pi_i \in \breve\pi$.
\end{remark}

De agora em diante, as definições e conceitos apresentados referem-se às instâncias intergênicas flexíveis balanceadas cujos modelos são compostos exclusivamente por eventos de rearranjo conservativos. Note que, dada uma instância intergênica flexível balanceada $\mathcal{I} = ((\pi,\breve\pi),(\iota,\breve\iota^{\min},\breve\iota^{\max}))$, todos os ciclos instáveis devem ser removidos e $G(\mathcal{I})$ deve possuir $n+1$ ciclos estáveis para transformar $(\pi,\breve\pi)$ em $(\iota,\breve\pi^{\prime})$, tal que $\forall \breve\pi^{\prime}_i \in \breve\pi^{\prime}, \breve\iota^{\min}_i \le \breve\pi^{\prime}_i \le \breve\iota^{\max}_i$. Dependendo da distribuição dos nucleotídeos e das restrições de tamanho mínimo e máximo nas arestas cinzas, alguns dos ciclos estáveis também devem ser afetados para realizar essa tarefa. Na verdade, existem dois cenários em que isso ocorre:

$$\texttt{(i)}~\sum_{C \in \mathcal{S}_i(G(\mathcal{I}))} W_p(C) < \sum_{C \in \mathcal{S}_i(G(\mathcal{I}))} W^{\min}_c(C)$$
$$\texttt{(ii)}\sum_{C \in \mathcal{S}_i(G(\mathcal{I}))} W_p(C) > \sum_{C \in \mathcal{S}_i(G(\mathcal{I}))} W^{\max}_c(C)$$

No cenário \texttt{(i)}, chamado de \emph{fonte}, a soma do peso total de todos os ciclos instáveis é menor que a soma do peso mínimo total dos mesmos ciclos, ou seja, não é possível atender todas as restrições de peso mínimo e máximo nas arestas cinzas dos ciclos instáveis sem que alguns ciclos estáveis transfiram uma determinada quantidade de peso de suas arestas pretas para os ciclos instáveis. No cenário \texttt{(ii)}, chamado de \emph{sorvedouro}, a soma do peso total de todos os ciclos instáveis é maior que a soma do peso máximo total dos mesmos ciclos. Nesse caso, alguns ciclos instáveis precisam transferir uma determinada quantidade de peso de suas arestas pretas para alguns dos ciclos estáveis. Uma instância intergênica flexível balanceada que não pertence ao cenário fonte ou sorvedouro, pertence ao cenário de \emph{equilíbrio}. Neste caso, é possível transformar os ciclos instáveis em estáveis sem afetar os demais ciclos estáveis da instância.

\begin{definition}
Dada uma instância intergênica flexível balanceada $\mathcal{I} = ((\pi,\breve\pi),\break(\iota,\breve\iota^{\min},\breve\iota^{\max}))$, um ciclo estável $C \in G(\mathcal{I})$ é chamado de \emph{auxiliar} se ele deve receber ou transferir peso de suas arestas pretas para outro ciclo, e é chamado de \emph{definitivo} caso contrário.
\end{definition}

Definimos os conjuntos de ciclos auxiliares e definitivos em $G(\mathcal{I})$ como $\mathcal{S}_a(G(\mathcal{I}))$ e $\mathcal{S}_d(G(\mathcal{I}))$, respectivamente. Observe que os cenários fonte e sorvedouro não ocorrem se $\sum_{C \in \mathcal{S}_i(G(\mathcal{I}))} W^{\min}_c(C) \le \sum_{C \in \mathcal{S}_i(G(\mathcal{I}))} W_p(C) \le \sum_{C \in \mathcal{S}_i(G(\mathcal{I}))} W^{\max}_c(C)$, onde temos que $\mathcal{S}_a(G(\mathcal{I})) = \emptyset$ e $\mathcal{S}_d(G(\mathcal{I})) = \mathcal{S}_e(G(\mathcal{I}))$ (Exemplo~\ref{example:VBQSYHZS}). Note que os cenários fonte e sorvedouro não podem ocorrer simultaneamente. Caso um deles ocorra, então é necessário determinar os conjuntos $\mathcal{S}_a(G(\mathcal{I}))$ e $\mathcal{S}_d(G(\mathcal{I}))$, que são dependentes do cenário em que a instância se encaixa. 


Considerando o cenário fonte, um conjunto $\mathcal{S}_a(G(\mathcal{I}))$ de tamanho mínimo pode ser composto do menor número de ciclos, de forma que a seguinte restrição seja cumprida: 

$$\sum_{C \in \mathcal{S}_a(G(\mathcal{I}))} gap_{\min}(C) + \sum_{C \in \mathcal{S}_i(G(\mathcal{I}))} gap_{\min}(C) \ge 0.$$

Considerando o cenário sorvedouro, um conjunto $\mathcal{S}_a(G(\mathcal{I}))$ de tamanho mínimo pode ser composto do menor número de ciclos, de forma  que a seguinte restrição seja cumprida: 

$$\sum_{C \in \mathcal{S}_a(G(\mathcal{I}))} gap_{\max}(C) + \sum_{C \in \mathcal{S}_i(G(\mathcal{I}))} gap_{\max}(C) \ge 0.$$

Observe que, em ambos os cenários, o conjunto $\mathcal{S}_a(G(\mathcal{I}))$ pode ser facilmente obtido após a ordenação, de forma decrescente, dos ciclos estáveis pelos valores $gap_{\min}$ e $gap_{\max}$, considerando os casos fonte e sorvedouro, respectivamente. Então, seguindo a ordem decrescente, os ciclos são rotulados como auxiliares até que a restrição seja satisfeita. O conjunto de ciclos definitivos $\mathcal{S}_d(G(\mathcal{I}))$ é obtido por $\mathcal{S}_e(G(\mathcal{I})) - \mathcal{S}_a(G(\mathcal{I}))$, note que $\mathcal{S}_a(G(\mathcal{I})) \cup \mathcal{S}_d(G(\mathcal{I})) = \mathcal{S}_e(G(\mathcal{I}))$.

Dada uma instância intergênica flexível balanceada $\mathcal{I} = ((\pi,\breve\pi),(\iota,\breve\iota^{\min},\breve\iota^{\max}))$, denotamos por $c_d(G(\mathcal{I}))$ o número de ciclos definitivos em $G(\mathcal{I})$. Dada uma sequência de eventos de rearranjo $S$, denotamos por $\Delta c_d(G(\mathcal{I}), S) = c_d(G(\mathcal{I^{\prime}})) - c_d(G(\mathcal{I}))$, tal que $\mathcal{I^{\prime}} = ((\pi,\breve\pi) \cdot S,(\iota,\breve\iota))$, a variação no número de ciclos definitivos após a aplicação da sequência $S$ no genoma de origem $(\pi,\breve\pi)$ de $\mathcal{I}$.

\begin{remark}\label{remark:HLVDQLCE}
Seja $\mathcal{I} = ((\pi,\breve\pi),(\iota,\breve\iota^{\min},\breve\iota^{\max}))$ uma instância intergênica flexível balanceada tal que $c(G(\mathcal{I})) = c_d(G(\mathcal{I})) = n+1$, então temos que $\pi = \iota$ e $\breve\iota^{\min}_i \le \breve\pi_i \le \breve\iota^{\max}_i$ para todo $\breve\pi_i \in \breve\pi$.
\end{remark}

O Exemplo~\ref{example:SSQOHQDY} mostra o grafo de ciclos ponderado flexível construído a partir da instância intergênica flexível $\mathcal{I} = ((({+0}~{+3}~{+2}~{+1}~{+4}~{+5}~{+6}),(1,2,0,2,6,2)),(({+0}~{+1}~{+2}\break{+3}~{+4}~{+5}~{+6}),(3,2,4,0,2,1),(4,3,5,4,6,2)))$.

\input{examples/SSQOHQDY}

No Exemplo~\ref{example:SSQOHQDY}, $G(\mathcal{I})$ possui quatro ciclos, sendo eles: $C_1 = (3,1)$, $C_2 = (4,2)$, $C_3 = (5)$ e $C_4=(6)$. Além disso, temos os conjuntos $\mathcal{S}_i(G(\mathcal{I})) = \{C_1\}$ e $\mathcal{S}_e(G(\mathcal{I})) = \{C_2,C_3,C_4\}$. Observe que a instância intergênica flexível $\mathcal{I}$ do Exemplo~\ref{example:SSQOHQDY} pertence ao caso fonte, onde apenas o ciclo instável $C_1$ precisa aumentar o seu peso total para ser transformado em um ciclo estável ($1 = W_p(C_1) < W^{\min}_c(C_1) = 7$). Note que $gap_{\min}(C_2) = 2$, $gap_{\min}(C_3) = 4$ e $gap_{\min}(C_4) = 1$. Portanto, temos que $\mathcal{S}_a(G(\mathcal{I})) = \{C_2, C_3\}$ e $\mathcal{S}_d(G(\mathcal{I})) = \{C_4\}$.

O Exemplo~\ref{example:XSRSUPBR} mostra o grafo de ciclos ponderado flexível construído a partir da instância intergênica flexível $\mathcal{I} = ((({+0}~{+5}~{+4}~{+3}~{+2}~{+1}~{+6}),(1,2,4,2,6,2)),(({+0}~{+1}~{+2}\break{+3}~{+4}~{+5}~{+6}),(3,2,1,0,2,1),(4,3,1,4,3,4)))$.

\input{examples/XSRSUPBR}

No Exemplo~\ref{example:XSRSUPBR}, $G(\mathcal{I})$ possui dois ciclos, sendo eles: $C_1 = (5,3,1)$ e $C_2 = (6,4,2)$. Além disso, temos os conjuntos $\mathcal{S}_i(G(\mathcal{I})) = \{C_1\}$ e $\mathcal{S}_e(G(\mathcal{I})) = \{C_2\}$. Observe que a instância intergênica flexível $\mathcal{I}$ do Exemplo~\ref{example:XSRSUPBR} pertence ao caso sorvedouro, onde apenas o ciclo instável $C_1$ precisa reduzir o seu peso total para ser transformado em um ciclo estável ($11 = W_p(C_1) > W^{\max}_c(C_1) = 8$). Note que $gap_{\max}(C_2) = 5$. Portanto, temos que $\mathcal{S}_a(G(\mathcal{I})) = \{C_2\}$ e $\mathcal{S}_d(G(\mathcal{I})) = \emptyset$.
\chapter{Modelos com Porporção entre Operações}\label{chapter:JWIGFELF}

Os problemas de distância entre genomas podem utilizar uma abordagem \emph{não ponderada}, ou seja, cada evento de rearranjo utilizado para transformar o genoma de origem no genoma alvo contribui em uma unidade para a distância. Essa abordagem tem como característica que cada tipo de evento de rearranjo, pertencente ao modelo de rearranjo adotado, possui a mesma probabilidade de ocorrer em um cenário evolutivo. Outra abordagem que surgiu para possibilitar uma contribuição diferente para cada evento de rearranjo é chamada de \emph{ponderada}. Nesse abordagem, cada tipo de evento de rearranjo possui um peso associado que é contabilizado na distância evolutiva entre os genomas. A abordagem ponderada geralmente é utilizada para mapear um cenário em que queremos que determinado eventos de rearrajo tenham uma possibilidade maior de ocorrer do que outros. Para isso, basta atribuir um peso menor nos eventos de rearranjo que esperados que ocorram mais. Esses pesos podem ser atribuídos com base em observações empíricas de determinados organismos ou através de análises realizadas especificamente para esse objetivo~\cite{2008-bader-etal,2001-eriksen}. 

Os eventos de rearranjo de reversão e transposição são dois dos eventos mais estudados na literatura~\cite{2002-berman-etal,2006-elias-hartman,2022-silva-etal}. Considerando uma representação clássica e uma abordagem não ponderada, temos o problema de Ordenção de Permutações por Reversões e Transposições (\SbRT), sendo que o problema possui a variação com e sem sinais. Ambas as variações pertencem à classe NP-difícil de problemas~\cite{2019b-oliveira-etal}, para a variação com sinais do problema existe um algoritmo de aproximação com fator 2~\cite{1998-walter-etal}. Para a variação sem sinais, existe um algoritmo de aproximação com fator $2k$~\cite{2008-rahman-etal}, onde $k$~\cite{2013-chen} é o fator de aproximação do algoritmo utilizado para a decomposição de ciclos do Grafo de Ciclos~\cite{1999-caprara}.

Considerando um abordagem poderada, temos o problema de Ordenção de Permutações por Reversões e Transposições Ponderadas (\SbWRT) na variação com e sem sinais. In 2002, Eriksen~\cite{2002-eriksen} apresentou um algoritmo com factor de aproximação $7/6$ para a variação com sinais do problema utilizando os pesos $1$ e $2$ para os eventos de reversão e transposição, respectivamente. Oliveira \textit{et al.}\cite{2019a-oliveira-etal} desenvolveram um algoritmo de aproximação com fator $1.5$ para a variação com sinais do problema \SbWRT{} utilizando os pesos $2$ e $3$ para os eventos de reversão e transposição, respectivamente. Além disso, os autores mostraram que as variações com e sem sinais do problema \SbWRT{} pertencem à classe NP-difícil quando a razão entre os pesos dos eventos de transposição e reversão é maior ou igual a $1.5$.  

Em 2007, Bader e Ohlebusch~\cite{2007-bader-ohlebusch} apresentaram o problema de Ordenção de Permutações por Reversões, Transposições e Transposições Inversa Ponderadas (\SbWRTIT). A transposição inversa é um evento similar ao evento de transposição, mas com um dos segmentos adjacentes afetados sendo invertido. Para a variação com sinais do problema os autores apresentaram um algoritmo de aproximação com fator $1.5$ utilizando o peso $1$ para o evento de reversão e o mesmo peso, no intervalo $[1..2]$, para os eventos de transposição e transposição inversa. Em 2020, Alexandrino \textit{et al.}\cite{2020b-alexandrino-etal} mostraram que as variações com e sem sinais do problema \SbWRTIT{} pertencem à classe NP-difícil quando os eventos de transposição e transposição inversa possuem o mesmo peso e a razão entre os pesos dos eventos de transposição e reversão é maior ou igual a $1.5$.

A abordagem ponderada possui vantagens em comparação com a abordagem não ponderada quando queremos mapear um cenário evolutivo dando mais prioridade para determinados tipos de eventos de rearranjo. Entretanto, ela não garantem que os rearranjos de menor custo, que são supostamente os mais frequentes em um cenário evolutivo, serão os mais utilizados pelos algoritmos. Para contornar esse ponto, propomos e investigamos o problema de Ordenção de Permutações por Reversões e Transposições com Restrição de Proporção (\SbPRT) em instâncias clássicas com e sem sinais. Neste cenário, buscamos uma sequência de reversões e transposições $S$ capaz de transformar o genoma de origem no genoma alvo com uma restrição adicional na qual a relação entre o número de reversões e o tamanho da sequência $S$ deve ser maior ou igual a um determinado parâmetro $k \in [0..1]$. 

Observe que tanto as abordagens ponderada e proporcional tentam incorporar no modelo a frequência na qual os eventos de rearranjo afetam o genoma de um determinado organismo. É importante notar que, do ponto de vista biológico, a frequência e o conjunto de eventos de rearranjo podem variar dependendo do organismo considerado. De um ponto de vista teórico, as abordagens possuem objetivos diferentes, apesar de compartilharem características comuns. Uma característica que difere da abordagem de proporção é que uma vez conhecida a frequência na qual os eventos afetam o genoma, a proporção pode ser facilmente derivada dessa informação, enquanto que na abordagem ponderada o peso associado a cada tipo de evento precisa ser ajustado e validado através de testes experimentais.

O Exemplo~\ref{example:DFGJNHTP} mostra uma solução ótimo $S$ para a instância clássica com sinais $(({+0}~{-1}~{+4}~{-8}~{+3}~{+5}~{+2}~{-7}~{-6}~{+9}),({+0}~{+1}~{+2}~{+3}~{+4}~{+5}~{+6}~{+7}~{+8}~{+9}))$ considerando os problemas \SbRT{} e \SbWRT{} (utilizando os pesos $2$ e $3$ para os eventos de reversão e transposição, respectivamente). Note que metade dos eventos de rearranjo de $S$ são reversões e a outra metade transposições, mesmo utilizando um custo maior para o evento de transposição.

\begin{example}\label{example:DFGJNHTP}
  \hfill \break
  % \scriptsize
  \begin{tabular}{lllll}
    $\pi    $ & $=$ &                                                             &     & $({+0}~{-1}~{+4}~{-8}~{+3}~{+5}~{+2}~{-7}~{-6}~{+9})                         $  \\
    $\pi^{1}$ & $=$ & $\pi \cdot \rho^{(1,5)}$                                    & $=$ & $({+0}~\underline{{-5}~{-3}~{+8}~{-4}~{+1}}~{+2}~{-7}~{-6}~{+9})             $  \\
    $\pi^{2}$ & $=$ & $\pi^{1} \cdot \tau^{(2,4,9)}$                              & $=$ & $({+0}~{-5}~\underline{{-4}~{+1}~{+2}~{-7}~{-6}}~\underline{{-3}~{+8}}~{+9}) $  \\
    $\pi^{3}$ & $=$ & $\pi^{2}\cdot \tau^{(1,3,7)}$                               & $=$ & $({+0}~\underline{{+1}~{+2}~{-7}~{-6}}~\underline{{-5}~{-4}}~{-3}~{+8}~{+9}) $  \\
    $\pi^{4}$ & $=$ & $\pi^{3} \cdot \rho^{(3,7)}$                                & $=$ & $({+0}~{+1}~{+2}~\underline{{+3}~{+4}~{+5}~{+6}~{+7}}~{+8}~{+9})             $  \\
    $S      $ & $=$ & $(\rho^{(1,5)},\tau^{(2,4,9)},\tau^{(1,3,7)},\rho^{(3,7)})$ &     &                                                                                 
  \end{tabular}
\end{example}


O Exemplo~\ref{example:MODRXOJQ} mostra uma solução ótima $S'$ para a mesma instância clássica com sinais apresentada no Exemplo~\ref{example:DFGJNHTP} considerando o problem \SbPRT{} e adotando um valor de $k = 0.6$, ou seja, pelo menos 60\% dos eventos de rearranjo em $S'$ devem ser reversões. Quando comparamos com o Exemplo~\ref{example:DFGJNHTP}, podemos perceber que $S'$ possui apenas um evento a mais que $S$, mas a proporção mínima de reversões em relação ao tamanho da sequência $S'$ é garantida.

\begin{example}\label{example:MODRXOJQ}
  \hfill
  \begin{\position}
    \begin{tabular}{lllll}
      $\pi    $ & $=$ &                                                                         &     & $({+0}~{-1}~{+4}~{-8}~{+3}~{+5}~{+2}~{-7}~{-6}~{+9})                         $ \\
      $\pi^{1}$ & $=$ & $\pi \cdot \rho^{(2,8)}$                                                & $=$ & $({+0}~{-1}~\underline{{+6}~{+7}~{-2}~{-5}~{-3}~{+8}~{-4}}~{+9})             $ \\
      $\pi^{2}$ & $=$ & $\pi^{1}\cdot \rho^{(2,4)}$                                             & $=$ & $({+0}~{-1}~\underline{{+2}~{-7}~{-6}}~{-5}~{-3}~{+8}~{-4}~{+9})             $ \\
      $\pi^{3}$ & $=$ & $\pi^{2} \cdot \tau^{(6,8,9)}$                                          & $=$ & $({+0}~{-1}~{+2}~{-7}~{-6}~{-5}~\underline{{-4}}~\underline{{-3}~{+8}}~{+9}) $ \\
      $\pi^{4}$ & $=$ & $\pi^{3} \cdot \rho^{(1,1)}$                                            & $=$ & $({+0}~\underline{{+1}}~{+2}~{-7}~{-6}~{-5}~{-4}~{-3}~{+8}~{+9})             $ \\
      $\pi^{5}$ & $=$ & $\pi^{4} \cdot \rho^{(3,7)}$                                            & $=$ & $({+0}~{+1}~{+2}~\underline{{+3}~{+4}~{+5}~{+6}~{+7}}~{+8}~{+9})             $ \\
      $S'     $ & $=$ & $(\rho^{(2,8)},\rho^{(2,4)},\tau^{(6,8,9)},\rho^{(1,1)},\rho^{(3,7)})$  &     &                                                                               
    \end{tabular}
  \end{\position}
\end{example}

Dada uma sequência de eventos de rearranjo $S$, denotamos por $|S|$ o tamanho da sequência $S$, ou seja, a quantidade de eventos em $S$. Além disso, denotamos por $|S_{\rho}|$ a quantidade de eventos de reversão em $S$. A seguir, descrevemos formalmente o problema de Ordenção de Permutações por Reversões e Transposições com Restrição de Proporção.

\begin{task}
  \problemtitle{Ordenção de Permutações por Reversões e Transposições com Restrição de Proporção (\SbPRT)}
  \probleminput{Uma instância clássica com ou sem sinais $\mathcal{I}=(\pi,\iota)$ e um número racional $k \in [0..1]$.}
  \problemtask{Com base no modelo de rearranjo $\mathcal{M}=\{\rho,\tau\}$, determinar uma sequência de eventos de rearranjo $S$ de tamanho mínimo capaz de transformar $\pi$ em $\iota$, tal que $\frac{|S_{\rho}|}{|S|} \ge k$.}
\end{task}

Dada uma instância clássica com ou sem sinais $\mathcal{I}=(\pi,\iota)$ e um número racional $k \in [0..1]$, a \emph{distância de propoção} entre $\pi$ e $\iota$, denotada por $dp_{k}(\mathcal{I})$, é o tamanho da menor sequências de eventos de rearranjo $S$, tal que todo evento de $S$ pertence ao modelo $\mathcal{M}=\{\rho,\tau\}$, $\pi \cdot S = \iota$ e $\frac{|S_{\rho}|}{|S|} \ge k$. Por praticidade, nesse capítulo iremos nos referir a um breakpoint clássico simplesmente como um breakpoint.

Nesse capítulo, provaremos que o problema \SbPRT{} pertence à classe NP-difícil em instâncias clássicas sem sinais para qualquer valor de $k$. Em instâncias clássicas com sinais mostraremos que existe um algoritmo exato polinomial para o problema quando $k=1$ e provaremos que o problema pertence à classe NP-difícil quando $k < 1$. Para as variações com e sem sinais do problema \SbPRT{} apresentaremos algoritmos de aproximação com fatores $3 - \frac{3k}{2}$ e $3-k$, respectivamente. Além disso, apresentaremos um algoritmo de aproximação assintótico com um fator teórico melhor para instâncias clássicas com sinais. Por fim, realizaremos experimentos comparando o desempenho práticos dos algoritmo propostos.

Estes resultados foram publicados em 2021 na revista \emph{Journal of Bioinformatics and Computational Biology}~\cite{2021a-brito-etal}.

% ------------------------------------------------------------------ %
\section{Limitantes Inferiores}
% ------------------------------------------------------------------ %

Nessa seção, apresentamos limitantes inferiores para as variações com e sem sinais do problema \SbPRT{}.

\begin{lemma}[Kececioglu e Sankoff~\cite{1995-kececioglu-sankoff}]\label{lemma:QIRAVPQT}
Dada uma instância clássica sem sinais $\mathcal{I} = (\pi,\iota)$, para qualquer reversão $\rho$ temos que $\Delta b_1(\mathcal{I}, S = (\rho)) \ge -2$.
\end{lemma}

\begin{lemma}[Walter \textit{et al.}~\cite{1998-walter-etal}]\label{lemma:NJATEDCC}
Dada uma instância clássica sem sinais $\mathcal{I} = (\pi,\iota)$, para qualquer transposição $\tau$ temos que $\Delta b_1(\mathcal{I}, S = (\tau)) \ge -3$.
\end{lemma}

\begin{lemma}\label{lemma:JYYZBREC}
Dada uma instância clássica sem sinais $\mathcal{I} = (\pi,\iota)$ para o problema $\SbPRT{}$ considerando a proporção $k \in [0..1]$ e seja $S$ uma sequência ótima de eventos de rearranjo para o problema. O número de breakpoints tipo um removidos por cada evento de $S$, em média, é menor ou igual a $3-k$.
\end{lemma}
\begin{proof}
Como $S$ é uma sequência ótima para a instância $\mathcal{I}$ com base na proporção $k$, temos que pelo menos $|S|k$ eventos presentes em $S$ são reversões. Pelos lemas~\ref{lemma:QIRAVPQT} e \ref{lemma:NJATEDCC}, temos que uma reversão pode remover até dois breakpoints tipo um enquanto uma transposição pode remover até três. Seja $\phi b(S)$ o número médio de breakpoints tipo um removidos por um evento de $S$, temos que:
$$\phi b(S) \le \frac{(2 |S| k) + (3 |S| (1 - k))}{|S|} = 2k + 3(1 - k) = 3 - k.$$ 
\end{proof}

\begin{lemma}[Hannenhalli e Pevzner~\cite{1999-hannenhalli-pevzner}]\label{lemma:MYFTFTWE}
Dada uma instância clássica com sinais $\mathcal{I} = (\pi,\iota)$, para qualquer reversão $\rho$ temos que $\Delta b_2(\mathcal{I}, S = (\rho)) \ge -2$.
\end{lemma}

\begin{lemma}\label{lemma:AKEQRMOY}
Dada uma instância clássica com sinais $\mathcal{I} = (\pi,\iota)$, para qualquer transposição $\tau$ temos que $\Delta b_2(\mathcal{I}, S = (\tau)) \ge -3$.
\end{lemma}
\begin{proof}
Note que uma transposição pode afetar no máximo três breakpoints tipo dois de $\mathcal{I}$. Logo, no melhor cenário, os três breakpoints são removidos e o lema segue.
\end{proof}

\begin{lemma}\label{lemma:ZZLNPRWJ}
Dada uma instância clássica com sinais $\mathcal{I} = (\pi,\iota)$ para o problema $\SbPRT{}$ considerando a proporção $k \in [0..1]$ e seja $S$ uma sequência ótima de eventos de rearranjo para o problema. O número de breakpoints tipo dois removidos por cada evento de $S$, em média, é menor ou igual a $3-k$.
\end{lemma}
\begin{proof}
A prova é similar a descrita no Lema~\ref{lemma:JYYZBREC}, mas considerando os lemas~\ref{lemma:MYFTFTWE} e \ref{lemma:AKEQRMOY}.
\end{proof}

\begin{lemma}[Hannenhalli e Pevzner~\cite{1999-hannenhalli-pevzner}]\label{lemma:WMUBEYJS}
Dada uma instância clássica com sinais $\mathcal{I} = (\pi,\iota)$, para qualquer reversão $\rho$ temos que $\Delta c(G(\mathcal{I}), S = (\rho)) \le 1$.
\end{lemma}

\begin{lemma}[Bafna e Pevzner~\cite{1995b-bafna-pevzner}; Walter \textit{et al.}~\cite{1998-walter-etal}]\label{lemma:WITSEXYZ}
Dada uma instância clássica com ou sem sinais $\mathcal{I} = (\pi,\iota)$, para qualquer transposição $\tau$ temos que $\Delta c(G(\mathcal{I}), S = (\tau)) \le 2$.
\end{lemma}

\begin{lemma}\label{lemma:IMSCPWKN}
Dada uma instância clássica com sinais $\mathcal{I} = (\pi,\iota)$ para o problema $\SbPRT{}$ considerando a proporção $k \in [0..1]$ e seja $S$ uma sequência ótima de eventos de rearranjo para o problema. A variação no número de ciclos para cada evento de $S$, em média, é menor ou igual a $2-k$.
\end{lemma}
\begin{proof}
Como $S$ é uma sequência ótima para a instância $\mathcal{I}$ com base na proporção $k$, temos que pelo menos $|S|k$ eventos presentes em $S$ são reversões. Pelos lemas~\ref{lemma:WMUBEYJS} e \ref{lemma:WITSEXYZ}, temos que uma reversão pode criar até um novo ciclo enquanto uma transposição pode criar até dois. Seja $\phi c(S)$ o número médio de ciclos criados por um evento de $S$, temos que:
$$\phi c(S) \leq \frac{(1 |S| k) + (2 |S| (1 - k))}{|S|} = 1k + 2(1 - k) = 2 - k.$$
\end{proof}

\begin{theorem}\label{theorem:KJXGKJIP}
Dada uma instância clássica sem sinais $\mathcal{I} = (\pi,\iota)$ para o problema $\SbPRT{}$ e uma proporção $k \in [0..1]$, temos que $dp_{k}(\mathcal{I}) \ge \frac{b_1(\mathcal{I})}{3-k}$.
\end{theorem}
\begin{proof}
Como $b_1(\mathcal{I})$ breakpoints tipo um devem ser removidos para transformar $\pi$ em $\iota$ e, pelo Lema~\ref{lemma:JYYZBREC}, até $3-k$ breakpoints tipo um são removidos, em média, por cada operação de uma sequência ótima para o problema. Logo, o teorema segue.
\end{proof}

\begin{theorem}\label{theorem:ABGZQHIL}
Dada uma instância clássica com sinais $\mathcal{I} = (\pi,\iota)$ para o problema $\SbPRT{}$ e uma proporção $k \in [0..1]$, temos que $dp_{k}(\mathcal{I}) \ge \frac{b_2(\mathcal{I})}{3-k}$.
\end{theorem}
\begin{proof}
A prova é similar a descrita no Teorema~\ref{theorem:KJXGKJIP}, mas considerando o número de breakpoints tipo dois em $\mathcal{I}$ e o Lema~\ref{lemma:ZZLNPRWJ}.
\end{proof}

\begin{theorem}\label{theorem:WSTPPSMD}
Dada uma instância clássica com sinais $\mathcal{I} = (\pi,\iota)$ para o problema $\SbPRT{}$ e uma proporção $k \in [0..1]$, temos que $dp_{k}(\mathcal{I}) \ge \frac{n + 1 - c(G(\mathcal{I}))}{2-k}$.
\end{theorem}
\begin{proof}
Note que, pela Obervação~\ref{remark:OYRVGHTB}, $n+1 - c(G(\mathcal{I}))$ novos ciclos precisam ser criados para transformar $\pi$ em $\iota$. Pelo Lema~\ref{lemma:IMSCPWKN}, até $2-k$ novos ciclos são criados, em média, por cada operação de uma sequência ótima para o problema. Logo, o teorema segue.
\end{proof}

% ------------------------------------------------------------------ %
\section{Análise de Complexidade}
% ------------------------------------------------------------------ %

Nessa seção, apresentamos uma análise de complexidade do problema \SbPRT{} em instâncias clássicas com e sem sinais para todos os possível valores de $k$. A seguir descrevemos formalmente a versão de decisão do problema \SbPRT{}.

\begin{decision}
  \problemtitle{\SbPRT (Versão de Decisão)}
  \probleminput{Uma instância clássica com ou sem sinais $\mathcal{I}=(\pi,\iota)$, um número racional $k \in [0..1]$ e um número natural $t$.}
  \problemquestion{Existe uma sequência de eventos de rearranjo $S$, com base no modelo de rearranjo $\mathcal{M}=\{\rho,\tau\}$, capaz de transformar $\pi$ em $\iota$, tal que $\frac{|S_{\rho}|}{|S|} \ge k$ e $|S| = t$?}
\end{decision}

Note que para o problema \SbPRT{} é possível fornecer como entrada diferentes valores para $k$. Entretanto, quando utilizamos o valor de $k=0$ obtemos o problema \SbRT{}, uma vez que estipulamos que em uma solução não é necessário obter uma porcentagem mínima de eventos de reversão em comparação ao tamanho da sequência de eventos de rearranjo. Por outro lado, quando adotamos o valor de $k=1$, obtemos o problema de Ordenação de Permutações por Reversões (\SbR{}). Note que, nesse caso, toda solução para o problema deve ser composta exclusivamente por eventos de reversão. Com base nessa característica do problema obtemos os seguintes lemmas.

\begin{lemma}
O problema \SbPRT{} em instâncias clássicas com sinais pertence à classe NP-difícil quando $k=1$ e existe um algoritmo exato polinomial quando $k=0$.
\end{lemma}
\begin{proof}
Quando $k=0$ o problema \SbPRT{} em instâncias clássicas com sinais torna-se a variação com sinais do problema \SbRT{}, que é NP-difícil~\cite{2019b-oliveira-etal}. Por outro lado, quando $k=1$ o problema \SbPRT{} em instâncias clássicas com sinais torna-se a variação com sinais do problema \SbR{}, que possui um algoritmo exato polinomial~\cite{1999-hannenhalli-pevzner}.
\end{proof}

\begin{lemma}
O problema \SbPRT{} em instâncias clássicas sem sinais pertence à classe NP-difícil quando $k \in \{0,1\}$.
\end{lemma}
\begin{proof}
Quando $k=0$ e $k=1$ o problema \SbPRT{} em instâncias clássicas sem sinais torna-se a variação sem sinais dos problemas \SbRT{} e \SbR{}, respectivamente. Ambos os problemas pertencem à classe NP-difícil~\cite{2019b-oliveira-etal,1999-caprara}.
\end{proof}

A seguir investigamos a complexidade do problema \SbPRT{} quando $k$ pertence ao intervalo $(0..1)$. Para isso, apresentamos definições que serão utilizadas para provar a complexidade do problema para esse intervalo de valores de $k$. As transformações de \emph{duplicação}, \emph{orientação}, \emph{extensão bridge} e \emph{extensão gadget} descritas a seguir utilizam uma representação clássica de um genoma na sua forma não estendida. Caso a representação esteja na forma estendida, os elementos $\pi_0$ e $\pi_{n+1}$ são ignorados, a transformação é aplicada e a nova representação clássica resultante é então estendida.

\begin{definition}
Dada uma representação clássica sem sinais $\pi$ de tamanho $n$, a \emph{duplicação} cria uma representação clássica sem sinais $\pi'$ de tamanho $2n$ de forma que cada elemento $\pi_i \in \pi$ é mapeado em dois novos valores, com $\pi'_{2i-1} = 2\pi_i-1$ e $\pi'_{2i} = 2\pi_i$, para $i \in [1..n]$.
\end{definition}

O Exemplo~\ref{example:QJCKPQSS} mostra o transformação de duplicação sendo aplicado na representação clássica sem sinais $\pi=(4~1~5~3~2)$.

\begin{example}\label{example:QJCKPQSS}
  \hfill
  \begin{\position}
    \begin{tabular}{lll}
      $\pi$  & $=$ & $(4~1~5~3~2)$ \\
      $\pi'$ & $=$ & $(7~8~1~2~9~10~5~6~3~4)$ \\
    \end{tabular}
  \end{\position}
\end{example}

\begin{definition}
Dada uma representação clássica sem sinais $\pi$ de tamanho $n$, a \emph{orientação} cria uma representação clássica com sinais $\pi'$ também de tamanho $n$ de forma que $\pi'_{i} = +\pi_i$, para $i \in [1..n]$.
\end{definition}

O Exemplo~\ref{example:GUELXUJE} mostra a transformação de orientação sendo aplicado na representação clássica sem sinais $\pi=(4~1~5~3~2)$.

\begin{example}\label{example:GUELXUJE}
  \hfill
  \begin{\position}
    \begin{tabular}{lll}
      $\pi$  & $=$ & $(4~1~5~3~2)$ \\
      $\pi'$ & $=$ & $({+4}~{+1}~{+5}~{+3}~{+2})$ \\
    \end{tabular}
  \end{\position}
\end{example}

\begin{definition}
Dada uma representação clássica com ou sem sinais $\pi$ de tamanho $n$, a \emph{extensão bridge} cria uma representação clássica $\pi'$ de tamanho $n + 3$. Caso $\pi$ seja uma representação com sinais, $\pi'$ é gerado da seguinte forma: (i) $\pi'_{i} = \pi_i$ e (ii) $\pi'_{n+j} = +(n{+j})$, para $i \in [1..n]$ e $j \in [1..3]$. Caso contrário, $\pi'$ é gerado da seguinte forma: (i) $\pi'_{i} = \pi_i$ e (ii) $\pi'_{n+j} = n{+j}$, para $i \in [1..n]$ e $j \in [1..3]$.
\end{definition}

O Exemplo~\ref{example:AWNIOTEZ} mostra a transformação de extensão bridge sendo aplicada na representação clássica com sinais $\pi=({+4}~{+1}~{+5}~{-3}~{-2})$.

\begin{example}\label{example:AWNIOTEZ}
  \hfill
  \begin{\position}
    \begin{tabular}{lll}
      $\pi$  & $=$ & $({+4}~{+1}~{+5}~{-3}~{-2})$ \\
      $\pi'$ & $=$ & $({+4}~{+1}~{+5}~{-3}~{-2}~{+6}~{+7}~{+8})$ \\
    \end{tabular}
  \end{\position}
\end{example}

O Exemplo~\ref{example:BYQQHUAS} mostra a transformação de extensão bridge sendo aplicada na representação clássica sem sinais $\pi=(4~1~5~3~2)$.

\begin{example}\label{example:BYQQHUAS}
  \hfill \break
  % \centering
  \begin{tabular}{lll}
    $\pi$  & $=$ & $(4~1~5~3~2)$ \\
    $\pi'$ & $=$ & $(4~1~5~3~2~6~7~8)$ \\
  \end{tabular}
\end{example}

\begin{definition}
Dada uma representação clássica com ou sem sinais $\pi$ de tamanho $n$, a \emph{extensão gadget} cria uma representação clássica $\pi'$ de tamanho $n + 6$. Caso $\pi$ seja uma representação com sinais, $\pi'$ é gerado da seguinte forma: (i) $\pi'_{i} = \pi_i$; (ii) $\pi'_j = -(n+4-j)$; (iii) $\pi'_{n+k} = +(n{+k})$, para $i \in [1..n]$, $j \in [1..3]$ e $k \in [4..6]$. Caso contrário, $\pi'$ é gerado da seguinte forma: (i) $\pi'_{i} = \pi_i$; (ii) $\pi'_j = n+4-j$; (iii) $\pi'_{n+k} = n{+k}$, para $i \in [1..n]$, $j \in [1..3]$ e $k \in [4..6]$.
\end{definition}

O Exemplo~\ref{example:TCTQPMWV} mostra a transformação de extensão gadget sendo aplicada na representação clássica com sinais $\pi=({+4}~{+1}~{+5}~{-3}~{-2})$.

\begin{example}\label{example:TCTQPMWV}
  \hfill \break
  % \centering
  \begin{tabular}{lll}
    $\pi$  & $=$ & $({+4}~{+1}~{+5}~{-3}~{-2})$ \\
    $\pi'$ & $=$ & $({+4}~{+1}~{+5}~{-3}~{-2}~{-8}~{-7}~{-6}~{+9}~{+10}~{+11})$ \\
  \end{tabular}
\end{example}

O Exemplo~\ref{example:ZMGTJRFE} mostra a transformação de extensão gadget sendo aplicada na representação clássica sem sinais $\pi=(4~1~5~3~2)$.

\begin{example}\label{example:ZMGTJRFE}
  \hfill
  \begin{\position}
    \begin{tabular}{lll}
      $\pi$  & $=$ & $(4~1~5~3~2)$ \\
      $\pi'$ & $=$ & $(4~1~5~3~2~8~7~6~9~10~11)$ \\
    \end{tabular}
  \end{\position}
\end{example}

A seguir descrevemos formalmente a versão de decisão do problema de Ordenação de Permutações por 3-Transposições (\textbf{B3T}).

\begin{decision}
  \problemtitle{\textbf{B3T} (Versão de Decisão)}
  \probleminput{Uma instância clássica sem sinais $\mathcal{I}=(\pi,\iota)$, tal que $b_2(\mathcal{I}) = 3s$ e $s$ é um número natural não nulo.}
  \problemquestion{Existe uma sequência de eventos de rearranjo $S$, com base no modelo de rearranjo $\mathcal{M}=\{\tau\}$, capaz de transformar $\pi$ em $\iota$, tal que $|S| = \frac{b_2(\mathcal{I})}{3}$?}
\end{decision}

Bulteau e coautores~\cite{2012-bulteau-etal} provaram que o problem \textbf{B3T} pertence à classe NP-difícil. Utilizaremos uma redução do problema \textbf{B3T} para provar que o problema \SbPRT{} é NP-difícil quando $k$ pertence ao intervalo $(0..1)$.

\begin{lemma}[Oliveira~\textit{et al.}~\cite{2019b-oliveira-etal}]\label{lemma:CWNRJAPM}
Se uma instância clássica com sinais $\mathcal{I}=(\pi,\iota)$ possui apenas strips positivas, para qualquer reversão $\rho$ temos que $\Delta b_2(\mathcal{I}, S=(\rho)) \ge 0$.
\end{lemma}

\begin{theorem}\label{theorem:NSWQYFLG}
O problema \SbPRT{} em instâncias clássicas com sinais pertence à classe NP-difícil quando $k \in (0..1)$.
\end{theorem}
\begin{proof}
Dada uma instância clássica sem sinais $\mathcal{I}=(\pi,\iota)$ para o problema \textbf{B3T}, definimos $\ell = \frac{b_2(\mathcal{I})}{3} \ge 1$. Criamos uma instância clássica com sinais $\mathcal{I'}=(\pi',\iota')$ para o problema \SbPRT{} da seguinte maneira:

\begin{enumerate}
  \item Seja $\sigma$ uma representação clássica com sinais de tamanho $n+3$ obtida através do processo de orientação aplicado em $\pi$ e seguido da extenção bridge.
  \item Seja $k$ um número racional no intervalo $(0..1)$, definimos $p = \lceil\frac{\ell k}{1-k}\rceil \ge 1$, ou seja, $p$ é o menor número inteiro tal que $\frac{p}{p+\ell} \ge k$.
  \item Seja $\pi'$ uma representação clássica com sinais de tamanho $n+3+6p$ obtida através da aplicação consecutiva de $p$ extensões gadget em $\sigma$.
  \item Seja $\iota'$ uma representação clássica com sinais de tamanho $n+3+6p$. Caso $\pi$ esteja na sua forma estendida, $\iota'_i = +i$ para $i \in [1..(n+3+6p)]$. Caso contrário,  $\iota'_i = +i$ para $i \in [0..(n+3+6p-1)]$.
\end{enumerate}

O Exemplo~\ref{example:NDFPEMFC} mostra o processo de criação de uma instância clássica com sinais $\mathcal{I'}=(\pi',\iota')$ para o problema \SbPRT{} a partir de uma instância clássica sem sinais $\mathcal{I}=(\pi,\iota)$ para o problema \textbf{B3T}. Note que em ambas as instâncias os genomas de origem e alvo são representados na forma estendida. Além disso, é importante lembrar que o problema \textbf{B3T} e a variação com sinais do problema \SbPRT{} utilizam breakpoints tipo dois. Note que a transformação de orientação preserva o número de breakpoints tipo dois, já que adicionamos apenas um sinal positivo aos elementos da permutação. A extensão bridge também preserva o número breakpoints tipo dois, já que adiciona apenas três elementos consecutivos ao final da permutação. Por outro lado, cada extensão gadget adiciona dois novos breakpoints tipo dois (ou seja, as extremidades de cada strip negativa), então $b_2(\mathcal{I'}) = b_2(\mathcal{I}) +2p$.

Agora mostramos que a instância $\mathcal{I}$ do problema \textbf{B3T} é satisfeita se e somente se $dp_k(\mathcal{I'}) \le \ell+p$.

($\Rightarrow$) Suponha que existe uma sequência $S$ com $\ell$ transposições, tal que $\pi \cdot S = \iota$. Isso significa que cada transposição de $S$ remove exatamente três breakpoints tipo dois de $\mathcal{I}$. Considere a sequência $S'$ como sendo uma cópia de $S$ e incluindo $p$ reversões, de forma que cada reversão é aplicada sobre uma strip negativa de $\mathcal{I'}$. Como $\pi'_i = +\pi_i$ para $i \in [1..n]$, cada transposição de $S'$ também remove exatamente três breakpoints tipo dois, restando apenas $2p$ breakpoints tipo dois para serem removidos. Contúdo, cada reversão $\rho \in S'$ remove dois breakpoints tipo dois (criados pela extensão gadget). Logo, $|S'| = \ell+p$, $\pi' \cdot S' = \iota'$ e $\frac{|S_{\rho}'|}{|S'|}$.

($\Leftarrow$) Pelo Teorema~\ref{theorem:ABGZQHIL}, temos que $dp_k(\mathcal{I'}) \ge \frac{b_2(\mathcal{I'})}{3-k} = \frac{b_2(\mathcal{I})+2p}{3-k}$. Como temos por construção que $b_2(\mathcal{I}) = 3\ell$ e $\frac{p-1}{\ell+(p-1)} < k \leq \frac{p}{\ell+p}$, segue que $dp_k(\mathcal{I'}) > \frac{(\ell+p-1)(3\ell+2p)}{3\ell+2p-2}$. Além disso, $\ell \geq 1$ e $p \geq 1$, então $\frac{3\ell+2p}{3\ell+2p-2} > 1$ e $dp_k(\mathcal{I'}) > \ell+p-1$, o que resulta em $d_k(\mathcal{I'}) \ge \ell + p$. Suponha que existe uma sequência de eventos de rearranjo $S'$ de tamanho $\ell + p$, tal que $\pi' \cdot S' = \iota'$ e $\frac{|S_{\rho}'|}{|S'|} \ge k$.

Como $b_2(\mathcal{I'}) = 3\ell+2p$, então deve existir pelo menos $\ell$ transposições em $S'$ com cada uma removendo três breakpoints tipo dois. Caso contrário, $S'$ não seria capaz de transformar $\pi'$ em $\iota'$. Além disso, deve exitir no máximo $\ell$ transpositions in $S'$. Caso contrário, a proporção $\frac{|S_{\rho}'|}{|S'|}$ não seria satisfeita. Dessa forma, temos que existe $\ell$ transposições em $S'$ com cada uma removendo três breakpoints tipo dois. Logo, restam $|S'| - \ell = \ell+p - \ell = p$ reversões em $S'$, e cada reversão deve remover dois breakpoints tipo dois. Caso contrário, $S'$ não seria capaz de transformar $\pi'$ em $\iota'$.

Vamos definir três tipos de elementos em $\pi'$. Dizemos que um dado elemento $\pi'_i$ é (i) original se $i \in [1..n]$; (ii) transitório se $i \in [n{+1}..n{+3}]$; e (iii) estendido se $i > n{+3}$. Como os elementos originais e transitórios são todos positivos, as strips nas primeiras $n+3$ posições são todas positivas. Pelo Lemma~\ref{lemma:CWNRJAPM}, nenhuma reversão $\rho$ aplicada nesses elementos remove breakpoints tipo dois, e isto permanece verdadeiro enquanto as transposições afetam apenas os elementos originais.

Como não é aplicada nenhuma reversão aos elementos originais, os $3\ell$ breakpoints tipo dois $(\pi'_i,\pi'_{i+1})$, tal que pelo menos $\pi'_i$ é um elemento original, devem ser removidos por transposições. Dessa forma, $S'$ possui $\ell$ transposições $\tau^{(i,j,k)}$ de tal maneira que $1 \le i < j < k \le n+1$ (ou seja, as transposições afetam apenas os elementos originais). 

Os restantes eventos de rearranjo de $S'$, ou seja, as $p$ reversões, devem remover $2p$ breakpoints tipo dois $(\pi'_i,\pi'_{i+1})$, de tal forma que pelo menos $\pi'_{i+1}$ seja um elemento estendido (ou seja, $i \ge n+3$). A cada iteração, as únicas reversões que removem dois breakpoints tipo dois são aquelas aplicadas nas duas extremidades de uma strip negativa, implicando que cada reversão de $S'$ é aplicada em uma das $p$ strips negativas adicionadas pelas extensões gadget.

Perceba que $S'$ possui $\ell$ transposições que removem $3\ell$ breakpoints tipo dois $(\pi'_i,\pi'_{i+1})$, tal que $i \le n$. Seja $S$ uma sequência de transposições criada a partir das transposições de $S'$ mantendo a mesma ordem relativa. Como $\pi'_i = +\pi_i$ para $i \in [1..n]$, $\pi \cdot S = \iota$, e o teorema segue.
\end{proof}

\begin{example}\label{example:NDFPEMFC}
  Dada a instância clássica sem sinais $\mathcal{I} = ((0 \; 3 \; 5 \; 1 \; 4 \; 2 \; 6),(0 \; 1 \; 2 \; 3 \; 4 \; 5 \; 6))$ para o problema \textbf{B3T}, temos que $b_2(\mathcal{I}) = 6$. Para a criação da instância clássica com sinais $\mathcal{I'}=(\pi',\iota')$ para o problema \SbPRT{} temos no passo 1 a obteção da representação clássica com sinais $\sigma = ({+0} \; {+3} \; {+5} \; {+1} \; {+4} \; {+2} \; {+6} \; {+7} \; {+8} \; {+9})$. Usando $k = 0.3$, temos que $p = \lceil\frac{2\times 0.3}{1 - 0.3}\rceil = \lceil\frac{0.6}{0.7}\rceil = 1$ no passo 2. No passo 3, obtemos a representação clássica com sinais $\pi' = ({+0} \; {+3} \; {+5} \; {+1} \; {+4} \; {+2} \; {+6} \; {+7} \; {+8} \; {-11} \; {-10} \; {-9} \; {+12} \; {+13} \; {+14} \; {+15})$ após aplicar $p = 1$ extenções gadget em $\sigma$. No passo 4, obtemos a representação clássica com sinais $\iota' = ({+0} \; {+1} \; {+2} \; {+3} \; {+4} \; {+5} \; {+6} \; {+7} \; {+8} \; {+9} \; {+10} \; {+11} \; {+12} \; {+13} \; {+14} \; {+15})$. Note que $b_2(\mathcal{I'}) = b_2(\mathcal{I}) + 2p = 6 + 2 = 8$. A sequência $S = (\tau^{(1,3,6)},\tau^{(2,3,5)})$ é tal que $\pi \cdot S = \iota$ e $|S| = 2 = \frac{b_2(\mathcal{I})}{3} = \ell$, e a sequência $S' = (\tau^{(1,3,6)},\tau^{(2,3,5)},\rho^{(9,11)})$ que possui a mesma sequência de transposições de $S$ é tal que (i) $\pi' \cdot S' = \iota'$; (ii) $\frac{|S'_\rho|}{|S'|} = 0.333 \ge 0.3 = k$; e (iii) $|S'| = 3 = \frac{b_2(\mathcal{I})}{3} + 1 = \ell+p$.
\end{example}

\begin{lemma}[Oliveira~\textit{et al.}~\cite{2019b-oliveira-etal}]\label{lemma:PXXMRMWO}
Se uma instância clássica sem sinais $\mathcal{I}=(\pi,\iota)$ possui apenas strips crescentes, para qualquer reversão $\rho$ temos que $\Delta b_1(\mathcal{I}, S=(\rho)) \ge 0$.
\end{lemma}

\begin{theorem}\label{theorem:QMHEKDLW}
O problema \SbPRT{} em instâncias clássicas sem sinais pertence à classe NP-difícil quando $k \in (0..1)$.
\end{theorem}
\begin{proof}
Dada uma instância clássica sem sinais $\mathcal{I}=(\pi,\iota)$ para o problema \textbf{B3T}, definimos $\ell = \frac{b_2(\mathcal{I})}{3} \ge 1$. Criamos uma instância clássica com sinais $\mathcal{I'}=(\pi',\iota')$ para o problema \SbPRT{} da seguinte maneira:

\begin{enumerate}
  \item Seja $\sigma$ uma representação clássica sem sinais de tamanho $2n+3$ obtida através do processo de duplicação aplicado em $\pi$ e seguido da extenção bridge.
  \item Seja $k$ um número racional no intervalo $(0..1)$, definimos $p = \lceil\frac{\ell k}{1-k}\rceil \ge 1$, ou seja, $p$ é o menor número inteiro tal que $\frac{p}{p+\ell} \ge k$.
  \item Seja $\pi'$ uma representação clássica sem sinais de tamanho $2n+3+6p$ obtida através da aplicação consecutiva de $p$ extensões gadget em $\sigma$.
  \item Seja $\iota'$ uma representação clássica com sinais de tamanho $2n+3+6p$. Caso $\pi$ esteja na sua forma estendida, $\iota'_i = i$ para $i \in [1..(n+3+6p)]$. Caso contrário,  $\iota'_i = i$ para $i \in [0..(n+3+6p-1)]$.
\end{enumerate}

O Exemplo~\ref{example:QBULCCOI} mostra o processo de criação de uma instância clássica sem sinais $\mathcal{I'}=(\pi',\iota')$ para o problema \SbPRT{} a partir de uma instância clássica sem sinais $\mathcal{I}=(\pi,\iota)$ para o problema \textbf{B3T}. Note que em ambas as instâncias os genomas de origem e alvo são representados na forma estendida. Note que, exceto por $\sigma_0$, cada elemento em posições pares de $\sigma$ é igual ao elemento à sua esquerda mais um. Isto significa que (i) exceto para a primeira e última strip, qualquer outra strip em $\sigma$ deve ter pelo menos dois elementos, ou seja, não existem singletons, e (ii) cada strip de $\sigma$ é crescente. Estas observações também são válidas para as primeiras $2n+3$ posições de $\pi'$. Além disso, é importante lembrar que o problema \textbf{B3T} utiliza breakpoints tipo dois enquanto a variação sem sinais do problema \SbPRT{} utiliza breakpoints tipo um. Note que (i) para cada breakpoint tipo dois $(\pi_i,\pi_{i+1})$ de $\mathcal{I}$ existe um breakpoint tipo um $(\pi'_{2i},\pi'_{2i+1})$ em $\mathcal{I'}$ (criado durante a transformação de duplicação), (ii) os pares $(\pi'_{2i-1},\pi'_{2i})$ não são breakpoints tipo um, para $i \in [1..n]$ e (iii) os pares $(\pi'_{2n+j},\pi'_{2n+j+1})$ não são breakpoints tipo um, para $j \in [1..3]$. Por outro lado, cada extensão gadget adiciona dois novos breakpoints tipo um (ou seja, as extremidades de cada strip decrescente), então $b_2(\mathcal{I'}) = b_1(\mathcal{I}) +2p$.

Agora mostramos que a instância $\mathcal{I}$ do problema \textbf{B3T} é satisfeita se e somente se $dp_k(\mathcal{I'}) \le \ell+p$.

($\Rightarrow$) Suponha que existe uma sequência $S$ com $\ell$ transposições, tal que $\pi \cdot S = \iota$. Isso significa que cada transposição de $S$ remove exatamente três breakpoints tipo dois de $\mathcal{I}$. Considere a sequência $S'$ criada da seguinte forma: (i) para cada transposição $\tau^{(i,j,k)}$ de $S$, seguindo a ordem relativa, adicione em $S'$ a transposição $\tau^{(2i-1,2j-1,2k-1)}$; (ii) Em seguida, adicione $p$ reversões em $S'$, de forma que cada reversão é aplicada sobre uma strip decrescente de $\mathcal{I'}$. Note que cada transposição de $S$ remove três breakpoints tipo dois de $\mathcal{I}$. Como temos que para cada breakpoint tipo dois $(\pi_i,\pi_{i+1})$ em $\mathcal{I}$ temos um breakpoint tipo um $(\pi'_{2i},\pi'_{2i+1})$ em $\mathcal{I'}$, isso significa que cada transposição de $S'$ remove três breakpoints tipo um de $\mathcal{I'}$. Com isso, restam apenas $2p$ breakpoints tipo um para serem removidos em $\mathcal{I'}$. Contúdo, cada reversão $\rho \in S'$ remove dois breakpoints tipo dois (criados pela extensão gadget). Logo, $|S'| = \ell+p$, $\pi' \cdot S' = \iota'$ e $\frac{|S_{\rho}'|}{|S'|}$.

($\Leftarrow$) Pelo Teorema~\ref{theorem:KJXGKJIP}, temos que $dp_k(\mathcal{I'}) \ge \frac{b_1(\mathcal{I'})}{3-k} = \frac{b_2(\mathcal{I})+2p}{3-k}$. Como temos por construção que $b_2(\mathcal{I}) = 3\ell$ e $\frac{p-1}{\ell+(p-1)} < k \leq \frac{p}{\ell+p}$, segue que $dp_k(\mathcal{I'}) > \frac{(\ell+p-1)(3\ell+2p)}{3\ell+2p-2}$. Além disso, $\ell \geq 1$ e $p \geq 1$, então $\frac{3\ell+2p}{3\ell+2p-2} > 1$ e $dp_k(\mathcal{I'}) > \ell+p-1$, o que resulta em $d_k(\mathcal{I'}) \ge \ell + p$. Suponha que existe uma sequência de eventos de rearranjo $S'$ de tamanho $\ell + p$, tal que $\pi' \cdot S' = \iota'$ e $\frac{|S_{\rho}'|}{|S'|} \ge k$.

Como $b_1(\mathcal{I'}) = 3\ell+2p$, então deve existir pelo menos $\ell$ transposições em $S'$ com cada uma removendo três breakpoints tipo um. Caso contrário, $S'$ não seria capaz de transformar $\pi'$ em $\iota'$. Além disso, deve exitir no máximo $\ell$ transpositions in $S'$. Caso contrário, a proporção $\frac{|S_{\rho}'|}{|S'|}$ não seria satisfeita. Dessa forma, temos que existe $\ell$ transposições em $S'$ com cada uma removendo três breakpoints tipo um. Logo, restam $|S'| - \ell = \ell+p - \ell = p$ reversões em $S'$, e cada reversão deve remover dois breakpoints tipo um. Caso contrário, $S'$ não seria capaz de transformar $\pi'$ em $\iota'$.

Vamos definir três tipos de elementos em $\pi'$. Dizemos que um dado elemento $\pi'_i$ é (i) original se $i \in [1..2n]$; (ii) transitório se $i \in [2n{+1}..2n{+3}]$; e (iii) estendido se $i > 2n{+3}$. Como todos elementos originais e transitórios fazem parte de uma strip crescente, pelo Lemma~\ref{lemma:PXXMRMWO}, nenhuma reversão $\rho$ aplicada nesses elementos remove breakpoints tipo um, e isto permanece verdadeiro enquanto as transposições afetam breakpoints tipo um entre os elementos originais.

Como não é aplicada nenhuma reversão aos elementos originais, os $3\ell$ breakpoints tipo um $(\pi'_i,\pi'_{i+1})$, tal que pelo menos $\pi'_i$ é um elemento original, devem ser removidos por transposições. Dessa forma, $S'$ possui $\ell$ transposições $\tau^{(i,j,k)}$ de tal maneira que $1 \le i < j < k \le 2n+1$ (ou seja, as transposições afetam apenas os elementos originais). 

Os restantes eventos de rearranjo de $S'$, ou seja, as $p$ reversões, devem remover $2p$ breakpoints tipo um $(\pi'_i,\pi'_{i+1})$, de tal forma que pelo menos $\pi'_{i+1}$ seja um elemento estendido (ou seja, $i \ge 2n+3$). A cada iteração, as únicas reversões que removem dois breakpoints tipo um são aquelas aplicadas nas duas extremidades de uma strip decrescente, implicando que cada reversão de $S'$ é aplicada em uma das $p$ strips decrescentes adicionadas pelas extensões gadget.

Perceba que $S'$ possui $\ell$ transposições que removem $3\ell$ breakpoints tipo um $(\pi'_i,\pi'_{i+1})$, tal que $i \le 2n$. Seja $S$ uma sequência de transposições criada a partir das transposições de $S'$ da seguinte forma: (i) mantendo a mesma ordem relativa, para cada transposição $\tau^{(i,j,k)}$ de $S'$ adicione em $S$ a transposição $\tau^{(\frac{i+1}{2},\frac{j+1}{2},\frac{k+1}{2})}$. Como mapeamento feito reflete que cada transposição em $S$ remove três breakpoints tipo dois de $\mathcal{I}$, temos que $\pi \cdot S = \iota$, e o teorema segue.
\end{proof}

\begin{example}\label{example:QBULCCOI}
  Dada a instância clássica sem sinais $\mathcal{I} = ((0 \;1 \; 3 \; 2 \; 4 \; 5),(0 \; 1 \; 2 \; 3 \; 4 \; 5))$ para o problema \textbf{B3T}, temos que $b_2(\mathcal{I}) = 3$. Para a criação da instância clássica sem sinais $\mathcal{I'}=(\pi',\iota')$ para o problema \SbPRT{} temos no passo 1 a obteção da representação clássica sem sinais $\sigma = ({0} \; {1} \; {2} \; {5} \; {6} \; {3} \; {4} \; {7} \; {8} \; {9} \; {10} \; {11} \; {12})$. Usando $k = 0.6$, temos que $p = \lceil\frac{1\times 0.6}{1 - 0.6}\rceil = \lceil\frac{0.6}{0.4}\rceil = 2$ no passo 2. No passo 3, obtemos a representação clássica sem sinais $\pi' = ({0} \; {1} \; {2} \; {5} \; {6} \; {3} \; {4} \; {7} \; {8} \; {9} \; {10} \; {11} \; {14} \; {13} \; {12} \; {15} \; {16} \; {17} \; {20} \; {19} \; {18} \; {21} \; {22} \; {23} \;{24})$ após aplicar $p = 2$ extenções gadget em $\sigma$. No passo 4, obtemos a representação clássica sem sinais $\iota' = ({0} \; {1} \; {2} \; {3} \; {4} \; {5} \; {6} \; {7} \; {8} \; {9} \; {10} \; {11} \; {12} \; {13} \; {14} \; {15} \; {16} \; {17} \; {18} \; {19} \; {20} \; {21} \; {22} \; {23} \; {24})$. Note que $b_1(\mathcal{I'}) = b_2(\mathcal{I}) + 2p = 3 + 4 = 7$. A sequência $S = (\tau^{(2,3,4)})$ é tal que $\pi \cdot S = \iota$ e $|S| = 1 = \frac{b_2(\mathcal{I})}{3} = \ell$, e a sequência $S' = (\tau^{(3,5,7)},\rho^{(12,14)},\rho^{(18,20)})$ que possui a mesma quantidade de transposições de $S$ é tal que (i) $\pi' \cdot S' = \iota'$; (ii) $\frac{|S'_\rho|}{|S'|} = 0.666 \ge 0.6 = k$; e (iii) $|S'| = 3 = \frac{b_2(\mathcal{I})}{3} + 2 = \ell+p$.
\end{example}


% ------------------------------------------------------------------ %
\section{Algoritmos de Aproximação}
% ------------------------------------------------------------------ %

Nessa seção, apresentamos algoritmos de aproximação para as variações com e sem sinais do problema \SbPRT{}.

% ------------------------------------------------------------------ %
\subsection{Instâncias Clássicas sem Sinais}
% ------------------------------------------------------------------ %

Com base no conceito de breakpoint, apresentamos algoritmos de aproximação com fatores de $3-k$ para o problema \SbPRT{} em instâncias clássicas sem sinais.

\begin{lemma}[Kececioglu e Sankoff~\cite{1995-kececioglu-sankoff}]\label{lemma:PZWHPXFL}
Dada uma instância clássica sem sinais $\mathcal{I} = (\pi,\iota)$, é possível transformar $\pi$ em $\iota$ utilizando no máximo $b_1(\mathcal{I})$ reversões.
\end{lemma}

\begin{theorem}\label{theorem:FTRSGXOZ}
Existe um algoritmo de aproximação com fator $3-k$ para o problema \SbPRT{} em instâncias clássicas sem sinais e para uma proporção $k \in [0..1]$.
\end{theorem}
\begin{proof}
Pelo Lema~\ref{lemma:PZWHPXFL}, dada uma instância clássica sem sinais $\mathcal{I} = (\pi,\iota)$, é possível transformar $\pi$ em $\iota$ utilizando no máximo $b_1(\mathcal{I})$ reversões. Como somente reversões são utilizadas na sequência de rearranjo $S$, então a restrição $\frac{|S_{\rho}|}{|S|} \ge k$ nunca é violada. Além disso, pelo Teorema~\ref{theorem:KJXGKJIP}, temos que $dp_{k}(\mathcal{I}) \ge \frac{b_1(\mathcal{I})}{3-k}$. Logo, $\frac{b_1(\mathcal{I})}{\frac{b_1(\mathcal{I})}{3-k}} = 3-k$, e o teorema segue.
\end{proof}

Note que o algoritmo de aproximação resultante do Teorema~\ref{theorem:FTRSGXOZ} utiliza somente reversões. Para evitar que as soluções sejam compostas exclusivamente por reversões, nós propomos o Algoritmo~\ref{algorithm:INBRWKCH}. Esse algoritmo algoritmo também garante um fator de aproximação de $3-k$ para instâncias clássicas sem sinais do problem \SbPRT{} e para qualquer valor de $k$. Além disso, a propoção entre a quantidade de reversões e o tamanho da sequência de eventos de rearranjo fornecida pelo o algoritmo tende a ser um valor próximo de $k$.

\begin{algorithm}[!tbh]
  \caption{Um algoritmo de aproximação para problema \SbPRT{} em instâncias clássicas sem sinais e $k \in [0..1]$.\label{algorithm:INBRWKCH}}
  \Entrada{Uma instância clássica sem sinais $\mathcal{I} = (\pi,\iota)$ e um valor de $k \in [0..1]$}
  \Saida{Uma sequência de reversões e transposições $S$, tal que $\pi \cdot S = \iota$ e $\frac{|S_{\rho}|}{|S|} \ge k$}
    Seja $S \gets ()$ \\
    \Enqto{$\pi \ne \iota$}{
      \Se {$\frac{|S_{\rho}|}{|S| + 1} \ge k$ e existe uma transposição $\tau$ que $\Delta b_1(\mathcal{I}, (\tau)) \le -1$}{
        $\pi \gets \pi \cdot \tau$ \\
        $S \gets S + (\tau)$ \\
      }\Senao{
        Seja $S'$ uma sequência de reversões (de tamanho um ou dois) que remove, na média, um breakpoint tipo um por operação~\cite{1995-kececioglu-sankoff} \\
        $\pi \gets \pi \cdot S'$ \\
        $S \gets S + S'$ \\
      }
    }
  \Retorna{S}
\end{algorithm}

Observe que o Algoritmo~\ref{algorithm:INBRWKCH} aplica uma transposição $\tau$ se duas restrições forem satisfeitas: (i) $\frac{|S_{\rho}|}{|S| + 1} \ge k$, que garante que a sequência de eventos de rearranjo $S$ construída pelo algoritmo obedecerá à restrição do problema que $\frac{|S_{\rho}|}{|S|} \ge k$; e (ii) $\Delta b_1(\mathcal{I}, (\tau)) \le -1$, que garante que a sequência de ordenação conterá no máximo $b_1(\mathcal{I})$ operações, pois cada sequência de reversões adicionada da sequência $S$ remove, em média, um ou mais breakpoints tipo um por operação. Como o Algoritmo~\ref{algorithm:INBRWKCH} remove, em média, um ou mais breakpoints tipo um por iteração, ele garante que $\pi$ será transformada em $\iota$. Além disso, não mais do que $b_1(\mathcal{I})$ operações serão usadas para isso, mantendo o fator de aproximação de $3-k$. Como a transposição $\tau$ (linhas 3-5) e a sequência de no máximo duas reversões $S^{\prime}$ (linhas 6-9) podem ser encontradas em tempo linear, o tempo de execução do Algoritmo~\ref{algorithm:INBRWKCH} é $\mathcal{O}(n^2)$, considerando que $|S|~\le~b_1(\pi)~\le~{n + 1}$.

% ------------------------------------------------------------------ %
\subsection{Instâncias Clássicas com Sinais}
% ------------------------------------------------------------------ %

Com base na estrutura de grafo de ciclos clássico, apresentamos um algoritmo de aproximação com fator de $3-\frac{3k}{2}$ para o problema \SbPRT{} em instâncias clássicas com sinais.

\begin{lemma}\label{lemma:ZUHMXSRH}
Dada uma instância clássica com sinais $\mathcal{I} = (\pi,\iota)$, existe uma sequência de reversões $S$ em que o número de ciclos criados por cada reversão, em média, é maior ou igual a $2/3$.
\end{lemma}
\begin{proof}
Se $G(\mathcal{I})$ possuir um ciclo divergente $C$, então existe uma reversão que quando aplicada em $C$ aumenta o número de ciclos em uma unidade (Teorema 5 de~\cite{1998-walter-etal}). Caso contrário, todos os ciclos não triviais devem ser convergentes. Isso significa que um dos seguintes cenários deve ocorrer obrigatoriamente~\cite{2019a-oliveira-etal}:
\begin{itemize}
  \item Existe em $G(\mathcal{I})$ um ciclo longo e orientado $C$ (Figura~\ref{figure:CQQWYGKH}, Caso 1);
  \item Existe em $G(\mathcal{I})$ um ciclo curto $C$ que os open gates são fechados por outro ciclo não trivial $D$ (Figura~\ref{figure:CQQWYGKH}, Caso 2);
  \item Existe em $G(\mathcal{I})$ um ciclo longo não orientado $C$ que os open gates são fechado por um ou mais ciclos não triviais (Figura~\ref{figure:CQQWYGKH}, Caso 3);
\end{itemize}

Se $G(\mathcal{I})$ possui um ciclo longo e orientado $C$, então podemos aplicar uma reversão em suas arestas pretas de maneira que $C$ é transformado em divergente. Como $C$ é um ciclo longo, então é possivel aplicar, pelo menos, duas reversões de forma que cada uma aumenta o número de ciclos em uma unidade (Figura~\ref{figure:CQQWYGKH}, Caso 1).

Se algum dos outros casos ocorrer, então podemos tornar o ciclo $C$ em divergente após aplicar uma reversão no(s) ciclo(s) que fecham os open gates de $C$. Se $C$ for um ciclo curto, então podemos aplicar uma reversão em suas arestas pretas quebrando-o em dois ciclos triviais, o que aumenta o número de ciclos em uma unidade. Como resultado da segunda reversão, o ciclo $D$ também passa a ser divergente, o que nos garante aplicar uma terceira reversão que aumenta o número de ciclos em uma unidade (Figura~\ref{figure:CQQWYGKH}, Caso 2). Se $C$ for um ciclo curto, então é possivel aplicar, pelo menos, duas reversões de forma que cada uma aumenta o número de ciclos em uma unidade (Figura~\ref{figure:CQQWYGKH}, Caso 3)

Nos três casos mencionados acima, aplicamos três reversões que aumentam em pelo menos duas unidades o número de ciclos, e o lema segue.
\end{proof}

\begin{figure}[!tbh]
\centering
\begin{tikzpicture}[scale=0.7]
\tiny
\begin{scope}[every node/.style={inner sep=0pt, minimum size = 0pt}]
    \node[label=below:\phantom{+}] (p0) at (0,0) {};
    \node[label=below:\phantom{+}] (m2) at (1,0) {};
    \node[label=below:\phantom{+}] (p2) at (2,0) {};
    \node[label=below:\phantom{+}] (m1) at (3,0) {};
    \node[label=below:\phantom{+}] (p1) at (4,0) {};
    \node[label=below:\phantom{+}] (m3) at (5,0) {};
\end{scope}

\begin{scope}[>={Stealth[black]},
              every edge/.style={draw=black}]
    \path [-] (p0) edge node [black, pos=0.5, sloped, above] {$1$} (m2);
    \path [-] (p2) edge node [black, pos=0.5, sloped, above] {$2$} (m1);
    \path [-] (p1) edge node [black, pos=0.5, sloped, above] {$3$} (m3);
\end{scope}

\begin{scope}[>={Stealth[black]},
              every edge/.style={draw=black}]
    \path [-] (p0) edge  [bend left=60] (m1);
    \path [-] (p1) edge  [bend right=60] (m2);
    \path [-] (p2) edge  [bend left=60] (m3);
\end{scope}

\begin{scope}[>={Stealth[black]},
              every edge/.style={draw=black}, every node/.style={inner sep=0pt, minimum size = 0pt}]
\node[label=\phantom{}] (bi1) at (2.5, -0.5) {};
\node[label=\phantom{}] (bi2) at (4.5, -0.5) {};
\path [{Bar}-{Bar}] (bi1) edge node [black, pos=0.5, sloped, below] {} (bi2);
\end{scope}

\begin{scope}[every node/.style={inner sep=0pt, minimum size = 0pt}]
    \node[label=below:\phantom{+}] (2p0) at (0,-2) {};
    \node[label=below:\phantom{+}] (2m2) at (1,-2) {};
    \node[label=below:\phantom{+}] (2p2) at (2,-2) {};
    \node[label=below:\phantom{+}] (2p1) at (3,-2) {};
    \node[label=below:\phantom{+}] (2m1) at (4,-2) {};
    \node[label=below:\phantom{+}] (2m3) at (5,-2) {};
\end{scope}

\begin{scope}[>={Stealth[black]},
              every edge/.style={draw=black}]
    \path [-] (2p0) edge node [black, pos=0.6, sloped, above] {${-1}$} (2m2);
    \path [-] (2p2) edge node [black, pos=0.5, sloped, above] {$2$} (2p1);
    \path [-] (2m1) edge node [black, pos=0.5, sloped, above] {$3$} (2m3);
\end{scope}

\begin{scope}[>={Stealth[black]},
              every edge/.style={draw=black}]
    \path [-] (2p0) edge  [bend left=60] (2m1);
    \path [-] (2p1) edge  [bend right=80] (2m2);
    \path [-] (2p2) edge  [bend left=70] (2m3);
\end{scope}

\begin{scope}[>={Stealth[black]},
              every edge/.style={draw=black}, every node/.style={inner sep=0pt, minimum size = 0pt}]
\node[label=\phantom{}] (bi1) at (0.5, -2.5) {};
\node[label=\phantom{}] (bi2) at (4.5, -2.5) {};
\path [{Bar}-{Bar}] (bi1) edge node [black, pos=0.5, sloped, below] {} (bi2);
\end{scope}


\begin{scope}[every node/.style={inner sep=0pt, minimum size = 0pt}]
    \node[label=below:\phantom{+}] (2p0) at (0,-4) {};
    \node[label=below:\phantom{+}] (2m1) at (1,-4) {};
    \node[label=below:\phantom{+}] (2p1) at (2,-4) {};
    \node[label=below:\phantom{+}] (2p2) at (3,-4) {};
    \node[label=below:\phantom{+}] (2m2) at (4,-4) {};
    \node[label=below:\phantom{+}] (2m3) at (5,-4) {};
\end{scope}

\begin{scope}[>={Stealth[black]},
              every edge/.style={draw=black}]
    \path [-] (2p0) edge node [black, pos=0.5, sloped, above] {$1$} (2m1);
    \path [-] (2p1) edge node [black, pos=0.5, sloped, above] {$-2$} (2p2);
    \path [-] (2m2) edge node [black, pos=0.5, sloped, above] {$3$} (2m3);
\end{scope}

\begin{scope}[>={Stealth[black]},
              every edge/.style={draw=black}]
    \path [-] (2p0) edge  [bend left=80, looseness=2] (2m1);
    \path [-] (2p1) edge  [bend left=80] (2m2);
    \path [-] (2p2) edge  [bend left=80] (2m3);
\end{scope}

\begin{scope}[>={Stealth[black]},
              every edge/.style={draw=black}, every node/.style={inner sep=0pt, minimum size = 0pt}]
\node[label=\phantom{}] (bi1) at (2.5, -4.5) {};
\node[label=\phantom{}] (bi2) at (4.5, -4.5) {};
\path [{Bar}-{Bar}] (bi1) edge node [black, pos=0.5, sloped, below] {} (bi2);
\end{scope}

\begin{scope}[every node/.style={inner sep=0pt, minimum size = 0pt}]
    \node[label=below:\phantom{+}] (2p0) at (0,-6) {};
    \node[label=below:\phantom{+}] (2m1) at (1,-6) {};
    \node[label=below:\phantom{+}] (2p1) at (2,-6) {};
    \node[label=below:\phantom{+}] (2m2) at (3,-6) {};
    \node[label=below:\phantom{+}] (2p2) at (4,-6) {};
    \node[label=below:\phantom{+}] (2m3) at (5,-6) {};
\end{scope}

\begin{scope}[>={Stealth[black]},
              every edge/.style={draw=black}]
    \path [-] (2p0) edge node [black, pos=0.5, sloped, above] {$1$} (2m1);
    \path [-] (2p1) edge node [black, pos=0.5, sloped, above] {$2$} (2m2);
    \path [-] (2p2) edge node [black, pos=0.5, sloped, above] {$3$} (2m3);
\end{scope}

\begin{scope}[>={Stealth[black]},
              every edge/.style={draw=black}]
    \path [-] (2p0) edge  [bend left=80, looseness=2] (2m1);
    \path [-] (2p1) edge  [bend left=80, looseness=2] (2m2);
    \path [-] (2p2) edge  [bend left=80, looseness=2] (2m3);
\end{scope}

\begin{scope}[>={Stealth[black]},
              every edge/.style={draw=black}, every node/.style={inner sep=0pt, minimum size = 0pt}]
\node[label={\small Caso 1}] (bi1) at (2.5, -7) {};
\end{scope}

\end{tikzpicture}
%
\qquad
%
\begin{tikzpicture}[scale=0.5]
\tiny
\begin{scope}[every node/.style={inner sep=0pt, minimum size = 0pt}]
    \node[label=below:\phantom{+}] (p0) at (0,0) {};
    \node[label=below:\phantom{+}] (m4) at (1,0) {};
    \node[label=below:\phantom{+}] (p4) at (2,0) {};
    \node[label=below:\phantom{+}] (m2) at (3,0) {};
    \node[label=below:\phantom{+}] (p3) at (4,0) {};
    \node[label=below:\phantom{+}] (m1) at (5,0) {};
    \node[label=below:\phantom{+}] (p1) at (6,0) {};
    \node[label=below:\phantom{+}] (m5) at (7,0) {};
\end{scope}

\begin{scope}[>={Stealth[black]},
              every edge/.style={draw=black}]
    \path [-] (p0) edge node [black, pos=0.5, sloped, above] {$1$} (m4);
    \path [-] (p4) edge node [black, pos=0.5, sloped, above] {$2$} (m2);
    \path [-] (p3) edge node [black, pos=0.5, sloped, above] {$3$} (m1);
    \path [-] (p1) edge node [black, pos=0.5, sloped, above] {$4$} (m5);
\end{scope}

\begin{scope}[>={Stealth[black]},
              every edge/.style={draw=black}]
    \path [-] (p0) edge  [bend left=60] (m1);
    \path [-] (p1) edge  [bend right=60] (m2);
    \path [-] (p3) edge  [bend right=60] (m4);
    \path [-] (p4) edge  [bend left=60] (m5);
\end{scope}

\begin{scope}[>={Stealth[black]},
              every edge/.style={draw=black}, every node/.style={inner sep=0pt, minimum size = 0pt}]
\node[label=\phantom{}] (bi1) at (2.5, -0.5) {};
\node[label=\phantom{}] (bi2) at (6.5, -0.5) {};
\path [{Bar}-{Bar}] (bi1) edge node [black, pos=0.5, sloped, below] {} (bi2);
\end{scope}

\begin{scope}[every node/.style={inner sep=0pt, minimum size = 0pt}]
    \node[label=below:\phantom{+}] (p0) at (0,-3) {};
    \node[label=below:\phantom{+}] (m4) at (1,-3) {};
    \node[label=below:\phantom{+}] (p4) at (2,-3) {};
    \node[label=below:\phantom{+}] (p1) at (3,-3) {};
    \node[label=below:\phantom{+}] (m1) at (4,-3) {};
    \node[label=below:\phantom{+}] (p3) at (5,-3) {};
    \node[label=below:\phantom{+}] (m2) at (6,-3) {};
    \node[label=below:\phantom{+}] (m5) at (7,-3) {};
\end{scope}

\begin{scope}[>={Stealth[black]},
              every edge/.style={draw=black}]
    \path [-] (p0) edge node [black, pos=0.6, sloped, above] {$-1$} (m4);
    \path [-] (p4) edge node [black, pos=0.5, sloped, above] {$2$} (p1);
    \path [-] (m1) edge node [black, pos=0.5, sloped, above] {$3$} (p3);
    \path [-] (m2) edge node [black, pos=0.5, sloped, above] {$4$} (m5);
\end{scope}

\begin{scope}[>={Stealth[black]},
              every edge/.style={draw=black}]
    \path [-] (p0) edge  [bend left=60] (m1);
    \path [-] (p1) edge  [bend left=60] (m2);
    \path [-] (p3) edge  [bend right=60] (m4);
    \path [-] (p4) edge  [bend left=60] (m5);
\end{scope}

\begin{scope}[>={Stealth[black]},
              every edge/.style={draw=black}, every node/.style={inner sep=0pt, minimum size = 0pt}]
\node[label=\phantom{}] (bi1) at (0.5, -3.5) {};
\node[label=\phantom{}] (bi2) at (4.5, -3.5) {};
\path [{Bar}-{Bar}] (bi1) edge node [black, pos=0.5, sloped, below] {} (bi2);
\end{scope}

\begin{scope}[every node/.style={inner sep=0pt, minimum size = 0pt}]
    \node[label=below:\phantom{+}] (p0) at (0,-6) {};
    \node[label=below:\phantom{+}] (m1) at (1,-6) {};
    \node[label=below:\phantom{+}] (p1) at (2,-6) {};
    \node[label=below:\phantom{+}] (p4) at (3,-6) {};
    \node[label=below:\phantom{+}] (m4) at (4,-6) {};
    \node[label=below:\phantom{+}] (p3) at (5,-6) {};
    \node[label=below:\phantom{+}] (m2) at (6,-6) {};
    \node[label=below:\phantom{+}] (m5) at (7,-6) {};
\end{scope}

\begin{scope}[>={Stealth[black]},
              every edge/.style={draw=black}]
    \path [-] (p0) edge node [black, pos=0.5, sloped, above] {$1$} (m1);
    \path [-] (p1) edge node [black, pos=0.6, sloped, above] {$-2$} (p4);
    \path [-] (m4) edge node [black, pos=0.5, sloped, above] {$3$} (p3);
    \path [-] (m2) edge node [black, pos=0.5, sloped, above] {$4$} (m5);
\end{scope}

\begin{scope}[>={Stealth[black]},
              every edge/.style={draw=black}]
    \path [-] (p0) edge  [bend left=90, looseness=2] (m1);
    \path [-] (p1) edge  [bend left=60] (m2);
    \path [-] (p3) edge  [bend right=90, looseness=2] (m4);
    \path [-] (p4) edge  [bend left=60] (m5);
\end{scope}

\begin{scope}[>={Stealth[black]},
              every edge/.style={draw=black}, every node/.style={inner sep=0pt, minimum size = 0pt}]
\node[label=\phantom{}] (bi1) at (2.5, -6.5) {};
\node[label=\phantom{}] (bi2) at (6.5, -6.5) {};
\path [{Bar}-{Bar}] (bi1) edge node [black, pos=0.5, sloped, below] {} (bi2);
\end{scope}


\begin{scope}[every node/.style={inner sep=0pt, minimum size = 0pt}]
    \node[label=below:\phantom{+}] (p0) at (0,-8.5) {};
    \node[label=below:\phantom{+}] (m1) at (1,-8.5) {};
    \node[label=below:\phantom{+}] (p1) at (2,-8.5) {};
    \node[label=below:\phantom{+}] (m2) at (3,-8.5) {};
    \node[label=below:\phantom{+}] (p3) at (4,-8.5) {};
    \node[label=below:\phantom{+}] (m4) at (5,-8.5) {};
    \node[label=below:\phantom{+}] (p4) at (6,-8.5) {};
    \node[label=below:\phantom{+}] (m5) at (7,-8.5) {};
\end{scope}

\begin{scope}[>={Stealth[black]},
              every edge/.style={draw=black}]
    \path [-] (p0) edge node [black, pos=0.5, sloped, above] {$1$} (m1);
    \path [-] (p1) edge node [black, pos=0.6, sloped, above] {$2$} (m2);
    \path [-] (p3) edge node [black, pos=0.5, sloped, above] {$3$} (m4);
    \path [-] (p4) edge node [black, pos=0.5, sloped, above] {$4$} (m5);
\end{scope}

\begin{scope}[>={Stealth[black]},
              every edge/.style={draw=black}]
    \path [-] (p0) edge  [bend left=90, looseness=2] (m1);
    \path [-] (p1) edge  [bend left=90, looseness=2] (m2);
    \path [-] (p3) edge  [bend left=90, looseness=2] (m4);
    \path [-] (p4) edge  [bend left=90, looseness=2] (m5);
\end{scope}

\begin{scope}[>={Stealth[black]},
              every edge/.style={draw=black}, every node/.style={inner sep=0pt, minimum size = 0pt}]
\node[label={\small Caso 2}] (bi1) at (3.5, -9.5) {};
\end{scope}
\end{tikzpicture}

%
\qquad
%
\begin{tikzpicture}[scale=0.5]
\tiny
\begin{scope}[every node/.style={inner sep=0pt, minimum size = 0pt}]
    \node[label=below:\phantom{+}] (p0) at (0,0) {};
    \node[label=below:\phantom{+}] (m3) at (1,0) {};
    \node[label=below:\phantom{+}] (p3) at (2,0) {};
    \node[label=below:\phantom{+}] (m2) at (3,0) {};
    \node[label=below:\phantom{+}] (p2) at (4,0) {};
    \node[label=below:\phantom{+}] (m1) at (5,0) {};
    \node[label=below:\phantom{+}] (p1) at (6,0) {};
    \node[label=below:\phantom{+}] (m6) at (7,0) {};
    \node[label=below:\phantom{+}] (p6) at (8,0) {};
    \node[label=below:\phantom{+}] (m5) at (9,0) {};
    \node[label=below:\phantom{+}] (p5) at (10,0) {};
    \node[label=below:\phantom{+}] (m4) at (11,0) {};
    \node[label=below:\phantom{+}] (p4) at (12,0) {};
    \node[label=below:\phantom{+}] (m7) at (13,0) {};
\end{scope}

\begin{scope}[>={Stealth[black]},
              every edge/.style={draw=black}]
    \path [-] (p0) edge  [bend left=60] (m1);
    \path [-] (p1) edge  [bend right=60] (m2);
    \path [-] (p2) edge  [bend right=60] (m3);
    \path [-] (p3) edge  [bend left=60] (m4);
    \path [-] (p4) edge  [bend right=60] (m5);
    \path [-] (p5) edge  [bend right=60] (m6);
    \path [-] (p6) edge  [bend left=60] (m7);
\end{scope}

\begin{scope}[>={Stealth[black]},
              every edge/.style={draw=black}]
    \path [-] (p0) edge node [black, pos=0.5, sloped, above] {$1$} (m3);
    \path [-] (p3) edge node [black, pos=0.5, sloped, above] {$2$} (m2);
    \path [-] (p2) edge node [black, pos=0.5, sloped, above] {$3$} (m1);
    \path [-] (p1) edge node [black, pos=0.5, sloped, above] {$4$} (m6);
    \path [-] (p6) edge node [black, pos=0.5, sloped, above] {$5$} (m5);
    \path [-] (p5) edge node [black, pos=0.5, sloped, above] {$6$} (m4);
    \path [-] (p4) edge node [black, pos=0.5, sloped, above] {$7$} (m7);
\end{scope}
\begin{scope}[>={Stealth[black]},
              every edge/.style={draw=black}, every node/.style={inner sep=0pt, minimum size = 0pt}]
\node[label=\phantom{}] (bi1) at (0.5, -0.5) {};
\node[label=\phantom{}] (bi2) at (4.5, -0.5) {};
\path [{Bar}-{Bar}] (bi1) edge node [black, pos=0.5, sloped, below] {} (bi2);
\end{scope}


\begin{scope}[every node/.style={inner sep=0pt, minimum size = 0pt}]
    \node[label=below:\phantom{+}] (p0) at (0,-2.5) {};
    \node[label=below:\phantom{+}] (p2) at (1,-2.5) {};
    \node[label=below:\phantom{+}] (m2) at (2,-2.5) {};
    \node[label=below:\phantom{+}] (p3) at (3,-2.5) {};
    \node[label=below:\phantom{+}] (m3) at (4,-2.5) {};
    \node[label=below:\phantom{+}] (m1) at (5,-2.5) {};
    \node[label=below:\phantom{+}] (p1) at (6,-2.5) {};
    \node[label=below:\phantom{+}] (m6) at (7,-2.5) {};
    \node[label=below:\phantom{+}] (p6) at (8,-2.5) {};
    \node[label=below:\phantom{+}] (m5) at (9,-2.5) {};
    \node[label=below:\phantom{+}] (p5) at (10,-2.5) {};
    \node[label=below:\phantom{+}] (m4) at (11,-2.5) {};
    \node[label=below:\phantom{+}] (p4) at (12,-2.5) {};
    \node[label=below:\phantom{+}] (m7) at (13,-2.5) {};
\end{scope}

\begin{scope}[>={Stealth[black]},
              every edge/.style={draw=black}]
    \path [-] (p0) edge  [bend left=60] (m1);
    \path [-] (p1) edge  [bend right=60] (m2);
    \path [-] (p2) edge  [bend left=60] (m3);
    \path [-] (p3) edge  [bend left=50] (m4);
    \path [-] (p4) edge  [bend right=60] (m5);
    \path [-] (p5) edge  [bend right=60] (m6);
    \path [-] (p6) edge  [bend left=60] (m7);
\end{scope}

\begin{scope}[>={Stealth[black]},
              every edge/.style={draw=black}]
    \path [-] (p0) edge node [black, pos=0.5, sloped, above] {$1$} (p2);
    \path [-] (m2) edge node [black, pos=0.6, sloped, above] {$-2$} (p3);
    \path [-] (m3) edge node [black, pos=0.5, sloped, above] {$3$} (m1);
    \path [-] (p1) edge node [black, pos=0.5, sloped, above] {$4$} (m6);
    \path [-] (p6) edge node [black, pos=0.5, sloped, above] {$5$} (m5);
    \path [-] (p5) edge node [black, pos=0.5, sloped, above] {$6$} (m4);
    \path [-] (p4) edge node [black, pos=0.5, sloped, above] {$7$} (m7);
\end{scope}

\begin{scope}[>={Stealth[black]},
              every edge/.style={draw=black}, every node/.style={inner sep=0pt, minimum size = 0pt}]
\node[label=\phantom{}] (bi1) at (2.5, -3) {};
\node[label=\phantom{}] (bi2) at (6.5, -3) {};
\path [{Bar}-{Bar}] (bi1) edge node [black, pos=0.5, sloped, below] {} (bi2);
\end{scope}

\begin{scope}[every node/.style={inner sep=0pt, minimum size = 0pt}]
    \node[label=below:\phantom{+}] (p0) at (0,-5) {};
    \node[label=below:\phantom{+}] (p2) at (1,-5) {};
    \node[label=below:\phantom{+}] (m2) at (2,-5) {};
    \node[label=below:\phantom{+}] (p1) at (3,-5) {};
    \node[label=below:\phantom{+}] (m1) at (4,-5) {};
    \node[label=below:\phantom{+}] (m3) at (5,-5) {};
    \node[label=below:\phantom{+}] (p3) at (6,-5) {};
    \node[label=below:\phantom{+}] (m6) at (7,-5) {};
    \node[label=below:\phantom{+}] (p6) at (8,-5) {};
    \node[label=below:\phantom{+}] (m5) at (9,-5) {};
    \node[label=below:\phantom{+}] (p5) at (10,-5) {};
    \node[label=below:\phantom{+}] (m4) at (11,-5) {};
    \node[label=below:\phantom{+}] (p4) at (12,-5) {};
    \node[label=below:\phantom{+}] (m7) at (13,-5) {};
\end{scope}

\begin{scope}[>={Stealth[black]},
              every edge/.style={draw=black}]
    \path [-] (p0) edge  [bend left=60] (m1);
    \path [-] (p1) edge  [bend right=90, looseness=2] (m2);
    \path [-] (p2) edge  [bend left=60] (m3);
    \path [-] (p3) edge  [bend left=50] (m4);
    \path [-] (p4) edge  [bend right=60] (m5);
    \path [-] (p5) edge  [bend right=60] (m6);
    \path [-] (p6) edge  [bend left=60] (m7);
\end{scope}

\begin{scope}[>={Stealth[black]},
              every edge/.style={draw=black}]
    \path [-] (p0) edge node [black, pos=0.6, sloped, above] {$-1$} (p2);
    \path [-] (m2) edge node [black, pos=0.5, sloped, above] {$2$} (p1);
    \path [-] (m1) edge node [black, pos=0.5, sloped, above] {$3$} (m3);
    \path [-] (p3) edge node [black, pos=0.5, sloped, above] {$4$} (m6);
    \path [-] (p6) edge node [black, pos=0.5, sloped, above] {$5$} (m5);
    \path [-] (p5) edge node [black, pos=0.5, sloped, above] {$6$} (m4);
    \path [-] (p4) edge node [black, pos=0.5, sloped, above] {$7$} (m7);
\end{scope}

\begin{scope}[>={Stealth[black]},
              every edge/.style={draw=black}, every node/.style={inner sep=0pt, minimum size = 0pt}]
\node[label=\phantom{}] (bi1) at (0.5, -5.5) {};
\node[label=\phantom{}] (bi2) at (4.5, -5.5) {};
\path [{Bar}-{Bar}] (bi1) edge node [black, pos=0.5, sloped, below] {} (bi2);
\end{scope}

\begin{scope}[every node/.style={inner sep=0pt, minimum size = 0pt}]
    \node[label=below:\phantom{+}] (p0) at (0,-7.5) {};
    \node[label=below:\phantom{+}] (m1) at (1,-7.5) {};
    \node[label=below:\phantom{+}] (p1) at (2,-7.5) {};
    \node[label=below:\phantom{+}] (m2) at (3,-7.5) {};
    \node[label=below:\phantom{+}] (p2) at (4,-7.5) {};
    \node[label=below:\phantom{+}] (m3) at (5,-7.5) {};
    \node[label=below:\phantom{+}] (p3) at (6,-7.5) {};
    \node[label=below:\phantom{+}] (m6) at (7,-7.5) {};
    \node[label=below:\phantom{+}] (p6) at (8,-7.5) {};
    \node[label=below:\phantom{+}] (m5) at (9,-7.5) {};
    \node[label=below:\phantom{+}] (p5) at (10,-7.5) {};
    \node[label=below:\phantom{+}] (m4) at (11,-7.5) {};
    \node[label=below:\phantom{+}] (p4) at (12,-7.5) {};
    \node[label=below:\phantom{+}] (m7) at (13,-7.5) {};
\end{scope}

\begin{scope}[>={Stealth[black]},
              every edge/.style={draw=black}]
    \path [-] (p0) edge  [bend left=90, looseness=2] (m1);
    \path [-] (p1) edge  [bend left=90, looseness=2] (m2);
    \path [-] (p2) edge  [bend left=90, looseness=2] (m3);
    \path [-] (p3) edge  [bend left=50] (m4);
    \path [-] (p4) edge  [bend right=60] (m5);
    \path [-] (p5) edge  [bend right=60] (m6);
    \path [-] (p6) edge  [bend left=60] (m7);
\end{scope}

\begin{scope}[>={Stealth[black]},
              every edge/.style={draw=black}]
    \path [-] (p0) edge node [black, pos=0.5, sloped, above] {$1$} (m1);
    \path [-] (p1) edge node [black, pos=0.5, sloped, above] {$2$} (m2);
    \path [-] (p2) edge node [black, pos=0.5, sloped, above] {$3$} (m3);
    \path [-] (p3) edge node [black, pos=0.5, sloped, above] {$4$} (m6);
    \path [-] (p6) edge node [black, pos=0.5, sloped, above] {$5$} (m5);
    \path [-] (p5) edge node [black, pos=0.5, sloped, above] {$6$} (m4);
    \path [-] (p4) edge node [black, pos=0.5, sloped, above] {$7$} (m7);
\end{scope}
\begin{scope}[>={Stealth[black]},
              every edge/.style={draw=black}, every node/.style={inner sep=0pt, minimum size = 0pt}]
\node[label={\small Caso 3}] (bi1) at (6.5, -8.5) {};
\end{scope}
\end{tikzpicture}
\caption[Configurações e respectivas sequências de reversões.]{Configurações e respectivas sequências de reversões aplicadas em cada um dos casos do Lema~\ref{lemma:KZHQWACX}. Indicamos, para cada um dos casos, o par de arestas pretas afetadas por cada reversão. No Caso 1, o ciclo $C=(3,1,2)$ é longo e orientado. No Caso 2, o ciclo curto $C_1=(3,1)$ tem os open gates fechados pelo ciclo $C_2=(4,2)$. Por fim, no Caso 3, o ciclo longo não orientado $C_2=(6,4,2)$ tem os open gates fechados pelos ciclos $C_1=(3,1)$ e $C_3=(7,5)$. Para os três casos, é mostrado uma sequência de três reversões que aumenta o número de ciclos em duas unidades.}
\label{figure:CQQWYGKH}
\end{figure}

\begin{lemma}\label{lemma:KZHQWACX}
Dada uma instância clássica com sinais $\mathcal{I} = (\pi,\iota)$, é possível transforma $\pi$ em $\iota$ utilizando no máximo $\frac{3(n + 1 -c(G(\mathcal{I})))}{2}$ reversões.
\end{lemma}
\begin{proof}
O Lema~\ref{lemma:ZUHMXSRH} resulta em uma sequência de reversões que sempre aumenta o número de ciclos. Logo, podemos aplicarmos o Lema~\ref{lemma:ZUHMXSRH} até que $c(G(\mathcal{I}))$ seja igual a ${n+1}$. Consequentemente, $\pi$ será transformada em $\iota$. Além disso, cada sequência de reversões $S$ obtidas através do Lema~\ref{lemma:ZUHMXSRH} garante que o número de ciclos criados por cada reversão de $S$, em média, é maior ou igual a $2/3$. Logo, não mais do que $\frac{3(n + 1 -c(G(\mathcal{I})))}{2}$ reversões são utilizadas para transformar $\pi$ em $\iota$, e o lema segue.
\end{proof}

\begin{theorem}\label{theorem:HYRUPXPH}
Existe um algoritmo de aproximação com fator $3-\frac{3k}{2}$ para o problema \SbPRT{} em instâncias clássicas com sinais e para uma proporção $k \in [0..1]$.
\end{theorem}
\begin{proof}
Pelo Lema~\ref{lemma:KZHQWACX}, dada uma instância clássica com sinais $\mathcal{I} = (\pi,\iota)$, é possível transformar $\pi$ em $\iota$ utilizando no máximo $\frac{3(n + 1 -c(G(\mathcal{I})))}{2}$ reversões. Como somente reversões são utilizadas na sequência de rearranjo $S$, então a restrição $\frac{|S_{\rho}|}{|S|} \ge k$ nunca é violada. Além disso, pelo Teorema~\ref{theorem:WSTPPSMD}, temos que $dp_{k}(\mathcal{I}) \ge \frac{n + 1 - c(G(\mathcal{I}))}{2-k}$. Logo,
$$\frac{\frac{3(n + 1 - c(G(\mathcal{I})))}{2}}{\frac{n + 1 - c(G(\mathcal{I}))}{2-k}} = 3-\frac{3k}{2}.$$
\end{proof}


Note que o algoritmo de aproximação resultante do Teorema~\ref{theorem:HYRUPXPH} aplica somente reversões. Para evitar que as soluções sejam compostas exclusivamente por reversões, nós propomos o Algoritmo~\ref{algorithm:TNMWZNZU}. Esse algoritmo algoritmo também garante um fator de aproximação de $3-\frac{3k}{2}$ para instâncias clássicas com sinais do problem \SbPRT{} e para qualquer valor de $k$.

\begin{algorithm}[!tbh]
  \caption{Um algoritmo de aproximação para o problema \SbPRT{} em instâncias clássicas com sinais e $k \in [0..1]$.\label{algorithm:TNMWZNZU}}
  \Entrada{Uma instância clássica com sinais $\mathcal{I} = (\pi,\iota)$ e um valor de $k \in [0..1]$}
  \Saida{Uma sequência de reversões e transposições $S$, tal que $\pi \cdot S = \iota$ e $\frac{|S_{\rho}|}{|S|} \ge k$}
    Seja $S \gets ()$ \\
    \Enqto{$\pi \ne \iota$}{
      \Se {$\frac{|S_{\rho}|}{|S| + 1} \ge k$ e existe uma transposição $\tau$ tal que $\Delta c(G(\mathcal{I}), (\tau)) \ge 1$}{
        $\pi \gets \pi \cdot \tau$ \\
        $S \gets S + (\tau)$ \\
      }\Senao{
        Seja $S'$ uma sequência de reversões (de tamanho no máximo três), onde cada operação aumenta, em média, o número de ciclos em pelo menos $2/3$ unidades (Lema~\ref{lemma:ZUHMXSRH}) \\
        $\pi \gets \pi \cdot S'$ \\
        $S \gets S + S'$ \\
      }
    }
  \Retorna{S}
\end{algorithm}

Observe que o Algoritmo~\ref{algorithm:TNMWZNZU} aplica uma transposição $\tau$ se duas restrições forem satisfeitas: (i) $\frac{|S_{\rho}|}{|S| + 1} \ge k$, que garante que a sequência de eventos de rearranjo $S$ construída pelo algoritmo obedecerá à restrição do problema que $\frac{|S_{\rho}|}{|S|} \ge k$; e (ii) $\Delta c(G(\mathcal{I}), (\tau)) \ge 1$, que garante que a sequência de ordenação conterá no máximo $\frac{3(n + 1 -c(G(\mathcal{I})))}{2}$ operações, pois cada sequência de reversões adicionada da sequência $S$ aumenta, em média, o número de ciclos em pelo menos $2/3$ unidades. Dessa forma, o algoritmo garante o fator de aproximação de $3-\frac{3k}{2}$. A transposição $\tau$ (linhas 3-5) pode ser encontrada em tempo linear, já a sequência de no máximo três reversões $S^{\prime}$ (linhas 6-9) pode ser encontradas em tempo $\mathcal{O}(n^2)$. Com isso, o tempo de execução do Algoritmo~\ref{algorithm:TNMWZNZU} é $\mathcal{O}(n^3)$, considerando que $|S| \le \frac{3(n + 1 -c(G(\mathcal{I})))}{2} \le \frac{3({n+1})}{2}$.

% ------------------------------------------------------------------ %
\subsubsection{Algoritmo de Aproximação Assintótica}
% ------------------------------------------------------------------ %

Nessa seção, apresentamos um algoritmo de aproximação assintótica com fator de $\frac{2-k}{1-k/3}$ para o problema \SbPRT{} em instâncias clássicas com sinais.

\begin{definition}
Dada uma instância clássica com sinais $\mathcal{I} = (\pi,\iota)$, seja $\mathcal{A}_\rho$ o algoritmo descrito no Teorema~\ref{theorem:HYRUPXPH} que transforma $\pi$ em $\iota$ utilizando no máximo $\frac{3(n + 1 -c(G(\mathcal{I})))}{2}$ reversões. Denotamos por $\mathcal{A}_\rho(\mathcal{I})$ a sequência de reversões obtidas através de $\mathcal{A}_\rho$ e que transforma $\pi$ em $\iota$.
\end{definition}

\begin{remark}[Oliveira \textit{et al.}~\cite{2019a-oliveira-etal}]\label{remark:DNLEDNKT}
Dada uma representação clássica $\pi$, uma transposição $\tau^{(i,j,k)}$ pode ser reproduzida por uma sequência de três reversões consecutivas $S=(\rho^{(i,{j-1})},\rho^{(j,{k-1})},\rho^{(i,{k-1})})$, ou seja, $\pi \cdot \tau^{(i,j,k)} = \pi \cdot S$. 
\end{remark}

Agora considere o Algoritmo~\ref{algorithm:NTMUPIXY}. Note que podemos fazer uma análise considerando quatro sub-rotinas: i)  executar o algoritmo  $\mathcal{A}_\rho$ (linhas 2 e 12, tempo de execução $\mathcal{O}(n^3)$), ii) encontrar um ciclo divergente em $G(\mathcal{I})$ e determinar os parâmetros da reversão $\rho$ que aumenta o número de ciclos em uma unidade (linhas 3-6, em tempo linear), iii) determinar uma sequência de no máximo duas transposições que aumenta o número de ciclos em duas unidades (linhas 7-10, tempo de execução $\mathcal{O}(n^2)$) e iv) substituir até duas transposições de $S$ por uma sequência equivalente de reversões (linhas 13-14, tempo constante). Considerando que $|S| \le {n + 1}$, o tempo de execução de Algoritmo~\ref{algorithm:NTMUPIXY} é $\mathcal{O}(n^4)$.

\begin{algorithm}[!tbh]
  \caption{Um algoritmo de aproximação assintótica para problema \SbPRT{} em instâncias clássicas com sinais e $k \in [0..1]$.\label{algorithm:NTMUPIXY}}
  \Entrada{Uma instância clássica com sinais $\mathcal{I} = (\pi,\iota)$ e um valor de $k \in [0..1]$}
  \Saida{Uma sequência de reversões e transposições $S$, tal que $\pi \cdot S = \iota$ e $\frac{|S_{\rho}|}{|S|} \ge k$}
    Seja $S \gets ()$ \\
    \Enqto{$|\mathcal{A}_\rho(\mathcal{I})| > k(|S| + |\mathcal{A}_\rho(\mathcal{I})|)$}{
      \Se {em $G(\mathcal{I})$ existe um ciclo divergente}{
        Seja $\rho$ uma reversão que aumenta o número de ciclos em uma unidade (Teorema 5 de~\cite{1998-walter-etal}) \\
        $\mathcal{I} = (\pi \cdot \rho, \iota)$ \\
        $S \gets S + (\rho)$ \\
      }\Senao{
        Seja $S'$ uma sequência de no máximo duas transposições que aumenta o número de ciclos em duas unidade (Teorema 3.4 de~\cite{1998-bafna-pevzner})\\
        $\mathcal{I} = (\pi \cdot S', \iota)$ \\
        $S \gets S + S'$ \\
      }
    }
    $S \gets S + \mathcal{A}_\rho(\mathcal{I})$ \\
    \Se{$|S_\rho| < k|S|$}{
      Substitua até duas transposições de $S$ por uma sequência equivalente de reversões (Observação~\ref{remark:DNLEDNKT}) \\
    }
  \Retorna{S}
\end{algorithm}


\begin{lemma}\label{lemma:THMUHLVK}
Dada uma instância clássica com sinais $\mathcal{I} = (\pi,\iota)$ e um valor $k \in [0..1]$, o Algoritmo~\ref{algorithm:NTMUPIXY} fornece uma sequência de operações $S$ com no máximo $(n+1-c(G(\mathcal{I}))) / (1-k/3) + 4$ operações de reversões e transposições tal que $\pi \cdot S = \iota$ e $\frac{|S_{\rho}|}{|S|} \ge k$. 
\end{lemma}
\begin{proof}
Note que a sequência forneceida pelo algoritmo $\mathcal{A}_\rho$ (linha 12) trasforma $\pi$ em $\iota$. Consequentemente, o Algoritmo~\ref{algorithm:NTMUPIXY} também transforma $\pi$ em $\iota$.
Seja $S$ a sequência de operações gerada pelo Algoritmo~\ref{algorithm:NTMUPIXY} sem considerar a substituição de transposições por reversões feita na linha 14. Seja $S'$ a subsequência de $S$ criada durante o laço de repetição das linha 2 até 10. Note que, em média, cada operação em $S'$ aumenta o número de ciclos em pelo menos uma unidade. Além disso, em média, cada operação em $S \setminus S'$ (ou seja, as reversões utilizadas pelo algoritmo $\mathcal{A}_\rho$ na linha 12) aumenta o número de ciclos em pelo menos $2/3$ unidades. Pela condição na linha 2, temos que $|S'| \ge (1-k) |S|$. Além disso, em média, cada operação de $S$ aumenta o número de ciclos em pelo menos $\frac{(1-k)|S| + k|S|2/3}{|S|} = 1 - k/3$. Como para transformar $\pi$ em $\iota$ é necessário aumentar o número de ciclos em $n+1-c(G(\mathcal{I}))$ unidades, temos que $|S| \le \frac{n+1-c(G(\mathcal{I}))}{1-k/3}$. Note que a sequência $S$ pode não satisfazer a restrição $\frac{|S_{\rho}|}{|S|} \ge k$. Caso isso ocorra, sabemos que o Algoritmo~\ref{algorithm:NTMUPIXY} adiciona transposições em $S$ somente enquanto a condição da linha 2 for satisfeita e que, no máximo, duas transposições são adicionadas por iteração. No pior caso, garantimos que $\frac{|S_{\rho}|}{|S|} \ge k$ subtituindo até duas transposições de $S$ por seis reversões. Logo, $|S| \le \frac{n+1-c(G(\mathcal{I}))}{1-k/3} + 4$ e o lema segue.
\end{proof}

\begin{theorem}\label{theorem:VWZUZNOR}
O Algoritmo~\ref{algorithm:NTMUPIXY} é uma $\frac{2-k}{1-k/3}$-aproximação assintótica para o problema \SbPRT{} em instâncias clássicas com sinais e para uma proporção $k \in [0..1]$.
\end{theorem}
\begin{proof}
Pelo Lema~\ref{lemma:THMUHLVK} e Teorema~\ref{theorem:WSTPPSMD}, a sequência de operações $S$ obtida através do Algoritmo~\ref{algorithm:NTMUPIXY} satisfaz as seguintes condições: $\pi \cdot S = \iota$, $\frac{|S_{\rho}|}{|S|} \ge k$ e $|S| \le (n+1-c(G(\mathcal{I}))) / (1-k/3) + 4 \le \frac{2-k}{1-k/3} dp_k(\mathcal{I}) + 4$. Logo, o teorema segue.
\end{proof}

% ------------------------------------------------------------------ %
\section{Resultados Práticos}
% ------------------------------------------------------------------ %

Nesta seção, apresentamos os experimentos práticos e os resultados obtidos. Inicialmente, descrevemos os algoritmos utilizados como \emph{baseline}, bem como as modificações realizadas para garantir que suas soluções sejam válidas para o problema \SbPRT{}, ou seja, para garantir que a restrição de propoção seja satisfeita. Em seguida, apresentamos as bases de dados desenvolvidas e utilizadas como entrada pelos algoritmos. Por fim, discutimos os resultados.

% ------------------------------------------------------------------ %
\subsection{Algoritmos Comparados}\label{subsection:BBECOZCK}
% ------------------------------------------------------------------ %

Para fins de comparação, usamos seis algoritmos da literatura como baselines para comparar com os resultados fornecidos por nossos algoritmos. Os algoritmos que descreveremos a seguir foram desenvolvidos para problemas específicos que não consideram a restrição de proporção entre os eventos de rearranjo e podem fornecer soluções inviáveis para o problema \SbPRT{}. Para garantir que todas as soluções sejam viáveis, quando necessário, ajustamos a sequência de eventos de rearranjo substituindo transposições por reversões para atingir a proporção mínima dada como entrada. A seguir, apresentamos os algoritmos de baseline e o processo de modificações realizado.

\begin{itemize}
    \item Instâncias clássicas sem sinais:
    \begin{itemize}
        \item UR: Algoritmo de aproximação com fator $2$ para o problema de Ordenação de Permutações por Reversões~\cite{1995-kececioglu-sankoff}.
        \item UT: Algoritmo de aproximação com fator $1.5$ para o problema de Ordenação de Permutações por Transposições~\cite{1998-bafna-pevzner}.
        \item URT: Algoritmo de aproximação com fator $2\alpha$ para o problema de Ordenação de Permutações por Reversões e Transposições~\cite{2008-rahman-etal}, onde $\alpha$ é o fator de aproximação do algoritmo utilizado para a decomposição de ciclos do Grafo de Ciclos (valor adotado $\alpha = 1.4193 + \epsilon$~\cite{2004-lin-jiang}).
    \end{itemize}
    \item Instâncias clássicas com sinais:
    \begin{itemize}
        \item SR: Algoritmo exato e polinomial para o problema de Ordenação de Permutações por Reversões~\cite{1999-hannenhalli-pevzner}.
        \item SRT: Algoritmo de aproximação com fator $2$ para o problema de Ordenação de permutações por Reversões e Transposições~\cite{1998-walter-etal}.
        \item SWRT: Algoritmo de aproximação com fator $1.5$ para o problema de Ordenação de Permutações por Reversões e Transposições Ponderadas~\cite{2019a-oliveira-etal} (adotando os pesos 2 e 3 para os eventos de reversão e transposição, respectivamente).
    \end{itemize}
\end{itemize}

O processo de modificação realizado na sequência de eventos de rearranjo para satisfazer a restrição de proporção mínima difere entre instâncias clássicas com e sem sinais. No caso de uma instância clássica com sinais, enquanto a proporção mínima não for atingida, uma transposição é substituída por uma sequência de três reversões seguindo o processo descito na Observação~\ref{remark:DNLEDNKT}. No caso de uma instância clássica sem sinais, esse processo segue regras para evitar o crescimento desnecessário da sequência $S$ gerada pelos algoritmo de baseline: (i) se houver uma transposição $\tau^{(i, j, k)}$ tal que $k - i = 2$, então a substituição é realizada apenas por uma reversão $\rho^{(i, k - 1)}$; (ii) se houver uma transposição $\tau^{(i, j, k)}$ tal que $j - i = 1$ ou $k - j = 1$, então a substituição é realizada por uma sequência de duas reversões $S=(\rho^{(i, k - 1)},\rho^{(i, k - 2)})$ ou $S=(\rho^{(i, k - 1)},\rho^{(i + 1, k - 1)}$; e caso contrário, (iii) uma transposição é substituída por uma sequência de três reversões seguindo o processo descito na Observação~\ref{remark:DNLEDNKT}. Este processo se repete enquanto a proporção mínima não é atingida, seguindo a ordem das regras de substituição.

% ------------------------------------------------------------------ %
\subsection{Base de Dados}
% ------------------------------------------------------------------ %

Para verificar o desempenho dos algoritmos em diferentes cenários, criamos bases de dados de instâncias clássicas para simular cenários com proporções fixas entre eventos de reversão e transposição.

DB1 - Esta base de dados é dividida em grupos. Cada grupo tem um total de 10.000 instâncias clássicas de tamanho 200 (ou seja, $\pi$ e $\iota$ tem 200 elementos cada) e é identificado pela proporção $k$ adotada para criar as instâncias. Uma sequência com 40 operações é gerada de forma que seja composta por $40k$ de reversões e $40(1-k)$ de transposições. Os parâmetros das reversões e transposições geradas são escolhidos aleatoriamente entre os valores possíveis. Em seguida, a sequência de operações é embaralhada e aplicada na permutação identidade $\iota$. A permutação resultante $\pi$, a permutação identidade $\iota$ e a proporção $k$ formam uma instância do grupo. Este processo é repetido até que o grupo tenha um total de 10.000 instâncias. As proporções utilizadas variaram de 0 a 1, em intervalos de 0.1, totalizando 11 grupos. Esta base de dados tem as versões com instâncias clássicas com e sem sinais. Considerando instâncias clássicas com e sem sinais, essa base de dados possui um total de 220.000 instâncias.

DB2 - Esta base de dados foi desenvolvido para refletir cenários onde o número de reversões é 50\% maior que o número de transposições. Assim, no processo de criação das instâncias, foi mantida uma proporção de $k = 0,6$. A base de dados é dividida em grupos com 10.000 instâncias cada. Além disso, o identificador do grupo indica o tamanho das instâncias contidas nele e o número de operações utilizadas para criar as instâncias. Os tamanhos usados para as instâncias foram 100, 200, 300, 400 e 500. O número de operações foi baseado em uma porcentagem do tamanho da instância, adotamos: 10\%, 20\%, 30\%, 40\% e 50\%. As etapas finais do processo são semelhantes ao que descrevemos anteriormente na base de dados DB1. Esta base de dados possui uma versão apenas para instâncias clássicas com sinais e um total de 250.000 instâncias.

% ------------------------------------------------------------------ %
\subsection{Comparação dos Algoritmos}
% ------------------------------------------------------------------ %

Nesta seção, apresentamos os resultados fornecidos pelos algoritmos utilizando as bases de dados DB1 e DB2. Nas tabelas~\ref{table:AELVTSMQ}, \ref{table:CLTLBDUJ}, \ref{table:GZLFZWZB} e \ref{table:YDCKQRVK}, as colunas Min, Avg e Max representam mínimo, média e máximo, respectivamente.

O objetivo principal dos testes experimentais é a análise do desempenho prático dos algoritmos propostos comparando-os com as aproximações teóricas e com resultados fornecidos por outros algoritmos da literatura. As tabelas~\ref{table:AELVTSMQ} e~\ref{table:CLTLBDUJ} mostram os resultados dos algoritmos considerando diferentes cenários de proporção, o que é útil para estudar o comportamento dos algoritmos variando a proporção desejada. As siglas UPRT e SPRT referem-se aos algoritmos~\ref{algorithm:INBRWKCH} e~\ref{algorithm:TNMWZNZU}, respectivamente.

A Tabela~\ref{table:AELVTSMQ} mostra os resultados dos algoritmos UR, UT, URT e UPRT aplicados em instâncias sem sinal da base de dados DB1. Algumas soluções fornecidas por UT e URT foram modificadas seguindo o processo descrito na Seção~\ref{subsection:BBECOZCK} para ajustar a proporção mínima entre a quantidade de reversões e tamanho da sequência de rearranjo. Considerando todas as instâncias da base de dados, um total de 90.90\% e 34.39\% das soluções fornecidas por UT e URT, respectivamente, foram modificadas. A razão de aproximação foi calculada adotando-se o limite inferior apresentado no Teorema~\ref{theorem:KJXGKJIP}.

\begin{table}[!tbh]
\caption{Resultados dos algoritmos em instâncias clássicas sem sinais da base de dados DB1.}
\label{table:AELVTSMQ}
\begin{center}
\scriptsize
{\def\arraystretch{1.05}\tabcolsep=8pt
\begin{tabular}{|c|c|c|c|c|c|c|c|c|c|}
\hline
\multicolumn{10}{|c|}{\bf UR}                                                                                             \\ \hline
\multirow{2}{*}{k} & \multicolumn{3}{c|}{Proporção}  & \multicolumn{3}{c|}{Distância} & \multicolumn{3}{c|}{Aproximação}  \\ \cline{2-10}
                   & Min       & Avg      & Max      & Min      & Avg      & Max     & Min     & Avg   & Max              \\ \hline
0.0                & 1.000     & 1.000    & 1.000    & ~69      & ~84.746  & ~99     & 2.47    & 2.77  & 3.00             \\ \hline
0.1                & 1.000     & 1.000    & 1.000    & ~68      & ~81.358  & ~94     & 2.31    & 2.63  & 2.90             \\ \hline
0.2                & 1.000     & 1.000    & 1.000    & ~63      & ~78.096  & ~91     & 2.13    & 2.49  & 2.78             \\ \hline
0.3                & 1.000     & 1.000    & 1.000    & ~62      & ~74.598  & ~90     & 2.09    & 2.37  & 2.67             \\ \hline
0.4                & 1.000     & 1.000    & 1.000    & ~58      & ~71.054  & ~85     & 1.91    & 2.24  & 2.53             \\ \hline
0.5                & 1.000     & 1.000    & 1.000    & ~55      & ~67.264  & ~79     & 1.76    & 2.10  & 2.42             \\ \hline
0.6                & 1.000     & 1.000    & 1.000    & ~50      & ~63.265  & ~74     & 1.65    & 1.96  & 2.27             \\ \hline
0.7                & 1.000     & 1.000    & 1.000    & ~49      & ~59.089  & ~70     & 1.46    & 1.81  & 2.13             \\ \hline
0.8                & 1.000     & 1.000    & 1.000    & ~45      & ~54.682  & ~67     & 1.32    & 1.66  & 2.03             \\ \hline
0.9                & 1.000     & 1.000    & 1.000    & ~40      & ~50.249  & ~64     & 1.24    & 1.52  & 1.93             \\ \hline
1.0                & 1.000     & 1.000    & 1.000    & ~38      & ~45.613  & ~56     & 1.11    & 1.37  & 1.74             \\ \hline
\end{tabular}%
\vspace{5pt}
\begin{tabular}{|c|c|c|c|c|c|c|c|c|c|}
\hline
\multicolumn{10}{|c|}{\bf UT}                                                                                             \\ \hline
\multirow{2}{*}{k} & \multicolumn{3}{c|}{Proporção}  & \multicolumn{3}{c|}{Distância} & \multicolumn{3}{c|}{Aproximação}  \\ \cline{2-10}
                   & Min       & Avg      & Max      & Min      & Avg      & Max     & Min     & Avg   & Max              \\ \hline
0.0                & 0.000     & 0.000    & 0.000    & ~35      & ~40.475  & ~45     & 1.14    & 1.33  & 1.56             \\ \hline
0.1                & 0.100     & 0.117    & 0.156    & ~42      & ~80.488  & 125     & 1.27    & 2.60  & 4.24             \\ \hline
0.2                & 0.200     & 0.214    & 0.241    & ~51      & ~88.408  & 126     & 1.72    & 2.83  & 4.20             \\ \hline
0.3                & 0.300     & 0.312    & 0.333    & ~61      & ~96.265  & 140     & 1.94    & 3.06  & 4.67             \\ \hline
0.4                & 0.400     & 0.410    & 0.429    & ~65      & 104.689  & 147     & 2.13    & 3.30  & 4.93             \\ \hline
0.5                & 0.500     & 0.507    & 0.519    & ~75      & 115.007  & 155     & 2.39    & 3.60  & 5.12             \\ \hline
0.6                & 0.600     & 0.606    & 0.619    & ~78      & 127.210  & 170     & 2.52    & 3.94  & 6.00             \\ \hline
0.7                & 0.700     & 0.705    & 0.716    & ~87      & 142.344  & 211     & 2.75    & 4.36  & 6.56             \\ \hline
0.8                & 0.800     & 0.804    & 0.811    & 103      & 161.781  & 225     & 3.15    & 4.92  & 7.50             \\ \hline
0.9                & 0.900     & 0.903    & 0.909    & 119      & 187.906  & 257     & 3.56    & 5.67  & 8.76             \\ \hline
1.0                & 1.000     & 1.000    & 1.000    & 118      & 223.210  & 328     & 3.47    & 6.71  & 10.36            \\ \hline
\end{tabular}%
\vspace{5pt}
\begin{tabular}{|c|c|c|c|c|c|c|c|c|c|}
\hline
\multicolumn{10}{|c|}{\bf URT}                                                                                            \\ \hline
\multirow{2}{*}{k} & \multicolumn{3}{c|}{Proporção}  & \multicolumn{3}{c|}{Distância} & \multicolumn{3}{c|}{Aproximação}  \\ \cline{2-10}
                   & Min       & Avg      & Max      & Min      & Avg      & Max     & Min     & Avg   & Max              \\ \hline
0.0                & 0.188     & 0.642    & 0.919    & ~43      & ~59.872  & ~78     & 1.37    & 1.96  & 2.57             \\ \hline
0.1                & 0.267     & 0.651    & 0.899    & ~44      & ~57.611  & ~73     & 1.36    & 1.86  & 2.45             \\ \hline
0.2                & 0.222     & 0.657    & 0.901    & ~40      & ~55.277  & ~71     & 1.22    & 1.77  & 2.38             \\ \hline
0.3                & 0.306     & 0.660    & 0.909    & ~39      & ~52.658  & ~67     & 1.22    & 1.68  & 2.20             \\ \hline
0.4                & 0.400     & 0.663    & 0.905    & ~39      & ~49.953  & ~64     & 1.20    & 1.58  & 2.10             \\ \hline
0.5                & 0.500     & 0.667    & 0.933    & ~36      & ~47.097  & ~62     & 1.14    & 1.47  & 2.04             \\ \hline
0.6                & 0.600     & 0.683    & 0.930    & ~37      & ~44.624  & ~57     & 1.09    & 1.38  & 1.84             \\ \hline
0.7                & 0.700     & 0.732    & 0.925    & ~37      & ~43.719  & ~57     & 1.08    & 1.34  & 1.78             \\ \hline
0.8                & 0.800     & 0.815    & 0.918    & ~37      & ~44.750  & ~60     & 1.08    & 1.36  & 1.88             \\ \hline
0.9                & 0.900     & 0.912    & 0.977    & ~38      & ~47.017  & ~64     & 1.05    & 1.42  & 1.97             \\ \hline
1.0                & 1.000     & 1.000    & 1.000    & ~39      & ~48.809  & ~75     & 1.08    & 1.47  & 2.27             \\ \hline
\end{tabular}%
\vspace{5pt}
\begin{tabular}{|c|c|c|c|c|c|c|c|c|c|}
\hline
\multicolumn{10}{|c|}{\bf UPRT}                                                                                           \\ \hline
\multirow{2}{*}{k} & \multicolumn{3}{c|}{Proporção}  & \multicolumn{3}{c|}{Distância} & \multicolumn{3}{c|}{Aproximação}  \\ \cline{2-10}
                   & Min       & Avg      & Max      & Min      & Avg      & Max     & Min     & Avg   & Max              \\ \hline
0.0                & 0.024     & 0.251    & 0.491    & ~38      & ~46.595  & ~57     & 1.21    & 1.54  & 1.92             \\ \hline
0.1                & 0.143     & 0.345    & 0.590    & ~38      & ~47.496  & ~61     & 1.26    & 1.56  & 1.94             \\ \hline
0.2                & 0.200     & 0.385    & 0.596    & ~38      & ~46.415  & ~56     & 1.25    & 1.51  & 1.86             \\ \hline
0.3                & 0.302     & 0.444    & 0.660    & ~39      & ~45.713  & ~56     & 1.22    & 1.48  & 1.80             \\ \hline
0.4                & 0.400     & 0.510    & 0.667    & ~38      & ~45.114  & ~53     & 1.17    & 1.44  & 1.72             \\ \hline
0.5                & 0.500     & 0.579    & 0.723    & ~38      & ~44.385  & ~53     & 1.15    & 1.41  & 1.75             \\ \hline
0.6                & 0.600     & 0.662    & 0.787    & ~37      & ~43.976  & ~52     & 1.14    & 1.38  & 1.73             \\ \hline
0.7                & 0.700     & 0.744    & 0.844    & ~37      & ~43.302  & ~52     & 1.14    & 1.35  & 1.68             \\ \hline
0.8                & 0.800     & 0.829    & 0.889    & ~37      & ~42.536  & ~50     & 1.07    & 1.31  & 1.72             \\ \hline
0.9                & 0.900     & 0.918    & 0.956    & ~36      & ~41.857  & ~50     & 1.06    & 1.28  & 1.69             \\ \hline
1.0                & 1.000     & 1.000    & 1.000    & ~36      & ~40.818  & ~49     & 1.07    & 1.24  & 1.74             \\ \hline
\end{tabular}%
}
\end{center}
\end{table}


Pela Tabela~\ref{table:AELVTSMQ}, podemos ver que UR apresenta uma razão média de aproximação maior em valores menores de $k$. No entanto, à medida que o valor de $k$ aumenta, a razão de aproximação média tende a diminuir. A partir dos resultados práticos de UR, é possível notar que a razão de aproximação média é muitas vezes melhor do que o fator de aproximação teórica $3-k$ provado para o problema (Teorema~\ref{theorem:FTRSGXOZ}).

Analisando os resultados fornecidos por UT, é possível notar um comportamento oposto ao de UR, com a razão de aproximação média aumentando à medida que o valor de $k$ aumenta. A razão de aproximação média foi menor que três apenas nos grupos em que $k$ é menor ou igual a 0.2, e a razão de aproximação média no grupo em que $k = 1$ foi de 6.71. Vale ressaltar que todas as soluções fornecidas pela UT para grupos em que $0.1 \le k \le 1.0$ foram modificadas para se adequarem à proporção mínima exigida pelo problema \SbPRT.

Considerando os grupos onde $k \ge 0.7$, a distância máxima fornecida por UT foi supeior a cinco vezes o número de eventos utilizados para criar as instâncias (40 operações). Isso indica que a técnica de modificação de solução aplicada ao algoritmo desenvolvido para o problema considerando apenas transposições não fornece bons resultados para valores maiores de $k$.

Considerando os resultados de URT, podemos observar que a razão média de aproximação foi menor ou igual a 1.96 para todos os grupos. Comparado com UR e UT, o algoritmo URT apresentou melhores resultados para a aproximação média para grupos onde $0.0 < k < 1.0$. Os algoritmos UR e UT apresentaram melhores resultados quando $k = 1,0$ e $k = 0,0$, respectivamente. O desempenho superior de UR e UT nesses cenários particulares ocorre porque a sequência de ordenação composta apenas por reversões se encaixa perfeitamente no caso em que $k = 1,0$ e, quando $k = 0,0$, uma sequência de transposições não precisa passar o processo de modificação para respeitar a restrição de proporção. Nesse caso, modificar as soluções para se adequarem à proporção mínima exigida pela instância resultou em bons resultados independente do grupo.

Observando os resultados do algoritmo UPRT, podemos ver que a razão de aproximação média tende a diminuir à medida que o valor de $k$ aumenta e, em comparação com os demais algoritmos, sofre menor variação. Considerando a maior e a menor taxa de aproximação média entre todos os grupos, temos 1.56 e 1.24, respectivamente. Esta é uma variação de 0.32, o que mostra que o algoritmo é robusto independente da proporção adotada para o cenário. Observe que a variação da aproximação média mostra o quanto o algoritmo oscila de acordo com as diferentes proporções. Deseja-se obter uma variação tão pequena quanto possível. Isso ajuda a obter resultados práticos estáveis, independentemente da proporção desejada. A razão de aproximação média fornecida pelo algoritmo foi melhor que as demais, exceto no grupo onde $k = 0,0$. Isso provavelmente ocorre porque quando $k = 0,0$, uma sequência composta exclusivamente por transposições se enquadra na restrição de proporção. Considerando a razão de aproximação máxima (pior caso prático), podemos observar que em todos os grupos o valor foi menor ou igual a 1.94. Outra característica interessante dos resultados desse algoritmo está relacionada às proporções obtidas nas soluções. A proporção média para todos os grupos é sempre muito próxima do valor mínimo especificado para a instância.

A Tabela~\ref{table:CLTLBDUJ} mostra os resultados dos algoritmos SR, SRT, SWRT e SPRT aplicados em instâncias clássicas com sinais da base de dados DB1. Considerando todas as instâncias da base de dados, um total de 3.65\% e 39.11\% das soluções fornecidas por SRT e SWRT, respectivamente, foram modificadas para se adequarem à proporção mínima entre a quantidade de reversões e tamanho da sequência de rearranjo. A razão de aproximação foi calculada adotando-se o limite inferior apresentado no Teorema~\ref{theorem:WSTPPSMD}.

\begin{table}[!tbh]
\caption{Resultados dos algoritmos em instâncias clássicas com sinais da base de dados DB1.}
\label{table:CLTLBDUJ}
\begin{center}
\scriptsize
{\def\arraystretch{1.05}\tabcolsep=8pt
\begin{tabular}{|c|c|c|c|c|c|c|c|c|c|}
\hline
\multicolumn{10}{|c|}{\bf SR}                                                                                             \\ \hline
\multirow{2}{*}{k} & \multicolumn{3}{c|}{Proporção}  & \multicolumn{3}{c|}{Distância} & \multicolumn{3}{c|}{Aproximação}  \\ \cline{2-10}
                   & Min       & Avg      & Max      & Min      & Avg      & Max     & Min     & Avg    & Max             \\ \hline
0.0                & 1.000     & 1.000    & 1.000    & ~67      & ~78.870  & ~82     & 2.03    & 2.03   & 2.06            \\ \hline
0.1                & 1.000     & 1.000    & 1.000    & ~64      & ~74.485  & ~77     & 1.90    & 1.90   & 1.93            \\ \hline
0.2                & 1.000     & 1.000    & 1.000    & ~63      & ~70.867  & ~74     & 1.80    & 1.80   & 1.85            \\ \hline
0.3                & 1.000     & 1.000    & 1.000    & ~60      & ~67.167  & ~69     & 1.70    & 1.70   & 1.73            \\ \hline
0.4                & 1.000     & 1.000    & 1.000    & ~56      & ~63.386  & ~65     & 1.60    & 1.60   & 1.63            \\ \hline
0.5                & 1.000     & 1.000    & 1.000    & ~54      & ~59.532  & ~61     & 1.50    & 1.50   & 1.53            \\ \hline
0.6                & 1.000     & 1.000    & 1.000    & ~50      & ~55.654  & ~57     & 1.40    & 1.40   & 1.43            \\ \hline
0.7                & 1.000     & 1.000    & 1.000    & ~46      & ~51.767  & ~54     & 1.30    & 1.30   & 1.35            \\ \hline
0.8                & 1.000     & 1.000    & 1.000    & ~43      & ~47.837  & ~49     & 1.20    & 1.20   & 1.23            \\ \hline
0.9                & 1.000     & 1.000    & 1.000    & ~39      & ~43.882  & ~45     & 1.10    & 1.10   & 1.13            \\ \hline
1.0                & 1.000     & 1.000    & 1.000    & ~36      & ~39.920  & ~40     & 1.00    & 1.00   & 1.00            \\ \hline
\end{tabular}%
\vspace{5pt}
\begin{tabular}{|c|c|c|c|c|c|c|c|c|c|c|}
\hline
\multicolumn{10}{|c|}{\bf SRT}                                                                                            \\ \hline
\multirow{2}{*}{k} & \multicolumn{3}{c|}{Proporção}  & \multicolumn{3}{c|}{Distância} & \multicolumn{3}{c|}{Aproximação}  \\ \cline{2-10}
                   & Min       & Avg      & Max      & Min      & Avg      & Max     & Min     & Avg    & Max             \\ \hline
0.0                & 0.000     & 0.000    & 0.000    & ~38      & ~45.570  & ~54     & 1.00    & 1.17   & 1.35            \\ \hline
0.1                & 0.400     & 0.975    & 1.000    & ~55      & ~74.000  & ~76     & 1.38    & 1.88   & 1.90            \\ \hline
0.2                & 0.730     & 0.981    & 1.000    & ~62      & ~70.570  & ~72     & 1.59    & 1.79   & 1.80            \\ \hline
0.3                & 0.774     & 0.982    & 1.000    & ~60      & ~66.932  & ~68     & 1.55    & 1.69   & 1.70            \\ \hline
0.4                & 0.780     & 0.981    & 1.000    & ~54      & ~63.171  & ~64     & 1.47    & 1.59   & 1.60            \\ \hline
0.5                & 0.750     & 0.980    & 1.000    & ~53      & ~59.323  & ~60     & 1.35    & 1.49   & 1.50            \\ \hline
0.6                & 0.769     & 0.978    & 1.000    & ~50      & ~55.477  & ~56     & 1.30    & 1.39   & 1.40            \\ \hline
0.7                & 0.712     & 0.977    & 1.000    & ~46      & ~51.606  & ~52     & 1.18    & 1.29   & 1.30            \\ \hline
0.8                & 0.800     & 0.974    & 1.000    & ~43      & ~47.694  & ~52     & 1.10    & 1.19   & 1.30            \\ \hline
0.9                & 0.900     & 0.977    & 1.000    & ~39      & ~43.938  & ~59     & 1.00    & 1.10   & 1.48            \\ \hline
1.0                & 1.000     & 1.000    & 1.000    & ~36      & ~41.750  & ~60     & 1.00    & 1.05   & 1.53            \\ \hline
\end{tabular}%
\vspace{5pt}
\begin{tabular}{|c|c|c|c|c|c|c|c|c|c|c|}
\hline
\multicolumn{10}{|c|}{\bf SWRT}                                                                                           \\ \hline
\multirow{2}{*}{k} & \multicolumn{3}{c|}{Proporção}  & \multicolumn{3}{c|}{Distância} & \multicolumn{3}{c|}{Aproximação}  \\ \cline{2-10}
                   & Min       & Avg      & Max      & Min      & Avg      & Max     & Min     & Avg    & Max             \\ \hline
0.0                & 0.000     & 0.000    & 0.000    & ~35      & ~40.153  & ~45     & 1.00    & 1.03   & 1.14            \\ \hline
0.1                & 0.100     & 0.334    & 0.737    & ~37      & ~45.021  & ~57     & 1.00    & 1.15   & 1.50            \\ \hline
0.2                & 0.200     & 0.422    & 0.727    & ~37      & ~45.141  & ~55     & 1.00    & 1.14   & 1.41            \\ \hline
0.3                & 0.300     & 0.478    & 0.778    & ~37      & ~44.345  & ~54     & 1.00    & 1.12   & 1.39            \\ \hline
0.4                & 0.400     & 0.526    & 0.792    & ~37      & ~43.248  & ~53     & 1.00    & 1.09   & 1.33            \\ \hline
0.5                & 0.500     & 0.575    & 0.816    & ~37      & ~42.116  & ~50     & 1.00    & 1.06   & 1.26            \\ \hline
0.6                & 0.600     & 0.638    & 0.833    & ~37      & ~41.576  & ~48     & 1.00    & 1.04   & 1.20            \\ \hline
0.7                & 0.700     & 0.720    & 0.894    & ~36      & ~42.131  & ~54     & 1.00    & 1.06   & 1.35            \\ \hline
0.8                & 0.800     & 0.814    & 0.909    & ~37      & ~43.511  & ~56     & 1.00    & 1.09   & 1.40            \\ \hline
0.9                & 0.900     & 0.911    & 0.952    & ~38      & ~45.516  & ~60     & 1.00    & 1.14   & 1.50            \\ \hline
1.0                & 1.000     & 1.000    & 1.000    & ~38      & ~47.336  & ~66     & 1.00    & 1.19   & 1.65            \\ \hline
\end{tabular}%
\vspace{5pt}
\begin{tabular}{|c|c|c|c|c|c|c|c|c|c|c|}
\hline
\multicolumn{10}{|c|}{\bf SPRT}                                                                                           \\ \hline
\multirow{2}{*}{k} & \multicolumn{3}{c|}{Proporção}  & \multicolumn{3}{c|}{Distância} & \multicolumn{3}{c|}{Aproximação}  \\ \cline{2-10}
                   & Min       & Avg      & Max      & Min      & Avg      & Max     & Min     & Avg    & Max             \\ \hline
0.0                & 0.000     & 0.014    & 0.244    & ~35      & ~39.696  & 48      & 1.00    & 1.02   & 1.24            \\ \hline
0.1                & 0.125     & 0.334    & 0.582    & ~38      & ~45.691  & 55      & 1.02    & 1.17   & 1.43            \\ \hline
0.2                & 0.219     & 0.394    & 0.627    & ~39      & ~45.103  & 55      & 1.01    & 1.15   & 1.38            \\ \hline
0.3                & 0.300     & 0.455    & 0.640    & ~38      & ~44.406  & 54      & 1.01    & 1.12   & 1.35            \\ \hline
0.4                & 0.400     & 0.520    & 0.694    & ~38      & ~43.732  & 52      & 1.00    & 1.10   & 1.32            \\ \hline
0.5                & 0.500     & 0.587    & 0.735    & ~38      & ~42.965  & 50      & 1.00    & 1.08   & 1.25            \\ \hline
0.6                & 0.600     & 0.665    & 0.792    & ~37      & ~42.514  & 50      & 1.00    & 1.07   & 1.28            \\ \hline
0.7                & 0.700     & 0.745    & 0.844    & ~39      & ~42.010  & 47      & 1.00    & 1.06   & 1.20            \\ \hline
0.8                & 0.800     & 0.828    & 0.889    & ~37      & ~41.430  & 47      & 1.00    & 1.04   & 1.18            \\ \hline
0.9                & 0.900     & 0.918    & 0.954    & ~36      & ~40.941  & 46      & 1.00    & 1.03   & 1.15            \\ \hline
1.0                & 1.000     & 1.000    & 1.000    & ~36      & ~40.059  & 42      & 1.00    & 1.01   & 1.05            \\ \hline
\end{tabular}%
}
\end{center}
\end{table}


Pela Tabela~\ref{table:CLTLBDUJ}, podemos ver que o algoritmo SR apresentou um comportamento semelhante ao algoritmo UR no caso sem sinal. No entanto, a aproximação média obtida foi exatamente $2 - k$, exceto para o grupo com $k = 0,0$. Observe que quando $k=1$, temos o problema de Ordenando de Permutações por Reversões e, o algoritmo SR fornece uma solução exata em tempo polinomial para o problema. Mantivemos esse cenário de proporção em nossos experimentos para verificar o desempenho dos outros algoritmos neste caso específico.

Os algoritmos SRT e SWRT não apresentam tendência de aumentar ou diminuir a razão média de aproximação considerando o valor de $k$. Comparando ambos, podemos ver que a variação média de aproximação do algoritmo SRT ($1.88 - 1.05 = 0.83$) é maior que a variação do algoritmo SWRT ($1.19 - 1.03 = 0.16$). Em comparação com o SWRT, a proporção média de soluções fornecidas pelo algoritmo SRT é maior. Exceto para o grupo com $k = 0.0$, a proporção média foi superior a $0.97$. O fato do algoritmo SRT não ter aplicado nenhuma reversão no grupo em que $k = 0$ é explicado pelo próprio comportamento do algoritmo, pois ele aplica reversões apenas em ciclos divergentes, e as instâncias desses grupos foram geradas usando apenas transposições, o que não gera ciclos divergentes.

O algoritmo SPRT apresentou a aproximação média mais consistente para os diferentes valores de $k$. Observe que a aproximação média nos extremos quando $k$ é igual a $0,0$ e $1,0$ foi $1.02$ e $1.01$, respectivamente. Além disso, a máxima aproximação média para os diferentes valores de $k$ foi de $1.17$, mostrando a robustez do algoritmo considerando diferentes cenários de proporção.

As tabelas~\ref{table:GZLFZWZB} e \ref{table:YDCKQRVK} mostram, respectivamente, os resultados dos algoritmos SWRT e SPRT considerando o cenário de proporção específico onde $k=0.6$. Como o algoritmo SWRT adota pesos 2 e 3 para eventos de reversão e transposição, respectivamente, uma forma indireta de fornecer uma comparação justa é usar a proporção $k=0.6$ (o número de reversões em uma solução para o problema \SbPRT{} é pelo menos 50\% maior que o número de transposições).

A Tabela~\ref{table:GZLFZWZB} mostra os resultados do algoritmo SWRT aplicado em instâncias clássicas com sinais da base de dados DB2. A coluna OP mostra o número de operações para criar as instâncias. Considerando os grupos de instâncias de tamanho 100, 200, 300, 400 e 500, um total de 37.00\%, 69.46\%, 79.98\%, 84.81\% e 87.59\% das soluções fornecidas pelo algoritmo SWRT foram modificadas, respectivamente. Considerando todas as instâncias, um total de 71.76\% das soluções fornecidas pelo algoritmo SWRT foram modificadas. A razão de aproximação foi calculada adotando-se o limite inferior apresentado no Teorema~\ref{theorem:WSTPPSMD}.

\begin{table}[!tb]
\caption{Resultados do algoritmo SWRT em instâncias clássicas com sinais da base de dados DB2.}
\label{table:GZLFZWZB}
\begin{center}
\scriptsize
{\def\arraystretch{1.05}\tabcolsep=8pt
\begin{tabular}{|c|c|c|c|c|c|c|c|c|c|}
\hline
\multicolumn{10}{|c|}{\bf Tamanho da Instância = 100}                                                                          \\ \hline
\multirow{2}{*}{OP} & \multicolumn{3}{c|}{Proporção} & \multicolumn{3}{c|}{Distância} & \multicolumn{3}{c|}{Aproximação}   \\ \cline{2-10}
                    & Min       & Avg      & Max      & Min      & Avg      & Max     & Min     & Avg    & Max             \\ \hline
~10                 & 0.600     & 0.711    & 1.000    & ~~8      & ~11.197  & ~17     & 1.00    & 1.12   & 1.70            \\ \hline
~20                 & 0.600     & 0.697    & 0.963    & ~18      & ~21.624  & ~28     & 1.00    & 1.08   & 1.40            \\ \hline
~30                 & 0.600     & 0.663    & 0.921    & ~26      & ~31.134  & ~38     & 1.00    & 1.05   & 1.28            \\ \hline
~40                 & 0.600     & 0.637    & 0.878    & ~32      & ~39.931  & ~49     & 1.00    & 1.03   & 1.22            \\ \hline
~50                 & 0.600     & 0.624    & 0.818    & ~38      & ~47.398  & ~57     & 1.00    & 1.03   & 1.16            \\ \hline
\end{tabular}%
\vspace{5pt}
\begin{tabular}{|c|c|c|c|c|c|c|c|c|c|}
\hline
\multicolumn{10}{|c|}{\bf Tamanho da Instância = 200}                                                                          \\ \hline
\multirow{2}{*}{OP} & \multicolumn{3}{c|}{Proporção} & \multicolumn{3}{c|}{Distância} & \multicolumn{3}{c|}{Aproximação}   \\ \cline{2-10}
                    & Min       & Avg      & Max      & Min      & Avg      & Max     & Min     & Avg    & Max             \\ \hline
~20                 & 0.600     & 0.693    & 0.963    & ~19      & ~21.808  & ~28     & 1.00    & 1.09   & 1.40            \\ \hline
~40                 & 0.600     & 0.637    & 0.857    & ~37      & ~41.565  & ~49     & 1.00    & 1.04   & 1.22            \\ \hline
~60                 & 0.600     & 0.614    & 0.738    & ~55      & ~62.137  & ~71     & 1.00    & 1.04   & 1.18            \\ \hline
~80                 & 0.600     & 0.610    & 0.645    & ~70      & ~82.220  & ~91     & 1.00    & 1.06   & 1.15            \\ \hline
100                 & 0.600     & 0.608    & 0.618    & ~83      & ~98.919  & 111     & 1.00    & 1.07   & 1.17            \\ \hline
\end{tabular}%
\vspace{5pt}
\begin{tabular}{|c|c|c|c|c|c|c|c|c|c|}
\hline
\multicolumn{10}{|c|}{\bf Tamanho da Instância = 300}                                                                          \\ \hline
\multirow{2}{*}{OP} & \multicolumn{3}{c|}{Proporção} & \multicolumn{3}{c|}{Distância} & \multicolumn{3}{c|}{Aproximação}   \\ \cline{2-10}
                    & Min       & Avg      & Max      & Min      & Avg      & Max     & Min     & Avg    & Max             \\ \hline
~30                 & 0.600     & 0.667    & 0.895    & ~28      & ~31.947  & ~39     & 1.00    & 1.06   & 1.30            \\ \hline
~60                 & 0.600     & 0.615    & 0.746    & ~57      & ~62.345  & ~70     & 1.00    & 1.04   & 1.16            \\ \hline
~90                 & 0.600     & 0.608    & 0.618    & ~86      & ~95.273  & 104     & 1.00    & 1.06   & 1.15            \\ \hline
120                 & 0.600     & 0.606    & 0.613    & 115      & 126.313  & 135     & 1.03    & 1.08   & 1.14            \\ \hline
150                 & 0.600     & 0.605    & 0.612    & 136      & 151.915  & 166     & 1.04    & 1.09   & 1.15            \\ \hline
\end{tabular}%
\vspace{5pt}
\begin{tabular}{|c|c|c|c|c|c|c|c|c|c|}
\hline
\multicolumn{10}{|c|}{\bf Tamanho da Instância = 400}                                                                          \\ \hline
\multirow{2}{*}{OP} & \multicolumn{3}{c|}{Proporção} & \multicolumn{3}{c|}{Distância} & \multicolumn{3}{c|}{Aproximação}   \\ \cline{2-10}
                    & Min       & Avg      & Max      & Min      & Avg      & Max     & Min     & Avg    & Max             \\ \hline
~40                 & 0.600     & 0.647    & 0.857    & ~38      & ~41.928  & ~49     & 1.00    & 1.04   & 1.22            \\ \hline
~80                 & 0.600     & 0.610    & 0.698    & ~77      & ~83.992  & ~94     & 1.00    & 1.05   & 1.17            \\ \hline
120                 & 0.600     & 0.606    & 0.613    & 120      & 128.804  & 139     & 1.02    & 1.08   & 1.15            \\ \hline
160                 & 0.600     & 0.605    & 0.610    & 155      & 170.892  & 181     & 1.05    & 1.10   & 1.14            \\ \hline
200                 & 0.600     & 0.604    & 0.608    & 189      & 205.193  & 224     & 1.07    & 1.11   & 1.14            \\ \hline
\end{tabular}%
\vspace{5pt}
\begin{tabular}{|c|c|c|c|c|c|c|c|c|c|}
\hline
\multicolumn{10}{|c|}{\bf Tamanho da Instância = 500}                                                                          \\ \hline
\multirow{2}{*}{OP} & \multicolumn{3}{c|}{Proporção} & \multicolumn{3}{c|}{Distância} & \multicolumn{3}{c|}{Aproximação}   \\ \cline{2-10}
                    & Min       & Avg      & Max      & Min      & Avg      & Max     & Min     & Avg    & Max             \\ \hline
~50                 & 0.600     & 0.631    & 0.810    & ~48      & ~51.872  & ~61     & 1.00    & 1.03   & 1.22            \\ \hline
100                 & 0.600     & 0.607    & 0.620    & ~99      & 105.745  & 119     & 1.00    & 1.05   & 1.19            \\ \hline
150                 & 0.600     & 0.605    & 0.610    & 153      & 162.374  & 172     & 1.04    & 1.08   & 1.14            \\ \hline
200                 & 0.600     & 0.604    & 0.608    & 201      & 215.583  & 227     & 1.06    & 1.10   & 1.14            \\ \hline
250                 & 0.600     & 0.603    & 0.607    & 240      & 258.773  & 279     & 1.09    & 1.11   & 1.14            \\ \hline
\end{tabular}%
}
\end{center}
\end{table}


Pela Tabela~\ref{table:GZLFZWZB}, é possível notar que a variação da razão de aproximação média é muito pequena independente do tamanho da permutação ou do número de operações utilizadas para criar a instância. Considerando a maior e a menor razão de aproximação média, temos os valores $1.12$ e $1.03$, respectivamente. Além disso, a razão de aproximação máxima foi de $1.70$, observada no grupo de instâncias com tamanho 100. Note que o algoritmo apresenta bons resultados mesmo com 71.76\% das instâncias sendo modificadas para satisfazer a restrição de proporção mínima. Isso mostra que o processo de modificação pode produzir bons resultados dependendo do algoritmo utilizado.


A Tabela~\ref{table:YDCKQRVK} mostra os resultados do algoritmo SPRT aplicado aplicado em instâncias clássicas com sinais da base de dados DB2. A coluna OP mostra o número de operações para criar as instâncias. A razão de aproximação foi calculada adotando-se o limite inferior apresentado no Teorema~\ref{theorem:WSTPPSMD}.

\begin{table}[!tb]
\caption{Resultados do algoritmo SPRT em instâncias clássicas com sinais da base de dados DB2.}
\label{table:YDCKQRVK}
\begin{center}
\scriptsize
{\def\arraystretch{1.05}\tabcolsep=8pt
\begin{tabular}{|c|c|c|c|c|c|c|c|c|c|}
\hline
\multicolumn{10}{|c|}{\bf Tamanho da Instância = 100}                                                                          \\ \hline
\multirow{2}{*}{OP} & \multicolumn{3}{c|}{Proporção} & \multicolumn{3}{c|}{Distância} & \multicolumn{3}{c|}{Aproximação}   \\ \cline{2-10}
                    & Min       & Avg      & Max      & Min      & Avg      & Max     & Min     & Avg    & Max             \\ \hline
~10                 & 0.600     & 0.733    & 1.000    & ~~8      & ~11.349  & ~16     & 1.00    & 1.13   & 1.60            \\ \hline
~20                 & 0.600     & 0.701    & 0.889    & ~17      & ~21.882  & ~28     & 1.00    & 1.10   & 1.40            \\ \hline
~30                 & 0.600     & 0.681    & 0.824    & ~25      & ~31.725  & ~39     & 1.00    & 1.07   & 1.30            \\ \hline
~40                 & 0.600     & 0.666    & 0.800    & ~32      & ~40.592  & ~49     & 1.00    & 1.05   & 1.25            \\ \hline
~50                 & 0.600     & 0.659    & 0.769    & ~39      & ~47.919  & ~57     & 1.00    & 1.05   & 1.22            \\ \hline
\end{tabular}%
\vspace{5pt}
\begin{tabular}{|c|c|c|c|c|c|c|c|c|c|}
\hline
\multicolumn{10}{|c|}{\bf Tamanho da Instância = 200}                                                                          \\ \hline
\multirow{2}{*}{OP} & \multicolumn{3}{c|}{Proporção} & \multicolumn{3}{c|}{Distância} & \multicolumn{3}{c|}{Aproximação}   \\ \cline{2-10}
                    & Min       & Avg      & Max      & Min      & Avg      & Max     & Min     & Avg    & Max             \\ \hline
~20                 & 0.600     & 0.700    & 0.923    & ~19      & ~22.155  & ~29     & 1.00    & 1.10   & 1.45            \\ \hline
~40                 & 0.600     & 0.666    & 0.809    & ~38      & ~42.556  & ~50     & 1.00    & 1.06   & 1.25            \\ \hline
~60                 & 0.600     & 0.649    & 0.742    & ~55      & ~62.097  & ~69     & 1.00    & 1.04   & 1.16            \\ \hline
~80                 & 0.600     & 0.639    & 0.711    & ~70      & ~80.046  & ~89     & 1.00    & 1.03   & 1.14            \\ \hline
100                 & 0.600     & 0.633    & 0.697    & ~83      & ~94.773  & 105     & 1.00    & 1.03   & 1.10            \\ \hline
\end{tabular}%
\vspace{5pt}
\begin{tabular}{|c|c|c|c|c|c|c|c|c|c|}
\hline
\multicolumn{10}{|c|}{\bf Tamanho da Instância = 300}                                                                          \\ \hline
\multirow{2}{*}{OP} & \multicolumn{3}{c|}{Proporção} & \multicolumn{3}{c|}{Distância} & \multicolumn{3}{c|}{Aproximação}   \\ \cline{2-10}
                    & Min       & Avg      & Max      & Min      & Avg      & Max     & Min     & Avg    & Max             \\ \hline
~30                 & 0.600     & 0.679    & 0.842    & ~29      & ~32.601  & ~39     & 1.00    & 1.08   & 1.30            \\ \hline
~60                 & 0.600     & 0.650    & 0.758    & ~58      & ~62.951  & ~70     & 1.00    & 1.05   & 1.16            \\ \hline
~90                 & 0.600     & 0.635    & 0.704    & ~85      & ~92.262  & 100     & 1.00    & 1.03   & 1.12            \\ \hline
120                 & 0.600     & 0.628    & 0.681    & 110      & 119.250  & 129     & 1.00    & 1.02   & 1.09            \\ \hline
150                 & 0.600     & 0.623    & 0.664    & 128      & 141.327  & 153     & 1.00    & 1.02   & 1.08            \\ \hline
\end{tabular}%
\vspace{5pt}
\begin{tabular}{|c|c|c|c|c|c|c|c|c|c|}
\hline
\multicolumn{10}{|c|}{\bf Tamanho da Instância = 400}                                                                          \\ \hline
\multirow{2}{*}{OP} & \multicolumn{3}{c|}{Proporção} & \multicolumn{3}{c|}{Distância} & \multicolumn{3}{c|}{Aproximação}   \\ \cline{2-10}
                    & Min       & Avg      & Max      & Min      & Avg      & Max     & Min     & Avg    & Max             \\ \hline
~40                 & 0.600     & 0.666    & 0.816    & ~39      & ~42.867  & ~50     & 1.00    & 1.07   & 1.25            \\ \hline
~80                 & 0.600     & 0.640    & 0.727    & ~76      & ~83.148  & ~93     & 1.00    & 1.04   & 1.16            \\ \hline
120                 & 0.600     & 0.627    & 0.683    & 115      & 122.314  & 130     & 1.00    & 1.02   & 1.09            \\ \hline
160                 & 0.600     & 0.622    & 0.656    & 149      & 158.546  & 168     & 1.00    & 1.02   & 1.07            \\ \hline
200                 & 0.600     & 0.619    & 0.647    & 175      & 187.099  & 198     & 1.00    & 1.01   & 1.06            \\ \hline
\end{tabular}%
\vspace{5pt}
\begin{tabular}{|c|c|c|c|c|c|c|c|c|c|}
\hline
\multicolumn{10}{|c|}{\bf Tamanho da Instância = 500}                                                                          \\ \hline
\multirow{2}{*}{OP} & \multicolumn{3}{c|}{Proporção} & \multicolumn{3}{c|}{Distância} & \multicolumn{3}{c|}{Aproximação}   \\ \cline{2-10}
                    & Min       & Avg      & Max      & Min      & Avg      & Max     & Min     & Avg    & Max             \\ \hline
~50                 & 0.600     & 0.656    & 0.793    & ~49      & ~53.064  & ~60     & 1.00    & 1.06   & 1.20            \\ \hline
100                 & 0.600     & 0.633    & 0.697    & ~98      & 103.272  & 111     & 1.00    & 1.03   & 1.12            \\ \hline
150                 & 0.600     & 0.622    & 0.673    & 145      & 152.283  & 162     & 1.00    & 1.02   & 1.08            \\ \hline
200                 & 0.601     & 0.618    & 0.645    & 186      & 197.490  & 204     & 1.00    & 1.01   & 1.05            \\ \hline
250                 & 0.603     & 0.617    & 0.634    & 226      & 233.744  & 239     & 1.00    & 1.01   & 1.04            \\ \hline
\end{tabular}%
}
\end{center}
\end{table}


A partir da Tabela~\ref{table:YDCKQRVK}, é possível notar que o algoritmo SPRT apresenta uma tendência de diminuir a razão de aproximação média linearmente à medida que o tamanho da permutação e o número de operações utilizadas para criar as instâncias (OP) crescem. Outro fato importante é que em todos os grupos (considerando o tamanho da instância e o número de operações utilizadas para criar as instâncias), o algoritmo SPRT conseguiu encontrar, para pelo menos uma instância do grupo, uma solução ótima. Podemos confirmar esse comportamento observando a coluna da razão de aproximação mínima. Além disso, considerando os grupos de instâncias com tamanho maior que 100 e os casos em que foram utilizadas sequências de operações maiores que 20\% do tamanho das instâncias, o algoritmo SPRT, em comparação com o algoritmo SWRT, apresentou equivalente ou melhores resultados ao observar a razão de aproximação média.

A partir dos resultados, observamos que os algoritmos propostos apresentam um excelente desempenho na prática, tanto na variação sem sinais do problema \SbPRT, como também na variação com sinais. Vale ressaltar que o processo de modificação da solução proposto para viabilizar soluções obtidas atráves de algoritmo para outros problemas também apresentou bons resultados, principalmente quando aplicado ao algoritmo SWRT.

% ------------------------------------------------------------------ %
\section{Conclusões}
% ------------------------------------------------------------------ %

Neste capítulo, investigamos o problema de Ordenção de Permutações por Reversões e Transposições com Restrição de Proporção. Como resultado, apresentamos uma análise de complexidade do problema para qualquer valor permitido de $k$ e considerando as variações com e sem sinais. Apresentamos algoritmos de aproximação com fatores $3 - \frac{3k}{2}$ e $3-k$ para as variações com e sem sinais, respectivamente. Além disso, apresentamos um algoritmo de aproximação assintótico com fator $\frac{2-k}{1-k/3}$ para a variação com sinais do problema. Por fim, realizamos testes experimentais comparando os algoritmos propostos com outros algoritmo da literatura que fornecem uma solução válida para o problema ou que a solução foi modificada para tornar a comparação possível.
\chapter{Modelos Intergênicos Rígidos}\label{chapter:DOVAEMLI}

A representação de um genoma por meio de uma sequência de genes é bastante útil e amplamente utilizada em problemas de rearranjo de genomas. Entretanto, informações que não estão presentes ou associadas diretamente aos genes são descartadas, o que implica em uma perda de informação. Em particular, informações referente às regiões intergênicas, que são regiões entre cada par consecutivo de genes e nas extremidades de um genoma linear, acabam não sendo consideradas pelos modelos que adotam uma representação clássica de um genoma. Estudos~\cite{2016a-biller-etal, 2016b-biller-etal} sugerem que incorporar tais estruturas aos modelos pode resultar em resultados mais realistas para a distância evolutiva entre os organismos. Cada região intergênica possui uma quantidade de nucleotídeos, essa quantidade de nucleotídeos é denominada de \emph{tamanho}. Nesse capítulo, investigaremos as variações com e sem sinais dos seguintes problemas que consideram a informação dos genes e do tamanho das regiões intergênicas de um genoma:

\begin{itemize}
  \item Ordenação de Permutações por Reversões Intergênicas (\SbIR)
  \item Ordenação de Permutações por Operações Intergênicas de Reversão e Indel (\SbIRI)
  \item Ordenação de Permutações por Operações Intergênicas de Reversão e Move \break (\SbIRM)
  \item Ordenação de Permutações por Operações Intergênicas de Reversão, Move e Indel (\SbIRMI)
  \item Ordenação de Permutações por Operações Intergênicas de Reversão e Transposição (\SbIRT)
  \item Ordenação de Permutações por Operações Intergênicas de Reversão, Transposição e Indel (\SbIRTI)
  \item Ordenação de Permutações por Operações Intergênicas de Reversão, Transposição e Move (\SbIRTM)
  \item Ordenação de Permutações por Operações Intergênicas de Reversão, Transposição, Move e Indel (\SbIRTMI)
\end{itemize}

Neste capítulo, iremos nos referenciar aos eventos de rearranjo de reversão intergênica, transposição intergênica, move intergênico e indel intergênico simplesmente por reversão, transposição, move e indel, respectivamente. Além disso, iremos nos referir a um breakpoint intergênico simplesmente como um breakpoint.

Dada uma instância intergênica rígida com ou sem sinais $\mathcal{I}=((\pi,\breve\pi),(\iota,\breve\iota))$, a \emph{distância} entre $(\pi,\breve\pi)$ e $(\iota,\breve\iota)$, denotada por $d_{\mathcal{M}}(\mathcal{I})$, é o tamanho da menor sequências de eventos de rearranjo $S$, tal que todo evento de $S$ pertence ao modelo $\mathcal{M}$ e $(\pi,\breve\pi) \cdot S = (\iota,\breve\iota)$. Os modelos de rearranjo considerados neste capítulo são identificados por siglas apresentadas na Tabela~\ref{table:YQWDTZTK}.

\begin{table}[!htb]
  \caption[Siglas dos modelos de rearranjo considerados para instâncias intergênicas rígidas.]{Siglas dos modelos de rearranjo considerados para instâncias intergênicas rígidas.}
  \label{table:YQWDTZTK}
  \centering
  \begin{tabular}{|p{3cm}|p{8cm}|}
    \hline
    \textbf{Sigla}        & \textbf{Conjunto de Eventos de Rearranjo}          \\ \hline
    \SbIR                 & $\{\rho\}                              $           \\ \hline
    \SbIRI                & $\{\rho,\delta\}                       $           \\ \hline
    \SbIRM                & $\{\rho,\mu\}                          $           \\ \hline
    \SbIRMI               & $\{\rho,\mu,\delta\}                   $           \\ \hline
    \SbIRT                & $\{\rho,\tau\}                         $           \\ \hline
    \SbIRTI               & $\{\rho,\tau,\delta\}                  $           \\ \hline
    \SbIRTM               & $\{\rho,\tau,\mu\}                     $           \\ \hline
    \SbIRTMI              & $\{\rho,\tau,\mu,\delta\}              $           \\ \hline
  \end{tabular}
\end{table} 

Quando estivermos adotando um modelo de rearranjo composto exclusivamente por eventos de rarranjo conservativos assumimos que a instância intergênica rígida para o problema será sempre balanceada. Caso contrário, seria impossível transformar o genoma de origem no genoma alvo.

Parte dos resultados que serão apresentados neste capítulo foram publicados nas revistas \emph{Journal of Computational Biology}~\cite{2020a-brito-etal} e \emph{Algorithms for Molecular Biology}~\cite{2021b-brito-etal} em 2020 e 2021, respectivamente.

% ------------------------------------------------------------------ %
\section{Limitantes Inferiores}
% ------------------------------------------------------------------ %

Nesta seção, apresentaremos limitantes inferiores para as variações com e sem sinais dos problemas investigados neste capítulo.

Em instâncias intergênicas rígidas com e sem sinais utilizaremos o conceito de breakpoint tipo dois e um, respectivamente. Os eventos de rearranjo de reversão, transposição, move e indel afetam, respectivamente, a seguinte quantidade de regiões intergênicas: duas, três, duas e uma. No melhor cenário, cada uma das regiões intergênicas faz parte de um breakpoint que é removido após o evento de rearranjo ser aplicado. Com isso, obtemos os seguintes lemas.

\begin{lemma}\label{lemma:KFFPUBQG}
Dada uma instância intergênica rígida sem sinais $\mathcal{I}=((\pi,\breve\pi),(\iota,\breve\iota))$, para qualquer reversão $\rho$ temos que $\Delta ib_1(\mathcal{I}, S = (\rho)) \ge -2$.
\end{lemma}

\begin{lemma}\label{lemma:IUJZCMMV}
Dada uma instância intergênica rígida sem sinais $\mathcal{I}=((\pi,\breve\pi),(\iota,\breve\iota))$, para qualquer transposição $\tau$ temos que $\Delta ib_1(\mathcal{I}, S = (\tau)) \ge -3$.
\end{lemma}

\begin{lemma}\label{lemma:SYXLGTAP}
Dada uma instância intergênica rígida sem sinais $\mathcal{I}=((\pi,\breve\pi),(\iota,\breve\iota))$, para qualquer move $\mu$ temos que $\Delta ib_1(\mathcal{I}, S = (\mu)) \ge -2$.
\end{lemma}

\begin{lemma}\label{lemma:KWIVENLG}
Dada uma instância intergênica rígida sem sinais $\mathcal{I}=((\pi,\breve\pi),(\iota,\breve\iota))$, para qualquer indel $\delta$ temos que $\Delta ib_1(\mathcal{I}, S = (\delta)) \ge -1$.
\end{lemma}

\begin{lemma}\label{lemma:IKBNJWMY}
Dada uma instância intergênica rígida com sinais $\mathcal{I}=((\pi,\breve\pi),(\iota,\breve\iota))$, para qualquer reversão $\rho$ temos que $\Delta ib_2(\mathcal{I}, S = (\rho)) \ge -2$.
\end{lemma}

\begin{lemma}\label{lemma:MYVALTSG}
Dada uma instância intergênica rígida com sinais $\mathcal{I}=((\pi,\breve\pi),(\iota,\breve\iota))$, para qualquer transposição $\tau$ temos que $\Delta ib_2(\mathcal{I}, S = (\tau)) \ge -3$.
\end{lemma}

\begin{lemma}\label{lemma:LSPSMYMM}
Dada uma instância intergênica rígida com sinais $\mathcal{I}=((\pi,\breve\pi),(\iota,\breve\iota))$, para qualquer move $\mu$ temos que $\Delta ib_2(\mathcal{I}, S = (\mu)) \ge -2$.
\end{lemma}

\begin{lemma}\label{lemma:KXIYYHHL}
Dada uma instância intergênica rígida com sinais $\mathcal{I}=((\pi,\breve\pi),(\iota,\breve\iota))$, para qualquer indel $\delta$ temos que $\Delta ib_2(\mathcal{I}, S = (\delta)) \ge -1$.
\end{lemma}

\begin{theorem}\label{theorem:MPFPKHQO}
Dada uma instância intergênica rígida sem sinais $\mathcal{I}=((\pi,\breve\pi),(\iota,\breve\iota))$, temos que:

\begin{tabular}{lll}
  $d_{\SbIR}(\mathcal{I})$      & $ \ge $ & $\frac{ib_1(\mathcal{I})}{2}$, \\ 
  $d_{\SbIRI}(\mathcal{I})$     & $ \ge $ & $\frac{ib_1(\mathcal{I})}{2}$, \\
  $d_{\SbIRM}(\mathcal{I})$     & $ \ge $ & $\frac{ib_1(\mathcal{I})}{2}$, \\
  $d_{\SbIRMI}(\mathcal{I})$    & $ \ge $ & $\frac{ib_1(\mathcal{I})}{2}$, \\
  $d_{\SbIRT}(\mathcal{I})$     & $ \ge $ & $\frac{ib_1(\mathcal{I})}{3}$, \\
  $d_{\SbIRTI}(\mathcal{I})$    & $ \ge $ & $\frac{ib_1(\mathcal{I})}{3}$, \\
  $d_{\SbIRTM}(\mathcal{I})$    & $ \ge $ & $\frac{ib_1(\mathcal{I})}{3}$  \\
  e $d_{\SbIRTMI}(\mathcal{I})$ & $ \ge $ & $\frac{ib_1(\mathcal{I})}{3}$. \\
\end{tabular}
\end{theorem}
\begin{proof}
Pela Obervação~\ref{remark:UDYJTHAH}, para transformar $(\pi,\breve\pi)$ em $(\iota,\breve\iota)$ é necessário remover os $ib_1(\mathcal{I})$ breakpoints tipo um de $\mathcal{I}$. Dessa forma, obtemos um limitante inferior para cada um dos modelos através da divisão de $ib_1(\mathcal{I})$ pela maior quantidade de breakpoints tipo um que podem ser removidos por um evento permitido no modelo de rearranjo. Os lemas~\ref{lemma:KFFPUBQG}, \ref{lemma:IUJZCMMV}, \ref{lemma:SYXLGTAP} e \ref{lemma:KWIVENLG} mostram a quantidade máxima de breakpoints tipo um que podem ser removidos de uma instância intergênica rígida sem sinais pelos eventos de reversão, transposição, move e indel, respectivamente. Logo, o teorema segue.
\end{proof}

\begin{theorem}\label{theorem:JDOIUJLE}
Dada uma instância intergênica rígida sem sinais $\mathcal{I}=((\pi,\breve\pi),(\iota,\breve\iota))$. Se $\mathcal{I}$ for balanceada, então temos que $d_{\SbIRTI}(\mathcal{I}) \ge \frac{ib_1(\mathcal{I})}{3}$. Caso contrário, temos que $d_{\SbIRTI}(\mathcal{I}) \ge \frac{ib_1(\mathcal{I}) + 2}{3}$.
\end{theorem}
\begin{proof}
Pela Obervação~\ref{remark:UDYJTHAH}, para transformar $(\pi,\breve\pi)$ em $(\iota,\breve\iota)$ é necessário remover os $ib_1(\mathcal{I})$ breakpoints tipo um de $\mathcal{I}$. Note que se $\mathcal{I}$ for balanceada, então podemos aplicar o Teorema~\ref{theorem:MPFPKHQO}. Caso contrário, sabemos que para ser possível transformar $(\pi,\breve\pi)$ em $(\iota,\breve\iota)$ pelo menos um indel deve ser utilizado. Pelo Lema~\ref{lemma:KWIVENLG}, temos que no máximo um breakpoint tipo um pode ser removido utilizando uma operação de indel. No melhor caso, após aplicar apenas um indel, $\mathcal{I}$  torna-se uma instância balanceada e um breakpoint tipo um é removido. Assim, restam $ib_1(\mathcal{I}) - 1$ breakpoints para serem removidos de $\mathcal{I}$. Considerando os eventos de reversão, transposição e indel, no máximo três breakpoints tipo um podem ser removidos por operação (Lemas~\ref{lemma:KFFPUBQG}, \ref{lemma:IUJZCMMV} e \ref{lemma:KWIVENLG}). Logo, pelo menos $\frac{ib_1(\mathcal{I}) - 1}{3}$ eventos de reversão, transposição ou indel são necessários para remover o restante dos breakpoints tipo um. Dessa forma, pelo menos $\frac{ib_1(\mathcal{I}) - 1}{3} + 1 = \frac{ib_1(\mathcal{I}) + 2}{3}$ eventos de reversão, transposição e indel são necessários para transformar $(\pi,\breve\pi)$ em $(\iota,\breve\iota)$, e o teorema segue.
\end{proof}

\begin{theorem}\label{theorem:NFVKZGKW}
Dada uma instância intergênica rígida com sinais $\mathcal{I}=((\pi,\breve\pi),(\iota,\breve\iota))$, temos que:

\begin{tabular}{lll}
  $d_{\SbIR}(\mathcal{I})$      & $ \ge $ & $\frac{ib_2(\mathcal{I})}{2}$, \\ 
  $d_{\SbIRI}(\mathcal{I})$     & $ \ge $ & $\frac{ib_2(\mathcal{I})}{2}$, \\
  $d_{\SbIRM}(\mathcal{I})$     & $ \ge $ & $\frac{ib_2(\mathcal{I})}{2}$, \\
  $d_{\SbIRMI}(\mathcal{I})$    & $ \ge $ & $\frac{ib_2(\mathcal{I})}{2}$, \\
  $d_{\SbIRT}(\mathcal{I})$     & $ \ge $ & $\frac{ib_2(\mathcal{I})}{3}$, \\
  $d_{\SbIRTI}(\mathcal{I})$    & $ \ge $ & $\frac{ib_2(\mathcal{I})}{3}$, \\
  $d_{\SbIRTM}(\mathcal{I})$    & $ \ge $ & $\frac{ib_2(\mathcal{I})}{3}$  \\
  e $d_{\SbIRTMI}(\mathcal{I})$ & $ \ge $ & $\frac{ib_2(\mathcal{I})}{3}$. \\
\end{tabular}
\end{theorem}
\begin{proof}
A prova é similar a descrita no Teorema~\ref{theorem:MPFPKHQO}, mas considerando os lemas~\ref{lemma:IKBNJWMY}, \ref{lemma:MYVALTSG}, \ref{lemma:LSPSMYMM} e \ref{lemma:KXIYYHHL}.
\end{proof}

Considerando o grafo de ciclos ponderado rígido criado a partir de uma instância intergênica rígida com sinais, é possível notar que o evento de reversão afeta duas arestas pretas do grafo e pode aumentar tanto o número de ciclos como também o número de ciclos balanceados. O evento de move também afeta duas arestas pretas do grafo, mas pode aumentar somente o número de ciclos balanceados no grafo. Já o evento de indel afeta apenas uma aresta preta do grafo e pode aumentar somente o número de ciclos balanceados no grafo. Dessa forma, dada uma instância intergênica rígida com sinais $\mathcal{I} = ((\pi,\breve\pi),(\iota,\breve\iota))$, temos que $\Delta c(G(\mathcal{I}), S=(\rho)) \in \{1,0,-1\}$ e $\Delta c_b(G(\mathcal{I}), S=(\rho)) \in \{1,0,-1\}$ para qualquer reversão $\rho$. De maneira similar, temos que $\Delta c(G(\mathcal{I}), S=(\mu)) = 0$ e $\Delta c_b(G(\mathcal{I}), S=(\mu)) \in \{2,1,0,{-1},{-2}\}$ para qualquer move $\mu$, e $\Delta c(G(\mathcal{I}), S=(\delta)) = 0$ e $\Delta c_b(G(\mathcal{I}), S=(\delta)) \in \{1,0,{-1}\}$ para qualquer indel $\delta$. Com isso, obtemos os seguintes limitantes inferiores.

\begin{theorem}\label{theorem:OCNPWYNL}
Dada uma instância intergênica rígida com sinais $\mathcal{I}=((\pi,\breve\pi),(\iota,\breve\iota))$, temos que:

\begin{tabular}{lll}
  $d_{\SbIRM}(\mathcal{I})$     & $ \ge $ & ${n + 1} - \frac{c(G(\mathcal{I})) + c_b(G(\mathcal{I}))}{2}$, \\
  e $d_{\SbIRMI}(\mathcal{I})$    & $ \ge $ & ${n + 1} - \frac{c(G(\mathcal{I})) + c_b(G(\mathcal{I}))}{2}$. \\
\end{tabular}
\end{theorem}
\begin{proof}
Note que para atingir o genoma alvo é necessário aumentar tanto o número de ciclos quanto o de ciclos balanceados em $G(\mathcal{I})$ para $n+1$ (Observação~\ref{remark:WVLFPRDL}). Reversões, moves e indels podem aumentar $c(G(\mathcal{I})) + c_b(G(\mathcal{I}))$ em no máximo duas unidades, então o Teorema segue.
\end{proof}




% ------------------------------------------------------------------ %
\section{Análise de Complexidade}
% ------------------------------------------------------------------ %

Nesta seção realizamos uma análise de complexidade considerando as variações dos problemas resultantes dos modelos de rearranjo investigados neste capítulo.

Inicialmente descrevemos a versão de decisão da variação sem sinais dos problemas de Ordenação de Permutações por Reversões (\SbR) e Ordenação de Permutações por Reversões e Transposições(\SbRT), que pertencem à classe NP-difícil~\cite{1999-caprara,2019b-oliveira-etal}.

\begin{decision}
  \problemtitle{Ordenação de Permutações por Reversões (\SbR) (Versão de Decisão)}
  \probleminput{Uma instância clássica sem sinais $\mathcal{I}=(\pi,\iota)$ e um número natural $d$.}
  \problemquestion{Existe uma sequência de eventos de rearranjo $S$, com base no modelo de rearranjo $\mathcal{M}=\{\rho\}$, capaz de transformar $\pi$ em $\iota$, tal que $|S| \le d$?}
\end{decision}

\begin{decision}
  \problemtitle{Ordenação de Permutações por Reversões e Transposições (\SbRT) (Versão de Decisão)}
  \probleminput{Uma instância clássica sem sinais $\mathcal{I}=(\pi,\iota)$ e um número natural $d$.}
  \problemquestion{Existe uma sequência de eventos de rearranjo $S$, com base no modelo de rearranjo $\mathcal{M}=\{\rho,\tau\}$, capaz de transformar $\pi$ em $\iota$, tal que $|S| \le d$?}
\end{decision}

A seguir descrevemos a versão de decisão das variações sem sinais dos problemas que investigaremos neste capítulo.

\begin{decision}
  \problemtitle{\SbIR (Versão de Decisão)}
  \probleminput{Uma instância intergênica rígida sem sinais $\mathcal{I}=((\pi,\breve\pi),(\iota,\breve\iota))$ e um número natural $t$.}
  \problemquestion{Existe uma sequência de eventos de rearranjo $S$, com base no modelo de rearranjo $\mathcal{M}=\{\rho\}$, capaz de transformar $(\pi,\breve\pi)$ em $(\iota,\breve\iota)$, tal que $|S| \le t$?}
\end{decision}

\begin{decision}
  \problemtitle{\SbIRI (Versão de Decisão)}
  \probleminput{Uma instância intergênica rígida sem sinais $\mathcal{I}=((\pi,\breve\pi),(\iota,\breve\iota))$ e um número natural $t$.}
  \problemquestion{Existe uma sequência de eventos de rearranjo $S$, com base no modelo de rearranjo $\mathcal{M}=\{\rho,\delta\}$, capaz de transformar $(\pi,\breve\pi)$ em $(\iota,\breve\iota)$, tal que $|S| \le t$?}
\end{decision}

\begin{decision}
  \problemtitle{\SbIRM (Versão de Decisão)}
  \probleminput{Uma instância intergênica rígida sem sinais $\mathcal{I}=((\pi,\breve\pi),(\iota,\breve\iota))$ e um número natural $t$.}
  \problemquestion{Existe uma sequência de eventos de rearranjo $S$, com base no modelo de rearranjo $\mathcal{M}=\{\rho,\mu\}$, capaz de transformar $(\pi,\breve\pi)$ em $(\iota,\breve\iota)$, tal que $|S| \le t$?}
\end{decision}

\begin{decision}
  \problemtitle{\SbIRMI (Versão de Decisão)}
  \probleminput{Uma instância intergênica rígida sem sinais $\mathcal{I}=((\pi,\breve\pi),(\iota,\breve\iota))$ e um número natural $t$.}
  \problemquestion{Existe uma sequência de eventos de rearranjo $S$, com base no modelo de rearranjo $\mathcal{M}=\{\rho,\mu,\delta\}$, capaz de transformar $(\pi,\breve\pi)$ em $(\iota,\breve\iota)$, tal que $|S| \le t$?}
\end{decision}

\begin{decision}
  \problemtitle{\SbIRT (Versão de Decisão)}
  \probleminput{Uma instância intergênica rígida sem sinais $\mathcal{I}=((\pi,\breve\pi),(\iota,\breve\iota))$ e um número natural $t$.}
  \problemquestion{Existe uma sequência de eventos de rearranjo $S$, com base no modelo de rearranjo $\mathcal{M}=\{\rho,\tau\}$, capaz de transformar $(\pi,\breve\pi)$ em $(\iota,\breve\iota)$, tal que $|S| \le t$?}
\end{decision}

\begin{decision}
  \problemtitle{\SbIRTI (Versão de Decisão)}
  \probleminput{Uma instância intergênica rígida sem sinais $\mathcal{I}=((\pi,\breve\pi),(\iota,\breve\iota))$ e um número natural $t$.}
  \problemquestion{Existe uma sequência de eventos de rearranjo $S$, com base no modelo de rearranjo $\mathcal{M}=\{\rho,\tau,\delta\}$, capaz de transformar $(\pi,\breve\pi)$ em $(\iota,\breve\iota)$, tal que $|S| \le t$?}
\end{decision}

\begin{decision}
  \problemtitle{\SbIRTM (Versão de Decisão)}
  \probleminput{Uma instância intergênica rígida sem sinais $\mathcal{I}=((\pi,\breve\pi),(\iota,\breve\iota))$ e um número natural $t$.}
  \problemquestion{Existe uma sequência de eventos de rearranjo $S$, com base no modelo de rearranjo $\mathcal{M}=\{\rho,\tau,\mu\}$, capaz de transformar $(\pi,\breve\pi)$ em $(\iota,\breve\iota)$, tal que $|S| \le t$?}
\end{decision}

\begin{decision}
  \problemtitle{\SbIRTMI (Versão de Decisão)}
  \probleminput{Uma instância intergênica rígida sem sinais $\mathcal{I}=((\pi,\breve\pi),(\iota,\breve\iota))$ e um número natural $t$.}
  \problemquestion{Existe uma sequência de eventos de rearranjo $S$, com base no modelo de rearranjo $\mathcal{M}=\{\rho,\tau,\mu,\delta\}$, capaz de transformar $(\pi,\breve\pi)$ em $(\iota,\breve\iota)$, tal que $|S| \le t$?}
\end{decision}


\begin{theorem}\label{theorem:YARJETHG}
Os problemas \SbIR{}, \SbIRI{}, \SbIRM{} e \SbIRMI{} em instâncias intergênicas rígidas sem sinais pertencem à classe NP-difícil.
\end{theorem}
\begin{proof}
Dada uma instância clássica sem sinais $\mathcal{I}=(\pi,\iota)$ e um valor $d$ para a versão de decisão do problema \SbR, criaremos uma instância intergênica rígida sem sinais $\mathcal{I'}=((\pi',\breve\pi'),(\iota',\breve\iota'))$ e um valor $t$ para a versão de decisão do problema \SbIR{}, \SbIRI{}, \SbIRM{} ou \SbIRMI{} da seguinte maneira: i) $\pi' = \pi$, ii) $\iota' = \iota$, iii) $\breve\pi' = \breve\iota' = (0,0,\dots,0)$ e iv) $t = d$. Agora mostramos que a instância $(\mathcal{I},d)$ do problema \SbR{} é satisfeita se e somente se a instância $(\mathcal{I'},t)$ do problema \SbIR{}, \SbIRI{}, \SbIRM{} ou \SbIRMI{} é satisfeita.

($\Rightarrow$) Suponha que existe uma sequência $S$ com $d$ reversões, tal que $\pi \cdot S = \iota$. Considere a sequência $S'$ criada a partir da sequência $S$ mapeando cada reversão $\rho^{(i,j)}$ em uma reversão intergênica $\rho^{(i,j)}_{(0,0)}$. Note que $(\pi,\breve\pi) \cdot S' = (\iota,\breve\iota)$ e $|S| = t = d$, uma vez que o tamanho de todas as regiões intergênicas no genoma de origem e alvo é zero.

($\Leftarrow$) Agora suponha que existe uma sequência $S'$ com $t$ eventos de rearranjo, tal que $(\pi,\breve\pi) \cdot S' = (\iota,\breve\iota)$. Primeiramente mostraremos que a sequência $S'$ é composta exclusivamente por reversões intergênicas. Suponha por contradição que $S'$ é uma sequência ótima para a instância $(\mathcal{I'},t)$ do problema \SbIR{}, \SbIRI{}, \SbIRM{} ou \SbIRMI{} e não é composta exclusivamente por reversões intergênicas. Neste caso criaremos uma sequência $S''$, tal que  $|S''| < |S'|$ e $(\pi,\breve\pi) \cdot S'' = (\iota,\breve\iota)$. Para cada reversão intergênica $\rho^{(i,j)}_{(x,y)}$ de $S'$ adicione em $S''$ a reversão intergênica $\rho^{(i,j)}_{(0,0)}$. Note que os eventos de move e indel não afetam a ordem dos genes. Além disso, pela construção de $\mathcal{I'}$, temos que $\breve\pi' = \breve\iota'$. Logo, $(\pi,\breve\pi) \cdot S'' = (\iota,\breve\iota)$, o que contradiz a suposição de que $S'$ é uma sequência ótima para a instância $(\mathcal{I'},t)$ do problema \SbIR{}, \SbIRI{}, \SbIRM{} ou \SbIRMI{}. Sabendo que $S'$ é composta exclusivamente por reversões intergênicas, considere a sequência $S$ criada a partir da sequência $S'$ mapeando cada reversão intergênica $\rho^{(i,j)}_{(x,y)}$ em uma reversão $\rho^{(i,j)}$. Note que $\pi \cdot S = \iota$ e $|S| = d = t$. Logo, o teorema segue.
\end{proof}

\begin{theorem}\label{theorem:RDOZOOIB}
Os problemas \SbIRT, \SbIRTI, \SbIRTM{} e \SbIRTMI{} em instâncias intergênicas rígidas sem sinais pertencem à classe NP-difícil.
\end{theorem}
\begin{proof}
A prova é similar a descrita no Teorema~\ref{theorem:YARJETHG}, mas utilizando uma redução da versão de decisão da variação sem sinais do problema \SbRT{} e considerando que a sequência $S'$ para a instância $(\mathcal{I'},t)$ do problema \SbIRT{}, \SbIRTI{}, \SbIRTM{} ou \SbIRTMI{} é composta por reversões intergênicas e transposições intergênicas ao invés de reversões intergênicas exclusivamente.
\end{proof}

% ------------------------------------------------------------------ %
\section{Instâncias Intergênicas Rígidas sem Sinais}
% ------------------------------------------------------------------ %

Nesta seção apresentamos algoritmos para os problemas resultantes dos modelos de rearranjo investigados neste capítulo e considerando uma representação intergênica rígida sem sinais de um genoma. Inicialmente iremos apresentar alguns lemas que serão utilizados por múltiplos algoritmos. 

\begin{lemma}\label{lemma:WYEZMYTM}
Dada uma instância intergênica rígida sem sinais $\mathcal{I}=((\pi,\breve\pi),(\iota,\breve\iota))$, tal que $\sum_{i=1}^{n+1}\breve\pi_i \ge \sum_{i=1}^{n+1}\breve\iota_i$ e $ib_1(\mathcal{I}) > 1$, então sempre é possível encontrar um par conectado de breakpoints.
\end{lemma}
\begin{proof}
Como $ib_1(\mathcal{I}) > 1$, então deve existir pelo menos um par de breakpoints tipo um $(\pi_i,\pi_{i+1})$ e $(\pi_j,\pi_{j+1})$. Agora vamos mostrar que pelo menos um desses pares de breakpoints está conectado. Suponha por contradição que existe uma instância intergênica rígida sem sinais $\mathcal{I}=((\pi,\breve\pi),(\iota,\breve\iota))$, tal que $\sum_{i=1}^{n+1}\breve\pi_i \ge \sum_{i=1}^{n+1}\breve\iota_i$, $ib_1(\mathcal{I}) > 1$, e não existe nenhum par conectado de breakpoints em $\mathcal{I}$. Com isso, temos que avaliar dois possíveis casos:
\begin{itemize}
  \item Para todo par de breakpoints tipo um $(\pi_i,\pi_{i+1})$ e $(\pi_j,\pi_{j+1})$, os elementos $(\pi_i,\pi_{i+1})$, $(\pi_j,\pi_{j+1})$, $(\pi_i,\pi_{j})$, $(\pi_i,\pi_{j+1})$, $(\pi_{i+1},\pi_{j})$ e $(\pi_{i+1},\pi_{j+1})$ não são consecutivos na permutação identidade $\iota$. Entretanto, isso não pode acontecer uma vez que, por construção da instância, $\pi$ e $\iota$ são permutações que compartilham o mesmo conjunto de valores.
  \item Para todo par de breakpoints tipo um $(\pi_i,\pi_{i+1})$ e $(\pi_j,\pi_{j+1})$, a quantidade de nucleotídeos nas regiões intergênicas $\breve\pi_{i+1}$ e $\breve\pi_{j+1}$ não é sufuciente para remover qualquer breakpoint, ou seja, $\breve\pi_{i+1} + \breve\pi_{j+1} < \breve\iota_k$ onde é tamanho da região intergênica entre o par de elementos consecutivos correspondentes na permutação identidade $\iota$. Entretanto, se isso for verdade temos que $\sum_{i=1}^{n+1}\breve\pi_i < \sum_{i=1}^{n+1}\breve\iota_i$, o que contradiz a suposição inicial de que $\sum_{i=1}^{n+1}\breve\pi_i \ge \sum_{i=1}^{n+1}\breve\iota_i$.
\end{itemize}
\end{proof}

\begin{lemma}\label{lemma:WSPRPLAH}
Não existe uma instância intergênica rígida balanceada sem sinais $\mathcal{I}=((\pi,\breve\pi),(\iota,\breve\iota))$, tal que $ib_1(\mathcal{I}) = 1$.
\end{lemma}
\begin{proof}
Como $\mathcal{I}$ é uma instância intergênica rígida balanceada, temos que a seguinte condição é verdadeira: $\sum_{i=1}^{n+1}\breve\pi_i = \sum_{i=1}^{n+1}\breve\iota_i$. Agora vamos mostrar que não existe tal instância em que $ib_1(\mathcal{I}) = 1$. Suponha por contradição que existe uma instância intergênica rígida balanceada sem sinais $\mathcal{I}=((\pi,\breve\pi),(\iota,\breve\iota))$ com $ib_1(\mathcal{I}) = 1$. Como $ib_1(\mathcal{I}) = 1$, então o breakpoint tipo um $(\pi_i,\pi_{i+1})$ obrigatoriamente deve ser forte. Caso contrário, teríamos que $ib_1(\mathcal{I}) > 1$. Logo, temos que $\breve\pi_{i+1} \ne \breve\iota_{i+1}$, o que implica que $\sum_{i=1}^{n+1}\breve\pi_i \ne \sum_{i=1}^{n+1}\breve\iota_i$ e contradiz a suposição inicial de que $\mathcal{I}$ é balanceada.
\end{proof}


% ------------------------------------------------------------------ %
\subsection{Reversão}
% ------------------------------------------------------------------ %

Nesta seção apresentaremos um algoritmo de aproximação com fator $4$ para a variação sem sinais do problema \SbIR{}.

\begin{lemma}\label{lemma:IMYFBWDY}
Dada uma instância intergênica rígida sem sinais $\mathcal{I}=((\pi,\breve\pi),(\iota,\breve\iota))$ e sejam $(\pi_i,\pi_{i+1})$ e $(\pi_j,\pi_{j+1})$ breakpoints conectados, então é possível remover pelo menos um breakpoint tipo um de $\mathcal{I}$ utilizando no máximo duas reversões.
\end{lemma}
\begin{proof}
Sem perda de generalidade assuma que $i < j$, como os breakpoints $(\pi_i,\pi_{i+1})$ e $(\pi_j,\pi_{j+1})$ estão conectados, por definição, uma das seguintes possibilidades deve ocorrer:
\begin{enumerate}[i.]
    \item O par de elementos $(\pi_i,\pi_{j})$ ou $(\pi_{i+1},\pi_{j+1})$ não formam uma adjacência intergênica, são consecutivos em $\iota$ e $\breve\pi_{i+1} + \breve\pi_{j+1} \ge \breve\iota_k$, onde $\breve\iota_k$ é o tamanho da região intergênica entre o par de elementos consecutivos em $\iota$. Neste caso precisamos aplicar apenas uma reversão $\rho^{(i+1,j)}_{(x,y)}$ para posicionar o elemento $\pi_{j}$ no lado direito do elemento $\pi_{i}$ ou posicionar o elemento $\pi_{i+1}$ no lado esquerdo do elemento $\pi_{j+1}$. Como $\breve\pi_{i+1} + \breve\pi_{j+1} \ge \breve\iota_k$, então sempre é possível atribuir valores para os parâmetros $x$ e $y$ de forma que o tamanho da região intergênica entre os elementos consecutivos, posicionados pela reversão, tenha o mesmo tamanho do que a região intergênica entre os mesmo elementos no genoma alvo (Figura~\ref{figure:EMTPDAVS}a)).  
    \item O par  de elementos $(\pi_i,\pi_{j+1})$ não formam uma adjacência intergênica, são consecutivos em $\iota$ e $\breve\pi_{i+1} + \breve\pi_{j+1} \ge \breve\iota_k$, onde $\breve\iota_k$ é o tamanho da região intergênica entre o par de elementos consecutivos em $\iota$. Inicialmente iremos provar que deve existir um breakpoint tipo um $(\pi_k,\pi_{k+1})$, tal que $k < i$ ou $k > j$. Suponha por contradição que não existe um breakpoint tipo um $(\pi_k,\pi_{k+1})$, tal que $k < i$ ou $k > j$. Isso implica que os segmentos $(\pi_0,\dots,\pi_i)$ e $(\pi_{j+1},\dots,\pi_{n+1})$ são compostos por elementos consecutivos, ou seja, não existe breakpoints tipo um entre os elementos de ambos os segmentos. Sabemos também que $\pi_i$ e $\pi_{j+1}$ são elementos consecutivos em $\iota$. Entretando, se ambas afirmações forem verdadeiras, isso implica que os valores dos elementos no segmento $(\pi_{i+1},\dots,\pi_j)$ não estão presentes em $\iota$, isso contradiz a construção da instância $\mathcal{I}$, já que $\pi$ e $\iota$ são permutações que compartilham o mesmo conjunto de valores. Após identificar o breakpoint tipo um $(\pi_k,\pi_{k+1})$, temos duas possibilidades. Se $k < i$, aplicamos a reversão $\rho^{(k+1,i)}_{(0,\breve\pi_{i+1})}$, que não gera nenhum novo breakpoint, e obtemos o caso $(i)$ (Figura~\ref{figure:EMTPDAVS}b)). Se $k > j$, aplicamos a reversão $\rho^{(j+1,k)}_{(0,\breve\pi_{k+1})}$, que também não gera nenhum novo breakpoint, e obtemos o caso $(i)$ (Figura~\ref{figure:EMTPDAVS}c)).
    \item O par de elementos $(\pi_{i+1},\pi_{j})$ não formam uma adjacência intergênica, são consecutivos em $\iota$ e $\breve\pi_{i+1} + \breve\pi_{j+1} \ge \breve\iota_k$, onde $\breve\iota_k$ é o tamanho da região intergênica entre o par de elementos consecutivos em $\iota$. Inicialmente vamos provar que deve existir um breakpoint tipo um $(\pi_k,\pi_{k+1})$, tal que $i < k < j$. Suponha por contradição que não existe um breakpoint tipo um $(\pi_k,\pi_{k+1})$, tal que $i < k < j$. Isso implica que o segmento $(\pi_{i+1},\pi_{i+2},\dots,\pi_j)$ é composto por pelo menos três elementos consecutivos, ou seja, não existe breakpoints tipo um entre os elementos do segmento. Caso contrário, $(\pi_{i+1},\pi_{j})$ seria uma adjacência intergênica. Sabemos também que $\pi_{i+1}$ e $\pi_{j}$ são elementos consecutivos em $\iota$. Entretanto, se ambas afirmações forem verdadeiras, isso implica que $|\pi_j - \pi_{i+1}| > 1$ e contradiz a suposição de que $\pi_{i+1}$ e $\pi_{j}$ são elementos consecutivos em $\iota$. Após identificar o breakpoint tipo um $(\pi_k,\pi_{k+1})$, aplicamos a reversão $\rho^{(i+1,k)}_{(0,\breve\pi_{k+1})}$, que não gera nenhum novo breakpoint, e obtemos o caso $(i)$ (Figura~\ref{figure:EMTPDAVS}d)).
    \item O par de elementos $(\pi_{i},\pi_{i+1})$ ou $(\pi_{j},\pi_{j+1})$ não formam uma adjacência intergênica, são consecutivos em $\iota$ e $\breve\pi_{i+1} + \breve\pi_{j+1} \ge \breve\iota_k$, onde $\breve\iota_k$ é o tamanho da região intergênica entre o par de elementos consecutivos em $\iota$. Inicialmente aplicamos a reversão $\rho^{(i+1, j)}_{(0,\breve\pi_{j+1})}$, que não modifica o tamanho das regiões intergênicas $\breve\pi_{i+1}$ e $\breve\pi_{j+1}$, e obtemos o caso $(i) (Figura~\ref{figure:EMTPDAVS}e))$.
\end{enumerate}
Note que o caso $(i)$ aplica apenas um reversão e remove pelo menos um breakpoint tipo um. Os casos $(ii)$, $(iii)$ e $(iv)$ aplicam inicialmente uma reversão que não remove nenhum breakpoint tipo um, mas garatem que nenhum novo breakpoint é gerado e o caso $(i)$ poderá ser aplicado em seguida. No pior caso duas reversões são aplicadas e pelo menos um breakpoint tipo um é removido de $\mathcal{I}$. Logo, o lema segue. 
\end{proof}

\begin{figure}[!tbh]
  \resizebox{\linewidth}{!}{
    \centering
    \begin{tikzpicture}
      %%%%% a)

      \node[draw=none,fill=none, minimum height=1cm, minimum width=1cm] at (-3.0, 1.0) {a)};

      % intergenic regions
      \node[rectangle,draw,fill=white!20, minimum height=.7cm, minimum width=3cm, rounded corners=5pt] at (0.0, 0.0) {$\cdots$};
      \node[rectangle,draw,fill=white!20, minimum height=.7cm, minimum width=1cm, rounded corners=5pt] at (3.0, 0.0) {$\breve\pi_{i+1}$};
      \node[rectangle,draw,fill=white!20, minimum height=.7cm, minimum width=1cm, rounded corners=5pt] at (5.0, 0.0) {$\cdots$};
      \node[rectangle,draw,fill=white!20, minimum height=.7cm, minimum width=1cm, rounded corners=5pt] at (7.0, 0.0) {$\breve\pi_{j+1}$};
      \node[rectangle,draw,fill=white!20, minimum height=.7cm, minimum width=3cm, rounded corners=5pt] at (10.0, 0.0) {$\cdots$};

      % genes
      \node[draw, circle, minimum height=1.2cm, minimum width=1.2cm, fill=white!20] at ({(2*0 - 2)}, 0.0) {$\pi_{0}$};
      \node[draw, circle, minimum height=1.2cm, minimum width=1.2cm, fill=black!20] at ({(2*2 - 2)}, 0.0) {$\pi_{i}$};
      \node[draw, circle, minimum height=1.2cm, minimum width=1.2cm, fill=black!50] at ({(2*3 - 2)}, 0.0) {$\pi_{i+1}$};
      \node[draw, circle, minimum height=1.2cm, minimum width=1.2cm, fill=black!20] at ({(2*4 - 2)}, 0.0) {$\pi_{j}$};
      \node[draw, circle, minimum height=1.2cm, minimum width=1.2cm, fill=black!50] at ({(2*5 - 2)}, 0.0) {$\pi_{j+1}$};
      \node[draw, circle, minimum height=1.2cm, minimum width=1.2cm, fill=white!20] at ({(2*7 - 2)}, 0.0) {$\pi_{n+1}$};

      % indexes
      \draw (3.0, 0.7) -- (3.0, 1.0);
      \draw (7.0, 0.7) -- (7.0, 1.0);
      \draw (3.0, 1.0) -- (7.0, 1.0);
      \node[draw=none,fill=none] at (2.75, 0.85) {$\rho_1$};

      %%%%% b)

      \node[draw=none,fill=none, minimum height=1cm, minimum width=1cm] at (-3.0, -2.0) {b)};

      % intergenic regions
      \node[rectangle,draw,fill=white!20, minimum height=.7cm, minimum width=1cm, rounded corners=5pt] at (-1.0, -3.0) {$\cdots$};
      \node[rectangle,draw,fill=white!20, minimum height=.7cm, minimum width=1cm, rounded corners=5pt] at (1.0, -3.0) {$\breve\pi_{k+1}$};
      \node[rectangle,draw,fill=white!20, minimum height=.7cm, minimum width=1cm, rounded corners=5pt] at (3.0, -3.0) {$\cdots$};
      \node[rectangle,draw,fill=white!20, minimum height=.7cm, minimum width=1cm, rounded corners=5pt] at (5.0, -3.0) {$\breve\pi_{i+1}$};
      \node[rectangle,draw,fill=white!20, minimum height=.7cm, minimum width=1cm, rounded corners=5pt] at (7.0, -3.0) {$\cdots$};
      \node[rectangle,draw,fill=white!20, minimum height=.7cm, minimum width=1cm, rounded corners=5pt] at (9.0, -3.0) {$\breve\pi_{j+1}$};
      \node[rectangle,draw,fill=white!20, minimum height=.7cm, minimum width=1cm, rounded corners=5pt] at (11.0, -3.0) {$\cdots$};

      % genes
      \node[draw, circle, minimum height=1.2cm, minimum width=1.2cm, fill=white!20] at ({(2*0 - 2)}, -3.0) {$\pi_{0}$};
      \node[draw, circle, minimum height=1.2cm, minimum width=1.2cm, fill=white!20] at ({(2*1 - 2)}, -3.0) {$\pi_{k}$};
      \node[draw, circle, minimum height=1.2cm, minimum width=1.2cm, fill=white!20] at ({(2*2 - 2)}, -3.0) {$\pi_{k+1}$};
      \node[draw, circle, minimum height=1.2cm, minimum width=1.2cm, fill=black!20] at ({(2*3 - 2)}, -3.0) {$\pi_{i}$};
      \node[draw, circle, minimum height=1.2cm, minimum width=1.2cm, fill=white!20] at ({(2*4 - 2)}, -3.0) {$\pi_{i+1}$};
      \node[draw, circle, minimum height=1.2cm, minimum width=1.2cm, fill=white!20] at ({(2*5 - 2)}, -3.0) {$\pi_{j}$};
      \node[draw, circle, minimum height=1.2cm, minimum width=1.2cm, fill=black!20] at ({(2*6 - 2)}, -3.0) {$\pi_{j+1}$};
      \node[draw, circle, minimum height=1.2cm, minimum width=1.2cm, fill=white!20] at ({(2*7 - 2)}, -3.0) {$\pi_{n+1}$};

      % indexes
      \draw (1.0, -2.3) -- (1.0, -2.0);
      \draw (5.0, -2.3) -- (5.0, -2.0);
      \draw (1.0, -2.0) -- (5.0, -2.0);
      \node[draw=none,fill=none] at (0.75, -2.15) {$\rho_1$};

      \draw (9.0, -1.8) -- (9.0, -1.5);
      \draw (1.0, -1.8) -- (1.0, -1.5);
      \draw (1.0, -1.5) -- (9.0, -1.5);
      \node[draw=none,fill=none] at (0.75, -1.65) {$\rho_2$};

      %%%%% c)

      \node[draw=none,fill=none, minimum height=1cm, minimum width=1cm] at (-3.0, -5.0) {c)};

      % intergenic regions
      \node[rectangle,draw,fill=white!20, minimum height=.7cm, minimum width=1cm, rounded corners=5pt] at (-1.0, -6.0) {$\cdots$};
      \node[rectangle,draw,fill=white!20, minimum height=.7cm, minimum width=1cm, rounded corners=5pt] at (1.0, -6.0) {$\breve\pi_{i+1}$};
      \node[rectangle,draw,fill=white!20, minimum height=.7cm, minimum width=1cm, rounded corners=5pt] at (3.0, -6.0) {$\cdots$};
      \node[rectangle,draw,fill=white!20, minimum height=.7cm, minimum width=1cm, rounded corners=5pt] at (5.0, -6.0) {$\breve\pi_{j+1}$};
      \node[rectangle,draw,fill=white!20, minimum height=.7cm, minimum width=1cm, rounded corners=5pt] at (7.0, -6.0) {$\cdots$};
      \node[rectangle,draw,fill=white!20, minimum height=.7cm, minimum width=1cm, rounded corners=5pt] at (9.0, -6.0) {$\breve\pi_{k+1}$};
      \node[rectangle,draw,fill=white!20, minimum height=.7cm, minimum width=1cm, rounded corners=5pt] at (11.0, -6.0) {$\cdots$};

      % genes
      \node[draw, circle, minimum height=1.2cm, minimum width=1.2cm, fill=white!20] at ({(2*0 - 2)}, -6.0) {$\pi_{0}$};
      \node[draw, circle, minimum height=1.2cm, minimum width=1.2cm, fill=black!20] at ({(2*1 - 2)}, -6.0) {$\pi_{i}$};
      \node[draw, circle, minimum height=1.2cm, minimum width=1.2cm, fill=white!20] at ({(2*2 - 2)}, -6.0) {$\pi_{i+1}$};
      \node[draw, circle, minimum height=1.2cm, minimum width=1.2cm, fill=white!20] at ({(2*3 - 2)}, -6.0) {$\pi_{j}$};
      \node[draw, circle, minimum height=1.2cm, minimum width=1.2cm, fill=black!20] at ({(2*4 - 2)}, -6.0) {$\pi_{j+1}$};
      \node[draw, circle, minimum height=1.2cm, minimum width=1.2cm, fill=white!20] at ({(2*5 - 2)}, -6.0) {$\pi_{k}$};
      \node[draw, circle, minimum height=1.2cm, minimum width=1.2cm, fill=white!20] at ({(2*6 - 2)}, -6.0) {$\pi_{k+1}$};
      \node[draw, circle, minimum height=1.2cm, minimum width=1.2cm, fill=white!20] at ({(2*7 - 2)}, -6.0) {$\pi_{n+1}$};

      % indexes
      \draw (5.0, -5.3) -- (5.0, -5.0);
      \draw (9.0, -5.3) -- (9.0, -5.0);
      \draw (5.0, -5.0) -- (9.0, -5.0);
      \node[draw=none,fill=none] at (9.35, -5.2) {$\rho_1$};

      \draw (9.0, -4.8) -- (9.0, -4.5);
      \draw (1.0, -4.8) -- (1.0, -4.5);
      \draw (1.0, -4.5) -- (9.0, -4.5);
      \node[draw=none,fill=none] at (9.35, -4.7) {$\rho_2$};

      %%%%% d)

      \node[draw=none,fill=none, minimum height=1cm, minimum width=1cm] at (-3.0, -8.0) {d)};

      % intergenic regions
      \node[rectangle,draw,fill=white!20, minimum height=.7cm, minimum width=1cm, rounded corners=5pt] at (-1.0, -9.0) {$\cdots$};
      \node[rectangle,draw,fill=white!20, minimum height=.7cm, minimum width=1cm, rounded corners=5pt] at (1.0, -9.0) {$\breve\pi_{i+1}$};
      \node[rectangle,draw,fill=white!20, minimum height=.7cm, minimum width=1cm, rounded corners=5pt] at (3.0, -9.0) {$\cdots$};
      \node[rectangle,draw,fill=white!20, minimum height=.7cm, minimum width=1cm, rounded corners=5pt] at (5.0, -9.0) {$\breve\pi_{k+1}$};
      \node[rectangle,draw,fill=white!20, minimum height=.7cm, minimum width=1cm, rounded corners=5pt] at (7.0, -9.0) {$\cdots$};
      \node[rectangle,draw,fill=white!20, minimum height=.7cm, minimum width=1cm, rounded corners=5pt] at (9.0, -9.0) {$\breve\pi_{j+1}$};
      \node[rectangle,draw,fill=white!20, minimum height=.7cm, minimum width=1cm, rounded corners=5pt] at (11.0, -9.0) {$\cdots$};

      % genes
      \node[draw, circle, minimum height=1.2cm, minimum width=1.2cm, fill=white!20] at ({(2*0 - 2)}, -9.0) {$\pi_{0}$};
      \node[draw, circle, minimum height=1.2cm, minimum width=1.2cm, fill=white!20] at ({(2*1 - 2)}, -9.0) {$\pi_{i}$};
      \node[draw, circle, minimum height=1.2cm, minimum width=1.2cm, fill=black!20] at ({(2*2 - 2)}, -9.0) {$\pi_{i+1}$};
      \node[draw, circle, minimum height=1.2cm, minimum width=1.2cm, fill=white!20] at ({(2*3 - 2)}, -9.0) {$\pi_{k}$};
      \node[draw, circle, minimum height=1.2cm, minimum width=1.2cm, fill=white!20] at ({(2*4 - 2)}, -9.0) {$\pi_{k+1}$};
      \node[draw, circle, minimum height=1.2cm, minimum width=1.2cm, fill=black!20] at ({(2*5 - 2)}, -9.0) {$\pi_{j}$};
      \node[draw, circle, minimum height=1.2cm, minimum width=1.2cm, fill=white!20] at ({(2*6 - 2)}, -9.0) {$\pi_{j+1}$};
      \node[draw, circle, minimum height=1.2cm, minimum width=1.2cm, fill=white!20] at ({(2*7 - 2)}, -9.0) {$\pi_{n+1}$};

      % indexes
      \draw (1.0, -8.3) -- (1.0, -8.0);
      \draw (4.9, -8.3) -- (4.9, -8.0);
      \draw (1.0, -8.0) -- (4.9, -8.0);
      \node[draw=none,fill=none] at (0.75, -8.15) {$\rho_1$};

      \draw (5.1, -8.3) -- (5.1, -8.0);
      \draw (9.0, -8.3) -- (9.0, -8.0);
      \draw (5.1, -8.0) -- (9.0, -8.0);
      \node[draw=none,fill=none] at (9.35, -8.15) {$\rho_2$};

      %%%%% e)

      \node[draw=none,fill=none, minimum height=1cm, minimum width=1cm] at (-3.0, -11.0) {e)};

      % intergenic regions
      \node[rectangle,draw,fill=white!20, minimum height=.7cm, minimum width=3cm, rounded corners=5pt] at (0.0, -12.0) {$\cdots$};
      \node[rectangle,draw,fill=white!20, minimum height=.7cm, minimum width=1cm, rounded corners=5pt] at (3.0, -12.0) {$\breve\pi_{i+1}$};
      \node[rectangle,draw,fill=white!20, minimum height=.7cm, minimum width=1cm, rounded corners=5pt] at (5.0, -12.0) {$\cdots$};
      \node[rectangle,draw,fill=white!20, minimum height=.7cm, minimum width=1cm, rounded corners=5pt] at (7.0, -12.0) {$\breve\pi_{j+1}$};
      \node[rectangle,draw,fill=white!20, minimum height=.7cm, minimum width=3cm, rounded corners=5pt] at (10.0, -12.0) {$\cdots$};

      % genes
      \node[draw, circle, minimum height=1.2cm, minimum width=1.2cm, fill=white!20] at ({(2*0 - 2)}, -12.0) {$\pi_{0}$};
      \node[draw, circle, minimum height=1.2cm, minimum width=1.2cm, fill=black!50] at ({(2*2 - 2)}, -12.0) {$\pi_{i}$};
      \node[draw, circle, minimum height=1.2cm, minimum width=1.2cm, fill=black!50] at ({(2*3 - 2)}, -12.0) {$\pi_{i+1}$};
      \node[draw, circle, minimum height=1.2cm, minimum width=1.2cm, fill=black!20] at ({(2*4 - 2)}, -12.0) {$\pi_{j}$};
      \node[draw, circle, minimum height=1.2cm, minimum width=1.2cm, fill=black!20] at ({(2*5 - 2)}, -12.0) {$\pi_{j+1}$};
      \node[draw, circle, minimum height=1.2cm, minimum width=1.2cm, fill=white!20] at ({(2*7 - 2)}, -12.0) {$\pi_{n+1}$};

      % indexes
      \draw (3.0, -11.3) -- (3.0, -11.0);
      \draw (7.0, -11.3) -- (7.0, -11.0);
      \draw (3.0, -11.0) -- (7.0, -11.0);
      \node[draw=none,fill=none] at (2.75, -11.15) {$\rho_1$};

      \draw (3.0, -10.8) -- (3.0, -10.5);
      \draw (7.0, -10.8) -- (7.0, -10.5);
      \draw (3.0, -10.5) -- (7.0, -10.5);
      \node[draw=none,fill=none] at (2.75, -10.65) {$\rho_2$};
    \end{tikzpicture}
  }
  \caption[Possibilidades de remoção de, pelo menos, um breakpoint tipo um a partir de pares de breakpoints conectados.]{Possibilidades que podem surgir quando existe um par de breakpoints conectados e reversões que podem ser aplicadas para remover, pelo menos, um breakpoint tipo um. O par de elementos que são consecutivos no genoma alvo está representado por tons de cinza.}
  \label{figure:EMTPDAVS}
\end{figure}

A seguir apresentamos o Algoritmo~\ref{algorithm:AKKUXQNR} para a variação sem sinais do problema \SbIR{}.  

\input{algorithms/AKKUXQNR}

\begin{lemma}\label{lemma:RBHACFIP}
Dada uma instância intergênica rígida balanceada sem sinais $\mathcal{I}=((\pi,\breve\pi),\break(\iota,\breve\iota))$, o Algoritmo~\ref{algorithm:AKKUXQNR} transforma $(\pi,\breve\pi)$ em $(\iota,\breve\iota)$ utilizando no máximo $2ib_1(\mathcal{I})$ reversões.
\end{lemma}
\begin{proof}
  No Algoritmo~\ref{algorithm:AKKUXQNR}, temos que enquanto $ib_1(\mathcal{I})$ for maior que um, ou seja, $(\pi,\breve\pi)$ for diferente de $(\iota,\breve\iota)$ (pela Observação~\ref{remark:UDYJTHAH} e Lema~\ref{lemma:WSPRPLAH}), o seguinte procedimento é aplicado: pelos lemas~\ref{lemma:WYEZMYTM} e~\ref{lemma:IMYFBWDY}, sempre podemos encontrar um par conectado de breakpoints e remover pelo menos um breakpoint tipo um após aplicar no máximo duas reversões. A cada iteração do algoritmo pelo menos um breakpoint tipo um é removido. Dessa forma, o genoma alvo eventualmente será alcançado. No pior caso, cada breakpoint tipo um é removido utilizando duas reversões. Logo, $2ib_1(\mathcal{I})$ reversões, no máximo, são utilizadas para transformar $(\pi,\breve\pi)$ em $(\iota,\breve\iota)$ e o lema segue.
\end{proof}

Note que o Algoritmo~\ref{algorithm:AKKUXQNR} pode ser analisado considerando os seguintes pontos: 
\begin{itemize}
  \item Encontrar o par conectado de breakpoints, que pode ser feito em tempo linear com o auxílio da permutação inversa de $\pi$.
  \item Aplicação dos casos do Lema~\ref{lemma:IMYFBWDY}, que no pior caso, também pode levar um tempo linear se for necessário encontrar o breakpoint tipo um $(\pi_k,\pi_{k+1})$ nos casos $ii$ e $iii$. 
\end{itemize}
Como esse processo é repetido no máximo $n$ vezes, então o tempo de execução do Algoritmo~\ref{algorithm:AKKUXQNR} é $\mathcal{O}(n^2)$.

\begin{theorem}\label{theorem:BLJAGNDZ}
Dada uma instância intergênica rígida balanceada sem sinais $\mathcal{I}=\break((\pi,\breve\pi),(\iota,\breve\iota))$, o Algoritmo~\ref{algorithm:AKKUXQNR} é uma $4$-aproximação para o problema \SbIR{}.
\end{theorem}
\begin{proof}
Pelo Lema~\ref{lemma:RBHACFIP}, o Algoritmo~\ref{algorithm:AKKUXQNR} transforma $(\pi,\breve\pi)$ em $(\iota,\breve\iota)$ utilizando no máximo $2ib_1(\mathcal{I})$ reversões. Pelo Teorema~\ref{theorem:MPFPKHQO}, temos o seguinte limitante inferior $d_{\SbIR}(\mathcal{I}) \ge \frac{ib_1(\mathcal{I})}{2}$. Logo, o teorema segue. 
\end{proof}

% ------------------------------------------------------------------ %
\subsection{Reversão e Indel}
% ------------------------------------------------------------------ %

Nesta seção apresentaremos um algoritmo de aproximação com fator $4$ para a variação sem sinais do problema \SbIRI{}.

\begin{lemma}\label{lemma:QGOIQLZD}
Dada uma instância intergênica rígida desbalanceada sem sinais $\mathcal{I}=((\pi,\breve\pi),\break(\iota,\breve\iota))$, tal que $\sum_{i=1}^{n+1}\breve\pi_i < \sum_{i=1}^{n+1}\breve\iota_i$, então sempre é possível aplicar um indel $\delta$ de forma que $\Delta ib_1(\mathcal{I}, S=(\delta)) \le 0$ e $\mathcal{I}$ é tranformada em uma instância intergênica rígida balanceada.
\end{lemma}
\begin{proof}
Como $\mathcal{I}$ é desbalanceada, então $ib_1(\mathcal{I}) > 0$. Seja $(\pi_i,\pi_{i+1})$ um breakpoint tipo um de $\mathcal{I}$. Aplique o indel $\delta_{(x)}^{(i+1)}$, tal que $x = \sum_{i=1}^{n+1}\breve\iota_i - \sum_{i=1}^{n+1}\breve\pi_i$. Note que o indel insere a quantidade necessária de nucleotídeos na região intergênica $\breve\pi_{i+1}$ para tornar $\mathcal{I}$ uma instância balanceada. No pior caso, $(\pi_i,\pi_{i+1})$ continua sendo um breakpoint tipo um e o lema segue.
\end{proof}

\begin{lemma}\label{lemma:QNHGBLYF}
Dada uma instância intergênica rígida desbalanceada sem sinais $\mathcal{I}=((\pi,\breve\pi),\break(\iota,\breve\iota))$, tal que $ib_1(\mathcal{I}) = 1$, então sempre é possível aplicar um indel $\delta$ de forma que $\Delta ib_1(\mathcal{I}, S=(\delta)) = -1$.
\end{lemma}
\begin{proof}
Seja $(\pi_i,\pi_{i+1})$ o único breakpoint tipo um de $\mathcal{I}$. Como $(\pi_i,\pi_{i+1})$ é único breakpoint tipo um de $\mathcal{I}$, então obrigatoriamente ele deve ser um breakpoint forte. Aplique o indel $\delta_{(x)}^{(i+1)}$, tal que $x = \breve\iota_{i+1} - \breve\pi_{i+1}$. Note que o indel insere ou remove a quantidade necessária de nucleotídeos na região intergênica $\breve\pi_{i+1}$ para remover o breakpoint $(\pi_i,\pi_{i+1})$ caso ele seja subcarregado ou sobrecarregado, respectivamente. Como o breakpoint $(\pi_i,\pi_{i+1})$ acaba sendo removido após a aplicação do evento de indel, o lema segue.
\end{proof}

A seguir apresentamos o Algoritmo~\ref{algorithm:LHOPSFVN} para a variação sem sinais do problema \SbIRI{}.

\begin{algorithm}[!tbh]
  \caption{Um algoritmo de aproximação para o problema \SbIRI{}.\label{algorithm:LHOPSFVN}}
  \Entrada{Uma instância intergênica rígida sem sinais $\mathcal{I}=((\pi,\breve\pi),(\iota,\breve\iota))$}
  \Saida{Uma sequência de reversões e indels $S$, tal que $(\pi,\breve\pi) \cdot S = (\iota,\breve\iota)$}
    Seja $S \gets ()$ \\
    \tcp{Lema~\ref{lemma:QGOIQLZD}}
    \Se{$\sum_{i=1}^{n+1}\breve\pi_i < \sum_{i=1}^{n+1}\breve\iota_i$}{
      $S' \gets (\delta_1)$ \\
      $S \gets S + S'$ \\
      $\mathcal{I} \gets ((\pi, \breve\pi) \cdot S',(\iota,\breve\iota))$ \\
    }
    \Enqto{$ib(\mathcal{I}) > 1$}{
      \tcp{Lema~\ref{lemma:WYEZMYTM}}
      $(\pi_i,\pi_{i+1})$, $(\pi_j,\pi_{j+1}) \gets $ encontre um par de breakpoints conectados \\
      \tcp{Lema~\ref{lemma:IMYFBWDY}}
      \Se{$(\pi_i,\pi_{i+1})$, $(\pi_j,\pi_{j+1})$ pertence ao caso $i$}{
        $S' \gets (\rho_1)$ \\
      }\SenaoSe{$(\pi_i,\pi_{i+1})$, $(\pi_j,\pi_{j+1})$ pertence ao caso $ii$}{
        $S' \gets (\rho_1,\rho_2)$ \\
      }\SenaoSe{$(\pi_i,\pi_{i+1})$, $(\pi_j,\pi_{j+1})$ pertence ao caso $iii$}{
        $S' \gets (\rho_1,\rho_2)$ \\
      }\SenaoSe{$(\pi_i,\pi_{i+1})$, $(\pi_j,\pi_{j+1})$ pertence ao caso $iv$}{
        $S' \gets (\rho_1,\rho_2)$ \\
      }
      $S \gets S + S'$ \\
      $\mathcal{I} \gets ((\pi, \breve\pi) \cdot S',(\iota,\breve\iota))$ \\
    }
    \tcp{Lema~\ref{lemma:QNHGBLYF}}
    \Se{$ib(\mathcal{I}) = 1$}{
      $S' \gets (\delta_1)$ \\
      $S \gets S + S'$ \\
      $\mathcal{I} \gets ((\pi, \breve\pi) \cdot S',(\iota,\breve\iota))$ \\
    }
  \Retorna{S}
\end{algorithm}

\begin{lemma}\label{lemma:XUDIVWPC}
Dada uma instância intergênica rígida sem sinais $\mathcal{I}=((\pi,\breve\pi),(\iota,\breve\iota))$, o Algoritmo~\ref{algorithm:LHOPSFVN} transforma $(\pi,\breve\pi)$ em $(\iota,\breve\iota)$ utilizando no máximo $2ib_1(\mathcal{I})$ eventos de reversão e indel.
\end{lemma}
\begin{proof}
  Podemos analisar o Algoritmo~\ref{algorithm:LHOPSFVN} considerando três cenários:
  \begin{itemize}
    \item $\sum_{i=1}^{n+1}\breve\pi_i < \sum_{i=1}^{n+1}\breve\iota_i$, neste cenários o Algoritmo~\ref{algorithm:LHOPSFVN} aplica um indel (linhas 2-5) que pode não remover nenhum breakpoint tipo um, mas torna $\mathcal{I}$ em uma instância balanceada. Caso ainda existam breakpoints em $\mathcal{I}$, então o laço de repetição (linhas 6-17) remove, por iteração, pelo menos um breakpoint tipo um utilizando no máximo duas reversões. Esse processo repete-se até que todos os breakpoints tipo um de $\mathcal{I}$ sejam removidos. Como todos os breakpoints tipo um são removidos, então $(\pi,\breve\pi)$ é transformada em $(\iota,\breve\iota)$. Note que se o indel aplicado inicialmente não remover nenhum breakpoint tipo um, então pelo menos uma reversão é aplicada em seguida. Além disso, pelo Lema~\ref{lemma:WSPRPLAH}, podemos deduzir que a última reversão que transforma $(\pi,\breve\pi)$ em $(\iota,\breve\iota)$ deve obrigatoriamente remover dois breakpoints tipo um. Isso implica que no máximo $2ib_1(\mathcal{I})$ reversões e indels são utilizadas pelo Algoritmo~\ref{algorithm:LHOPSFVN} para transformar $(\pi,\breve\pi)$ em $(\iota,\breve\iota)$.
    \item $\sum_{i=1}^{n+1}\breve\pi_i = \sum_{i=1}^{n+1}\breve\iota_i$, para esse cenários o Algoritmo~\ref{algorithm:LHOPSFVN} comporta-se exatamente como o Algoritmo~\ref{algorithm:AKKUXQNR}, que transforma $(\pi,\breve\pi)$ em $(\iota,\breve\iota)$ utilizando no máximo $2ib_1(\mathcal{I})$ reversões.
    \item $\sum_{i=1}^{n+1}\breve\pi_i > \sum_{i=1}^{n+1}\breve\iota_i$, neste último cenário enquanto $ib(\mathcal{I})$ for maior que um, o Algoritmo~\ref{algorithm:LHOPSFVN} aplica no máximo duas reversões a cada iteração do laço de repetição (linhas 6-17) que removem pelo menos um breakpoint tipo um. Por fim, um indel é aplicado (linhas 19-22) transformando $(\pi,\breve\pi)$ em $(\iota,\breve\iota)$. Note que no pior caso deste cenário cada breakpoint tipo um é removido após a aplicação de duas reversões.
  \end{itemize}
  Nos três cenários o Algoritmo~\ref{algorithm:LHOPSFVN} transforma $(\pi,\breve\pi)$ em $(\iota,\breve\iota)$ utilizando no máximo $2ib_1(\mathcal{I})$ reversões e indels e o lema segue.
\end{proof}

Note que o Algoritmo~\ref{algorithm:LHOPSFVN} difere do Algoritmo~\ref{algorithm:AKKUXQNR} pelos trechos responsáveis por aplicar uma operção de indel (linhas 2-5 e 19-22). Ambos os trechos podem ser realizar em tempo linear. Dessa forma, o tempo de execução do Algoritmo~\ref{algorithm:LHOPSFVN} também é $\mathcal{O}(n^2)$.

\begin{theorem}\label{theorem:AFAHUIUF}
Dada uma instância intergênica rígida sem sinais $\mathcal{I}=((\pi,\breve\pi),(\iota,\breve\iota))$, o Algoritmo~\ref{algorithm:LHOPSFVN} é uma $4$-aproximação para o problema \SbIRI{}.
\end{theorem}
\begin{proof}
Pelo Lema~\ref{lemma:XUDIVWPC}, o Algoritmo~\ref{algorithm:LHOPSFVN} transforma $(\pi,\breve\pi)$ em $(\iota,\breve\iota)$ utilizando no máximo $2ib_1(\mathcal{I})$ eventos de reversão e indel. Pelo Teorema~\ref{theorem:MPFPKHQO}, temos o seguinte limitante inferior $d_{\SbIRI}(\mathcal{I}) \ge \frac{ib_1(\mathcal{I})}{2}$. Logo, o teorema segue. 
\end{proof}

% ------------------------------------------------------------------ %
\subsection{Reversão e Move}
% ------------------------------------------------------------------ %

Nesta seção apresentaremos um algoritmo de aproximação com fator $4$ para a variação sem sinais do problema \SbIRM{}.

\begin{lemma}\label{lemma:NWNNZGXH}
Dada uma instância intergênica rígida sem sinais $\mathcal{I}=((\pi,\breve\pi),(\iota,\breve\iota))$ e sejam $(\pi_i,\pi_{i+1})$ e $(\pi_j,\pi_{j+1})$ breakpoints conectados, então é possível remover pelo menos um breakpoint tipo um de $\mathcal{I}$ utilizando no máximo duas reversões ou um move.
\end{lemma}
\begin{proof}
Note a aplicação do Lema~\ref{lemma:IMYFBWDY} já é suficiente para provar este lema. Entretanto, iremos melhorar o caso $iv$ para utilizarmos apenas um evento de move ao invés de duas reversões. Sem perda de generalidade assuma que $i < j$, como o par de breakpoints $(\pi_i,\pi_{i+1})$ e $(\pi_j,\pi_{j+1})$ está conectado, por definição, uma das seguintes possibilidades deve ocorrer:
\begin{enumerate}[i.]
  \item O par de elementos $(\pi_i,\pi_{j})$ ou $(\pi_{i+1},\pi_{j+1})$ não formam uma adjacência intergênica, são consecutivos em $\iota$ e $\breve\pi_{i+1} + \breve\pi_{j+1} \ge \breve\iota_k$, onde $\breve\iota_k$ é o tamanho da região intergênica entre o par de elementos consecutivos em $\iota$. Aplique uma reversão como descrito no caso $i$ do Lema~\ref{lemma:IMYFBWDY}.   
  \item O par  de elementos $(\pi_i,\pi_{j+1})$ não formam uma adjacência intergênica, são consecutivos em $\iota$ e $\breve\pi_{i+1} + \breve\pi_{j+1} \ge \breve\iota_k$, onde $\breve\iota_k$ é o tamanho da região intergênica entre o par de elementos consecutivos em $\iota$. Aplique uma sequência de duas reversões como descrito no caso $ii$ do Lema~\ref{lemma:IMYFBWDY}.
  \item O par de elementos $(\pi_{i+1},\pi_{j})$ não formam uma adjacência intergênica, são consecutivos em $\iota$ e $\breve\pi_{i+1} + \breve\pi_{j+1} \ge \breve\iota_k$, onde $\breve\iota_k$ é o tamanho da região intergênica entre o par de elementos consecutivos em $\iota$. Aplique uma sequência de duas reversões como descrito no caso $iii$ do Lema~\ref{lemma:IMYFBWDY}.
  \item O par de elementos $(\pi_{i},\pi_{i+1})$ ou $(\pi_{j},\pi_{j+1})$ não formam uma adjacência intergênica, são consecutivos em $\iota$ e $\breve\pi_{i+1} + \breve\pi_{j+1} \ge \breve\iota_k$, onde $\breve\iota_k$ é o tamanho da região intergênica entre o par de elementos consecutivos em $\iota$. Neste caso, obrigatoriamente $(\pi_{i},\pi_{i+1})$ ou $(\pi_{j},\pi_{j+1})$ deve ser um breakpoint forte. Se $(\pi_{i},\pi_{i+1})$ for um breakpoint forte, então ele pode ser sobrecarregado ou subcarregado. Caso ele seja sobrecarregado, então aplicamos um move $\mu^{(i+1,j+1)_{(x)}}$, tal que $x = \breve\pi_{i+1} - \breve\iota_{max(\pi_{i},\pi_{i+1})}$. Caso contrário, aplicamos um move $\mu^{(j+1,i+1)_{(x)}}$, tal que $x = \breve\iota_{max(\pi_{i},\pi_{i+1})} - \breve\pi_{i+1}$. De maneira similar, se $(\pi_{j},\pi_{j+1})$ for um breakpoint forte, então ele pode ser sobrecarregado ou subcarregado. Caso ele seja sobrecarregado, então aplicamos um move $\mu^{(j+1,i+1)_{(x)}}$, tal que $x = \breve\pi_{j+1} - \breve\iota_{max(\pi_{j},\pi_{j+1})}$. Caso contrário, aplicamos um move $\mu^{(i+1,j+1)_{(x)}}$, tal que $x = \breve\iota_{max(\pi_{j},\pi_{j+1})} - \breve\pi_{j+1}$. Em ambos os cenários pelo menos um breakpoint tipo um é removido após a aplicação do move (Figura~\ref{figure:CAIZFSWA}).
\end{enumerate}
Note que o caso $(i)$ aplica apenas um reversão e remove pelo menos um breakpoint tipo um. Os casos $(ii)$ e $(iii)$ aplicam inicialmente uma reversão que não remove nenhum breakpoint tipo um, mas garatem que nenhum novo breakpoint é gerado e o caso $(i)$ poderá ser aplicado em seguida. Por fim, o caso $(iv)$ remove pelo menos um breakpoint tipo um utilizando um move. No pior caso duas reversões são aplicadas e pelo menos um breakpoint tipo um é removido de $\mathcal{I}$. Logo, o lema segue. 
\end{proof}

\begin{figure}[!tbh]
  \resizebox{\linewidth}{!}{
    \centering
    \begin{tikzpicture}
      %%%%% e)

      \node[draw=none,fill=none, minimum height=1cm, minimum width=1cm] at (-3.0, -11.0) {};

      % intergenic regions
      \node[rectangle,draw,fill=white!20, minimum height=.7cm, minimum width=3cm, rounded corners=5pt] at (0.0, -12.0) {$\cdots$};
      \node[rectangle,draw,fill=white!20, minimum height=.7cm, minimum width=1cm, rounded corners=5pt] at (3.0, -12.0) {$\breve\pi_{i+1}$};
      \node[rectangle,draw,fill=white!20, minimum height=.7cm, minimum width=1cm, rounded corners=5pt] at (5.0, -12.0) {$\cdots$};
      \node[rectangle,draw,fill=white!20, minimum height=.7cm, minimum width=1cm, rounded corners=5pt] at (7.0, -12.0) {$\breve\pi_{j+1}$};
      \node[rectangle,draw,fill=white!20, minimum height=.7cm, minimum width=3cm, rounded corners=5pt] at (10.0, -12.0) {$\cdots$};

      % genes
      \node[draw, circle, minimum height=1.2cm, minimum width=1.2cm, fill=white!20] at ({(2*0 - 2)}, -12.0) {$\pi_{0}$};
      \node[draw, circle, minimum height=1.2cm, minimum width=1.2cm, fill=black!50] at ({(2*2 - 2)}, -12.0) {$\pi_{i}$};
      \node[draw, circle, minimum height=1.2cm, minimum width=1.2cm, fill=black!50] at ({(2*3 - 2)}, -12.0) {$\pi_{i+1}$};
      \node[draw, circle, minimum height=1.2cm, minimum width=1.2cm, fill=black!20] at ({(2*4 - 2)}, -12.0) {$\pi_{j}$};
      \node[draw, circle, minimum height=1.2cm, minimum width=1.2cm, fill=black!20] at ({(2*5 - 2)}, -12.0) {$\pi_{j+1}$};
      \node[draw, circle, minimum height=1.2cm, minimum width=1.2cm, fill=white!20] at ({(2*7 - 2)}, -12.0) {$\pi_{n+1}$};

      % indexes
      \draw (3.0, -11.3) -- (3.0, -11.0);
      \draw (7.0, -11.3) -- (7.0, -11.0);
      \draw (3.0, -11.0) -- (7.0, -11.0);
      \node[draw=none,fill=none] at (2.75, -11.15) {$\mu_1$};
    \end{tikzpicture}
  }
  \caption[Exemplo de remoção de pelo menos um breakpoint tipo um após a aplicação de um move $\mu$.]{Exemplo de remoção de pelo menos um breakpoint tipo um após a aplicação de um move $\mu$.}
  \label{figure:CAIZFSWA}
\end{figure}

A seguir apresentamos o Algoritmo~\ref{algorithm:OLSRUEFZ} para a variação sem sinais do problema \SbIRM{}.  

\input{algorithms/OLSRUEFZ}

\begin{lemma}\label{lemma:TZYVWBRT}
Dada uma instância intergênica rígida balanceada sem sinais $\mathcal{I}=((\pi,\breve\pi),\break(\iota,\breve\iota))$, o Algoritmo~\ref{algorithm:OLSRUEFZ} transforma $(\pi,\breve\pi)$ em $(\iota,\breve\iota)$ utilizando no máximo $2ib_1(\mathcal{I})$ eventos de reversão e move.
\end{lemma}
\begin{proof}
  No Algoritmo~\ref{algorithm:OLSRUEFZ}, temos que enquanto $ib_1(\mathcal{I})$ for maior que um, ou seja, $(\pi,\breve\pi)$ for diferente de $(\iota,\breve\iota)$ (pela Observação~\ref{remark:UDYJTHAH} e Lema~\ref{lemma:WSPRPLAH}), o seguinte procedimento é aplicado: pelos lemas~\ref{lemma:WYEZMYTM} e~\ref{lemma:NWNNZGXH}, sempre podemos encontrar um par conectado de breakpoints e remover pelo menos um breakpoint tipo um após aplicar no máximo duas reversões ou um move. A cada iteração do algoritmo pelo menos um breakpoint tipo um é removido. Dessa forma, o genoma alvo eventualmente será alcançado. No pior caso, cada breakpoint tipo um é removido utilizando dois eventos de rearranjo. Logo, uma sequência com, no máximo, $2ib_1(\mathcal{I})$ reversões e moves é utilizada para transformar $(\pi,\breve\pi)$ em $(\iota,\breve\iota)$ e o lema segue.
\end{proof}

Note que o Algoritmo~\ref{algorithm:OLSRUEFZ} difere do Algoritmo~\ref{algorithm:AKKUXQNR} apenas pelo caso $iv$ de um par conectado de breakpoints, que também pode ser realizado em tempo constante. Logo, o tempo de execução do Algoritmo~\ref{algorithm:OLSRUEFZ} também é $\mathcal{O}(n^2)$.

\begin{theorem}\label{theorem:PNTKLAHZ}
Dada uma instância intergênica rígida balanceada sem sinais $\mathcal{I}=\break((\pi,\breve\pi),(\iota,\breve\iota))$, o Algoritmo~\ref{algorithm:OLSRUEFZ} é uma $4$-aproximação para o problema \SbIRM{}.
\end{theorem}
\begin{proof}
Pelo Lema~\ref{lemma:TZYVWBRT}, o Algoritmo~\ref{algorithm:OLSRUEFZ} transforma $(\pi,\breve\pi)$ em $(\iota,\breve\iota)$ utilizando no máximo $2ib_1(\mathcal{I})$ eventos de reversão e move. Pelo Teorema~\ref{theorem:MPFPKHQO}, temos o seguinte limitante inferior $d_{\SbIRM}(\mathcal{I}) \ge \frac{ib_1(\mathcal{I})}{2}$. Logo, o teorema segue. 
\end{proof}

% ------------------------------------------------------------------ %
\subsection{Reversão, Move e Indel}
% ------------------------------------------------------------------ %

Nesta seção apresentaremos um algoritmo de aproximação com fator $4$ para a variação sem sinais do problema \SbIRMI{}. 

A seguir apresentamos o Algoritmo~\ref{algorithm:JAJGNYWD} para a variação sem sinais do problema \SbIRMI{}.

\input{algorithms/JAJGNYWD}

\begin{lemma}\label{lemma:SINGKSVU}
Dada uma instância intergênica rígida sem sinais $\mathcal{I}=((\pi,\breve\pi),(\iota,\breve\iota))$, o Algoritmo~\ref{algorithm:JAJGNYWD} transforma $(\pi,\breve\pi)$ em $(\iota,\breve\iota)$ utilizando no máximo $2ib_1(\mathcal{I})$ eventos de reversão, move e indel.
\end{lemma}
\begin{proof}
  A prova é similar a descrita no Lema~\ref{lemma:XUDIVWPC}.
\end{proof}

Note que o Algoritmo~\ref{algorithm:JAJGNYWD} difere do Algoritmo~\ref{algorithm:LHOPSFVN} apenas pelo caso $iv$ de um par conectado de breakpoints, que também pode ser realizado em tempo constante. Logo, o tempo de execução do Algoritmo~\ref{algorithm:JAJGNYWD} também é $\mathcal{O}(n^2)$.

\begin{theorem}\label{theorem:WSCHLXXJ}
Dada uma instância intergênica rígida sem sinais $\mathcal{I}=((\pi,\breve\pi),(\iota,\breve\iota))$, o Algoritmo~\ref{algorithm:JAJGNYWD} é uma $4$-aproximação para o problema \SbIRMI{}.
\end{theorem}
\begin{proof}
Pelo Lema~\ref{lemma:SINGKSVU}, o Algoritmo~\ref{algorithm:JAJGNYWD} transforma $(\pi,\breve\pi)$ em $(\iota,\breve\iota)$ utilizando no máximo $2ib_1(\mathcal{I})$ eventos de reversão, move e indel. Pelo Teorema~\ref{theorem:MPFPKHQO}, temos o seguinte limitante inferior $d_{\SbIRMI}(\mathcal{I}) \ge \frac{ib_1(\mathcal{I})}{2}$. Logo, o teorema segue. 
\end{proof}

% ------------------------------------------------------------------ %
\subsection{Reversão e Transposição}
% ------------------------------------------------------------------ %

Nesta seção apresentaremos algoritmos de aproximação para a variação sem sinais do problema \SbIRT{} com fatores $6$, $4.5$ e $4$.


\begin{lemma}\label{lemma:SIAFJFDO}
Dada uma instância intergênica rígida sem sinais $\mathcal{I}=((\pi,\breve\pi),(\iota,\breve\iota))$ e sejam $(\pi_i,\pi_{i+1})$ e $(\pi_j,\pi_{j+1})$ breakpoints conectados, então é possível remover pelo menos um breakpoint tipo um de $\mathcal{I}$ utilizando no máximo duas reversões ou uma transposição.
\end{lemma}
\begin{proof}
Note a aplicação do Lema~\ref{lemma:IMYFBWDY} já é suficiente para provar este lema. Entretanto, iremos melhorar os casos $ii$ e $iii$ para utilizarmos apenas um evento de transposição ao invés de duas reversões. Sem perda de generalidade assuma que $i < j$, como os breakpoints $(\pi_i,\pi_{i+1})$ e $(\pi_j,\pi_{j+1})$ estão conectados, por definição, uma das seguintes possibilidades deve ocorrer:
\begin{enumerate}[i.]
  \item O par de elementos $(\pi_i,\pi_{j})$ ou $(\pi_{i+1},\pi_{j+1})$ não formam uma adjacência intergênica, são consecutivos em $\iota$ e $\breve\pi_{i+1} + \breve\pi_{j+1} \ge \breve\iota_k$, onde $\breve\iota_k$ é o tamanho da região intergênica entre o par de elementos consecutivos em $\iota$. Aplique uma reversão como descrito no caso $i$ do Lema~\ref{lemma:IMYFBWDY}.
  \item O par  de elementos $(\pi_i,\pi_{j+1})$ não formam uma adjacência intergênica, são consecutivos em $\iota$ e $\breve\pi_{i+1} + \breve\pi_{j+1} \ge \breve\iota_k$, onde $\breve\iota_k$ é o tamanho da região intergênica entre o par de elementos consecutivos em $\iota$. Nesta caso sabemos que deve existir um breakpoint tipo um $(\pi_k, \pi_{k+1})$, tal que $k <i$ ou $k > j$ (caso $ii$, Lema~\ref{lemma:IMYFBWDY}). Se $k < i$, aplicamos uma transposição $\tau^{(k+1,i+1,j+1)}_{(x,y,z)}$ para posicionar o elemento $\pi_{i}$ no lado esquerdo do elemento $\pi_{j+1}$ (Figura~\ref{figure:WDJFPAXN}(a)). Se $k > j$, aplicamos uma transposição $\tau^{(i+1,j+1,k+1)}_{(x,y,z)}$ para posicionar o elemento $\pi_{j+1}$ no lado direito do elemento $\pi_{i}$ (Figura~\ref{figure:WDJFPAXN}(b)). Em ambos os cenários, temos que $\breve\pi_{i+1} + \breve\pi_{j+1} \ge \breve\iota_k$. Logo, os parâmetros $x$, $y$ e $z$ sempre podem ser escolhidos de forma que ao posicionar lado a lado o par de elemento $(\pi_i,\pi_{j+1})$ o tamanho da região entre eles seja igual no genoma de origem e alvo.
  \item O par de elementos $(\pi_{i+1},\pi_{j})$ não formam uma adjacência intergênica, são consecutivos em $\iota$ e $\breve\pi_{i+1} + \breve\pi_{j+1} \ge \breve\iota_k$, onde $\breve\iota_k$ é o tamanho da região intergênica entre o par de elementos consecutivos em $\iota$. Nesta caso sabemos que deve existir um breakpoint tipo um $(\pi_k, \pi_{k+1})$, tal que $i < k < j$ (caso $iii$, Lema~\ref{lemma:IMYFBWDY}). Após identificar o breakpoint tipo um $(\pi_k, \pi_{k+1})$, aplicamos a transposição $\tau^{(i+1,k+1,j+1)}_{(x,y,z)}$ para posicionar o elemento $\pi_{j}$ no lado esquerdo do elemento $\pi_{i+1}$ (Figura~\ref{figure:WDJFPAXN}(c)). Como $\breve\pi_{i+1} + \breve\pi_{j+1} \ge \breve\iota_k$, então os parâmetros $x$, $y$ e $z$ sempre podem ser escolhidos de forma que ao posicionar lado a lado o par de elemento $(\pi_{i+1},\pi_{j})$ o tamanho da região entre eles seja igual no genoma de origem e alvo.
  \item O par de elementos $(\pi_{i},\pi_{i+1})$ ou $(\pi_{j},\pi_{j+1})$ não formam uma adjacência intergênica, são consecutivos em $\iota$ e $\breve\pi_{i+1} + \breve\pi_{j+1} \ge \breve\iota_k$, onde $\breve\iota_k$ é o tamanho da região intergênica entre o par de elementos consecutivos em $\iota$. Aplique uma sequência de duas reversões como descrito no caso $iv$ do Lema~\ref{lemma:IMYFBWDY}.
\end{enumerate}
Note que o caso $(i)$ aplica apenas um reversão e remove pelo menos um breakpoint tipo um. Os casos $(ii)$ e $(iii)$ aplicam apenas um transposição e remove pelo menos um breakpoint tipo um. Por fim, o caso $(iv)$ remove pelo menos um breakpoint tipo após aplicar duas reversões. No pior caso, duas reversões são aplicadas e pelo menos um breakpoint tipo um é removido de $\mathcal{I}$. Logo, o lema segue. 
\end{proof}

\begin{figure}[!tbh]
  \resizebox{\linewidth}{!}{
    \centering
    \begin{tikzpicture}
      %%%%% a)

      \node[draw=none,fill=none, minimum height=1cm, minimum width=1cm] at (-3.0, -2.0) {a)};

      % intergenic regions
      \node[rectangle,draw,fill=white!20, minimum height=.7cm, minimum width=1cm, rounded corners=5pt] at (-1.0, -3.0) {$\cdots$};
      \node[rectangle,draw,fill=white!20, minimum height=.7cm, minimum width=1cm, rounded corners=5pt] at (1.0, -3.0) {$\breve\pi_{k+1}$};
      \node[rectangle,draw,fill=white!20, minimum height=.7cm, minimum width=1cm, rounded corners=5pt] at (3.0, -3.0) {$\cdots$};
      \node[rectangle,draw,fill=white!20, minimum height=.7cm, minimum width=1cm, rounded corners=5pt] at (5.0, -3.0) {$\breve\pi_{i+1}$};
      \node[rectangle,draw,fill=white!20, minimum height=.7cm, minimum width=1cm, rounded corners=5pt] at (7.0, -3.0) {$\cdots$};
      \node[rectangle,draw,fill=white!20, minimum height=.7cm, minimum width=1cm, rounded corners=5pt] at (9.0, -3.0) {$\breve\pi_{j+1}$};
      \node[rectangle,draw,fill=white!20, minimum height=.7cm, minimum width=1cm, rounded corners=5pt] at (11.0, -3.0) {$\cdots$};

      % genes
      \node[draw, circle, minimum height=1.2cm, minimum width=1.2cm, fill=white!20] at ({(2*0 - 2)}, -3.0) {$\pi_{0}$};
      \node[draw, circle, minimum height=1.2cm, minimum width=1.2cm, fill=white!20] at ({(2*1 - 2)}, -3.0) {$\pi_{k}$};
      \node[draw, circle, minimum height=1.2cm, minimum width=1.2cm, fill=white!20] at ({(2*2 - 2)}, -3.0) {$\pi_{k+1}$};
      \node[draw, circle, minimum height=1.2cm, minimum width=1.2cm, fill=black!20] at ({(2*3 - 2)}, -3.0) {$\pi_{i}$};
      \node[draw, circle, minimum height=1.2cm, minimum width=1.2cm, fill=white!20] at ({(2*4 - 2)}, -3.0) {$\pi_{i+1}$};
      \node[draw, circle, minimum height=1.2cm, minimum width=1.2cm, fill=white!20] at ({(2*5 - 2)}, -3.0) {$\pi_{j}$};
      \node[draw, circle, minimum height=1.2cm, minimum width=1.2cm, fill=black!20] at ({(2*6 - 2)}, -3.0) {$\pi_{j+1}$};
      \node[draw, circle, minimum height=1.2cm, minimum width=1.2cm, fill=white!20] at ({(2*7 - 2)}, -3.0) {$\pi_{n+1}$};

      % indexes
      \draw (1.0, -2.3) -- (1.0, -2.0);
      \draw (5.0, -2.3) -- (5.0, -2.0);
      \draw (9.0, -2.3) -- (9.0, -2.0);
      \draw (1.0, -2.0) -- (9.0, -2.0);
      \node[draw=none,fill=none] at (0.75, -2.15) {$\tau_1$};

      %%%%% b)

      \node[draw=none,fill=none, minimum height=1cm, minimum width=1cm] at (-3.0, -5.0) {b)};

      % intergenic regions
      \node[rectangle,draw,fill=white!20, minimum height=.7cm, minimum width=1cm, rounded corners=5pt] at (-1.0, -6.0) {$\cdots$};
      \node[rectangle,draw,fill=white!20, minimum height=.7cm, minimum width=1cm, rounded corners=5pt] at (1.0, -6.0) {$\breve\pi_{i+1}$};
      \node[rectangle,draw,fill=white!20, minimum height=.7cm, minimum width=1cm, rounded corners=5pt] at (3.0, -6.0) {$\cdots$};
      \node[rectangle,draw,fill=white!20, minimum height=.7cm, minimum width=1cm, rounded corners=5pt] at (5.0, -6.0) {$\breve\pi_{j+1}$};
      \node[rectangle,draw,fill=white!20, minimum height=.7cm, minimum width=1cm, rounded corners=5pt] at (7.0, -6.0) {$\cdots$};
      \node[rectangle,draw,fill=white!20, minimum height=.7cm, minimum width=1cm, rounded corners=5pt] at (9.0, -6.0) {$\breve\pi_{k+1}$};
      \node[rectangle,draw,fill=white!20, minimum height=.7cm, minimum width=1cm, rounded corners=5pt] at (11.0, -6.0) {$\cdots$};

      % genes
      \node[draw, circle, minimum height=1.2cm, minimum width=1.2cm, fill=white!20] at ({(2*0 - 2)}, -6.0) {$\pi_{0}$};
      \node[draw, circle, minimum height=1.2cm, minimum width=1.2cm, fill=black!20] at ({(2*1 - 2)}, -6.0) {$\pi_{i}$};
      \node[draw, circle, minimum height=1.2cm, minimum width=1.2cm, fill=white!20] at ({(2*2 - 2)}, -6.0) {$\pi_{i+1}$};
      \node[draw, circle, minimum height=1.2cm, minimum width=1.2cm, fill=white!20] at ({(2*3 - 2)}, -6.0) {$\pi_{j}$};
      \node[draw, circle, minimum height=1.2cm, minimum width=1.2cm, fill=black!20] at ({(2*4 - 2)}, -6.0) {$\pi_{j+1}$};
      \node[draw, circle, minimum height=1.2cm, minimum width=1.2cm, fill=white!20] at ({(2*5 - 2)}, -6.0) {$\pi_{k}$};
      \node[draw, circle, minimum height=1.2cm, minimum width=1.2cm, fill=white!20] at ({(2*6 - 2)}, -6.0) {$\pi_{k+1}$};
      \node[draw, circle, minimum height=1.2cm, minimum width=1.2cm, fill=white!20] at ({(2*7 - 2)}, -6.0) {$\pi_{n+1}$};

      % indexes
      \draw (1.0, -5.3) -- (1.0, -5.0);
      \draw (5.0, -5.3) -- (5.0, -5.0);
      \draw (9.0, -5.3) -- (9.0, -5.0);
      \draw (1.0, -5.0) -- (9.0, -5.0);
      \node[draw=none,fill=none] at (0.75, -5.15) {$\tau_1$};

      %%%%% c)

      \node[draw=none,fill=none, minimum height=1cm, minimum width=1cm] at (-3.0, -8.0) {c)};

      % intergenic regions
      \node[rectangle,draw,fill=white!20, minimum height=.7cm, minimum width=1cm, rounded corners=5pt] at (-1.0, -9.0) {$\cdots$};
      \node[rectangle,draw,fill=white!20, minimum height=.7cm, minimum width=1cm, rounded corners=5pt] at (1.0, -9.0) {$\breve\pi_{i+1}$};
      \node[rectangle,draw,fill=white!20, minimum height=.7cm, minimum width=1cm, rounded corners=5pt] at (3.0, -9.0) {$\cdots$};
      \node[rectangle,draw,fill=white!20, minimum height=.7cm, minimum width=1cm, rounded corners=5pt] at (5.0, -9.0) {$\breve\pi_{k+1}$};
      \node[rectangle,draw,fill=white!20, minimum height=.7cm, minimum width=1cm, rounded corners=5pt] at (7.0, -9.0) {$\cdots$};
      \node[rectangle,draw,fill=white!20, minimum height=.7cm, minimum width=1cm, rounded corners=5pt] at (9.0, -9.0) {$\breve\pi_{j+1}$};
      \node[rectangle,draw,fill=white!20, minimum height=.7cm, minimum width=1cm, rounded corners=5pt] at (11.0, -9.0) {$\cdots$};

      % genes
      \node[draw, circle, minimum height=1.2cm, minimum width=1.2cm, fill=white!20] at ({(2*0 - 2)}, -9.0) {$\pi_{0}$};
      \node[draw, circle, minimum height=1.2cm, minimum width=1.2cm, fill=white!20] at ({(2*1 - 2)}, -9.0) {$\pi_{i}$};
      \node[draw, circle, minimum height=1.2cm, minimum width=1.2cm, fill=black!20] at ({(2*2 - 2)}, -9.0) {$\pi_{i+1}$};
      \node[draw, circle, minimum height=1.2cm, minimum width=1.2cm, fill=white!20] at ({(2*3 - 2)}, -9.0) {$\pi_{k}$};
      \node[draw, circle, minimum height=1.2cm, minimum width=1.2cm, fill=white!20] at ({(2*4 - 2)}, -9.0) {$\pi_{k+1}$};
      \node[draw, circle, minimum height=1.2cm, minimum width=1.2cm, fill=black!20] at ({(2*5 - 2)}, -9.0) {$\pi_{j}$};
      \node[draw, circle, minimum height=1.2cm, minimum width=1.2cm, fill=white!20] at ({(2*6 - 2)}, -9.0) {$\pi_{j+1}$};
      \node[draw, circle, minimum height=1.2cm, minimum width=1.2cm, fill=white!20] at ({(2*7 - 2)}, -9.0) {$\pi_{n+1}$};

      % indexes
      \draw (1.0, -8.3) -- (1.0, -8.0);
      \draw (5.0, -8.3) -- (5.0, -8.0);
      \draw (9.0, -8.3) -- (9.0, -8.0);
      \draw (1.0, -8.0) -- (9.0, -8.0);
      \node[draw=none,fill=none] at (0.75, -8.15) {$\tau_1$};
    \end{tikzpicture}
  }
  \caption[Exemplo de remoção de, pelo menos, um breakpoint tipo um após a aplicação de uma transposição $\tau$.]{Exemplo de remoção de, pelo menos, um breakpoint tipo um após a aplicação de uma transposição $\tau$.}
  \label{figure:WDJFPAXN}
\end{figure}

A seguir apresentamos o Algoritmo~\ref{algorithm:SAAUGXYG} para a variação sem sinais do problema \SbIRT{}.

\input{algorithms/SAAUGXYG}

\begin{lemma}\label{lemma:QMLCZMMK}
Dada uma instância intergênica rígida balanceada sem sinais $\mathcal{I}=((\pi,\breve\pi),\break(\iota,\breve\iota))$, o Algoritmo~\ref{algorithm:SAAUGXYG} transforma $(\pi,\breve\pi)$ em $(\iota,\breve\iota)$ utilizando no máximo $2ib_1(\mathcal{I})$ eventos de reversão e transposição.
\end{lemma}
\begin{proof}
  No Algoritmo~\ref{algorithm:SAAUGXYG}, temos que enquanto $ib_1(\mathcal{I})$ for maior que um, ou seja, $(\pi,\breve\pi)$ for diferente de $(\iota,\breve\iota)$ (pela Observação~\ref{remark:UDYJTHAH} e Lema~\ref{lemma:WSPRPLAH}), o seguinte procedimento é aplicado: pelos lemas~\ref{lemma:WYEZMYTM} e~\ref{lemma:SIAFJFDO}, sempre podemos encontrar um par conectado de breakpoints e remover pelo menos um breakpoint tipo um após aplicar no máximo duas reversões ou uma transposição. A cada iteração do algoritmo pelo menos um breakpoint tipo um é removido. Dessa forma, o genoma alvo eventualmente será alcançado. No pior caso, cada breakpoint tipo um é removido utilizando dois eventos de rearranjo. Logo, uma sequência com, no máximo, $2ib_1(\mathcal{I})$ reversões e transposições é utilizada para transformar $(\pi,\breve\pi)$ em $(\iota,\breve\iota)$ e o lema segue.
\end{proof}

Note que o Algoritmo~\ref{algorithm:SAAUGXYG} difere do Algoritmo~\ref{algorithm:AKKUXQNR} apenas pelos casos $ii$ e $iii$ de um par conectado de breakpoints, que também podem ser realizados em tempo linear. Logo, o tempo de execução do Algoritmo~\ref{algorithm:SAAUGXYG} também é $\mathcal{O}(n^2)$.

\begin{theorem}\label{theorem:ZEFRNBIE}
Dada uma instância intergênica rígida balanceada sem sinais $\mathcal{I}=\break((\pi,\breve\pi),(\iota,\breve\iota))$, o Algoritmo~\ref{algorithm:SAAUGXYG} é uma $6$-aproximação para o problema \SbIRT{}.
\end{theorem}
\begin{proof}
Pelo Lema~\ref{lemma:QMLCZMMK}, o Algoritmo~\ref{algorithm:SAAUGXYG} transforma $(\pi,\breve\pi)$ em $(\iota,\breve\iota)$ utilizando no máximo $2ib_1(\mathcal{I})$ eventos de reversão e transposição. Pelo Teorema~\ref{theorem:MPFPKHQO}, temos o seguinte limitante inferior $d_{\SbIRT}(\mathcal{I}) \ge \frac{ib_1(\mathcal{I})}{3}$. Logo, o teorema segue. 
\end{proof}

A seguir apresentaremos lemas que serão utilizados para obtermos um algoritmo de aproximação com fator $4.5$ para a variação sem sinais do problema \SbIRT{}.

\begin{lemma}\label{lemma:RAJPFOWJ}
Dada uma representação intergênica rígida $(\pi,\breve\pi)$ e duas transposições consecutivas no formato:
$$(\pi,\breve\pi)\cdot\tau^{(i,j,k)}_{(\varphi_i,\varphi_j,\varphi_k)}\cdot\tau^{(i,i+k-j,k)}_{(\varphi^\prime_i,\varphi^\prime_{i+k-j},\varphi^\prime_k)},$$
então é possível realizar qualquer redistribuição de nucleotídeos nas regiões intergênicas $\breve\pi_i$, $\breve\pi_j$ e $\breve\pi_k$.
\end{lemma} 
\begin{proof}
Temos que mostrar que sempre é possível encontrar valores para as triplas $(\varphi_i,\varphi_j,\varphi_k)$ e $(\varphi^\prime_i,\varphi^\prime_{i+k-j},\varphi^\prime_k)$ para qualquer redistribuição de nucleotídeos nas regiões intergênicas $\breve\pi_i$, $\breve\pi_j$ e $\breve\pi_k$.

Como usamos apenas transposições, sabemos que $\breve\pi_i + \breve\pi_j + \breve\pi_k = \breve\pi^{\prime\prime}_i + \breve\pi^{\prime\prime}_j + \breve\pi^{\prime\prime}_k$, onde $\breve\pi^{\prime\prime}_i$, $\breve\pi^{\prime\prime}_j$ e $\breve \pi^{\prime\prime}_k$ representam os tamanhos das regiões intergênicas após a aplicação das duas transposições consecutivas.

Com base no fluxo de nucleotídeos entre as regiões intergênicas, realizaremos uma redução para uma instância $I_{MF}$ do problema de fluxo máximo, e mostraremos que sempre é possível encontrar uma solução para esta instância que satisfaça as restrições de redistribuição das regiões intergênicas $\breve\pi_i$, $\breve\pi_j$ e $\breve\pi_k$.

A Figura~\ref{figure:XUQWDNIW} (esquerda) mostra o fluxo que os nucleotídeos nas regiões intergênicas podem seguir ao aplicar as duas transposições consecutivas (declaradas no enunciado do Lemma~\ref{lemma:RAJPFOWJ}). Note que cada região intergênica pode enviar nucleotídeos para duas regiões intergênicas distintas. Assim, podemos criar um grafo, onde cada vértice corresponde a uma região intergênica, e onde existe um arco $(i,j)$ com capacidade ilimitada se a região intergênica $i$ pode enviar nucleotídeos para a região intergênica $j$.

Por fim, para obter a instância $I_{MF}$ do problema de fluxo máximo, adicionaremos os vértices 0 (origem) e 10 (destino), juntamente com os seguintes arcos: $(0,1)$, $(0, 2)$, $(0,3)$, $(7,10)$, $(8,10)$ e $(9,10)$ com suas respectivas capacidades: $a=\breve\pi_i$, $b=\breve\pi_j$, $c=\breve\pi_k$, $x=\breve\pi^{\prime\prime}_i$, $y=\breve\pi^{\prime\prime}_j $ e $z=\breve\pi^{\prime\prime}_k$. Todos os outros arcos têm capacidade infinita atribuída. Figura~\ref{figure:XUQWDNIW} (direita) mostra a instância $I_{MF}$ do problema de fluxo máximo obtido.

\begin{figure}[!tbh]
  \resizebox{.45\linewidth}{!}{
    \centering  
    \begin{tikzpicture}
      \tikzstyle{STY1} = [draw, rectangle, minimum height=.8cm, minimum width=1.5cm, rounded corners=10pt]
      \tikzstyle{STY2} = [->]

      %%%%% VERTEX

      \node[draw=none, fill=none, rectangle, minimum height=.8cm, minimum width=1.5cm, rounded corners=10pt](v0) at ({(2*2 - 2)}, 2.0) {};

      \node[STY1](v1) at ({(2*1 - 2)}, 0.0) {$\breve\pi_i$};
      \node[STY1](v2) at ({(2*2 - 2)}, 0.0) {$\breve\pi_j$};
      \node[STY1](v3) at ({(2*3 - 2)}, 0.0) {$\breve\pi_k$};

      \node[STY1](v4) at ({(2*1 - 2)}, -2.0) {$\breve\pi^{\prime}_i$};
      \node[STY1](v5) at ({(2*2 - 2)}, -2.0) {$\breve\pi^{\prime}_{i+k-j}$};
      \node[STY1](v6) at ({(2*3 - 2)}, -2.0) {$\breve\pi^{\prime}_k$};


      \node[STY1](v7) at ({(2*1 - 2)}, -4.0) {$\breve\pi^{\prime\prime}_i$};
      \node[STY1](v8) at ({(2*2 - 2)}, -4.0) {$\breve\pi^{\prime\prime}_j$};
      \node[STY1](v9) at ({(2*3 - 2)}, -4.0) {$\breve\pi^{\prime\prime}_k$};

      \node[draw=none, fill=none, rectangle, minimum height=.8cm, minimum width=1.5cm, rounded corners=10pt](v10) at ({(2*2 - 2)}, -6.0) {};

      %%%%% EDGES

      \draw[STY2]  (v1) edge (v4);
      \draw[STY2]  (v1) edge (v5);
      \draw[STY2]  (v2) edge (v4);
      \draw[STY2]  (v2) edge (v6);
      \draw[STY2]  (v3) edge (v5);
      \draw[STY2]  (v3) edge (v6);

      \draw[STY2]  (v4) edge (v7);
      \draw[STY2]  (v4) edge (v8);
      \draw[STY2]  (v5) edge (v7);
      \draw[STY2]  (v5) edge (v9);
      \draw[STY2]  (v6) edge (v8);
      \draw[STY2]  (v6) edge (v9);
      %     \node[draw=none,fill=none, minimum height=1cm, minimum width=1cm] at (0.0, 2.0) {a)};
    \end{tikzpicture}
  }
  \hfill
  \resizebox{.45\linewidth}{!}{
    \centering
    \begin{tikzpicture}
      \tikzstyle{STY1} = [draw, rectangle, minimum height=.8cm, minimum width=1.5cm, rounded corners=10pt]
      \tikzstyle{STY2} = [->]

      %%%%% VERTEX

      \node[STY1](v0) at ({(2*2 - 2)}, 2.0) {0};

      \node[STY1](v1) at ({(2*1 - 2)}, 0.0) {1};
      \node[STY1](v2) at ({(2*2 - 2)}, 0.0) {2};
      \node[STY1](v3) at ({(2*3 - 2)}, 0.0) {3};

      \node[STY1](v4) at ({(2*1 - 2)}, -2.0) {4};
      \node[STY1](v5) at ({(2*2 - 2)}, -2.0) {5};
      \node[STY1](v6) at ({(2*3 - 2)}, -2.0) {6};


      \node[STY1](v7) at ({(2*1 - 2)}, -4.0) {7};
      \node[STY1](v8) at ({(2*2 - 2)}, -4.0) {8};
      \node[STY1](v9) at ({(2*3 - 2)}, -4.0) {9};

      \node[STY1](v10) at ({(2*2 - 2)}, -6.0) {10};

      %%%%% EDGES
      \draw[STY2]  (v0) edge node[anchor=center, left, midway] {$a$} (v1);
      \draw[STY2]  (v0) edge node[anchor=center, left, midway] {$b$} (v2);
      \draw[STY2]  (v0) edge node[anchor=center, left, midway] {$c$} (v3);

      \draw[STY2]  (v1) edge node[anchor=center, left, midway] {$\infty$} (v4);
      \draw[STY2]  (v1) edge node[anchor=center, left, midway] {$\infty$} (v5);
      \draw[STY2]  (v2) edge node[anchor=center, right, midway] {$\infty$} (v4);
      \draw[STY2]  (v2) edge node[anchor=center, left, midway] {$\infty$} (v6);
      \draw[STY2]  (v3) edge node[anchor=center, right, midway] {$\infty$} (v5);
      \draw[STY2]  (v3) edge node[anchor=center, right, midway] {$\infty$} (v6);

      \draw[STY2]  (v4) edge node[anchor=center, left, midway] {$\infty$} (v7);
      \draw[STY2]  (v4) edge node[anchor=center, left, midway] {$\infty$} (v8);
      \draw[STY2]  (v5) edge node[anchor=center, right, midway] {$\infty$} (v7);
      \draw[STY2]  (v5) edge node[anchor=center, left, midway] {$\infty$} (v9);
      \draw[STY2]  (v6) edge node[anchor=center, right, midway] {$\infty$} (v8);
      \draw[STY2]  (v6) edge node[anchor=center, right, midway] {$\infty$} (v9);

      \draw[STY2]  (v7) edge node[anchor=center, left, midway] {$x$} (v10);
      \draw[STY2]  (v8) edge node[anchor=center, left, midway] {$y$} (v10);
      \draw[STY2]  (v9) edge node[anchor=center, left, midway] {$z$} (v10);

      %     \node[draw=none,fill=none, minimum height=1cm, minimum width=1cm] at (0.0, 2.0) {b)};
    \end{tikzpicture}
  }
  \caption[Exemplo de fluxo de nucleotídeos entre as regiões intergênicas.]{No lado esquerdo, uma representação do fluxo de nucleotídeos entre as regiões intergênicas. No lado direito, a instância $I_{MF}$ do problema de fluxo máximo com os vértices de origem (0) e destino (10).}
  \label{figure:XUQWDNIW}
\end{figure}

Se analisarmos a Figura~\ref{figure:XUQWDNIW} (direita), podemos ver que o fluxo máximo da instância $I_{MF}$ é limitado a $\max\{(a+b+c),(x+ y+z)\}$, mas sabemos que $(a+b+c) = (x+y+z) = F$. Observe também que se houver uma redistribuição das regiões intergênicas $\breve\pi_i$, $\breve\pi_j$ e $\breve\pi_k$ isso significa que a instância $I_{MF}$ tem uma solução onde o máximo fluxo é $F$. Por outro lado, podemos ver que se a instância $I_{MF}$ tiver uma solução com fluxo máximo de $F$ e todas as variáveis da solução forem inteiras, isso significa que é possível redistribuir as regiões intergênicas $\breve \pi_i$, $\breve\pi_j$ e $\breve\pi_k$.

Agora, mostraremos que a instância $I_{MF}$ sempre tem uma solução com fluxo máximo $F$, onde todas as variáveis da solução são inteiras. Vamos provar este resultado fornecendo uma solução para a instância $I_{MF}$, que é obtida em três etapas.

A etapa 1 consiste em remover os vértices 8 e 9 de $I_{MF}$ (Figura~\ref{figure:NCRGBSMG} (esquerda)), e resolver a instância usando Programação Linear (PL) para obter uma possível solução fracionária. Observe que o fluxo máximo para esta etapa é menor ou igual a $x$, pois $(a+b+c) \ge x$. Além disso, existe uma solução que atinge exatamente $x$ (por exemplo, envie $a$ pelo caminho $(1,4,7)$; envie $b$ pelo caminho $(2,4,7)$; envie $c$ pelo caminho $(3,5,7)$). Seja $X^{\prime}$ a matriz da solução, na qual $X^{\prime}_{i,j}$ representa o fluxo que vai do vértice $i$ ao vértice $j$ na solução. Sabemos que $(X^{\prime}_{0,1}+X^{\prime}_{0,2}+X^{\prime}_{0,3}) = x$ e $x \in \mathbb{N}$. Se $\{X^{\prime}_{0,1},X^{\prime}_{0,2},X^{\prime}_{0,3}\} \not\subset \mathbb {N}$, podemos obter valores inteiros para as variáveis $X^{\prime}_{0,1}$, $X^{\prime}_{0,2}$ e $X^{\prime }_{0,3}$ redistribuindo a parte fracionária entre os arcos, isso é possível porque sabemos que $(a+b+c) \ge x$. Como todos os caminhos que partem dos vértices 1, 2 e 3 e chegam ao vértice 7 têm capacidade ilimitada, podemos obter uma solução inteira para as variáveis restantes. Após este processo, obtemos uma solução inteira $X^{\prime}$ para a etapa 1.

A etapa 2 consiste em remover os vértices 7 e 9 de $I_{MF}$, e atualizar a capacidade dos arcos $(0,1)$, $(0,2)$ e $(0,3)$ para $a ^{\prime}=a-X^{\prime}_{0,1}$, $b^{\prime}=b-X^{\prime}_{0,2}$ e $c^{\prime} =c-X^{\prime}_{0,3}$, respectivamente (Figura~\ref{figure:NCRGBSMG} (centro)). Em outras palavras, levamos em consideração, para os arcos $(0,1)$, $(0,2)$ e $(0,3)$, as capacidades que já foram utilizadas na etapa 1. Observe que $a^{\prime}+b^{\prime}+c^{\prime} = a+b+c-x$, mas também sabemos que $a+b+c = x+y+z$, assim $a^{\prime}+b^{\prime}+c^{\prime} = a+b+c-x = y+z \geq y$. Observe que o fluxo máximo para esta etapa é menor ou igual a $y$, pois $a^{\prime}+b^{\prime}+c^{\prime} \ge y$, e existe uma solução que atinge exatamente $y$ (por exemplo, envie $a^{\prime}$ pelo caminho $(1,4,8)$; envie $b^{\prime}$ pelo caminho $(2,4,8) $; envie $c^{\prime}$ pelo caminho $(3,6,8)$). Similarmente ao processo realizado na etapa 1, resolvemos o problema para obter uma solução $X^{\prime\prime}$ onde o fluxo máximo é $y$ e todas as variáveis são inteiras.

A etapa 3 consiste em remover os vértices 7 e 8 de $I_{MF}$, e atualizar a capacidade dos arcos $(0,1)$, $(0,2)$ e $(0,3)$ para $a ^{\prime\prime}=a^{\prime}-X^{\prime\prime}_{0,1}$, $b^{\prime\prime}=b^{\prime}-X^ {\prime\prime}_{0,2}$ e $c^{\prime\prime}=c^{\prime}-X^{\prime\prime}_{0,3}$, respectivamente (Figura~\ref{figure:NCRGBSMG} (direita)). Em outras palavras, levamos em consideração, para os arcos $(0,1)$, $(0,2)$ e $(0,3)$, as capacidades que já foram utilizadas nas etapas~1 e~2. Observe que $a^{\prime\prime}+b^{\prime\prime}+c^{\prime\prime} = a+b+c-x-y$, mas também sabemos que $a+b+c = x +y+z$, portanto $a^{\prime\prime}+b^{\prime\prime}+c^{\prime\prime} = a+b+c-x-y = z$. Observe que o fluxo máximo para esta etapa é igual a $z$, pois $a^{\prime\prime}+b^{\prime\prime}+c^{\prime\prime} = z$, e existe uma solução que atinja exatamente $z$ (por exemplo, envie $a^{\prime\prime}$ pelo caminho $(1,5,9)$; envie $b^{\prime\prime}$ pelo caminho $ (2,6,9)$; envie $c^{\prime\prime}$ pelo caminho $(3,6,9)$). Da mesma forma que o processo realizado na etapa 1, resolvemos o problema para obter uma solução $X^{\prime\prime\prime}$ onde o fluxo máximo é $z$ e todas as variáveis são números inteiros.

\begin{figure}[!tbh]
  \resizebox{.3\linewidth}{!}{
    \centering
    \begin{tikzpicture}
      \tikzstyle{STY1} = [draw, rectangle, minimum height=.8cm, minimum width=1.5cm, rounded corners=10pt]
      \tikzstyle{STY2} = [->]

      %%%%% VERTEX

      \node[STY1](v0) at ({(2*2 - 2)}, 2.0) {0};

      \node[STY1](v1) at ({(2*1 - 2)}, 0.0) {1};
      \node[STY1](v2) at ({(2*2 - 2)}, 0.0) {2};
      \node[STY1](v3) at ({(2*3 - 2)}, 0.0) {3};

      \node[STY1](v4) at ({(2*1 - 2)}, -2.0) {4};
      \node[STY1](v5) at ({(2*2 - 2)}, -2.0) {5};
      \node[STY1](v6) at ({(2*3 - 2)}, -2.0) {6};


      \node[STY1](v7) at ({(2*1 - 2)}, -4.0) {7};

      \node[STY1](v10) at ({(2*2 - 2)}, -6.0) {10};

      %%%%% EDGES
      \draw[STY2]  (v0) edge node[anchor=center, left, midway] {$a$} (v1);
      \draw[STY2]  (v0) edge node[anchor=center, left, midway] {$b$} (v2);
      \draw[STY2]  (v0) edge node[anchor=center, left, midway] {$c$} (v3);

      \draw[STY2]  (v1) edge node[anchor=center, left, midway] {$\infty$} (v4);
      \draw[STY2]  (v1) edge node[anchor=center, left, midway] {$\infty$} (v5);
      \draw[STY2]  (v2) edge node[anchor=center, right, midway] {$\infty$} (v4);
      \draw[STY2]  (v2) edge node[anchor=center, left, midway] {$\infty$} (v6);
      \draw[STY2]  (v3) edge node[anchor=center, right, midway] {$\infty$} (v5);
      \draw[STY2]  (v3) edge node[anchor=center, right, midway] {$\infty$} (v6);

      \draw[STY2]  (v4) edge node[anchor=center, left, midway] {$\infty$} (v7);
      \draw[STY2]  (v5) edge node[anchor=center, right, midway] {$\infty$} (v7);

      \draw[STY2]  (v7) edge node[anchor=center, left, midway] {$x$} (v10);

      %     \node[draw=none,fill=none, minimum height=1cm, minimum width=1cm] at (0.0, 2.0) {a)};
    \end{tikzpicture}
  }
  \hfill
  \resizebox{.3\linewidth}{!}{
    \centering
    \begin{tikzpicture}
      \tikzstyle{STY1} = [draw, rectangle, minimum height=.8cm, minimum width=1.5cm, rounded corners=10pt]
      \tikzstyle{STY2} = [->]

      %%%%% VERTEX

      \node[STY1](v0) at ({(2*2 - 2)}, 2.0) {0};

      \node[STY1](v1) at ({(2*1 - 2)}, 0.0) {1};
      \node[STY1](v2) at ({(2*2 - 2)}, 0.0) {2};
      \node[STY1](v3) at ({(2*3 - 2)}, 0.0) {3};

      \node[STY1](v4) at ({(2*1 - 2)}, -2.0) {4};
      \node[STY1](v5) at ({(2*2 - 2)}, -2.0) {5};
      \node[STY1](v6) at ({(2*3 - 2)}, -2.0) {6};

      \node[STY1](v8) at ({(2*2 - 2)}, -4.0) {8};

      \node[STY1](v10) at ({(2*2 - 2)}, -6.0) {10};

      %%%%% EDGES
      \draw[STY2]  (v0) edge node[anchor=center, left, midway] {$a^{\prime}$} (v1);
      \draw[STY2]  (v0) edge node[anchor=center, left, midway] {$b^{\prime}$} (v2);
      \draw[STY2]  (v0) edge node[anchor=center, left, midway] {$c^{\prime}$} (v3);

      \draw[STY2]  (v1) edge node[anchor=center, left, midway] {$\infty$} (v4);
      \draw[STY2]  (v1) edge node[anchor=center, left, midway] {$\infty$} (v5);
      \draw[STY2]  (v2) edge node[anchor=center, right, midway] {$\infty$} (v4);
      \draw[STY2]  (v2) edge node[anchor=center, left, midway] {$\infty$} (v6);
      \draw[STY2]  (v3) edge node[anchor=center, right, midway] {$\infty$} (v5);
      \draw[STY2]  (v3) edge node[anchor=center, right, midway] {$\infty$} (v6);

      \draw[STY2]  (v4) edge node[anchor=center, left, midway] {$\infty$} (v8);
      \draw[STY2]  (v6) edge node[anchor=center, right, midway] {$\infty$} (v8);
      \draw[STY2]  (v8) edge node[anchor=center, left, midway] {$y$} (v10);

      %     \node[draw=none,fill=none, minimum height=1cm, minimum width=1cm] at (0.0, 2.0) {b)};
    \end{tikzpicture}
  }
  \hfill
  \resizebox{.3\linewidth}{!}{
    \centering
    \begin{tikzpicture}
      \tikzstyle{STY1} = [draw, rectangle, minimum height=.8cm, minimum width=1.5cm, rounded corners=10pt]
      \tikzstyle{STY2} = [->]

      %%%%% VERTEX

      \node[STY1](v0) at ({(2*2 - 2)}, 2.0) {0};

      \node[STY1](v1) at ({(2*1 - 2)}, 0.0) {1};
      \node[STY1](v2) at ({(2*2 - 2)}, 0.0) {2};
      \node[STY1](v3) at ({(2*3 - 2)}, 0.0) {3};

      \node[STY1](v4) at ({(2*1 - 2)}, -2.0) {4};
      \node[STY1](v5) at ({(2*2 - 2)}, -2.0) {5};
      \node[STY1](v6) at ({(2*3 - 2)}, -2.0) {6};

      \node[STY1](v9) at ({(2*3 - 2)}, -4.0) {9};

      \node[STY1](v10) at ({(2*2 - 2)}, -6.0) {10};

      %%%%% EDGES
      \draw[STY2]  (v0) edge node[anchor=center, left, midway] {$a^{\prime\prime}$} (v1);
      \draw[STY2]  (v0) edge node[anchor=center, left, midway] {$b^{\prime\prime}$} (v2);
      \draw[STY2]  (v0) edge node[anchor=center, left, midway] {$c^{\prime\prime}$} (v3);

      \draw[STY2]  (v1) edge node[anchor=center, left, midway] {$\infty$} (v4);
      \draw[STY2]  (v1) edge node[anchor=center, left, midway] {$\infty$} (v5);
      \draw[STY2]  (v2) edge node[anchor=center, right, midway] {$\infty$} (v4);
      \draw[STY2]  (v2) edge node[anchor=center, left, midway] {$\infty$} (v6);
      \draw[STY2]  (v3) edge node[anchor=center, right, midway] {$\infty$} (v5);
      \draw[STY2]  (v3) edge node[anchor=center, right, midway] {$\infty$} (v6);

      \draw[STY2]  (v5) edge node[anchor=center, left, midway] {$\infty$} (v9);
      \draw[STY2]  (v6) edge node[anchor=center, right, midway] {$\infty$} (v9);
      \draw[STY2]  (v9) edge node[anchor=center, left, midway] {$z$} (v10);

      %     \node[draw=none,fill=none, minimum height=1cm, minimum width=1cm] at (0.0, 2.0) {c)};
    \end{tikzpicture}
  }
  \caption[Divisão da instância $I_{MF}$ do problema de fluxo máximo em três etapas.]{Divisão da instância $I_{MF}$ do problema de fluxo máximo em três etapas.}
  \label{figure:NCRGBSMG}
\end{figure}

A solução final $X$ consiste na soma de todas as capacidades utilizadas pelas soluções nas etapas 1, 2 e 3, ou seja, $\forall$~$1\leq i,j\leq 10$, $X_{i, j} = X^{\prime}_{i,j} + X^{\prime\prime}_{i,j} + X^{\prime\prime\prime}_{i,j}$. Observe que a solução $X$ não viola nenhuma restrição de capacidade, todas as variáveis são inteiras e o fluxo máximo é $F=a+b+c=x+y+z$.

\end{proof}

Em resumo, o Lema~\ref{lemma:RAJPFOWJ} nos permite, com duas transposições consecutivas, redistribuir o tamanho de três regiões intergênicas mantendo os genes na mesma ordem e orientação.

\begin{lemma}\label{lemma:FSGHLWJU}
Dada uma instância intergênica rígida balanceada sem sinais $\mathcal{I}=((\pi,\breve\pi),\break(\iota,\breve\iota))$ com pelo menos dois breakpoints sobrecarregados, então existe uma sequência de duas transposições que remove pelo menos dois breakpoints tipo um de $\mathcal{I}$.
\end{lemma}
\begin{proof}
Seja $(\pi_i,\pi_{i+1})$ e $(\pi_j,\pi_{j+1})$ dois breakpoints sobrecarregados de $\mathcal{I}$.
Agora, observe que deve existir um terceiro breakpoint tipo um $(\pi_k,\pi_{k+1})$ em $\mathcal{I}$. Caso contrário, $\mathcal{I}$ seria uma instância desbalanceada. Pelo Lema~\ref{lemma:RAJPFOWJ}, sabemos que é possível realizar qualquer redistribuição de nucleotídeos em três regiões intergênicas utilizando duas transposições consecutivas. Dessa forma, podemos realizar a redistribuição do tamanho das regiões intergênicas $\breve\pi_{i+1}$, $\breve\pi_{j+1}$ e $\breve\pi_{k+1}$ para $\breve\iota_{\max(\pi_i,\pi_{i+1})}$, $\breve\iota_{\max(\pi_j,\pi_{j+1})}$ e $\breve\pi_{k+1} + (\breve\pi_{i+1} - \breve\iota_{\max(\pi_i,\pi_{i+1})}) + (\breve\pi_{j+1} - \breve\iota_{\max(\pi_j,\pi_{j+1})})$, respectivamente. Neste caso, o excesso de nucleotídeos nos breakpoits sobrecarregados é transferido para o breakpoint $(\pi_k,\pi_{k+1})$. Como resultado, pelo menos dois breakpoints tipo um são removidos após a aplicação de duas transposições, e o lema segue.
\end{proof}

\begin{lemma}\label{lemma:RHTVEKOL}
Dada uma instância intergênica rígida balanceada sem sinais $\mathcal{I}=((\pi,\breve\pi),\break(\iota,\breve\iota))$ com apenas um breakpoint sobrecarregado $(\pi_i,\pi_{i+1})$ e pelo menos um breakpoint subcarregado $(\pi_j,\pi_{j+1})$, tal que $\breve\pi_{i+1} + \breve\pi_{j+1} \ge \breve\iota_{x} + \breve\iota_{y}$, onde $x = \max(\pi_i,\pi_{i+1})$ e $y=\max(\pi_j,\pi_{j+1})$, então existe uma sequência de duas transposições que remove pelo menos dois breakpoints tipo um de $\mathcal{I}$.
\end{lemma}
\begin{proof}
Note que o excesso de nucleotídeos na região intergênica $\breve\pi_{i+1}$ é maior ou igual a quantidade de nucleotídeos que falta na região intergênica $\breve\pi_{j+1}$. Caso $ib_1(\mathcal{I} \ge 3)$, utilizaremos, para selecionar o terceiro breakpoint tipo um $(\pi_k,\pi_{k+1})$ ($k \notin \{i,j\}$), a seguinte ordem de prioridade: breakpoint suave, breakpoint sobrecarregado e breakpoint subcarregado. Note que o terceiro breakpoint tipo um deve obrigatoriamente ser de um dos tipos da lista de prioridade. Em seguida, aplicamos o mesmo processo descrito no Lema~\ref{lemma:FSGHLWJU}. Caso contrário, selecionamos uma região intergênica $\breve\pi_{k+1}$ de forma que $k+1 \notin \{i+1,j+1\}$ e, pelo Lema~\ref{lemma:RAJPFOWJ}, realizamos a redistribuição do tamanho das regiões intergênicas $\breve\pi_{i+1}$, $\breve\pi_{j+1}$ e $\breve\pi_{k+1}$ para $\breve\iota_{\max(\pi_i,\pi_{i+1})}$, $\breve\iota_{\max(\pi_j,\pi_{j+1})}$ e $\breve\pi_{k+1}$, respectivamente. Note que a região intergênica $\breve\pi_{k+1}$ é utilizada apenas para transferir o excesso de nucleotídeos de $\breve\pi_{i+1}$ para $\breve\pi_{j+1}$, mas após a aplicação das duas transposições consecutivas seu tamanho permanece o mesmo. Como resultado, pelo menos dois breakpoints tipo um são removidos após a aplicação de duas transposições, e o lema segue.
\end{proof}

\begin{lemma}\label{lemma:ICDGSTEE}
Dada uma instância intergênica rígida balanceada sem sinais $\mathcal{I}=((\pi,\breve\pi),\break(\iota,\breve\iota))$ com apenas um breakpoint sobrecarregado $(\pi_i,\pi_{i+1})$ e sem nenhum breakpoint subcarregado $(\pi_j,\pi_{j+1})$, tal que $\breve\pi_{i+1} + \breve\pi_{j+1} \ge \breve\iota_{x} + \breve\iota_{y}$, onde $x = \max(\pi_i,\pi_{i+1})$ e $y=\max(\pi_j,\pi_{j+1})$, então existe uma sequência de duas reversões que remove o breakpoint sobrecarregado $(\pi_i,\pi_{i+1})$ de $\mathcal{I}$ e não gera nenhum outro.
\end{lemma}
\begin{proof}
Note que um breakpoint sobrecarregado sempre vai estar conectado com qualquer outro breakpoint tipo um. Além disso, um segundo breakpoint tipo um $(\pi_k,\pi_{k+1})$ deve existir (subcarregado ou suave). Dessa forma, pelo caso $iv$ do Lema~\ref{lemma:IMYFBWDY}, temos que os breakpoints $(\pi_i,\pi_{i+1})$ e $(\pi_k,\pi_{k+1})$ estão conectados e o breakpoint sobrecarregado $(\pi_i,\pi_{i+1})$ é removido por uma sequência de duas reversões. Como não existe nenhum breakpoint subcarregado $(\pi_j,\pi_{j+1})$ em $\mathcal{I}$, tal que $\breve\pi_{i+1} + \breve\pi_{j+1} \ge \breve\iota_{\max(\pi_i,\pi_{i+1})} + \breve\iota_{\max(\pi_j,\pi_{j+1})}$, isso implica que a aplicação das duas reversões não gera breakpoints sobrecarregados, e o lema segue.
\end{proof}

\begin{lemma}\label{lemma:GZNXMCLB}
Dada uma instância intergênica rígida balanceada sem sinais $\mathcal{I}=((\pi,\breve\pi),\break(\iota,\breve\iota))$ sem breakpoints sobrecarregados e com $ib_1(\mathcal{I}) > 0$, então deve existir em $\mathcal{I}$ pelo menos um par suavemente conectado de breakpoints.
\end{lemma}
\begin{proof}
Suponha por contradição que $\mathcal{I}=((\pi,\breve\pi),(\iota,\breve\iota))$ sem breakpoints sobrecarregados e $ib_1(\mathcal{I}) > 0$ é uma instância intergênica rígida balanceada sem sinais, sem breakpoints sobrecarregados, com $ib_1(\mathcal{I}) > 0$ e não existe em  $\mathcal{I}$ um par suavemente conectado de breakpoints. Como $\mathcal{I}$ não possui breakpoints sobrecarregados, devem existir pelo menos dois breakpoints suaves. Caso contrário, $\mathcal{I}$ teria apenas breakpoints subcarregados e isso implicaria que $\mathcal{I}$ é uma instância desbalanceada, ou seja, $\sum_{i=1}^{n+1}\breve\pi_i < \sum_{i=1}^{n+1}\breve\iota_i$, o que contradiz a suposição de que $\mathcal{I}$ é uma instância intergênica rígida balanceada. Entretanto, como não existe em  $\mathcal{I}$ um par suavemente conectado de breakpoints, isso significa que os nucleotídeos presentes nas regiões intergênicas dos breakpoints suaves não é suficiente para removê-los sem torná-los em breakpoints subcarregados. Logo, temos que $\sum_{i=1}^{n+1}\breve\pi_i < \sum_{i=1}^{n+1}\breve\iota_i$, o que contradiz a suposição de que $\mathcal{I}$ é uma instância intergênica rígida balanceada.
\end{proof}

\begin{lemma}\label{lemma:LRCEAVRZ}
Dada uma instância intergênica rígida sem sinais $\mathcal{I}=((\pi,\breve\pi),(\iota,\breve\iota))$ e sejam $(\pi_i,\pi_{i+1})$ e $(\pi_j,\pi_{j+1})$ breakpoints suavemente conectados, então é possível remover pelo menos um breakpoint tipo um de $\mathcal{I}$ utilizando no máximo uma reversão ou uma transposição.
\end{lemma}
\begin{proof}
O Lema~\ref{lemma:SIAFJFDO} apresenta os quatro casos que abrangem todas as possibilidades a partir de um par conectado de breakpoints. Em particular, os casos $i$, $ii$ e $iii$ são os únicos em que é possível que ambos os breakpoints tipo um sejam suaves. Nos três casos apenas uma reversão ou uma transposição é utilizada para remover pelo menos um breakpoint tipo um de $\mathcal{I}$. Logo, o lema segue.
\end{proof}

A seguir apresentamos o Algoritmo~\ref{algorithm:JQHVZACM} para a variação sem sinais do problema \SbIRT{}.

\begin{algorithm}[!tbh]
  \caption{Um algoritmo de aproximação para o problema \SbIRT{}.\label{algorithm:JQHVZACM}}
  \Entrada{Uma instância intergênica rígida balanceada sem sinais $\mathcal{I}=((\pi,\breve\pi),(\iota,\breve\iota))$}
  \Saida{Uma sequência de reversões e transposições $S$, tal que $(\pi,\breve\pi) \cdot S = (\iota,\breve\iota)$}
    Seja $S \gets ()$ \\
    \Enqto{$ib(\mathcal{I}) > 1$}{
      \Se{existir pelo menos dois breakpoints sobrecarregados em $\mathcal{I}$}{
        \tcp{Lema~\ref{lemma:FSGHLWJU}}
        $S' \gets (\tau_1,\tau_2)$ \\
      }
      \SenaoSe{existir apenas um breakpoint sobrecarregado $(\pi_i,\pi_{i+1})$ em $\mathcal{I}$}{
        \Se{existir um breakpoint subcarregado $(\pi_j,\pi_{j+1})$ em $\mathcal{I}$, tal que $\breve\pi_{i+1} + \breve\pi_{j+1} \ge \breve\iota_{x} + \breve\iota_{y}$, onde $x = \max(\pi_i,\pi_{i+1})$ e $y=\max(\pi_j,\pi_{j+1})$}{
          \tcp{Lema~\ref{lemma:RHTVEKOL}}
          $S' \gets (\tau_1,\tau_2)$ ou $(\rho_1,\rho_2)$ \\
        }\Senao{
          \tcp{Lema~\ref{lemma:ICDGSTEE}}
          $S' \gets (\rho_1,\rho_2)$ \\
        }
      }\Senao{
        \tcp{Lemas~\ref{lemma:GZNXMCLB} e~\ref{lemma:LRCEAVRZ}}
        $(\pi_i,\pi_{i+1})$, $(\pi_j,\pi_{j+1}) \gets $ encontre um par suavemente conectado de breakpoints \\
        \Se{$(\pi_i,\pi_{i+1})$, $(\pi_j,\pi_{j+1})$ pertence ao caso $(i)$}{
          $S' \gets (\rho_1)$ \\
        }\SenaoSe{$(\pi_i,\pi_{i+1})$, $(\pi_j,\pi_{j+1})$ pertence ao caso $(ii)$}{
          $S' \gets (\tau_1)$ \\
        }\SenaoSe{$(\pi_i,\pi_{i+1})$, $(\pi_j,\pi_{j+1})$ pertence ao caso $(iii)$}{
          $S' \gets (\tau_1)$ \\
        }
      }
      $S \gets S + S'$ \\
      $\mathcal{I} \gets ((\pi, \breve\pi) \cdot S',(\iota,\breve\iota))$ \\

    }
  \Retorna{S}
\end{algorithm}

\begin{lemma}\label{lemma:RNJHXOWZ}
Dada uma instância intergênica rígida balanceada sem sinais $\mathcal{I}=((\pi,\breve\pi),\break(\iota,\breve\iota))$, o Algoritmo~\ref{algorithm:JQHVZACM} transforma $(\pi,\breve\pi)$ em $(\iota,\breve\iota)$ utilizando no máximo $\frac{3ib_1(\mathcal{I})}{2}$ eventos de reversão e transposição.
\end{lemma}
\begin{proof}
Podemos realizar a análise de cada iteração do Algoritmo~\ref{algorithm:JQHVZACM} considerando duas fases:
\begin{itemize}
  \item A remoção de breakpoints sobrecarregados: Caso existam dois ou mais breakpoints sobrecarregados em $\mathcal{I}$ duas transposições são aplicadas removendo dois breakpoints sobrecarregados (linhas 3-4). Caso exista apenas um breakpoint sobrecarregado em $\mathcal{I}$, então é verificado se existe um breakpoint subcarregado de forma que o excesso de nucleotídeos na região intergênica do breakpoint sobrecarregado seja suficiente para remover o breakpoint subcarregado. Caso exista, duas transposições são aplicadas removendo o tanto o breakpoint sobrecarregado como o subcarregado (linhas 6-7). Caso contrário, o breakpoint sobrecarregado é removido com duas reversões sem gerar nenhum breakpoint sobrecarregado (linhas 8-9).
  \item A remoção de breakpoints suaves: Se algoritmo chegou até esse ponto isso significa que não existe nenhum breakpoint sobrecarregado em $\mathcal{I}$ e deve existir pelo menos um par suavemente conectado de breakpoints. Dado um par suavemente conectado de breakpoints, então é possível remover um breakpoint tipo um utilizando no máximo uma reversão ou uma transposição.
\end{itemize}
Note que, a cada iteração do Algoritmo~\ref{algorithm:JQHVZACM}, pelo menos um breakpoint tipo um é removido. Dessa forma, o genoma alvo eventualmente será alcançado. Além disso, observe que, no pior caso, pelo menos um breakpoint tipo um é removido por duas reversões na fase de remoção de breakpoints sobrecarregados e pelo menos um breakpoint tipo um é removido por uma reversão ou uma transposição na fase de remoção de breakpoints suaves. Entretanto, se o pior caso da fase de remoção de breakpoints sobrecarregados ocorrer, sabemos que: i) o genoma alvo ainda não foi alcançado, ou seja, $(\pi,\breve\pi)$ é diferente de $(\iota,\breve\iota)$; ii) $\mathcal{I}$ não possui mais nenhum breakpoint sobrecarregado. Com essas duas constatações temos que o pior caso da fase de remoção de breakpoints sobrecarregados é obrigatoriamente seguido por uma fase de remoção de breakpoints suaves. Logo, no pior caso, temos que pelo menos dois breakpoints tipo um são removidos após a aplicação de no máximo três eventos de reverão e transposição. Como inicialmente $\mathcal{I}$ possui $ib_1(\mathcal{I})$ breakpoints tipo um, então no máximo $\frac{3ib_1(\mathcal{I})}{2}$ eventos de reversão e transposição são utilizados pelo Algoritmo~\ref{algorithm:JQHVZACM} para transformar $(\pi,\breve\pi)$ em $(\iota,\breve\iota)$, e o lema segue.
\end{proof}

Note que as fases de remoção de breakpoints sobrecarregados e suaves podem ser realizadas em tempo linear. Como a quantidade máxima de breakpoints tipo um em uma instância é $n+1$ e algoritmo, a cada iteração, remove pelo menos um breakpoint tipo um, então o tempo de execução do Algoritmo~\ref{algorithm:JQHVZACM} é $\mathcal{O}(n^2)$.

\begin{theorem}\label{theorem:QKJNIMOI}
Dada uma instância intergênica rígida balanceada sem sinais $\mathcal{I}=\break((\pi,\breve\pi),(\iota,\breve\iota))$, o Algoritmo~\ref{algorithm:JQHVZACM} é uma $4.5$-aproximação para o problema \SbIRT{}.
\end{theorem}
\begin{proof}
Pelo Lema~\ref{lemma:RNJHXOWZ}, o Algoritmo~\ref{algorithm:JQHVZACM} transforma $(\pi,\breve\pi)$ em $(\iota,\breve\iota)$ utilizando no máximo $\frac{3ib_1(\mathcal{I})}{2}$ eventos de reversão e transposição. Pelo Teorema~\ref{theorem:MPFPKHQO}, temos o seguinte limitante inferior $d_{\SbIRT}(\mathcal{I}) \ge \frac{ib_1(\mathcal{I})}{3}$. Logo, o teorema segue. 
\end{proof}

A seguir apresentaremos lemas que serão utilizados para obtermos um algoritmo de aproximação com fator $4$ para a variação sem sinais do problema \SbIRT{}.

\begin{lemma}\label{lemma:XPQZERDR}
Dada uma instância intergênica rígida balanceada sem sinais $\mathcal{I}=((\pi,\breve\pi),\break(\iota,\breve\iota))$, se $ib_1(\mathcal{I}) > 0$ e não existir nenhum par suavemente conectado de breakpoints em $\mathcal{I}$, então deve existir pelo menos um breakpoint sobrecarregado em $\mathcal{I}$.
\end{lemma}
\begin{proof}
Suponha por contradição que não existe um breakpoint sobrecarregado em $\mathcal{I}$. Note que $\mathcal{I}$ não pode ter apenas breakpoints subcarregados, pois isso implica que $\sum_{i=1}^{n+1}\breve\pi_i < \sum_{i=1}^{n+1}\breve\iota_i$, o que contradiz a suposição de que $\mathcal{I}$ é uma instância intergênica rígida balanceada. Neste caso, devem existir pelo menos dois breakpoints suaves. Entretanto, como não existe em  $\mathcal{I}$ um par suavemente conectado de breakpoints, isso significa que os nucleotídeos presentes nas regiões intergênicas dos breakpoints suaves não é suficiente para removê-los sem torná-los em breakpoints subcarregados. Logo, temos que $\sum_{i=1}^{n+1}\breve\pi_i < \sum_{i=1}^{n+1}\breve\iota_i$, o que contradiz a suposição de que $\mathcal{I}$ é uma instância intergênica rígida balanceada.
\end{proof}

\begin{lemma}\label{lemma:DWXIBBXO}
Dada uma instância intergênica rígida balanceada sem sinais $\mathcal{I}=((\pi,\breve\pi),\break(\iota,\breve\iota))$, se $\mathcal{I}$ possui apenas um breakpoint sobrecarregado $(\pi_i,\pi_{i+1})$, pelo menos um breakpoint subcarregado $(\pi_j,\pi_{j+1})$ e nenhum par suavemente conectado de breakpoints, então $\breve\pi_{i+1} + \breve\pi_{j+1} \ge \breve\iota_{x} + \breve\iota_{y}$, onde $x = \max(\pi_i,\pi_{i+1})$ e $y=\max(\pi_j,\pi_{j+1})$.
\end{lemma}
\begin{proof}
Suponha por contradição que $\breve\pi_{i+1} + \breve\pi_{j+1} < \breve\iota_{x} + \breve\iota_{y}$. Como não existe nenhum par suavemente conectado de breakpoints em $\mathcal{I}$, temos que $\mathcal{I}$ não possui breakpoints suaves ou a quantidade de nucleotídeos presente em suas regiões intergênicas é insuficiente para removê-los. Em ambos os casos, ao mover o excesso de nucleotídeos da região intergênica $\breve\pi_{i+1}$ do breakpoint sobrecarregado $(\pi_i,\pi_{i+1})$ para a região intergênica $\breve\pi_{j+1}$ do breakpoint subcarregado $(\pi_j,\pi_{j+1})$, temos que $\mathcal{I}$ ainda permanece com pelo menos um breakpoint subcarregado e possivelmente breakpoits suaves que não estão suavemente conectados. Logo, temos que $\sum_{i=1}^{n+1}\breve\pi_i < \sum_{i=1}^{n+1}\breve\iota_i$, o que contradiz a suposição de que $\mathcal{I}$ é uma instância intergênica rígida balanceada.
\end{proof}

\begin{lemma}\label{lemma:QSQPQMYH}
Dada uma instância intergênica rígida balanceada sem sinais $\mathcal{I}=((\pi,\breve\pi),\break(\iota,\breve\iota))$, se $\mathcal{I}$ possui apenas um breakpoint sobrecarregado $(\pi_i,\pi_{i+1})$, pelo menos um breakpoint subcarregado $(\pi_j,\pi_{j+1})$ e nenhum par suavemente conectado de breakpoints, então existe uma sequência de duas transposições que remove pelo menos dois breakpoints tipo um de $\mathcal{I}$.
\end{lemma}
\begin{proof}
Pelo Lema~\ref{lemma:DWXIBBXO}, sabemos que $\breve\pi_{i+1} + \breve\pi_{j+1} \ge \breve\iota_{x} + \breve\iota_{y}$. Logo, podemos aplicar o Lema~\ref{lemma:RHTVEKOL}, e o lema segue.
\end{proof}

Note que a sequência de transposições aplicadas pelo Lema~\ref{lemma:QSQPQMYH}, gera no máximo um breakpoint sobrecarregado. Entretanto, se isso ocorrer implica que a instância $\mathcal{I}$ não possui breakpoints suaves. Este fato é decorrente da lista de prioridade para a seleção do terceito breakpoint no Lema~\ref{lemma:RHTVEKOL}. Uma vez que adicionar ou remover nucleotídeos em uma região intergênica de um breakpoint suave não transforma-o em um breakpoint forte.

\begin{remark}\label{remark:QVNWZDDQ}
Nenhum breakpoint super forte pode ser removido por uma operação de reversão ou transposição resultante do Lema~\ref{lemma:LRCEAVRZ}.
\end{remark}

\begin{lemma}\label{lemma:OJTODIFY}
Dada uma instância intergênica rígida balanceada sem sinais $\mathcal{I}=((\pi,\breve\pi),\break(\iota,\breve\iota))$, se $\mathcal{I}$ não possui um par suavemente conectado de breakpoints, então é possível, após aplicar uma reversão ou um transposição, criar um breakpoint subcarregado mantendo $\mathcal{I}$ sem um par suavemente conectado de breakpoints ou criar um breakpoint super forte subcarregado.
\end{lemma}
\begin{proof}
Se houver pelo menos uma strip suave decrescente em $\mathcal{I}$, deve existir um par de breakpoints suaves $(\pi_{i},\pi_{i+1})$ e $(\pi_{j},\pi_ {j+1})$, com $i <j$, tal que $(\pi_{i},\pi_{j})$ ou $(\pi_{i+1},\pi_{j+1} )$ são consecutivos em $\iota$~\cite{1995-kececioglu-sankoff}. Se $(\pi_{i},\pi_{j})$ são consecutivos em $\iota$, então aplicamos uma reversão $\rho_{(\breve\pi_{i+1},\breve\pi_{j +1})}^{(i+1, j)}$. Caso contrário, aplicamos uma reversão $\rho_{(0, 0)}^{(i+1, j)}$. Observe que em ambos os casos todos os nucleotídeos são movidos para o breakpoint forte subcarregado criado, o que garante que a instância permaneça sem um par suavemente conectado de breakpoints. Se não houver uma strip decrescente em $\mathcal{I}$, sempre é possível encontrar três breakpoints suaves $(\pi_{i},\pi_{i+1})$, $(\pi_{j},\pi_{ j+1})$ e $(\pi_{k},\pi_{k+1})$, de modo que uma transposição $\tau_{(0,0,0)}^{(i+1,j +1,k+1)}$ cria um breakpoint forte subcarregado e nenhum breakpoint forte é removido~\cite{1998-walter-etal}. Além disso, como a instância possui apenas strips suaves crescentes, temos a garantia de que o breakpoint forte subcarregado criado (unindo duas strips suaves crescentes) seja um breakpoint super forte subcarregado, e o lema segue.
\end{proof}

\begin{lemma}\label{lemma:WGHTDURW}
Dada uma instância intergênica rígida balanceada sem sinais $\mathcal{I}=((\pi,\breve\pi),\break(\iota,\breve\iota))$ com apenas um breakpoint sobrecarregado, sem nenhum breakpoint subcarregado e não existe nenhum par suavemente conectado de breakpoints em $\mathcal{I}$, então existe uma sequência com no máximo três operações que remove pelo menos dois breakpoints tipo um ou uma sequência com no máximo quatro operações que remove pelo menos três breakpoints tipo um.
\end{lemma}
\begin{proof}
Inicialmente podemos notar que $ib_1(\mathcal{I}) \ge 3$, uma vez que é impossível criar uma instância intergênica rígida balanceada com apenas um breakpoint sobrecarregado e um breakpoint suave. Aplicando o Lema~\ref{lemma:OJTODIFY}, temos duas possibilidades: (i) um breakpoint subcarregado é criado mantendo $\mathcal{I}$ sem um par suavemente conectado de breakpoints, então podemos aplicar o Lema~\ref{lemma:QSQPQMYH} (resultando em uma sequência com três operações que remove pelo menos dois breakpoints tipo um); (ii) um breakpoint super forte subcarregado é criado. Neste caso, se $\mathcal{I}$ continuar sem nenhum par suavemente conectado de breakpoints, então podemos aplicar o Lema~\ref{lemma:QSQPQMYH} (resultando também em uma sequência com três operações que remove pelo menos dois breakpoints tipo um). Caso contrário, o Lema~\ref{lemma:LRCEAVRZ} pode ser aplicado. Pela Observação~\ref{remark:QVNWZDDQ}, nenhum breakpoint super forte pode ser removido por uma operação de reversão ou transposição resultante do Lema~\ref{lemma:LRCEAVRZ}. Logo um dos seguintes casos pode ocorrer:
\begin{itemize}
  \item Um novo breakpoint sobrecarregado é criado, e podemos aplicar o Lema~\ref{lemma:FSGHLWJU} (resultando em uma sequência com quatro operações que remove pelo menos três breakpoints tipo um).
  \item Um par suavemente conectado de breakpoints é criado, e o Lema~\ref{lemma:LRCEAVRZ} é aplicado (resultando em uma sequência com três operações que remove pelo menos dois breakpoints tipo um).
  \item Não existe nenhum par suavemente conectado de breakpoints em $\mathcal{I}$, e o Lema~\ref{lemma:QSQPQMYH} é aplicado (resultando em uma sequência com quatro operações que remove pelo menos três breakpoints tipo um).
\end{itemize}
\end{proof}

\begin{remark}\label{remark:GWXLDIIV}
Note que se apenas dois breakpoints tipo um forem removidos pelo Lema~\ref{lemma:WGHTDURW}, então isso implica que o genoma alvo ainda não foi alcançado, ou seja, $(\pi,\breve\pi)$ é diferente de $(\iota,\breve\iota)$.
\end{remark}

Agora considere Algoritmo~\ref{algorithm:LCPCUFNZ}, que consiste em quatro casos dependendo do número de breakpoints sobrecarregados ou da existência de um par suavemente conectado de breakpoints.

\begin{algorithm}[!tbh]
  \caption{Um algoritmo de aproximação para o problema \SbIRT{}.\label{algorithm:LCPCUFNZ}}
  \Entrada{Uma instância intergênica rígida balanceada sem sinais $\mathcal{I}=((\pi,\breve\pi),(\iota,\breve\iota))$}
  \Saida{Uma sequência de reversões e transposições $S$, tal que $(\pi,\breve\pi) \cdot S = (\iota,\breve\iota)$}
    Seja $S \gets ()$ \\
    \Enqto{$ib_1(\mathcal{I}) > 1$}{
      \Se{existir pelo menos dois breakpoints sobrecarregados em $\mathcal{I}$}{
        \tcp{Lema~\ref{lemma:FSGHLWJU}}
        Seja $S'$ uma sequência com duas transposições que remove, pelo menos, dois breakpoints tipo um de $\mathcal{I}$\\
      }
      \SenaoSe{existir um par suavemente conectado de breakpoints em $\mathcal{I}$}{
        \tcp{Lema~\ref{lemma:LRCEAVRZ}}
        Seja $S'$ uma sequência com uma reversão ou uma transposição que remove, pelo menos, um breakpoint tipo um de $\mathcal{I}$\\
      }\Senao{
        \tcp{Existe exatamente um breakpoint sobrecarregado (Lema~\ref{lemma:XPQZERDR}) }
        \Se{existir pelo menos um breakpoint subcarregado em $\mathcal{I}$}{
          \tcp{Lema~\ref{lemma:QSQPQMYH}}
          Seja $S'$ uma sequência com duas transposições ou duas reversões que remove, pelo menos, dois breakpoints tipo um de $\mathcal{I}$\\
        }\Senao{
          \tcp{Lema~\ref{lemma:WGHTDURW}}
          Seja $S'$ uma sequência com, no máximo, três operações que remove, pelo menos, dois breakpoints tipo um de $\mathcal{I}$ ou uma sequência com, no máximo, quatro operações que remove, pelo menos, três breakpoints tipo um de $\mathcal{I}$\\
        }
      }
      $S \gets S + S'$ \\
      $\mathcal{I} \gets ((\pi, \breve\pi) \cdot S',(\iota,\breve\iota))$ \\
    }
  \Retorna{S}
\end{algorithm}

\begin{lemma}\label{lemma:HIIRAXUH}
Dada uma instância intergênica rígida balanceada sem sinais $\mathcal{I}=((\pi,\breve\pi),\break(\iota,\breve\iota))$, o Algoritmo~\ref{algorithm:LCPCUFNZ} transforma $(\pi,\breve\pi)$ em $(\iota,\breve\iota)$ utilizando no máximo $\frac{4b_1(\mathcal{I})}{3}$ eventos de reversão e transposição.
\end{lemma}
\begin{proof}
O Algoritmo~\ref{algorithm:LCPCUFNZ} pode ser analisado considerando os seguintes casos:
\begin{itemize}
  \item $\mathcal{I}$ tem pelo menos dois breakpoints sobrecarregados (linhas 3-4).
  \item $\mathcal{I}$ tem pelo menos um par suavemente conectado de breakpoints (linhas 5-6).
  \item $\mathcal{I}$ tem apenas um breakpoint sobrecarregado, pelo menos um breakpoint subcarregado e sem nenhum um par suavemente conectado de breakpoints (linhas 8-9).
  \item $\mathcal{I}$ tem apenas um breakpoint sobrecarregado, nenhum breakpoint subcarregado e sem nenhum um par suavemente conectado de breakpoints (linhas 10-11).
\end{itemize}
Note que, a cada iteração do Algoritmo~\ref{algorithm:JQHVZACM}, pelo menos um dos quatro casos deve obrigatoriamente ser aplicado e pelo menos um breakpoint tipo um é removido. Dessa forma, o genoma alvo eventualmente será alcançado e o algoritmo para. Observe que, se o algoritmo atingir os casos 3 ou 4, há exatamente um breakpoint sobrecarregado em $\mathcal{I}$ (Lema~\ref{lemma:XPQZERDR}) e sem nenhum um par suavemente conectado de breakpoints. Caso contrário, o caso 1 ou 2 seria aplicado.

Os casos 1, 2 e 3 removem, em média, um breakpoint tipo um por operação. No pior cenário do caso 4 (onde dois breakpoints tipo um são removidos com três operações), temos pela Observação~\ref{remark:GWXLDIIV} que $(\pi,\breve\pi) \neq (\iota ,\breve\iota)$, e o caso 1, 2 ou 3 será aplicado posteriormente, com todos eles garantindo uma sequência de operações que remove, em média, um breakpoint tipo um por operação. Assim, em média, cada breakpoint tipo um é removido usando no máximo $\frac{4}{3}$ operações, e segue o lema.
\end{proof}

Note que cada caso do Algoritmo~\ref{algorithm:LCPCUFNZ} é realizado em tempo linear utilizando as estruturas auxiliares de lista de breakpoints e a permutação inversa de $\pi$ (ou seja, uma permutação que indica a posição de cada elemento $i$ em $\pi$). Como $ib_1(\mathcal{I}) \le n + 1$, o tempo de execução do Algoritmo~\ref{algorithm:LCPCUFNZ} é $\mathcal{O}(n^2)$.

\begin{theorem}\label{theorem:USRSHCGH}
Dada uma instância intergênica rígida balanceada sem sinais $\mathcal{I}=\break((\pi,\breve\pi),(\iota,\breve\iota))$, o Algoritmo~\ref{algorithm:JQHVZACM} é uma $4$-aproximação para o problema \SbIRT{}.
\end{theorem}
\begin{proof}
Pelo Lema~\ref{lemma:HIIRAXUH}, o Algoritmo~\ref{algorithm:LCPCUFNZ} transforma $(\pi,\breve\pi)$ em $(\iota,\breve\iota)$ utilizando no máximo $\frac{4ib_1(\mathcal{I})}{3}$ eventos de reversão e transposição. Pelo Teorema~\ref{theorem:MPFPKHQO}, temos o seguinte limitante inferior $d_{\SbIRT}(\mathcal{I}) \ge \frac{ib_1(\mathcal{I})}{3}$. Logo, o teorema segue. 
\end{proof}

% ------------------------------------------------------------------ %
\subsection{Reversão, Transposição e Indel}
% ------------------------------------------------------------------ %

Nesta seção apresentaremos um algoritmo de aproximação com fator $4$ para a variação sem sinais do problema \SbIRTI{}. 

\begin{lemma}\label{lemma:UWWIFTBQ}
Dada uma instância intergênica rígida sem sinais $\mathcal{I}=((\pi,\breve\pi),(\iota,\breve\iota))$ e seja $S$ uma sequência de reversões, transposições e indels que transforma $(\iota,\breve\iota)$ em $(\pi,\breve\pi)$, então é possível construir uma sequência $S'$ que transforma $(\pi,\breve\pi)$ em $(\iota,\breve\iota)$, tal que $|S| = |S'|$.
\end{lemma}
\begin{proof}
Criaremos a sequência $S'$ com base nas operações da sequência $S$. Para cada evento de rearranjo $\beta$ em $S$, mantendo a ordem relativa, utilize o seguinte mapeamento:
\begin{itemize}
  \item Se $\beta$ for um reversão $\rho^{(i,j)}_{(x,y)}$, então adicione em $S'$ a reversão $\rho^{(i,j)}_{(x,x^{\prime})}$.
  \item Se $\beta$ for um transposição $\tau^{(i,j,k)}_{(x,y,z)}$, então adicione em $S'$ a transposição $\rho^{(i,i+k-j,k)}_{(x,z,y)}$.
  \item Se $\beta$ for um indel $\delta^{(i)}_{(x)}$, então adicione em $S'$ o indel $\delta^{(i)}_{(-x)}$. 
\end{itemize}  
Por fim, inverta a a sequência $S'$. Note que cada operação adicionada na sequência $S'$ desfaz a mudança realizada por sua respectiva operação $\beta$ da sequência $S$. Além disso, ao inverter a sequência $S'$ temos que a ordem em que as operações são aplicadas também são invertidas. Logo, $(\pi,\breve\pi) \cdot S' = (\iota,\breve\iota)$ e $|S| = |S'|$, e o lema segue.
\end{proof}

A Figura~\ref{figure:MINEYNFC} mostra um exemplo de uma sequência $S'=(\tau^{(3,6,7)}_{(1,1,0)},\rho^{(1,5)}_{(0,1)},\delta^{(0)}_{({+5})})$ sendo construída a partir de uma instância $\mathcal{I} = (((0,5,4,2,1,6,3,7),(3,3,1,2,3,3,0)),((0,1,2,\break3,4,5,6,7),(6,2,0,3,3,5,1)))$ e uma sequência $S=(\delta^{(0)}_{({-5})},\rho^{(1,5)}_{(0,3)},\tau^{(3,4,7)}_{(1,0,1)})$ que transforma o genoma alvo no genoma de origem.

\begin{figure}[!tbh]
  \scriptsize
  \hfill \break
  \resizebox{\linewidth}{!}{
    \centering
    \begin{tikzpicture}
      \node[fill = white!10, align = left, text width = 25mm, minimum width = 25mm] at (-0.5, 6.0) {$(\iota,\breve\iota) = $};
      \draw pic at  (1, 6) {ir = {$6$, black!10}};
      \draw pic at  (3, 6) {ir = {$2$, black!10}};
      \draw pic at  (5, 6) {ir = {$0$, black!10}};
      \draw pic at  (7, 6) {ir = {$3$, black!10}};
      \draw pic at  (9, 6) {ir = {$3$, black!10}};
      \draw pic at (11, 6) {ir = {$5$, black!10}};
      \draw pic at (13, 6) {ir = {$1$, black!10}};
      \draw pic at  (0, 6) {gene = {$0$, red!50}};
      \draw pic at  (2, 6) {gene = {$1$, orange!50}};
      \draw pic at  (4, 6) {gene = {$2$, blue!50}};
      \draw pic at  (6, 6) {gene = {$3$, teal!50}};
      \draw pic at  (8, 6) {gene = {$4$, green!50}};
      \draw pic at (10, 6) {gene = {$5$, brown!50}};
      \draw pic at (12, 6) {gene = {$6$, violet!50}};
      \draw pic at (14, 6) {gene = {$7$, purple!50}};

      \node[fill = white!10, align = left, text width = 25mm, minimum width = 25mm] at (4.0, 5.0) {$\delta^{(0)}_{({-5})}$};
      \node[fill = white!10, align = left, text width = 25mm, minimum width = 25mm] at (12.0, 5.0) {$\delta^{(0)}_{({+5})}$};
      \node [draw, single arrow, minimum width = 5mm, minimum height = 8mm, rotate = -90] at (4.0, 5.0) {};
      \node [draw, single arrow, minimum width = 5mm, minimum height = 8mm, rotate = 90] at (10.0, 5.0) {};
      \draw pic at  (1, 4) {ir = {$1$, black!10}};
      \draw pic at  (3, 4) {ir = {$2$, black!10}};
      \draw pic at  (5, 4) {ir = {$0$, black!10}};
      \draw pic at  (7, 4) {ir = {$3$, black!10}};
      \draw pic at  (9, 4) {ir = {$3$, black!10}};
      \draw pic at (11, 4) {ir = {$5$, black!10}};
      \draw pic at (13, 4) {ir = {$1$, black!10}};
      \draw pic at  (0, 4) {gene = {$0$, red!50}};
      \draw pic at  (2, 4) {gene = {$1$, orange!50}};
      \draw pic at  (4, 4) {gene = {$2$, blue!50}};
      \draw pic at  (6, 4) {gene = {$3$, teal!50}};
      \draw pic at  (8, 4) {gene = {$4$, green!50}};
      \draw pic at (10, 4) {gene = {$5$, brown!50}};
      \draw pic at (12, 4) {gene = {$6$, violet!50}};
      \draw pic at (14, 4) {gene = {$7$, purple!50}};

      \node[fill = white!10, align = left, text width = 25mm, minimum width = 25mm] at (4.0, 3.0) {$\rho^{(1,5)}_{(0,3)}$};
      \node[fill = white!10, align = left, text width = 25mm, minimum width = 25mm] at (12.0, 3.0) {$\rho^{(1,5)}_{(0,1)}$};
      \node [draw, single arrow, minimum width = 5mm, minimum height = 8mm, rotate = -90] at (4.0, 3.0) {};
      \node [draw, single arrow, minimum width = 5mm, minimum height = 8mm, rotate = 90] at (10.0, 3.0) {};
      \draw pic at  (1, 2) {ir = {$3$, black!10}};
      \draw pic at  (3, 2) {ir = {$3$, black!10}};
      \draw pic at  (5, 2) {ir = {$3$, black!10}};
      \draw pic at  (7, 2) {ir = {$0$, black!10}};
      \draw pic at  (9, 2) {ir = {$2$, black!10}};
      \draw pic at (11, 2) {ir = {$3$, black!10}};
      \draw pic at (13, 2) {ir = {$1$, black!10}};
      \draw pic at  (0, 2) {gene = {$0$, red!50}};
      \draw pic at  (2, 2) {gene = {$5$, brown!50}};
      \draw pic at  (4, 2) {gene = {$4$, green!50}};
      \draw pic at  (6, 2) {gene = {$3$, teal!50}};
      \draw pic at  (8, 2) {gene = {$2$, blue!50}};
      \draw pic at (10, 2) {gene = {$1$, orange!50}};
      \draw pic at (12, 2) {gene = {$6$, violet!50}};
      \draw pic at (14, 2) {gene = {$7$, purple!50}};

      \node[fill = white!10, align = left, text width = 25mm, minimum width = 25mm] at (4.0, 1.0) {$\tau^{(3,4,7)}_{(1,0,1)}$};
      \node[fill = white!10, align = left, text width = 25mm, minimum width = 25mm] at (12.0, 1.0) {$\tau^{(3,6,7)}_{(1,1,0)}$};
      \node [draw, single arrow, minimum width = 5mm, minimum height = 8mm, rotate = -90] at (4.0, 1.0) {};
      \node [draw, single arrow, minimum width = 5mm, minimum height = 8mm, rotate = 90] at (10.0, 1.0) {};
      \node[fill = white!10, align = left, text width = 25mm, minimum width = 25mm] at (-0.5, 0.0) {$(\pi,\breve\pi) = $};
      \draw pic at  (1, 0) {ir = {$3$, black!10}};
      \draw pic at  (3, 0) {ir = {$3$, black!10}};
      \draw pic at  (5, 0) {ir = {$1$, black!10}};
      \draw pic at  (7, 0) {ir = {$2$, black!10}};
      \draw pic at  (9, 0) {ir = {$3$, black!10}};
      \draw pic at (11, 0) {ir = {$3$, black!10}};
      \draw pic at (13, 0) {ir = {$0$, black!10}};
      \draw pic at  (0, 0) {gene = {$0$, red!50}};
      \draw pic at  (2, 0) {gene = {$5$, brown!50}};
      \draw pic at  (4, 0) {gene = {$4$, green!50}};
      \draw pic at  (6, 0) {gene = {$2$, blue!50}};
      \draw pic at  (8, 0) {gene = {$1$, orange!50}};
      \draw pic at (10, 0) {gene = {$6$, violet!50}};
      \draw pic at (12, 0) {gene = {$3$, teal!50}};
      \draw pic at (14, 0) {gene = {$7$, purple!50}};
    \end{tikzpicture}
  }
  \caption[Exemplo de construção de uma sequência $S'$ que transforma $(\pi,\breve\pi)$ em $(\iota,\breve\iota)$ a partir de uma sequência $S$ que transforma $(\iota,\breve\iota)$ em $(\pi,\breve\pi)$.]{Exemplo de construção de uma sequência $S'$ que transforma $(\pi,\breve\pi)$ em $(\iota,\breve\iota)$ a partir de uma sequência $S$ que transforma $(\iota,\breve\iota)$ em $(\pi,\breve\pi)$.}
  \label{figure:MINEYNFC}
\end{figure}

\begin{remark}\label{remark:MSXQJZFR}
Note que se temos uma instância intergênica rígida sem sinais $\mathcal{I}=((\pi,\breve\pi),(\iota,\breve\iota))$ e um algoritmo $\mathcal{A}$ que fornece uma sequência de eventos que transforma $(\pi,\breve\pi)$ em $(\iota,\breve\iota)$, então podemos utilizar o algoritmo $\mathcal{A}$ para obter uma sequência de eventos que transforma $(\iota,\breve\iota)$ em $(\pi,\breve\pi)$. Para isso, basta criarmos uma instância intergênica rígida sem sinais $\mathcal{I'}=((\pi^{-1} \circ \iota,\breve\iota),(\pi^{-1} \circ \pi,\breve\pi))$, onde $\alpha \circ \sigma$ representa a operação de composição entre as permutações $\alpha$ e $\sigma$. A sequência fornecida pelo algoritmo $\mathcal{A}$ para a instância $\mathcal{I'}$ também transforma $(\iota,\breve\iota)$ em $(\pi,\breve\pi)$.
\end{remark}

A composição entre as permutações $\alpha=(\alpha_1,\alpha_2,\dots,\alpha_n)$ e $\sigma=(\sigma_1,\sigma_2,\dots,\sigma_n)$ resulta em uma nova permutação $\alpha \circ \sigma = (\alpha_{\sigma_1},\alpha_{\sigma_2},\dots,\alpha_{\sigma_n})$. Além disso, $\sigma \circ \sigma^{-1} = \sigma^{-1} \circ \sigma = \iota$. Em outras palavras, ao criar a instância $\mathcal{I'}$ estamos mapeando o genes do genoma de origem (elementos da permutação $\pi$) com os valores padrão da permutação identidade e também alterando os valores dos genes do genoma alvo (elementos da permutação $\iota$) para refletir que os genes iguais no genoma de origem e alvo possuam o mesmo valor associado.

A seguir apresentamos o Algoritmo~\ref{algorithm:YIZYUGZZ} para a variação sem sinais do problema \SbIRTI{}.

\begin{algorithm}[!tbh]
  \caption{Um algoritmo de aproximação para o problema \SbIRTI{}.\label{algorithm:YIZYUGZZ}}
  \Entrada{Uma instância intergênica rígida sem sinais $\mathcal{I}=((\pi,\breve\pi),(\iota,\breve\iota))$}
  \Saida{Uma sequência de reversões, transposições e indels $S$, tal que $(\pi,\breve\pi) \cdot S = (\iota,\breve\iota)$}
    \Se{$\mathcal{I}$ for uma instância balanceada}{
      Seja $S$ uma sequência de operações fornecida pelo Algoritmo~\ref{algorithm:LCPCUFNZ} para a instância $\mathcal{I}$ \\
      \Retorna{$S$} \\
    }\Senao{
      \Se{$\sum_{i=1}^{n+1}\breve\pi_i < \sum_{i=1}^{n+1}\breve\iota_i$}{
        \tcp{Lema~\ref{lemma:QGOIQLZD}}
        Seja $\delta_1$ um indel que torna $\mathcal{I}$ uma instância balanceada \\
        $\mathcal{I} \gets ((\pi, \breve\pi) \cdot \delta_1,(\iota,\breve\iota))$ \\
        Seja $S$ uma sequência de operações fornecida pelo Algoritmo~\ref{algorithm:LCPCUFNZ} para a instância $\mathcal{I}$ \\
        \Retorna{$(\delta_1) + S$} \\
      }\Senao{
        $\mathcal{I'} \gets ((\pi^{-1} \circ \iota,\breve\iota),(\pi^{-1} \circ \pi,\breve\pi))$ \\
        \tcp{Lema~\ref{lemma:QGOIQLZD}}
        Seja $\delta_1$ um indel que torna $\mathcal{I'}$ uma instância balanceada \\
        $\mathcal{I'} \gets ((\pi^{-1} \circ  \iota,\breve\iota) \cdot \delta_1,(\pi^{-1} \circ \pi,\breve\pi))$ \\
        Seja $S$ uma sequência de operações fornecida pelo Algoritmo~\ref{algorithm:LCPCUFNZ} para a instância $\mathcal{I'}$ \\
        $S \gets (\delta_1) + S$ \\
        \tcp{Lema~\ref{lemma:UWWIFTBQ}}
        Seja $S'$ uma sequência, criada a partir de $S$ e $\mathcal{I}$, que transforma $(\pi,\breve\pi)$ em $(\iota,\breve\iota)$ \\
        \Retorna{$S'$} \\
      }
    }
\end{algorithm}

\begin{lemma}\label{lemma:MUTXDAUG}
Dada uma instância intergênica rígida sem sinais $\mathcal{I}=((\pi,\breve\pi),(\iota,\breve\iota))$, o Algoritmo~\ref{algorithm:YIZYUGZZ} transforma $(\pi,\breve\pi)$ em $(\iota,\breve\iota)$. Caso $\mathcal{I}$ seja uma instância intergênica rígida balanceada, então são utilizados no máximo $\frac{4b_1(\mathcal{I})}{3}$ eventos de reversão e transposição. Caso contrário, são utilizados no máximo $\frac{4b_1(\mathcal{I})}{3} + 1$ eventos de reversão, transposição e indel.
\end{lemma}
\begin{proof}
O Algoritmo~\ref{algorithm:YIZYUGZZ} pode ser analisado considerando dois cenários. Caso $\mathcal{I}$ seja uma instância intergênica rígida balanceada, então o Algoritmo~\ref{algorithm:LCPCUFNZ} é aplicado e, pelo Lema~\ref{lemma:HIIRAXUH}, a sequência eventos fornecida pelo Algoritmo~\ref{algorithm:LCPCUFNZ} transnforma $(\pi,\breve\pi)$ em $(\iota,\breve\iota)$ utilizando no máximo $\frac{4b_1(\mathcal{I})}{3}$ eventos de reversão e transposição. Caso contrário, temos duas possibilidades:
\begin{itemize}
  \item $\sum_{i=1}^{n+1}\breve\pi_i < \sum_{i=1}^{n+1}\breve\iota_i$: Neste caso, pelo Lema~\ref{lemma:QGOIQLZD}, existe um indel que torna $\mathcal{I}$ uma instância intergênica rígida balanceada e em seguida o Algoritmo~\ref{algorithm:LCPCUFNZ} pode ser aplicado, resultado no máximo em $\frac{4b_1(\mathcal{I})}{3} + 1$ eventos de reversão, transposição e indel para transnformar $(\pi,\breve\pi)$ em $(\iota,\breve\iota)$.
  \item $\sum_{i=1}^{n+1}\breve\pi_i > \sum_{i=1}^{n+1}\breve\iota_i$: Neste caso, pela Observação~\ref{remark:MSXQJZFR},uma instância auxiliar $\mathcal{I'}=((\pi',\breve\pi'),(\iota',\breve\iota'))$ é criada, onde $\breve\pi' = \breve\iota$, $\breve\iota' = \breve\pi$ e $\sum_{i=1}^{n+1}\breve\pi^{'}_i < \sum_{i=1}^{n+1}\breve\iota^{'}_i$. Pelo Lema~\ref{lemma:QGOIQLZD}, existe um indel que torna $\mathcal{I'}$ uma instância intergênica rígida balanceada e em seguida o Algoritmo~\ref{algorithm:LCPCUFNZ} pode ser aplicado, resultado em uma sequência $S$ com no máximo $\frac{4b_1(\mathcal{I})}{3} + 1$ eventos de reversão, transposição e indel que transnforma $(\iota,\breve\iota)$ em $(\pi,\breve\pi)$. Pelo Lema~\ref{lemma:UWWIFTBQ}, podemos criar uma sequência $S'$, com o mesmo tamanho de $S$, e que transforma $(\pi,\breve\pi)$ em $(\iota,\breve\iota)$.
\end{itemize}
Em ambos os cenários $(\pi,\breve\pi)$ é transformado em $(\iota,\breve\iota)$ e a quantidade de eventos utilizados para tal não ultrapassa o limite estabelecido, e o lema segue.
\end{proof}

Note que o algoritmo Algoritmo~\ref{algorithm:YIZYUGZZ} não possui laços de repetição e a subrotina com maior gasto computacional de tempo é decorrente do uso do Algoritmo~\ref{algorithm:LCPCUFNZ} ($\mathcal{O}(n^2)$). Logo, o tempo de execução do algoritmo Algoritmo~\ref{algorithm:YIZYUGZZ} também é $\mathcal{O}(n^2)$.

\begin{theorem}\label{theorem:ZEIGUWRR}
Dada uma instância intergênica rígida sem sinais $\mathcal{I}=((\pi,\breve\pi),\break(\iota,\breve\iota))$, o Algoritmo~\ref{algorithm:YIZYUGZZ} é uma $4$-aproximação para o problema \SbIRTI{}.
\end{theorem}
\begin{proof}
Pelo Lema~\ref{lemma:MUTXDAUG}, o Algoritmo~\ref{algorithm:LCPCUFNZ} transforma $(\pi,\breve\pi)$ em $(\iota,\breve\iota)$. Além disso, caso $\mathcal{I}$ seja uma instância intergênica rígida balanceada, então são utilizados no máximo $\frac{4b_1(\mathcal{I})}{3}$ eventos de reversão e transposição. Caso contrário, são utilizados no máximo $\frac{4b_1(\mathcal{I})}{3} + 1$ eventos de reversão, transposição e indel. Pelo Teorema~\ref{theorem:JDOIUJLE}, temos o seguinte limitante inferior. Caso $\mathcal{I}$ seja uma instância intergênica rígida balanceada, $d_{\SbIRTI}(\mathcal{I}) \ge \frac{ib_1(\mathcal{I})}{3}$. Caso contrário, $d_{\SbIRTI}(\mathcal{I}) \ge \frac{ib_1(\mathcal{I}) + 2}{3}$. Se $\mathcal{I}$ for balanceada, temos que $\frac{\frac{4b_1(\mathcal{I})}{3}}{\frac{ib_1(\mathcal{I})}{3}}=4$. Se $\mathcal{I}$ for desbalanceada,  temos que $\frac{\frac{4b_1(\mathcal{I})}{3} + 1}{\frac{ib_1(\mathcal{I}) + 2}{3}}=\frac{\frac{4b_1(\mathcal{I})+3}{3}}{\frac{ib_1(\mathcal{I}) + 2}{3}}=\frac{4b_1(\mathcal{I})+3}{b_1(\mathcal{I})+2}$. Entretanto, como $\frac{4b_1(\mathcal{I})+3}{b_1(\mathcal{I})+2}<\frac{4(b_1(\mathcal{I})+2)}{b_1(\mathcal{I})+2}=4$, o teorema segue.
\end{proof}

% ------------------------------------------------------------------ %
\subsection{Reversão, Transposição e Move}
% ------------------------------------------------------------------ %

Nesta seção apresentaremos um algoritmo de aproximação com fator $3$ para a variação sem sinais do problema \SbIRTM{}. 

\begin{lemma}\label{lemma:YLNUFQYG}
Dada uma instância intergênica rígida sem sinais $\mathcal{I}=((\pi,\breve\pi),(\iota,\breve\iota))$ e sejam $(\pi_i,\pi_{i+1})$ e $(\pi_j,\pi_{j+1})$ breakpoints conectados, então é possível remover pelo menos um breakpoint tipo um de $\mathcal{I}$ utilizando no máximo um evento de reversão, transposição ou move.
\end{lemma}
\begin{proof}
Sem perda de generalidade assuma que $i < j$, como os breakpoints $(\pi_i,\pi_{i+1})$ e $(\pi_j,\pi_{j+1})$ estão conectados, por definição, uma das seguintes possibilidades deve ocorrer:
\begin{enumerate}[i.]
  \item O par de elementos $(\pi_i,\pi_{j})$ ou $(\pi_{i+1},\pi_{j+1})$ não formam uma adjacência intergênica, são consecutivos em $\iota$ e $\breve\pi_{i+1} + \breve\pi_{j+1} \ge \breve\iota_k$, onde $\breve\iota_k$ é o tamanho da região intergênica entre o par de elementos consecutivos em $\iota$. Aplique uma reversão como descrito no caso $i$ do Lema~\ref{lemma:IMYFBWDY}.
  \item O par  de elementos $(\pi_i,\pi_{j+1})$ não formam uma adjacência intergênica, são consecutivos em $\iota$ e $\breve\pi_{i+1} + \breve\pi_{j+1} \ge \breve\iota_k$, onde $\breve\iota_k$ é o tamanho da região intergênica entre o par de elementos consecutivos em $\iota$. Aplique uma transposição como descrito no caso $ii$ do Lema~\ref{lemma:SIAFJFDO}.
  \item O par de elementos $(\pi_{i+1},\pi_{j})$ não formam uma adjacência intergênica, são consecutivos em $\iota$ e $\breve\pi_{i+1} + \breve\pi_{j+1} \ge \breve\iota_k$, onde $\breve\iota_k$ é o tamanho da região intergênica entre o par de elementos consecutivos em $\iota$. Aplique uma transposição como descrito no caso $iii$ do Lema~\ref{lemma:SIAFJFDO}.
  \item O par de elementos $(\pi_{i},\pi_{i+1})$ ou $(\pi_{j},\pi_{j+1})$ não formam uma adjacência intergênica, são consecutivos em $\iota$ e $\breve\pi_{i+1} + \breve\pi_{j+1} \ge \breve\iota_k$, onde $\breve\iota_k$ é o tamanho da região intergênica entre o par de elementos consecutivos em $\iota$. Aplique um move como descrito no caso $iv$ do Lema~\ref{lemma:NWNNZGXH}.
\end{enumerate}
Note que os casos $i$, $ii$, $iii$ e $iv$ removem pelo menos um breakpoint tipo um de $\mathcal{I}$ utilizando no máximo um eventos de reversões, transposição ou move. Logo, o lema segue. 
\end{proof}

A seguir apresentamos, para a variação sem sinais do problema \SbIRTM{}, o Algoritmo~\ref{algorithm:UZWADMNZ}.

\begin{algorithm}[!tbh]
  \caption{Um algoritmo de aproximação para o problema \SbIRTM{}.\label{algorithm:UZWADMNZ}}
  \Entrada{Uma instância intergênica rígida balanceada sem sinais $\mathcal{I}=((\pi,\breve\pi),(\iota,\breve\iota))$}
  \Saida{Uma sequência de reversões, transposições e moves $S$, tal que $(\pi,\breve\pi) \cdot S = (\iota,\breve\iota)$}
    Seja $S \gets ()$ \\
    \Enqto{$ib(\mathcal{I}) > 1$}{
      \tcp{Lema~\ref{lemma:WYEZMYTM}}
      $(\pi_i,\pi_{i+1})$, $(\pi_j,\pi_{j+1}) \gets $ encontre um par de breakpoints conectados \\
      \tcp{Lema~\ref{lemma:YLNUFQYG}}
      \Se{$(\pi_i,\pi_{i+1})$, $(\pi_j,\pi_{j+1})$ pertence ao caso $i$}{
        $S' \gets (\rho_1)$ \\
      }\SenaoSe{$(\pi_i,\pi_{i+1})$, $(\pi_j,\pi_{j+1})$ pertence ao caso $ii$}{
        $S' \gets (\tau_1)$ \\
      }\SenaoSe{$(\pi_i,\pi_{i+1})$, $(\pi_j,\pi_{j+1})$ pertence ao caso $iii$}{
        $S' \gets (\tau_1)$ \\
      }\SenaoSe{$(\pi_i,\pi_{i+1})$, $(\pi_j,\pi_{j+1})$ pertence ao caso $iv$}{
        $S' \gets (\mu_1)$ \\
      }
      $S \gets S + S'$ \\
      $\mathcal{I} \gets ((\pi, \breve\pi) \cdot S',(\iota,\breve\iota))$ \\
    }
  \Retorna{S}
\end{algorithm}

\begin{lemma}\label{lemma:UUWLBHHA}
Dada uma instância intergênica rígida balanceada sem sinais $\mathcal{I}=((\pi,\breve\pi),\break(\iota,\breve\iota))$, o Algoritmo~\ref{algorithm:UZWADMNZ} transforma $(\pi,\breve\pi)$ em $(\iota,\breve\iota)$ utilizando no máximo $b_1(\mathcal{I})$ eventos de reversão, transposição e move.
\end{lemma}
\begin{proof}
  No Algoritmo~\ref{algorithm:UZWADMNZ}, temos que enquanto $ib_1(\mathcal{I})$ for maior que um, ou seja, $(\pi,\breve\pi)$ for diferente de $(\iota,\breve\iota)$ (pela Observação~\ref{remark:UDYJTHAH} e Lema~\ref{lemma:WSPRPLAH}), o seguinte procedimento é aplicado: pelos lemas~\ref{lemma:WYEZMYTM} e~\ref{lemma:YLNUFQYG}, sempre podemos encontrar um par conectado de breakpoints e remover pelo menos um breakpoint tipo um após aplicar no máximo uma operação de reversão, transposição ou move. A cada iteração do algoritmo pelo menos um breakpoint tipo um é removido. Dessa forma, o genoma alvo eventualmente será alcançado. No pior caso, cada breakpoint tipo um é removido utilizando evento de rearranjo. Logo, $ib_1(\mathcal{I})$ operações de reversão, transposição ou move são utilizadas, no máximo, para transformar $(\pi,\breve\pi)$ em $(\iota,\breve\iota)$, e o lema segue.
\end{proof}

Note que, no pior caso, cada iteração do Algoritmo~\ref{algorithm:UZWADMNZ} pode levar um tempo linear para ser aplicada. Como pelo menos um breakpoint tipo um é removido por iteração e $ib_1(\mathcal{I}) \le {n+1}$, então o tempo de execução do Algoritmo~\ref{algorithm:UZWADMNZ} é $\mathcal{O}(n^2)$.  

\begin{theorem}\label{theorem:EANLWIUO}
Dada uma instância intergênica rígida balanceada sem sinais $\mathcal{I}=\break((\pi,\breve\pi),(\iota,\breve\iota))$, o Algoritmo~\ref{algorithm:JQHVZACM} é uma $3$-aproximação para o problema \SbIRTM{}.
\end{theorem}
\begin{proof}
Pelo Lema~\ref{lemma:UUWLBHHA}, o Algoritmo~\ref{algorithm:UZWADMNZ} transforma $(\pi,\breve\pi)$ em $(\iota,\breve\iota)$ utilizando no máximo $ib_1(\mathcal{I})$ eventos de reversão, transposição e move. Pelo Teorema~\ref{theorem:MPFPKHQO}, temos o seguinte limitante inferior $d_{\SbIRTM}(\mathcal{I}) \ge \frac{ib_1(\mathcal{I})}{3}$. Logo, o teorema segue. 
\end{proof}

% ------------------------------------------------------------------ %
\subsection{Reversão, Transposição, Move e Indel}
% ------------------------------------------------------------------ %

Nesta seção apresentaremos um algoritmo de aproximação com fator $3$ para a variação sem sinais do problema \SbIRTMI{}. A seguir apresentamos o Algoritmo~\ref{algorithm:FMDPGQTJ}.

\input{algorithms/FMDPGQTJ}

\begin{lemma}\label{lemma:GCEWGEBP}
Dada uma instância intergênica rígida sem sinais $\mathcal{I}=((\pi,\breve\pi),(\iota,\breve\iota))$, o Algoritmo~\ref{algorithm:FMDPGQTJ} transforma $(\pi,\breve\pi)$ em $(\iota,\breve\iota)$ utilizando no máximo $ib_1(\mathcal{I})$ eventos de reversão, transposição, move e indel.
\end{lemma}
\begin{proof}
  A prova é similar a descrita no Lema~\ref{lemma:XUDIVWPC}.
\end{proof}

Podemos notar que Algoritmo~\ref{algorithm:FMDPGQTJ}, em comparação com o Algoritmo~\ref{algorithm:JQHVZACM}, possui adicionalmente apenas subrotinas que podem ser feitas em tempo constante. Logo, o tempo de execução do Algoritmo~\ref{algorithm:FMDPGQTJ} é $\mathcal{O}(n^2)$.  

\begin{theorem}\label{theorem:EANLWIUO}
Dada uma instância intergênica rígida sem sinais $\mathcal{I}=((\pi,\breve\pi),(\iota,\breve\iota))$, o Algoritmo~\ref{algorithm:FMDPGQTJ} é uma $3$-aproximação para o problema \SbIRTMI{}.
\end{theorem}
\begin{proof}
Pelo Lema~\ref{lemma:GCEWGEBP}, o Algoritmo~\ref{algorithm:FMDPGQTJ} transforma $(\pi,\breve\pi)$ em $(\iota,\breve\iota)$ utilizando no máximo $ib_1(\mathcal{I})$ eventos de reversão, transposição, move e indel. Pelo Teorema~\ref{theorem:MPFPKHQO}, temos o seguinte limitante inferior $d_{\SbIRTMI}(\mathcal{I}) \ge \frac{ib_1(\mathcal{I})}{3}$. Logo, o teorema segue. 
\end{proof}

% ------------------------------------------------------------------ %
\section{Instâncias Intergênicas Rígidas com Sinais}
% ------------------------------------------------------------------ %

% ------------------------------------------------------------------ %
\section{Conclusões}
% ------------------------------------------------------------------ %
\chapter{Modelos Intergênicos Flexíveis}\label{chapter:GMJBMTWF}

Neste capítulo, investigaremos problemas que levam em conta tanto a ordem dos genes como o tamanho das regiões intergênicas, mas considerando um grau de flexibilidade em relação ao tamanho das regiões intergênicas no genoma alvo que é desejado. Neste contexto, nós consideramos os eventos de reversão intergênica, transposição intergênica, move intergênico e indel intergênico, e investigaremos as variações com e sem sinais dos seguintes problemas.

\begin{itemize}
  \item Ordenação de Permutações por Reversões Intergênicas com Regiões Intergênicas Flexíveis (\SbFIR)
  \item Ordenação de Permutações por Operações Intergênicas de Reversão e Indel com Regiões Intergênicas Flexíveis (\SbFIRI)
  \item Ordenação de Permutações por Operações Intergênicas de Reversão e Move com Regiões Intergênicas Flexíveis (\SbFIRM)
  \item Ordenação de Permutações por Operações Intergênicas de Reversão, Move e Indel com Regiões Intergênicas Flexíveis (\SbFIRMI)
  \item Ordenação de Permutações por Operações Intergênicas de Reversão e Transposição com Regiões Intergênicas Flexíveis (\SbFIRT)
  \item Ordenação de Permutações por Operações Intergênicas de Reversão, Transposição e Indel com Regiões Intergênicas Flexíveis (\SbFIRTI)
  \item Ordenação de Permutações por Operações Intergênicas de Reversão, Transposição e Move com Regiões Intergênicas Flexíveis (\SbFIRTM)
  \item Ordenação de Permutações por Operações Intergênicas de Reversão, Transposição, Move e Indel com Regiões Intergênicas Flexíveis (\SbFIRTMI)
\end{itemize}

Além disso, investigaremos as variações sem sinais dos seguintes problemas.

\begin{itemize}
  \item Ordenação de Permutações por Transposições Intergênicas com Regiões Intergênicas Flexíveis (\SbFIT)
  \item Ordenação de Permutações por Operações Intergênicas de Transposição e Move com Regiões Intergênicas Flexíveis (\SbFITM)
\end{itemize}

Note que nos dois problemas apresentados anteriormente tanto o evento de transposição intergênica como o evento de move intergênico não alteram a orientação dos genes. Por esse motivo, apenas a variação sem sinais será investigada.

Neste capítulo, iremos nos referenciar aos eventos de rearranjo de reversão intergênica, transposição intergênica, move intergênico e indel intergênico simplesmente por reversão, transposição, move e indel, respectivamente. Além disso, iremos nos referir a um breakpoint intergênico simplesmente como um breakpoint. Dada uma sequência de eventos de rearranjo $S$, denotamos por $|S|$ o tamanho da sequência $S$, ou seja, a quantidade de eventos em $S$.

Dada uma instância intergênica flexível com ou sem sinais $\mathcal{I} = ((\pi,\breve\pi),(\iota,\breve\iota^{\min},\breve\iota^{\max}))$, a \emph{distância flexível} entre $(\pi,\breve\pi)$ e $(\iota,\breve\iota^{\min},\breve\iota^{\max})$, denotada por $df_{\mathcal{M}}(\mathcal{I})$, é o tamanho da menor sequência de eventos de rearranjo $S$, tal que todo evento de $S$ pertence ao modelo $\mathcal{M}$ e $(\pi,\breve\pi) \cdot S = (\iota,\breve\pi^{\prime})$, onde $\breve\iota^{\min}_i \le \breve\pi^{\prime}_i \le \breve\iota^{\max}_i$ para $i \in [1..n+1]$. Os modelos de rearranjo considerados neste capítulo são identificados por siglas apresentadas na Tabela~\ref{table:CGOLSOYF}.

\begin{table}[!htb]
  \caption{Siglas dos modelos de rearranjo considerados para instâncias intergênicas flexíveis.}
  \label{table:CGOLSOYF}
  \centering
  \begin{tabular}{|p{2.5cm}|p{3.5cm}|p{8cm}|}
    \hline
    \textbf{Problema}     & \textbf{Sigla do Modelo} & \textbf{Conjunto de Eventos de Rearranjo}          \\ \hline
    \SbFIR                & \R                       & $\{\rho\}                              $           \\ \hline
    \SbFIRI               & \RI                      & $\{\rho,\delta\}                       $           \\ \hline
    \SbFIRM               & \RM                      & $\{\rho,\mu\}                          $           \\ \hline
    \SbFIRMI              & \RMI                     & $\{\rho,\mu,\delta\}                   $           \\ \hline
    \SbFIT                & \T                       & $\{\tau\}                              $           \\ \hline
    \SbFITM               & \TM                      & $\{\tau,\mu\}                          $           \\ \hline
    \SbFIRT               & \RT                      & $\{\rho,\tau\}                         $           \\ \hline
    \SbFIRTI              & \RTI                     & $\{\rho,\tau,\delta\}                  $           \\ \hline
    \SbFIRTM              & \RTM                     & $\{\rho,\tau,\mu\}                     $           \\ \hline
    \SbFIRTMI             & \RTMI                    & $\{\rho,\tau,\mu,\delta\}              $           \\ \hline
  \end{tabular}
\end{table}

Dada uma instância intergênica flexível com ou sem sinais $\mathcal{I} = ((\pi,\breve\pi),(\iota,\breve\iota^{\min},\breve\iota^{\max}))$, nós utilizaremos a expressão \emph{atingir o genoma alvo} quanto $\pi = \iota$ e $\forall i \in \{1,2,\dots,({n+1})\}: \breve\iota^{\min}_i \le \breve\pi_i \le \breve\iota^{\max}_i$.

Parte dos resultados que serão apresentados neste capítulo foram aceitos para publicação na revista \emph{IEEE/ACM Transactions on Computational Biology and Bioinformatics}~\cite{2022a-brito-etal} em 2022.

% ------------------------------------------------------------------ %
\section{Análise de Complexidade}
% ------------------------------------------------------------------ %

Nesta seção, realizaremos uma análise de complexidade dos problemas que consideram um grau de flexibilidade em relação ao tamanho das regiões intergênicas no genoma alvo que é desejado. Note que todos os problemas investigados neste capítulo generalizam suas respectivas versões considerando um tamanho estrito (rígido) para o tamanho das regiões intergênicas desejadas no genoma alvo.

Isso pode ser facilmente constatado por meio de uma redução. Sejam $\mathcal{P}_f$ e $\mathcal{P}_r$ problemas com base no mesmo modelo de rearranjo $\mathcal{M}$, mas $\mathcal{P}_f$ e $\mathcal{P}_r$ possuem, respectivamente, uma característica de flexibilidade e rígidez em relação ao tamanho das regiões intergênicas no genoma alvo. Seja $\mathcal{I}=((\pi,\breve\pi),(\iota,\breve\iota))$ uma instância intergênica rígida para o problema $\mathcal{P}_r$, então podemos criar uma instância intergênica flexível $\mathcal{I'} = ((\pi',\breve\pi'),(\iota',\breve\iota'^{\min},\breve\iota'^{\max}))$ para o problema $\mathcal{P}_f$ da seguinte forma:

\begin{itemize}
  \item $(\pi',\breve\pi') = (\pi,\breve\pi)$
  \item $\iota' = \iota$
  \item $\breve\iota'^{\min} = \breve\iota'^{\max} = \breve\iota$
\end{itemize}

Note que pela construção da instância $\mathcal{I'}$ e pelo fato de que $\mathcal{P}_f$ e $\mathcal{P}_r$ adotam o mesmo modelo de rearranjo $\mathcal{M}$, temos que $df_{\mathcal{M}}(\mathcal{I'}) = d_{\mathcal{M}}(\mathcal{I})$.

No Capítulo~\ref{chapter:DOVAEMLI} foi mostrado que a variação sem sinais dos problemas \SbIR{}, \SbIRI{}, \SbIRM{}, \SbIRMI{}, \SbIRT{}, \SbIRTI{}, \SbIRTM{} e \SbIRTMI{} pertencem à classe NP-difícil. Adicionalmente, temos que a variação sem sinais dos problemas \SbIT{} e \SbITM{} também pertencem à classe NP-difícil~\cite{2021a-oliveira-etal}. Dessa forma, temos o seguinte lema.

\begin{lemma}\label{lemma:BEBGUYUB}
Os problemas \SbFIR{}, \SbFIRI{}, \SbFIRM{}, \SbFIRMI{}, \SbFIRT{}, \SbFIRTI{}, \SbFIRTM{}, \SbFIRTMI{}, \SbFIT{} e \SbFITM{} em instâncias intergênicas flexíveis sem sinais pertencem à classe NP-difícil.
\end{lemma}

Além disso, o Capítulo~\ref{chapter:DOVAEMLI} apresenta a informação de que a variação com sinais dos problemas \SbIR{}, \SbIRM{}, \SbIRMI{}, \SbIRT{}, \SbIRTI{}, \SbIRTM{} e \SbIRTMI{} também pertencem à classe NP-difícil. Com isso, obtemos os seguinte lema.

\begin{lemma}\label{lemma:XPRZJZES}
Os problemas \SbFIR{},\SbFIRM{}, \SbFIRMI{}, \SbFIRT{}, \SbFIRTI{}, \SbFIRTM{} e \SbFIRTMI{} em instâncias intergênicas flexíveis com sinais pertencem à classe NP-difícil.
\end{lemma}

% ------------------------------------------------------------------ %
\section{Limitantes Inferiores}
% ------------------------------------------------------------------ %

Nesta seção, apresentaremos limitantes inferiores para as variações com e sem sinais dos problemas investigados neste capítulo.

Para a variação sem sinais dos problemas \SbFIR{}, \SbFIRI{}, \SbFIRM{}, \SbFIRMI{}, \SbFIRT{}, \SbFIRTI{}, \SbFIRTM{} e  \SbFIRTMI{} utilizaremos o conceito de região intergênica. Note que os eventos de rearranjo de reversão, transposição, move e indel afetam, respectivamente, a seguinte quantidade de regiões intergênicas: duas, três, duas e uma. No melhor cenário, cada uma das regiões intergênicas afetadas pode ser instável ou auxiliar, e são removidas após o evento de rearranjo ser aplicado. Com isso, obtemos os seguintes lemas.

\begin{lemma}\label{lemma:VJKGLBQG}
Dada uma instância intergênica flexível sem sinais $\mathcal{I}$, para qualquer reversão $\rho$ temos que $\Delta ir_i(\mathcal{I}, S = (\rho)) \ge -2$.
\end{lemma}

\begin{lemma}\label{lemma:XLUTQDGV}
Dada uma instância intergênica flexível sem sinais $\mathcal{I}$, para qualquer tranposição $\tau$ temos que $\Delta ir_i(\mathcal{I}, S = (\tau)) \ge -3$.
\end{lemma}

\begin{lemma}\label{lemma:ZOCGWWGV}
Dada uma instância intergênica flexível sem sinais $\mathcal{I}$, para qualquer move $\mu$ temos que $\Delta ir_i(\mathcal{I}, S = (\mu)) \ge -2$.
\end{lemma}

\begin{lemma}\label{lemma:HQJMMZCU}
Dada uma instância intergênica flexível sem sinais $\mathcal{I}$, para qualquer indel $\delta$ temos que $\Delta ir_i(\mathcal{I}, S = (\delta)) \ge -1$.
\end{lemma}

Além disso, considerando uma instância intergênica flexível balanceada sem sinais e com base em um modelo composto exclusivamente por eventos conservativos, temos os seguintes lemas.

\begin{lemma}\label{lemma:IERALSKC}
Dada uma instância intergênica flexível balanceada sem sinais $\mathcal{I}$, para qualquer reversão $\rho$ temos que $\Delta ir_i(\mathcal{I}, S = (\rho)) + \Delta ir_a(\mathcal{I}, S = (\rho)) \ge -2$.
\end{lemma}

\begin{lemma}\label{lemma:FOXQSODF}
Dada uma instância intergênica flexível balanceada sem sinais $\mathcal{I}$, para qualquer tranposição $\tau$ temos que $\Delta ir_i(\mathcal{I}, S = (\tau)) + \Delta ir_a(\mathcal{I}, S = (\tau)) \ge -3$.
\end{lemma}

\begin{lemma}\label{lemma:AXMNYRLB}
Dada uma instância intergênica flexível balanceada sem sinais $\mathcal{I}$, para qualquer move $\mu$ temos que $\Delta ir_i(\mathcal{I}, S = (\mu)) + \Delta ir_a(\mathcal{I}, S = (\mu)) \ge -2$.
\end{lemma}

Com base na quantidade máxima de regiões intergênicas instáveis e auxiliares que cada evento pode remover de uma instância intergênica flexível, obtemos os seguintes limitantes inferiores.

\begin{theorem}\label{theorem:BOTBXFZQ}
Dada uma instância intergênica flexível sem sinais $\mathcal{I}$, temos que:

\begin{tabular}{lll}
  $df_{\SbFIRI}(\mathcal{I})$     & $ \ge $ & $\frac{ir_i(\mathcal{I})}{2}$, \\
  $df_{\SbFIRMI}(\mathcal{I})$    & $ \ge $ & $\frac{ir_i(\mathcal{I})}{2}$, \\
  $df_{\SbFIRTI}(\mathcal{I})$    & $ \ge $ & $\frac{ir_i(\mathcal{I})}{3}$, \\
  e $df_{\SbFIRTMI}(\mathcal{I})$ & $ \ge $ & $\frac{ir_i(\mathcal{I})}{3}$. \\
\end{tabular}
\end{theorem}
\begin{proof}
Pela Observação~\ref{remark:EUSNDMWS}, temos que todas as regiões intergênicas instáveis devem ser removidas para que o genoma alvo seja alcançado. Pelos lemas~\ref{lemma:VJKGLBQG}, \ref{lemma:XLUTQDGV}, \ref{lemma:ZOCGWWGV} e \ref{lemma:HQJMMZCU}, temos que os eventos de reversão, transposição, move e indel podem remover, no máximo, $2$, $3$, $2$ e $1$ região intergênica instável, respectivamente. Como a instância $\mathcal{I}$ possui $ir_i(\mathcal{I})$ regiões intergênicas instáveis e considerando o máximo de regiões intergênicas instávéis que podem ser removidas por cada evento nos modelos de rearranjo \SbFIRI{}, \SbFIRMI{}, \SbFIRTI{} e \SbFIRTMI{}, o teorema segue.
\end{proof}

\begin{theorem}\label{theorem:KKKUCDHN}
Dada uma instância intergênica flexível balanceada sem sinais $\mathcal{I}$, temos que:

\begin{tabular}{lll}
  $df_{\SbFIR}(\mathcal{I})$      & $ \ge $ & $\frac{ir_i(\mathcal{I}) + ir_a(\mathcal{I})}{2}$, \\ 
  $df_{\SbFIRM}(\mathcal{I})$     & $ \ge $ & $\frac{ir_i(\mathcal{I}) + ir_a(\mathcal{I})}{2}$, \\
  $df_{\SbFIRT}(\mathcal{I})$     & $ \ge $ & $\frac{ir_i(\mathcal{I}) + ir_a(\mathcal{I})}{3}$, \\
  e $df_{\SbFIRTM}(\mathcal{I})$  & $ \ge $ & $\frac{ir_i(\mathcal{I}) + ir_a(\mathcal{I})}{3}$. \\
\end{tabular}
\end{theorem}
\begin{proof}
Note que em todos os problemas possuem um modelo de rearranjo é composto exclusivamente por eventos conservativos. Pela Observação~\ref{remark:PGEYZJME}, temos que todas as regiões intergênicas instáveis e auxiliares devem ser removidas para que o genoma alvo seja alcançado. Pelos lemas~\ref{lemma:IERALSKC}, \ref{lemma:FOXQSODF} e \ref{lemma:AXMNYRLB}, temos que variação no número de regiões intergênicas instáveis mais a variação no número de regiões intergênicas auxiliares após aplicar um evento de reversão, transposição e move é maior ou igual que $-2$, $-3$ e $-2$, respectivamente. Como a instância $\mathcal{I}$ possui $ir_i(\mathcal{I}) + ir_a(\mathcal{I})$ regiões intergênicas instáveis e auxiliares, considerando o máximo de regiões intergênicas instávéis e auxiliares que podem ser removidas por cada evento nos modelos de rearranjo \SbFIR{}, \SbFIRM{}, \SbFIRT{} e \SbFIRTM{}, o teorema segue.
\end{proof}

A seguir, com base na estrutura de grafo de ciclos ponderado flexível, apresentamos limitantes inferiores para a variação sem sinais dos problemas \SbFIT{} e \SbFITM{}, e para a variação com sinais dos problemas \SbFIR{}, \SbFIRI{}, \SbFIRM{}, \SbFIRMI{}, \SbFIRT{} e \SbFIRTM{}.

Note que os evento de reversão e tranposição afetam, respectivamente, duas e três arestas pretas do grafo de ciclos ponderado flexível e podem aumentar tanto o número de ciclos como também o número de ciclos estáveis e definitivos. O evento de move afeta duas arestas pretas do grafo, mas pode aumentar somente o número de ciclos estáveis e definitivos. Já o evento de indel afeta apenas uma aresta preta do grafo e pode aumentar apenas o número de ciclos estáveis.

Dessa forma, dada uma instância intergênica flexível $\mathcal{I}$, temos que $\Delta c(G(\mathcal{I}), S=(\rho)) \in \{1,0,-1\}$, $\Delta c_e(G(\mathcal{I}), S=(\rho)) \in \{1,0,-1\}$ e $\Delta c_d(G(\mathcal{I}), S=(\rho)) \in \{1,0,-1\}$ para qualquer reversão $\rho$. Para qualquer transposição $\tau$, temos que $\Delta c(G(\mathcal{I}), S=(\tau)) \in \{2,0,-2\}$, $\Delta c_e(G(\mathcal{I}), S=(\tau)) \in \{2,1,0,-1,-2\}$ e $\Delta c_d(G(\mathcal{I}), S=(\tau)) \in \{2,1,0,-1,-2\}$. Para qualquer move $\mu$, temos que $\Delta c(G(\mathcal{I}), S=(\mu)) = 0$, $\Delta c_e(G(\mathcal{I}), S=(\mu)) \in \{2,1,0,-1,\break-2\}$ e $\Delta c_d(G(\mathcal{I}), S=(\mu)) \in \{2,1,0,-1,-2\}$. Por fim, para qualquer indel $\delta$, temos que $\Delta c(G(\mathcal{I}), S=(\delta)) = 0$ e $\Delta c_e(G(\mathcal{I}), S=(\delta)) \in \{1,0,{-1}\}$. Com isso, obtemos os seguintes limitantes inferiores.

\begin{theorem}\label{theorem:PQQUYBMS}
Dada uma instância intergênica flexível balanceada sem sinais $\mathcal{I}$, temos que:

\begin{tabular}{lll}
  $df_{\SbFIT}(\mathcal{I})$      & $ \ge $ & $\frac{{n+1} - c_d(G(\mathcal{I} ))}{2}$, \\
  e $df_{\SbFITM}(\mathcal{I})$   & $ \ge $ & $\frac{{n+1} - c_d(G(\mathcal{I} ))}{2}$. \\
\end{tabular}
\end{theorem}
\begin{proof}
Pela Observação~\ref{remark:HLVDQLCE}, temos que para atingir o genoma alvo é necessário que $c(G(\mathcal{I})) = c_d(G(\mathcal{I})) = n+1$. Como $c_d(G(\mathcal{I})) \le c(G(\mathcal{I}))$, então se fizermos que $G(\mathcal{I})$ possua $n+1$ ciclos definitivos temos garantidamente que $c(G(\mathcal{I})) = c_d(G(\mathcal{I})) = n+1$. Tanto o evento de transposição como o evento de move podem aumentar o número de ciclos definitivos, no máximo, em duas unidades. Logo, são necessárias pelo menos $\frac{{n+1} - c_d(G(\mathcal{I} ))}{2}$ operações de transposição ou move para atingir o genoma alvo, e o teorema segue. 
\end{proof}

\begin{theorem}\label{theorem:EUNBEQEX}
Dada uma instância intergênica flexível balanceada com sinais $\mathcal{I}$, temos que: $df_{\SbFIR}(\mathcal{I}) \ge {n+1} - c_d(G(\mathcal{I} ))$.
\end{theorem}
\begin{proof}
A prova é similar a apresentada no Teorema~\ref{theorem:PQQUYBMS} considerando que o evento de reversão pode aumentar o número de ciclos definitivos, no máximo, em uma unidade.
\end{proof}

\begin{theorem}\label{theorem:SZNBDWOM}
Dada uma instância intergênica flexível com sinais $\mathcal{I}$, temos que:

\begin{tabular}{lll}
  $df_{\SbFIRI}(\mathcal{I})$       & $ \ge $ & ${n+1} - c_e(G(\mathcal{I} ))$, \\
  $df_{\SbFIRTI}(\mathcal{I})$      & $ \ge $ & $\frac{{n+1} - c_e(G(\mathcal{I} ))}{2}$, \\
  e $df_{\SbFIRTMI}(\mathcal{I})$   & $ \ge $ & $\frac{{n+1} - c_e(G(\mathcal{I} ))}{2}$. \\
\end{tabular}
\end{theorem}
\begin{proof}
Pela Observação~\ref{remark:IRNWKUZA}, temos que para atingir o genoma alvo é necessário que $c(G(\mathcal{I})) = c_e(G(\mathcal{I})) = n+1$. Como $c_e(G(\mathcal{I})) \le c(G(\mathcal{I}))$, então se fizermos que $G(\mathcal{I})$ possua $n+1$ ciclos estáveis temos garantidamente que $c(G(\mathcal{I})) = c_e(G(\mathcal{I})) = n+1$. Tanto o evento de reversão como o evento de indel podem aumentar o número de ciclos estáveis, no máximo, em uma unidade. Os eventos de transposição e move podem aumentar o número de ciclos estáveis, no máximo, em duas unidade. Logo, são necessárias pelo menos ${n+1} - c_e(G(\mathcal{I} ))$ operações de reversão ou indel para atingir o genoma alvo no problema \SbFIRI{}. Para os problemas \SbFIRTI{} e \SbFIRTMI{} são necessárias pelo menos $\frac{{n+1} - c_e(G(\mathcal{I} ))}{2}$ operações para atingir o genoma alvo, e o teorema segue. 
\end{proof}

\begin{theorem}\label{theorem:CNMFNKPK}
Dada uma instância intergênica flexível balanceada com sinais $\mathcal{I}$, temos que: $df_{\SbFIRM}(\mathcal{I}) \ge {n+1} - \frac{c(G(\mathcal{I} )) + c_d(G(\mathcal{I} ))}{2}$.
\end{theorem}
\begin{proof}
Pela Observação~\ref{remark:HLVDQLCE}, temos que para atingir o genoma alvo é necessário que $c(G(\mathcal{I})) = c_d(G(\mathcal{I})) = n+1$. Logo, temos que aumentar a quantidade de ciclos e ciclos definitivos em ${n+1} - c(G(\mathcal{I}))$ e ${n+1} - c_d(G(\mathcal{I}))$ unidades, respectivamente. Totalizando a quantidade de ciclos e ciclos definitivos que precisam ser criados temos o seguinte valor: $2(n+1) - (c(G(\mathcal{I})) + c_b(G(\mathcal{I})))$. Considerando os eventos de reversão e move, temos que para qualquer evento $\gamma \in \{\rho, \mu\}$ é verdade que $\Delta c(G(\mathcal{I}), S=(\gamma)) + \Delta c_d(G(\mathcal{I}), S=(\gamma)) \le 2$. Dessa forma, são necessárias pelo menos $\frac{2({n+1}) - (c(G(\mathcal{I})) + c_d(G(\mathcal{I})))}{2} = {n+1} - \frac{c(G(\mathcal{I} )) + c_d(G(\mathcal{I} ))}{2}$ operações de reversão ou move para atingir o genoma alvo, e o teorema segue. 
\end{proof}

\begin{theorem}\label{theorem:XQPRYMFX}
Dada uma instância intergênica flexível com sinais $\mathcal{I}$, temos que:\break $df_{\SbFIRMI}(\mathcal{I}) \ge {n+1} - \frac{c(G(\mathcal{I} )) + c_e(G(\mathcal{I} ))}{2}$.
\end{theorem}
\begin{proof}
A prova é similar a apresentada no Teorema~\ref{theorem:CNMFNKPK} considerando que para qualquer evento $\gamma \in \{\rho, \mu\,\delta\}$ é verdade que $\Delta c(G(\mathcal{I}), S=(\gamma)) + \Delta c_e(G(\mathcal{I}), S=(\gamma)) \le 2$.
\end{proof}

\begin{theorem}\label{theorem:HELIIGVZ}
Dada uma instância intergênica flexível balanceada com sinais $\mathcal{I}$, temos que:

\begin{tabular}{lll}
  $df_{\SbFIRT}(\mathcal{I})$     & $ \ge $ & $\frac{{n+1} - c_d(G(\mathcal{I} ))}{2}$, \\
  e $df_{\SbFIRTM}(\mathcal{I})$  & $ \ge $ & $\frac{{n+1} - c_d(G(\mathcal{I} ))}{2}$. \\
\end{tabular}
\end{theorem}
\begin{proof}
A prova é similar a apresentada no Teorema~\ref{theorem:PQQUYBMS} incluindo a consideração de que o evento de reversão pode aumentar o número de ciclos definitivos, no máximo, em uma unidade.
\end{proof}

% ------------------------------------------------------------------ %
\section{Algoritmos de Aproximação}
% ------------------------------------------------------------------ %

Nesta seção, apresentaremos algoritmos de aproximação para as variações com e sem sinais dos problemas investigados neste capítulo. Inicialmente apresentaremos algumas funções de redução que criam uma instância intergênica rígida a partir de uma instância intergênica flexível.

Dada uma instância intergênica flexível sem sinais $\mathcal{I} = ((\pi,\breve\pi),(\iota,\breve\iota^{\min},\breve\iota^{\max}))$ a função $\mathcal{F}_{ir}^{'}$ cria uma instância intergênica rígida sem sinais $\mathcal{I'} = ((\pi',\breve\pi'),(\iota',\breve\iota'))$ da seguinte forma:

\begin{itemize}
  \item $(\pi',\breve\pi') = (\pi,\breve\pi)$
  \item $\iota' = \iota$
  \item Inicialmente, atribua em $\breve\iota_{i}'$ o valor $\breve\iota^{\min}_i$, para $i \in [1..({n+1})]$. Em seguida, para cada região intergênica estável $\breve\pi_i \in \mathcal{S}_{e}(\mathcal{I})$ atribua o valor $\breve\pi_i$ em $\breve\iota_{k}'$, onde $k = \max(\pi_{i-1},\pi_i)$.
\end{itemize}

Denotamos por $\mathcal{F}_{ir}^{'}(\mathcal{I})$ a instância intergênica rígida sem sinais criada pela função $\mathcal{F}_{ir}^{'}$ a partir de uma instância intergênica flexível sem sinais $\mathcal{I}$. Perceba que a função $\mathcal{F}_{ir}^{'}$ apenas define valores estritos para os tamanhos das regiões intergênicas no genoma alvo $\breve\iota'$ da instância intergênica rígida $\mathcal{I'}$, uma vez que $(\pi',\breve\pi') = (\pi,\breve\pi)$ e $\iota' = \iota$. Além disso, a função $\mathcal{F}_{ir}^{'}$ pode ser executada em tempo linear. O Exemplo~\ref{example:DXEJSITZ} mostra uma instância intergênica rígida sem sinais $\mathcal{I}' = ((0~1~2~5~4~3~6),(5,0,3,1,6,2)),((0~1~2~3~4~5~6),(5,3,3,6,2,1))$ criada pela função $\mathcal{F}_{ir}^{'}$ a partir de uma instância intergênica flexível sem sinais $\mathcal{I}=((0~1~2~5~4~3~6),(5,0,3,1,6,2)),((0~1~2~3~4~5~6),(4,3,3,2,2,1),(6,4,8,7,3,3)))$. Perceba que no exemplo apenas as regiões intergênicas $\breve\pi_1$ e $\breve\pi_5$ são estáveis. Por esse motivo, temos que $\breve\iota'_1 = 5$ e $\breve\iota'_4 = 6$.

\input{examples/DXEJSITZ}

\begin{lemma}\label{lemma:UFTVNRSX}
Seja $\mathcal{I'} = ((\pi',\breve\pi'),(\iota',\breve\iota'))$ uma instância intergênica rígida sem sinais tal que $\mathcal{I'} = \mathcal{F}_{ir}^{'}(\mathcal{I})$, onde $\mathcal{I} = ((\pi,\breve\pi),(\iota,\breve\iota^{\min},\breve\iota^{\max}))$ é uma instância intergênica flexível sem sinais, então temos que $ir_i(\mathcal{I}) = ib_1(\mathcal{I'})$.
\end{lemma}
\begin{proof}
Pela construção da função $\mathcal{F}_{ir}^{'}$ temos que cada região intergênica estável em $\mathcal{I}$ é mapeada em uma adjacência intergênica em $\mathcal{I'}$. Além disso, cada região intergênica instável acaba tornando-se um breakpoint tipo um em $\mathcal{I'}$, seja porque os elementos antes e depois da região intergênica não são adjacentes no genoma alvo ou porque o tamanho da região intergênica é menor do que o mínimo ou maior do que o máximo permitido no genoma alvo. Logo, $ir_i(\mathcal{I}) = ib_1(\mathcal{I'})$ e o lema segue.
\end{proof}

\begin{lemma}\label{lemma:SVKOAOXA}
Seja $\mathcal{I'} = ((\pi',\breve\pi'),(\iota',\breve\iota'))$ uma instância intergênica rígida sem sinais tal que $\mathcal{I'} = \mathcal{F}_{ir}^{'}(\mathcal{I})$, onde $\mathcal{I} = ((\pi,\breve\pi),(\iota,\breve\iota^{\min},\breve\iota^{\max}))$ é uma instância intergênica flexível sem sinais, e seja $S'$ uma sequência de eventos de rearranjo tal que $(\pi',\breve\pi') \cdot S' = (\iota',\breve\iota')$, então $S'$ é uma sequência que faz com que o genoma alvo em $\mathcal{I}$ seja atingido.
\end{lemma}
\begin{proof}
Diretamente pela construção da função $\mathcal{F}_{ir}^{'}$. Note que $(\pi',\breve\pi') = (\pi,\breve\pi)$, $\iota' = \iota$ e $\forall i \in \{1,2,\dots,({n+1})\}: \breve\iota^{\min}_i \le \breve\iota'_i \le \breve\iota^{\max}_i$.
\end{proof}

Agora considere a seguinte função de redução com base em um modelo de rearranjo composto exclusivamento por eventos de rearranjo conservativos. Dada uma instância intergênica flexível balanceada sem sinais $\mathcal{I} = ((\pi,\breve\pi),(\iota,\breve\iota^{\min},\breve\iota^{\max}))$ a função $\mathcal{F}_{ir}^{''}$ cria uma instância intergênica rígida balanceada sem sinais $\mathcal{I'} = ((\pi',\breve\pi'),(\iota',\breve\iota'))$ da seguinte forma:

\begin{itemize}
  \item $(\pi',\breve\pi') = (\pi,\breve\pi)$
  \item $\iota' = \iota$
  \item Os valores de $\breve\iota'$ são atribuídos de acordo com o cenário da instância $\mathcal{I}$:
  \begin{itemize}
    \item No cenário fonte, inicialmente atribua em $\breve\iota_{i}'$ o valor $\breve\iota^{\min}_i$, para $i \in [1..({n+1})]$. Em seguida, para cada região intergênica definitiva $\breve\pi_i \in \mathcal{S}_{d}(\mathcal{I})$ atribua o valor $\breve\pi_i$ em $\breve\iota_{k}'$, onde $k = \max(\pi_{i-1},\pi_i)$. Seja $\alpha$ o seguinte valor:
    $$\alpha = \sum_{\breve\pi_i \in \mathcal{S}_{a_{1}}(\mathcal{I})} gap_{\min}(\breve\pi_i) - \sum_{\breve\iota_{i}^{\min}  \in \breve\iota^{\min}} \breve\iota_{i}^{\min} - \sum_{\breve\pi_i \in \mathcal{S}_{e_{1}}(\mathcal{I})} (\breve\pi_i - gap_{\min}(\breve\pi_i)) - \sum_{\breve\pi_i \in \mathcal{S}_{i_{1}}(\mathcal{I})} \breve\pi_i$$
    Note que o valor de $\alpha$ neste cenário representa o total de nucleotídeos que podem ser transferidos das regiões intergênicas auxiliares sem torná-las instáveis menos o deficit total de nucleotídeos nas regiões intergênicas instáveis. Considerando que as regiões intergênicas auxiliares estão ordenadas de maneira decrescente pelo valor de $gap_{\min}$ e seja $\breve\pi_i$ a última região intergênica auxiliar, atribua em $\breve\iota_{k}'$, com $k = \max(\pi_{i-1},\pi_i)$, o valor $\breve\iota^{\min}_k + \alpha$. Note que $\breve\iota^{\min}_k \le \breve\iota_{k}' \le \breve\iota^{\max}_k$, pois por definição $\breve\pi_i$ também é uma região intergênica estável, foi a última região intergênica a ser adicionada no conjunto de regiões intergênicas auxiliares e obrigatoriamente temos que $gap_{\min}(\breve\pi_i) \ge \alpha$.

    \item No cenário sorvedouro, inicialmente atribua em $\breve\iota_{i}'$ o valor $\breve\iota^{\max}_i$, para $i \in [1..({n+1})]$. Em seguida, para cada região intergênica definitiva $\breve\pi_i \in \mathcal{S}_{d}(\mathcal{I})$ atribua o valor $\breve\pi_i$ em $\breve\iota_{k}'$, onde $k = \max(\pi_{i-1},\pi_i)$. Seja $\alpha$ o seguinte valor:
  $$\alpha = \sum_{\breve\pi_i \in \mathcal{S}_{a}(\mathcal{I})} gap_{\max}(\breve\pi_i) - \sum_{\breve\pi_i \in \mathcal{S}_{i}(\mathcal{I})} \breve\pi_i - \sum_{\breve\iota_{i}^{\max}  \in \breve\iota^{\max}} \breve\iota_{i}^{\max} - \sum_{\breve\pi_i \in \mathcal{S}_{e}(\mathcal{I})} (\breve\pi_i + gap_{\max}(\breve\pi_i))$$
    Neste cenário o valor de $\alpha$ representa o total de nucleotídeos que podem ser transferidos para as regiões intergênicas auxiliares sem torná-las instáveis menos a quantidade excedente total de nucleotídeos nas regiões intergênicas instáveis. Considerando que as regiões intergênicas auxiliares estão ordenadas de maneira decrescente pelo valor de $gap_{\max}$ e seja $\breve\pi_i$ a última região intergênica auxiliar, atribua em $\breve\iota_{k}'$, com $k = \max(\pi_{i-1},\pi_i)$, o valor $\breve\iota^{\max}_k - \alpha$. Note que $\breve\iota^{\min}_k \le \breve\iota_{k}' \le \breve\iota^{\max}_k$, pois por definição $\breve\pi_i$ também é uma região intergênica estável, foi a última região intergênica a ser adicionada no conjunto de regiões intergênicas auxiliares e obrigatoriamente temos que $gap_{\max}(\breve\pi_i) \ge \alpha$.

    \item No cenário de equilíbrio, temos que o conjunto de regiões intergênicas auxiliares é vazio e o total de nucleotídeos nas regiões intergênicas instáveis é suficiente para torná-las estáveis. Nesse caso, para cada região intergênica definitiva $\breve\pi_i \in \mathcal{S}_{d}(\mathcal{I})$ atribua o valor $\breve\pi_i$ em $\breve\iota_{k}'$, onde $k = \max(\pi_{i-1},\pi_i)$. Para as regiões intergênicas instáveis, sempre é possível encontrar uma lista de números inteiros não negativos que atenda ao tamanho mínimo e máximo permitido em cada região intergênica do genoma alvo e considerando o total de nucleotídeos disponíveis nas regiões intergênicas instáveis.
  \end{itemize}
\end{itemize}

Por construção da função $\mathcal{F}_{ir}^{''}$, temos que cada região intergênica definitiva de $\mathcal{I}$ é mapeada em uma adjacência intergênica em $\mathcal{I}'$ e cada região intergênica instável e auxiliar em $\mathcal{I}$ é mapeada em um breakpoint tipo um em $\mathcal{I}'$. Além disso, pela forma como os tamanho das regiões intergênicas são atribuídos em $\breve\iota'$ temos a garantia de que a instância intergênica rígida $\mathcal{I}'$ resultante é balanceada. Denotamos por $\mathcal{F}_{ir}^{''}(\mathcal{I})$ a instância intergênica rígida balanceada sem sinais criada pela função $\mathcal{F}_{ir}^{''}$ a partir de uma instância intergênica flexível balanceada sem sinais $\mathcal{I}$. A função $\mathcal{F}_{ir}^{''}$ pode ser executada em tempo $\mathcal{O}(n\log n)$, uma vez que, no pior caso, é necessário ordenar as regiões intergênicas estáveis para definir o conjunto de regiões intergênicas auxiliares. 

O Exemplo~\ref{example:AVAOAJHG} mostra uma instância intergênica rígida balanceada sem sinais $\mathcal{I}' = ((0~1~2~5~4~3~6),(5,0,3,1,6,2)),((0~1~2~3~4~5~6),(5,3,3,3,2,1))$ criada pela função $\mathcal{F}_{ir}^{''}$ a partir de uma instância intergênica flexível balanceada sem sinais $\mathcal{I} = (((0~1~2~5~4~3~6),(5,0,3,\break1,6,2)),((0~1~2~3~4~5~6),(4,3,3,2,2,1),(6,4,8,7,3,3)))$. Note que a instância $\mathcal{I}$ pertence ao cenário fonte com quatro regiões intergênicas instáveis ($ir_i(\mathcal{I}) = 4$ e $\mathcal{S}_{i}=\{\breve\pi_2,\breve\pi_3,\breve\pi_4,\breve\pi_6\}$) e duas regiões intergênicas estáveis ($\mathcal{S}_{e}=\{\breve\pi_1,\breve\pi_5\}$). No exemplo, temos apenas uma região intergênica auxiliar ($ir_a(\mathcal{I}) = 1$ e $\mathcal{S}_{a}=\{\breve\pi_5\}$). Note que $gap_{\min}(\breve\pi_1) = 1$ e $gap_{\min}(\breve\pi_5) = 4$. Logo, $ir_d(\mathcal{I}) = 1$ e $\mathcal{S}_{d}=\{\breve\pi_1\}$. Além disso, perceba que $\alpha = 1$. Como $\breve\pi_5$ é a única (e a última) região intergênica auxiliar, temos que $\breve\iota'_4 = \breve\iota^{\min}_4 +\alpha = 2 + 1 = 3$.

\input{examples/AVAOAJHG}

O Exemplo~\ref{example:OQWFSAKO} mostra uma instância intergênica rígida balanceada sem sinais $\mathcal{I}' = ((0~1~2~5~4~3~6),(5,0,3,1,6,2)),((0~1~2~3~4~5~6),(6,4,3,7,1,3))$ criada pela função $\mathcal{F}_{ir}^{''}$ a partir de uma instância intergênica flexível balanceada sem sinais $\mathcal{I} = (((0~1~2~5~4~3~6),(5,4,4,\break1,2,8)),((0~1~2~3~4~5~6),(4,3,1,2,0,1),(8,5,3,7,1,3)))$ que pertence ao cenário sorvedouro. Note que a instância $\mathcal{I}$ possui duas regiões intergênicas instáveis ($ir_i(\mathcal{I}) = 2$ e $\mathcal{S}_{i}=\{\breve\pi_3,\breve\pi_6\}$) e quatro regiões intergênicas estáveis ($\mathcal{S}_{e}=\{\breve\pi_1,\breve\pi_2,\breve\pi_4,\breve\pi_5\}$). No exemplo, temos duas região intergênica auxiliares ($ir_a(\mathcal{I}) = 2$ e $\mathcal{S}_{a}=\{\breve\pi_1,\breve\pi_5\}$). Note que $gap_{\max}(\breve\pi_1) = 3$, $gap_{\max}(\breve\pi_2) = 1$, $gap_{\max}(\breve\pi_4) = 0$ e $gap_{\max}(\breve\pi_5) = 5$. Logo, $ir_d(\mathcal{I}) = 2$ e $\mathcal{S}_{d}=\{\breve\pi_2,\breve\pi_4\}$. Além disso, perceba que $\alpha = 2$. Como $\breve\pi_1$ é a última região intergênica auxiliar considerando um ordenação decrescente pelo valor de $gap_{\max}$, temos que $\breve\iota'_1 = \breve\iota^{\max}_1 -\alpha = 8 - 2 = 6$.

\begin{example}\label{example:OQWFSAKO}
  \scriptsize
  \hfill \break
  \begin{tikzpicture}[arrow/.style={single arrow, draw=black, fill=#1, single arrow head extend=2mm}]
    \node[fill = white!10, align = left, text width = 25mm, minimum width = 25mm] at (-1.5, 3) {$(\pi,\breve\pi) = $};
    \node[minimum size = 10mm] at (0, 3.7) {$\pi_0$};
    \node[minimum size = 10mm] at (2, 3.7) {$\pi_1$};
    \node[minimum size = 10mm] at (4, 3.7) {$\pi_2$};
    \node[minimum size = 10mm] at (6, 3.7) {$\pi_3$};
    \node[minimum size = 10mm] at (8, 3.7) {$\pi_4$};
    \node[minimum size = 10mm] at (10, 3.7) {$\pi_5$};
    \node[minimum size = 10mm] at (12, 3.7) {$\pi_6$};
    \node[minimum size = 10mm] at (1, 3.7) {$\breve\pi_1$};
    \node[minimum size = 10mm] at (3, 3.7) {$\breve\pi_2$};
    \node[minimum size = 10mm] at (5, 3.7) {$\breve\pi_3$};
    \node[minimum size = 10mm] at (7, 3.7) {$\breve\pi_4$};
    \node[minimum size = 10mm] at (9, 3.7) {$\breve\pi_5$};
    \node[minimum size = 10mm] at (11, 3.7) {$\breve\pi_6$};
    \draw ( 1, 3) pic{ir = {$5$, black!10}};
    \draw ( 3, 3) pic{ir = {$4$, black!10}};
    \draw ( 5, 3) pic{ir = {$4$, black!10}};
    \draw ( 7, 3) pic{ir = {$1$, black!10}};
    \draw ( 9, 3) pic{ir = {$2$, black!10}};
    \draw (11, 3) pic{ir = {$8$, black!10}};
    \draw ( 0, 3) pic{gene = {$0$, red!50}};
    \draw ( 2, 3) pic{gene = {$1$, orange!50}};
    \draw ( 4, 3) pic{gene = {$2$, blue!50}};
    \draw ( 6, 3) pic{gene = {$5$, brown!50}};
    \draw ( 8, 3) pic{gene = {$4$, green!50}};
    \draw (10, 3) pic{gene = {$3$, teal!50}};
    \draw (12, 3) pic{gene = {$6$, violet!50}};
    \node[fill = white!10, align = left, text width = 25mm, minimum width = 25mm] at (-1.5, 1.5) {$(\iota,\breve\iota^{\min},\breve\iota^{\max}) = $};
    \node[minimum size = 10mm] at (0, 2.2) {$\iota_0$};
    \node[minimum size = 10mm] at (2, 2.2) {$\iota_1$};
    \node[minimum size = 10mm] at (4, 2.2) {$\iota_2$};
    \node[minimum size = 10mm] at (6, 2.2) {$\iota_3$};
    \node[minimum size = 10mm] at (8, 2.2) {$\iota_4$};
    \node[minimum size = 10mm] at (10, 2.2) {$\iota_5$};
    \node[minimum size = 10mm] at (12, 2.2) {$\iota_6$};
    \draw ( 1, 1.5) pic{flex ir = {$4$, $8$, black!10}};
    \draw ( 3, 1.5) pic{flex ir = {$3$, $5$, black!10}};
    \draw ( 5, 1.5) pic{flex ir = {$1$, $3$, black!10}};
    \draw ( 7, 1.5) pic{flex ir = {$2$, $7$, black!10}};
    \draw ( 9, 1.5) pic{flex ir = {$0$, $1$, black!10}};
    \draw (11, 1.5) pic{flex ir = {$1$, $3$, black!10}};
    \draw ( 0, 1.5) pic{gene = {$0$, red!50}};
    \draw ( 2, 1.5) pic{gene = {$1$, orange!50}};
    \draw ( 4, 1.5) pic{gene = {$2$, blue!50}};
    \draw ( 6, 1.5) pic{gene = {$3$, teal!50}};
    \draw ( 8, 1.5) pic{gene = {$4$, green!50}};
    \draw (10, 1.5) pic{gene = {$5$, brown!50}};
    \draw (12, 1.5) pic{gene = {$6$, violet!50}};

    \node[fill = white!10, align = center, text width = 25mm, minimum width = 25mm] at (6, 0.3) {$\mathcal{F}_{ir}^{''}(\mathcal{I})$};
    \node[fill = white!10, align = center, text width = 25mm, minimum width = 25mm] at (6, -1.2) {$\mathcal{I}'$};
    \node[arrow, rotate=270, draw=black, fill=white, minimum size=8mm](arrow) at (6, -0.5) {};

    \node[fill = white!10, align = left, text width = 25mm, minimum width = 25mm] at (-1.5, -2.5) {$(\pi',\breve\pi') = $};
    \node[minimum size = 10mm] at ( 0, -1.8) {$\pi'_0$};
    \node[minimum size = 10mm] at ( 2, -1.8) {$\pi'_1$};
    \node[minimum size = 10mm] at ( 4, -1.8) {$\pi'_2$};
    \node[minimum size = 10mm] at ( 6, -1.8) {$\pi'_3$};
    \node[minimum size = 10mm] at ( 8, -1.8) {$\pi'_4$};
    \node[minimum size = 10mm] at (10, -1.8) {$\pi'_5$};
    \node[minimum size = 10mm] at (12, -1.8) {$\pi'_6$};
    \node[minimum size = 10mm] at ( 1, -1.8) {$\breve\pi'_1$};
    \node[minimum size = 10mm] at ( 3, -1.8) {$\breve\pi'_2$};
    \node[minimum size = 10mm] at ( 5, -1.8) {$\breve\pi'_3$};
    \node[minimum size = 10mm] at ( 7, -1.8) {$\breve\pi'_4$};
    \node[minimum size = 10mm] at ( 9, -1.8) {$\breve\pi'_5$};
    \node[minimum size = 10mm] at (11, -1.8) {$\breve\pi'_6$};
    \draw ( 1, -2.5) pic{ir = {$5$, black!10}};
    \draw ( 3, -2.5) pic{ir = {$4$, black!10}};
    \draw ( 5, -2.5) pic{ir = {$4$, black!10}};
    \draw ( 7, -2.5) pic{ir = {$1$, black!10}};
    \draw ( 9, -2.5) pic{ir = {$2$, black!10}};
    \draw (11, -2.5) pic{ir = {$8$, black!10}};
    \draw ( 0, -2.5) pic{gene = {$0$, red!50}};
    \draw ( 2, -2.5) pic{gene = {$1$, orange!50}};
    \draw ( 4, -2.5) pic{gene = {$2$, blue!50}};
    \draw ( 6, -2.5) pic{gene = {$5$, brown!50}};
    \draw ( 8, -2.5) pic{gene = {$4$, green!50}};
    \draw (10, -2.5) pic{gene = {$3$, teal!50}};
    \draw (12, -2.5) pic{gene = {$6$, violet!50}};
    \node[fill = white!10, align = left, text width = 25mm, minimum width = 25mm] at (-1.5, -4.0) {$(\iota',\breve\iota') = $};
    \node[minimum size = 10mm] at ( 0, -3.3) {$\iota'_0$};
    \node[minimum size = 10mm] at ( 2, -3.3) {$\iota'_1$};
    \node[minimum size = 10mm] at ( 4, -3.3) {$\iota'_2$};
    \node[minimum size = 10mm] at ( 6, -3.3) {$\iota'_3$};
    \node[minimum size = 10mm] at ( 8, -3.3) {$\iota'_4$};
    \node[minimum size = 10mm] at (10, -3.3) {$\iota'_5$};
    \node[minimum size = 10mm] at (12, -3.3) {$\iota'_6$};
    \node[minimum size = 10mm] at ( 1, -3.3) {$\breve\iota'_1$};
    \node[minimum size = 10mm] at ( 3, -3.3) {$\breve\iota'_2$};
    \node[minimum size = 10mm] at ( 5, -3.3) {$\breve\iota'_3$};
    \node[minimum size = 10mm] at ( 7, -3.3) {$\breve\iota'_4$};
    \node[minimum size = 10mm] at ( 9, -3.3) {$\breve\iota'_5$};
    \node[minimum size = 10mm] at (11, -3.3) {$\breve\iota'_6$};
    \draw ( 1, -4.0) pic{ir = {$6$, black!10}};
    \draw ( 3, -4.0) pic{ir = {$4$, black!10}};
    \draw ( 5, -4.0) pic{ir = {$3$, black!10}};
    \draw ( 7, -4.0) pic{ir = {$7$, black!10}};
    \draw ( 9, -4.0) pic{ir = {$1$, black!10}};
    \draw (11, -4.0) pic{ir = {$3$, black!10}};
    \draw ( 0, -4.0) pic{gene = {$0$, red!50}};
    \draw ( 2, -4.0) pic{gene = {$1$, orange!50}};
    \draw ( 4, -4.0) pic{gene = {$2$, blue!50}};
    \draw ( 6, -4.0) pic{gene = {$3$, teal!50}};
    \draw ( 8, -4.0) pic{gene = {$4$, green!50}};
    \draw (10, -4.0) pic{gene = {$5$, brown!50}};
    \draw (12, -4.0) pic{gene = {$6$, violet!50}};
  \end{tikzpicture}
\end{example}

\begin{lemma}\label{lemma:KPGCUTDM}
Seja $\mathcal{I'} = ((\pi',\breve\pi'),(\iota',\breve\iota'))$ uma instância intergênica rígida balanceada sem sinais tal que $\mathcal{I'} = \mathcal{F}_{ir}^{''}(\mathcal{I})$, onde $\mathcal{I} = ((\pi,\breve\pi),(\iota,\breve\iota^{\min},\breve\iota^{\max}))$ é uma instância intergênica flexível balanceada sem sinais, então temos que $ir_i(\mathcal{I}) + ir_a(\mathcal{I}) = ib_1(\mathcal{I'})$.
\end{lemma}
\begin{proof}
Pela construção da função $\mathcal{F}_{1}^{''}$ temos que cada região intergênica definitiva tipo $\mathcal{I}$ é mapeada em uma adjacência intergênica em $\mathcal{I'}$. Além disso, cada região intergênica instável e auxiliar acaba tornando-se um breakpoint tipo um em $\mathcal{I'}$, seja porque os elementos antes e depois da região intergênica não são adjacentes no genoma alvo ou porque o tamanho da região intergênica é menor do que o mínimo ou maior do que o máximo permitido no genoma alvo. Logo, $ir_i(\mathcal{I}) + ir_a(\mathcal{I}) = ib_1(\mathcal{I'})$ e o lema segue.
\end{proof}

\begin{lemma}\label{lemma:KIVEWTOR}
Seja $\mathcal{I'} = ((\pi',\breve\pi'),(\iota',\breve\iota'))$ uma instância intergênica rígida balanceada sem sinais tal que $\mathcal{I'} = \mathcal{F}_{ir}^{''}(\mathcal{I})$, onde $\mathcal{I} = ((\pi,\breve\pi),(\iota,\breve\iota^{\min},\breve\iota^{\max}))$ é uma instância intergênica flexível balanceada sem sinais, e seja $S'$ uma sequência de eventos de rearranjo tal que $(\pi',\breve\pi') \cdot S' = (\iota',\breve\iota')$, então $S'$ é uma sequência que faz com que o genoma alvo em $\mathcal{I}$ seja atingido.
\end{lemma}
\begin{proof}
Diretamente pela construção da função $\mathcal{F}_{ir}^{''}$. Note que $(\pi',\breve\pi') = (\pi,\breve\pi)$, $\iota' = \iota$ e $\forall i \in \{1,2,\dots,({n+1})\}: \breve\iota^{\min}_i \le \breve\iota'_i \le \breve\iota^{\max}_i$.
\end{proof}

Agora definiremos, com base no grafo de ciclos ponderado flexível, duas funções de redução que criam uma instância intergênica rígida a partir de uma instância intergênica flexível. Note que tanto o grafo de ciclos ponderado rígido como o grafo de ciclos ponderado flexível são extensões do grafo de ciclos clássico que possuem as mesmas regras para a contrução dos conjuntos de vértices e arestas. A principal diferença na estrutura desses dois grafos é que nas arestas cinzas do grafo de ciclos ponderado rígido temos apenas um peso associado, que representa o tamanho estrito da região intergênica no genoma alvo que está entre os dois vértices conectados pela a aresta. Já no grafo de ciclos ponderado flexível, cada aresta cinza possui dois pesos associados, sendo eles: o tamanho mínimo e máximo permitido no genoma alvo para a região intergênica que está entre os dois vértices conectados pela a aresta. 

Observe que para criarmos um grafo de ciclos ponderado rígido a partir de um grafo de ciclos ponderado flexível basta desenvolver uma forma de associar um único peso para cada aresta cinza. Consequentemente, com o grafo de ciclos ponderado rígido construído podemos obter os valores da instância intergênica rígida que deu origem ao grafo.

Dada uma instância intergênica flexível $\mathcal{I} = ((\pi,\breve\pi),(\iota,\breve\iota^{\min},\breve\iota^{\max}))$ a função $\mathcal{F}_{c}^{'}$ cria uma instância intergênica rígida $\mathcal{I'} = ((\pi',\breve\pi'),(\iota',\breve\iota'))$ da seguinte forma:

\begin{itemize}
  \item Para cada ciclo estável $C=(c^1,\dots,c^k)$ em $G(\mathcal{I})$ encontre uma lista de números inteiros não negativos $L=(l_1,\dots,l_k)$, tal que $\sum_{i=1}^{k}l_i = W_p(C)$ e $\forall e^{\prime}_i \in E_c(C): w^{\min}_c(e^{\prime}_i) \le l_i \le w^{\max}_c(e^{\prime}_i)$. Note que pela definição de um ciclo estável $C$ temos que $W^{\min}_c(C) \le W_p(C) \le W^{\max}_c(C)$, então sempre é possível encontrar uma lista $L$ com tais características. Por fim, para cada aresta cinza $e^{\prime}_i$ de $C$ no grafo de ciclos ponderado flexível, atribua o peso $l_i$ na aresta cinza $e^{\prime}_i$ do grafo de ciclos ponderado rígido. Por construção, temos que cada ciclo estável em $G(\mathcal{I})$ torna-se um ciclo balanceado em $G(\mathcal{I}')$.
  \item Para cada ciclo instável $C$ em $G(\mathcal{I})$ realize a seguinte atribuição de peso: para cada aresta cinza $e^{\prime}_i$ de $C$ no grafo de ciclos ponderado flexível, atribua o peso $w^{\min}_c(e^{\prime}_i)$ na aresta cinza $e^{\prime}_i$ do grafo de ciclos ponderado rígido. Pela definição de um ciclo instável $C$, temos que $W_p(C)  < W^{\min}_c(C)$ ou $W_p(C) > W^{\max}_c(C)$. Por construção, temos que cada ciclo instável em $G(\mathcal{I})$ torna-se um ciclo desbalanceado em $G(\mathcal{I}')$.
\end{itemize}

Note que os conjuntos de vértices e arestas em $G(\mathcal{I})$ e $G(\mathcal{I}')$ são os mesmos. Logo, $c(G(\mathcal{I})) = c(G(\mathcal{I}'))$. Denotamos por $\mathcal{F}_{c}^{'}(\mathcal{I})$ a instância intergênica rígida criada pela função $\mathcal{F}_{c}^{'}$ a partir de uma instância intergênica flexível $\mathcal{I}$. Perceba que a função $\mathcal{F}_{c}^{'}$ apenas define o peso de cada aresta cinza no grafo de ciclos ponderado rígido. Além disso, a função $\mathcal{F}_{c}^{'}$ pode ser executada em tempo linear.

O Exemplo~\ref{example:MYLRZMXK} mostra uma instância intergênica rígida com sinais $\mathcal{I}' = (({+0}~{+3}~{+2}\break{+1}~{+4}~{+5}~{+6}),(1,3,0,2,5,2)),(({+0}~{+1}~{+2}~{+3}~{+4}~{+5}~{+6}),(3,2,4,0,5,2))$ criada pela função $\mathcal{F}_{c}^{'}$ a partir de uma instância intergênica flexível com sinais $\mathcal{I} = ((({+0}~{+3}~{+2}~{+1}~{+4}~{+5}\break{+6}),(1,3,0,2,5,2)),(({+0}~{+1}~{+2}~{+3}~{+4}~{+5}~{+6}),(3,2,4,0,2,1),(4,3,5,4,6,2)))$. Note que $\mathcal{S}_i(G(\mathcal{I})) = \{C_1=(3,1),C_2=(4,2)\}$ e $\mathcal{S}_e(G(\mathcal{I})) = \{C_3=(5),C_4=(6)\}$. Os ciclos estáveis $C_3$ e $C_4$ são mapeados em ciclos balanceados em $\mathcal{I}'$, enquanto os ciclos instáveis $C_1$ e $C_2$ são mapeados em ciclos desbalanceados.

\begin{example}\label{example:MYLRZMXK}
    \hfill \break
    \centering
    \begin{tikzpicture}[scale=0.7, gene/.style={single arrow,
            draw=black,
            fill=#1,
            single arrow head extend=2mm
        }]
    \scriptsize

        \begin{scope}
            \node[draw=none,fill=none, minimum height=1cm, minimum width=1cm, align=center] at (1.0, 3.0) {$G(\mathcal{I})$};
        \end{scope}

        \begin{scope}[every node/.style={inner sep=1.5pt, minimum size = 0pt}]
            \node[circle, draw] (p0) at (0,0) {$+0$};
            \node[circle, draw] (m3) at (1.5,0) {$-3$};
            \node[circle, draw] (p3) at (3,0) {$+3$};
            \node[circle, draw] (m2) at (4.5,0) {$-2$};
            \node[circle, draw] (p2) at (6,0) {$+2$};
            \node[circle, draw] (m1) at (7.5,0) {$-1$};
            \node[circle, draw] (p1) at (9,0) {$+1$};
            \node[circle, draw] (m4) at (10.5,0) {$-4$};
            \node[circle, draw] (p4) at (12,0) {$+4$};
            \node[circle, draw] (m5) at (13.5,0) {$-5$};
            \node[circle, draw] (p5) at (15,0) {$+5$};
            \node[circle, draw] (m6) at (16.5,0) {$-6$};
        \end{scope}

        \begin{scope}[>={Stealth[black]},
                      every edge/.style={draw=black}]
            \path [-] (p0) edge node [black, pos=0.5, sloped, below, yshift=-0.15cm] {$1$} (m3);
            \node[draw=none, fill=none, align=center, minimum width=1cm, text width=1cm] at (0.75, -1.0) {$\ell = 1$};
            \path [-] (p3) edge node [black, pos=0.5, sloped, below, yshift=-0.15cm] {$3$} (m2);
            \node[draw=none, fill=none, align=center, minimum width=1cm, text width=1cm] at (3.75, -1.0) {$\ell = 2$};
            \path [-] (p2) edge node [black, pos=0.5, sloped, below, yshift=-0.15cm] {$0$} (m1);
            \node[draw=none, fill=none, align=center, minimum width=1cm, text width=1cm] at (6.75, -1.0) {$\ell = 3$};
            \path [-] (p1) edge node [black, pos=0.5, sloped, below, yshift=-0.15cm] {$2$} (m4);
            \node[draw=none, fill=none, align=center, minimum width=1cm, text width=1cm] at (9.75, -1.0) {$\ell = 4$};
            \path [-] (p4) edge node [black, pos=0.5, sloped, below, yshift=-0.15cm] {$5$} (m5);
            \node[draw=none, fill=none, align=center, minimum width=1cm, text width=1cm] at (12.75, -1.0) {$\ell = 5$};
            \path [-] (p5) edge node [black, pos=0.5, sloped, below, yshift=-0.15cm] {$2$} (m6);
            \node[draw=none, fill=none, align=center, minimum width=1cm, text width=1cm] at (15.75, -1.0) {$\ell = 6$};
        \end{scope}

        \begin{scope}[>={Stealth[black]},
                      every edge/.style={draw=black}]
            \path [-] (p0) edge [bend left=70,  dashed] node [black, pos=0.5, sloped, below, yshift=-0.05cm] {$3$} node [black, pos=0.5, sloped, above, yshift=+0.05cm] {$4$} (m1);
            \path [-] (p1) edge [bend right=50, dashed] node [black, pos=0.5, sloped, below, yshift=-0.05cm] {$2$} node [black, pos=0.5, sloped, above, yshift=+0.05cm] {$3$} (m2);
            \path [-] (p2) edge [bend right=50, dashed] node [black, pos=0.5, sloped, below, yshift=-0.05cm] {$4$} node [black, pos=0.5, sloped, above, yshift=+0.05cm] {$5$} (m3);
            \path [-] (p3) edge [bend left=70,  dashed] node [black, pos=0.5, sloped, below, yshift=-0.05cm] {$0$} node [black, pos=0.5, sloped, above, yshift=+0.05cm] {$4$} (m4);
            \path [-] (p4) edge [bend left=70,  dashed] node [black, pos=0.5, sloped, below, yshift=-0.05cm] {$2$} node [black, pos=0.5, sloped, above, yshift=+0.05cm] {$6$} (m5);
            \path [-] (p5) edge [bend left=70,  dashed] node [black, pos=0.5, sloped, below, yshift=-0.05cm] {$1$} node [black, pos=0.5, sloped, above, yshift=+0.05cm] {$2$} (m6);
        \end{scope}

        \begin{scope}[every node/.style={draw=black, fill=white, minimum size=8mm}]
            \node[draw=none, fill=none, align=center, minimum width=3cm, text width=3cm] at (8.25, -1.75) {$\mathcal{F}'_{c}(\mathcal{I})$};
            \node[gene, rotate=270](arrow) at (8.25,-2.75) {};
        \end{scope}
            
        \begin{scope}
            \node[draw=none,fill=none, minimum height=1cm, minimum width=1cm, align=center] at (8.25, -4.0) {$\mathcal{I}'$};
            \node[draw=none,fill=none, minimum height=1cm, minimum width=1cm, align=center] at (1.0, -4.5) {$G(\mathcal{I}')$};
        \end{scope}

        \begin{scope}[every node/.style={inner sep=1.5pt, minimum size = 0pt}]
            \node[circle, draw] (p0) at (0,-7) {$+0$};
            \node[circle, draw] (m3) at (1.5,-7) {$-3$};
            \node[circle, draw] (p3) at (3,-7) {$+3$};
            \node[circle, draw] (m2) at (4.5,-7) {$-2$};
            \node[circle, draw] (p2) at (6,-7) {$+2$};
            \node[circle, draw] (m1) at (7.5,-7) {$-1$};
            \node[circle, draw] (p1) at (9,-7) {$+1$};
            \node[circle, draw] (m4) at (10.5,-7) {$-4$};
            \node[circle, draw] (p4) at (12,-7) {$+4$};
            \node[circle, draw] (m5) at (13.5,-7) {$-5$};
            \node[circle, draw] (p5) at (15,-7) {$+5$};
            \node[circle, draw] (m6) at (16.5,-7) {$-6$};
        \end{scope}

        \begin{scope}[>={Stealth[black]},
                      every edge/.style={draw=black}]
            \path [-] (p0) edge node [black, pos=0.5, sloped, below, yshift=-0.15cm] {$1$} (m3);
            \node[draw=none, fill=none, align=center, minimum width=1cm, text width=1cm] at (0.75, -8.0) {$\ell = 1$};
            \path [-] (p3) edge node [black, pos=0.5, sloped, below, yshift=-0.15cm] {$3$} (m2);
            \node[draw=none, fill=none, align=center, minimum width=1cm, text width=1cm] at (3.75, -8.0) {$\ell = 2$};
            \path [-] (p2) edge node [black, pos=0.5, sloped, below, yshift=-0.15cm] {$0$} (m1);
            \node[draw=none, fill=none, align=center, minimum width=1cm, text width=1cm] at (6.75, -8.0) {$\ell = 3$};
            \path [-] (p1) edge node [black, pos=0.5, sloped, below, yshift=-0.15cm] {$2$} (m4);
            \node[draw=none, fill=none, align=center, minimum width=1cm, text width=1cm] at (9.75, -8.0) {$\ell = 4$};
            \path [-] (p4) edge node [black, pos=0.5, sloped, below, yshift=-0.15cm] {$5$} (m5);
            \node[draw=none, fill=none, align=center, minimum width=1cm, text width=1cm] at (12.75, -8.0) {$\ell = 5$};
            \path [-] (p5) edge node [black, pos=0.5, sloped, below, yshift=-0.15cm] {$2$} (m6);
            \node[draw=none, fill=none, align=center, minimum width=1cm, text width=1cm] at (15.75, -8.0) {$\ell = 6$};
        \end{scope}

        \begin{scope}[>={Stealth[black]},
                      every edge/.style={draw=black}]
            \path [-] (p0) edge [bend left=70,  dashed] node [black, pos=0.5, sloped, below, yshift=+0.05cm] {$3$} (m1);
            \path [-] (p1) edge [bend right=50, dashed] node [black, pos=0.5, sloped, below, yshift=+0.05cm] {$2$} (m2);
            \path [-] (p2) edge [bend right=50, dashed] node [black, pos=0.5, sloped, below, yshift=+0.05cm] {$4$} (m3);
            \path [-] (p3) edge [bend left=70,  dashed] node [black, pos=0.5, sloped, below, yshift=+0.05cm] {$0$} (m4);
            \path [-] (p4) edge [bend left=70,  dashed] node [black, pos=0.5, sloped, below, yshift=+0.05cm] {$5$} (m5);
            \path [-] (p5) edge [bend left=70,  dashed] node [black, pos=0.5, sloped, below, yshift=+0.05cm] {$2$} (m6);
        \end{scope}
    \end{tikzpicture}
\end{example}

\begin{lemma}\label{lemma:AOKHMVAY}
Seja $\mathcal{I'} = ((\pi',\breve\pi'),(\iota',\breve\iota'))$ uma instância intergênica rígida tal que $\mathcal{I'} = \mathcal{F}_{c}^{'}(\mathcal{I})$, onde $\mathcal{I} = ((\pi,\breve\pi),(\iota,\breve\iota^{\min},\breve\iota^{\max}))$ é uma instância intergênica flexível, então temos que $c(G(\mathcal{I})) = c(G(\mathcal{I}'))$ e $c_e(G(\mathcal{I})) = c_b(G(\mathcal{I}'))$.
\end{lemma}
\begin{proof}
Diretamente pela construção da função $\mathcal{F}_{c}^{'}$.
\end{proof}

\begin{lemma}\label{lemma:TQUNQUGX}
Seja $\mathcal{I'} = ((\pi',\breve\pi'),(\iota',\breve\iota'))$ uma instância intergênica rígida tal que $\mathcal{I'} = \mathcal{F}_{c}^{'}(\mathcal{I})$, onde $\mathcal{I} = ((\pi,\breve\pi),(\iota,\breve\iota^{\min},\breve\iota^{\max}))$ é uma instância intergênica flexível, e seja $S'$ uma sequência de eventos de rearranjo tal que $(\pi',\breve\pi') \cdot S' = (\iota',\breve\iota')$, então $S'$ é uma sequência que faz com que o genoma alvo em $\mathcal{I}$ seja atingido.
\end{lemma}
\begin{proof}
Note que para uma aresta cinza $e^{\prime}_i$ em $G(\mathcal{I}')$ temos um peso $w_c(e^{\prime}_i)$ associado. Para a mesma aresta cinza $e^{\prime}_i$, mas em $G(\mathcal{I})$, temos os pesos mínimo $w^{\min}_c(e^{\prime}_i)$ e máximo $w^{\max}_c(e^{\prime}_i)$ associados. Pela forma como os pesos foram atribuídos em cada aresta cinza $e^{\prime}_i$ de $G(\mathcal{I}')$ temos a garantia de que $w^{\min}_c(e^{\prime}_i) \le w_c(e^{\prime}_i) \le w^{\max}_c(e^{\prime}_i)$. Dessa forma, isso implica que $\forall i \in \{1,2,\dots,({n+1})\}: \breve\iota^{\min}_i \le \breve\iota'_i \le \breve\iota^{\max}_i$. Além disso, o peso em cada aresta preta de $G(\mathcal{I}')$ é igual ao peso da mesma aresta preta em $G(\mathcal{I})$ e com os conjuntos de vértices e arestas sendo os mesmos em $G(\mathcal{I})$ e $G(\mathcal{I}')$. Logo, $(\pi',\breve\pi') = (\pi,\breve\pi)$, $\iota' = \iota$ e o lema segue.
\end{proof}

Agora considere a seguinte função de redução considerando um modelo composto exclusivamento por eventos de rearranjo conservativos. Dada uma instância intergênica flexível balanceada $\mathcal{I} = ((\pi,\breve\pi),(\iota,\breve\iota^{\min},\breve\iota^{\max}))$ a função $\mathcal{F}_{c}^{''}$ cria uma instância intergênica rígida balanceada $\mathcal{I'} = ((\pi',\breve\pi'),(\iota',\breve\iota'))$ da seguinte forma:

\begin{itemize}
  \item Para cada ciclo definitivo $C=(c^1,\dots,c^k)$ em $G(\mathcal{I})$ encontre uma lista de números inteiros não negativos $L=(l_1,\dots,l_k)$, tal que $\sum_{i=1}^{k}l_i = W_p(C)$ e $\forall e^{\prime}_i \in E_c(C): w^{\min}_c(e^{\prime}_i) \le l_i \le w^{\max}_c(e^{\prime}_i)$. Note que um ciclo definitivo $C$ também é um ciclo estável e temos que $W^{\min}_c(C) \le W_p(C) \le W^{\max}_c(C)$, então sempre é possível encontrar uma lista $L$ com tais características. Por fim, para cada aresta cinza $e^{\prime}_i$ de $C$ no grafo de ciclos ponderado flexível, atribua o peso $l_i$ na aresta cinza $e^{\prime}_i$ do grafo de ciclos ponderado rígido. Por construção, temos que cada ciclo definitivo em $G(\mathcal{I})$ torna-se um ciclo balanceado em $G(\mathcal{I}')$.
  \item O mapeamento dos ciclos restantes em $G(\mathcal{I})$ depende do cenário em que a instância $\mathcal{I}$ se encaixa: 

  \begin{itemize}
    \item Se for o cenário fonte, então para cada ciclo instável $C \in G(\mathcal{I})$ realize a seguinte atribuição de peso: para cada aresta cinza $e^{\prime}_i$ de $C$, atribua o peso $w^{\min}_c(e^{\prime}_i)$ na aresta cinza $e^{\prime}_i$ do grafo de ciclos ponderado rígido. Pela definição de um ciclo instável $C$, temos que $W_p(C)  < W^{\min}_c(C)$ ou $W_p(C) > W^{\max}_c(C)$. Por construção, temos que cada ciclo instável em $G(\mathcal{I})$ torna-se um ciclo desbalanceado em $G(\mathcal{I}')$. Considerando que os ciclos auxiliares estão ordenados pelo valor de $gap_{\min}$ de maneira decrescente e que $c_a(G(\mathcal{I})) = x$, então pra os primeiros $x-1$ ciclos auxiliares realize a seguinte atribuição de peso: para cada aresta cinza $e^{\prime}_i$ do ciclo, atribua o peso $w^{\min}_c(e^{\prime}_i)$ na aresta cinza $e^{\prime}_i$ do grafo de ciclos ponderado rígido. Para o último ciclo auxiliar $C=(c^1,\dots,c^k)$ encontre uma lista de números inteiros não negativos $L=(l_1,\dots,l_k)$, tal que $\forall e^{\prime}_i \in E_c(C): w^{\min}_c(e^{\prime}_i) \le l_i \le w^{\max}_c(e^{\prime}_i)$ e $\sum_{i=1}^{k}l_i = W^{\min}_c(C) + \alpha$, onde $\alpha = \sum_{C \in \mathcal{S}_a(G(\mathcal{I}))} gap_{\min}(C) + \sum_{C \in \mathcal{S}_i(G(\mathcal{I}))} gap_{\min}(C)$. Note que $C$ também é um ciclo estável e $gap_{\min}(C) \ge \alpha$, então sempre é possível encontrar uma lista $L$ com tais características. Por construção, temos que cada ciclo auxiliar em $G(\mathcal{I})$ também torna-se um ciclo desbalanceado em $G(\mathcal{I}')$.

    \item Se for o cenário sorvedouro, então para cada ciclo instável $C \in G(\mathcal{I})$ realize a seguinte atribuição de peso: para cada aresta cinza $e^{\prime}_i$ de $C$, atribua o peso $w^{\max}_c(e^{\prime}_i)$ na aresta cinza $e^{\prime}_i$ do grafo de ciclos ponderado rígido. Pela definição de um ciclo instável $C$, temos que $W_p(C)  < W^{\min}_c(C)$ ou $W_p(C) > W^{\max}_c(C)$. Por construção, temos que cada ciclo instável em $G(\mathcal{I})$ torna-se um ciclo desbalanceado em $G(\mathcal{I}')$. Considerando que os ciclos auxiliares estão ordenados pelo valor de $gap_{\max}$ de maneira decrescente e que $c_a(G(\mathcal{I})) = x$, então pra os primeiros $x-1$ ciclos auxiliares realize a seguinte atribuição de peso: para cada aresta cinza $e^{\prime}_i$ do ciclo no grafo de ciclos ponderado flexível, atribua o peso $w^{\max}_c(e^{\prime}_i)$ na aresta cinza $e^{\prime}_i$ do grafo de ciclos ponderado rígido. Para o último ciclo auxiliar $C=(c^1,\dots,c^k)$ encontre uma lista de números inteiros não negativos $L=(l_1,\dots,l_k)$, tal que $\forall e^{\prime}_i \in E_c(C): w^{\min}_c(e^{\prime}_i) \le l_i \le w^{\max}_c(e^{\prime}_i)$ e $\sum_{i=1}^{k}l_i = W^{\max}_c(C) - \alpha$, onde $\alpha = \sum_{C \in \mathcal{S}_a(G(\mathcal{I}))} gap_{\max}(C) + \sum_{C \in \mathcal{S}_i(G(\mathcal{I}))} gap_{\max}(C)$. Note que $C$ também é um ciclo estável e $gap_{\max}(C) \ge \alpha$, então sempre é possível encontrar uma lista $L$ com tais características. Por construção, temos que cada ciclo auxiliar em $G(\mathcal{I})$ também torna-se um ciclo desbalanceado em $G(\mathcal{I}')$.

    \item Se for o cenário de equilíbrio, então o conjunto de ciclos auxiliares é vazio e a soma do peso total dos ciclos instáveis é suficiente para torná-los em ciclos estáveis, ou seja, $$\sum_{C \in \mathcal{S}_i(G(\mathcal{I}))} W^{\min}_c(C) \le \sum_{C \in \mathcal{S}_i(G(\mathcal{I}))} W_p(C) \le \sum_{C \in \mathcal{S}_i(G(\mathcal{I}))} W^{\max}_c(C).$$ Para este caso basta atribuir pesos nas arestas cinzas que pertencem aos ciclos instáveis em $G(\mathcal{I})$ de maneira que o peso atribuído em cada aresta cinza no grafo de ciclos ponderado rígido não viole os pesos mínimo e máximo permitidos para a mesma aresta em $G(\mathcal{I})$. Além disso, a soma dos pesos atribuídos nas arestas cinzas deve ser igual a $\sum_{C \in \mathcal{S}_i(G(\mathcal{I}))} W_p(C)$ para garantir que a instância intergênica rígida $\mathcal{I}'$ resultante seja balanceada. Uma possível forma de realizar essa tarefa é: para cada aresta $e^{\prime}_i$ pertencente a um ciclo instávél em $G(\mathcal{I})$, atribua inicialmente o peso $w^{\min}_c(e^{\prime}_i)$ na aresta cinza $e^{\prime}_i$ do grafo de ciclos ponderado rígido. Perceba que ainda pode ser necessário distribuir um peso de $X = \sum_{C \in \mathcal{S}_i(G(\mathcal{I}))} W_p(C) - \sum_{C \in \mathcal{S}_i(G(\mathcal{I}))} W^{\min}_c(C)$ por essas mesmas arestas sem violar o peso mínimo e máximo permitido para cada uma delas em $G(\mathcal{I})$. Note que isso sempre é possível de ser realizado, pois $\sum_{C \in \mathcal{S}_i(G(\mathcal{I}))} W^{\min}_c(C) \le \sum_{C \in \mathcal{S}_i(G(\mathcal{I}))} W_p(C) \le \sum_{C \in \mathcal{S}_i(G(\mathcal{I}))} W^{\max}_c(C)$. Basta percorrer a mesmas arestas cinzas e, quando possível, incremente o peso atribuído na aresta até que o valor de $X$ seja igual a zero.
  \end{itemize}
\end{itemize}

Note que aqui também temos que os conjuntos de vértices e arestas em $G(\mathcal{I})$ e $G(\mathcal{I}')$ são os mesmos. Logo, $c(G(\mathcal{I})) = c(G(\mathcal{I}'))$. Denotamos por $\mathcal{F}_{c}^{''}(\mathcal{I})$ a instância intergênica rígida balanceada criada pela função $\mathcal{F}_{c}^{''}$ a partir de uma instância intergênica flexível balanceada $\mathcal{I}$. Perceba que a função $\mathcal{F}_{c}^{''}$ apenas define o peso de cada aresta cinza no grafo de ciclos ponderado rígido. Além disso, a função $\mathcal{F}_{c}^{''}$ pode ser executada em tempo $\mathcal{O}(n \log n)$, tendo em vista que, no pior caso, os ciclos estáveis precisam ser ordenados para definir o conjunto de ciclos auxiliares.

O Exemplo~\ref{example:VZQMESKH} mostra uma instância intergênica rígida balanceada sem sinais $\mathcal{I}' = ((0~3~2~1~4~5~6),(4,3,4,3,6,1)),((0~1~2~3~4~5~6),(3,2,5,0,5,6))$ criada pela função $\mathcal{F}_{c}^{''}$ a partir de uma instância intergênica flexível balanceada sem sinais $\mathcal{I} = (((0~3~2~1~4~5~6),(4,3,4,\break3,6,1)),((0~1~2~3~4~5~6),(3,2,4,0,4,6),(4,3,5,4,7,8)))$. Note que a instância $\mathcal{I}$ pertence ao cenário fonte com $\mathcal{S}_i(G(\mathcal{I})) = \{C_4=(6)\}$ e $\mathcal{S}_e(G(\mathcal{I})) = \{C_1=(3,1),C_2=(4,2),C_3=(5)\}$. Além disso, temos que $\mathcal{S}_a(G(\mathcal{I})) = \{C_2=(4,2),C_3=(5)\}$ e $\mathcal{S}_d(G(\mathcal{I})) = \{C_1=(3,1)\}$. Note que os valores de $gap_{\min}(C_1)$, $gap_{\min}(C_2)$ e $gap_{\min}(C_3)$ são 1, 4 e 2, respectivamente. Além disso, temos que $\alpha = gap_{\min}(C_2) + gap_{\min}(C_3) + gap_{\min}(C_4) = 4 + 2 - 5 = 1$. Como o ciclo $C_3$ é o último considerando uma ordenação decrescente pelo valor de $gap_{\min}$ e $W^{\min}_c(C_3) = 4$, temos que a soma dos pesos das arestas cinzas do ciclo $C_3$ em $G(\mathcal{I}')$ deve ser igual a $W^{\min}_c(C_3) + \alpha = 4 + 1 = 5$. Entretando, $C_3$ é um ciclo trivial. Logo, sua única aresta cinza possui um peso $5$ associado.

\input{examples/VZQMESKH}

O Exemplo~\ref{example:XQPNLJUW} mostra uma instância intergênica rígida balanceada com sinais $\mathcal{I}' = (({+0}~{+3}~{+2}~{+1}~{+4}~{+5}~{+6}),(4,1,4,2,4,5)),(({+0}~{+1}~{+2}~{+3}~{+4}~{+5}~{+6}),(5,3,4,4,1,3))$ criada pela função $\mathcal{F}_{c}^{''}$ a partir de uma instância intergênica flexível balanceada sem sinais $\mathcal{I} = ((({+0}~{+3}~{+2}~{+1}~{+4}~{+5}~{+6}),(4,3,4,3,6,1)),(({+0}~{+1}~{+2}~{+3}~{+4}~{+5}~{+6}),(3,2,4,0,0,\break1),(6,3,5,4,1,3)))$. Note que a instância $\mathcal{I}$ pertence ao cenário sorvedouro com os conjuntos $\mathcal{S}_i(G(\mathcal{I})) = \{C_3=(5),C_4=(6)\}$ e $\mathcal{S}_e(G(\mathcal{I})) = \{C_1=(3,1),C_2=(4,2)\}$. Além disso, temos que $\mathcal{S}_a(G(\mathcal{I})) = \{C_1=(3,1),C_2=(4,2)\}$ e $\mathcal{S}_d(G(\mathcal{I})) = \varnothing$. Note que os valores de $gap_{\max}(C_1)$ e $gap_{\max}(C_2)$ são, respectivamente, 3 e 4. Além disso, temos que $\alpha = gap_{\max}(C_1) + gap_{\max}(C_2) + gap_{\max}(C_3) + gap_{\max}(C_4) = 3 + 4 - 3 - 2 = 2$. Como o ciclo $C_1$ é o último considerando uma ordenação decrescente pelo valor de $gap_{\max}$ e $W^{\max}_c(C_1) = 11$, temos que a soma dos pesos das arestas cinzas do ciclo $C_1$ em $G(\mathcal{I}')$ deve ser igual a $W^{\max}_c(C_1) - \alpha = 11 - 2 = 9$. Além disso, o peso associado em cada aresta de $C_1$ em $G(\mathcal{I}')$ deve atender o peso mínimo e máximo da mesma aresta em $G(\mathcal{I})$. No exemplo, temos que as arestas cinzas $({+0},{-1})$ e $({+2},{-3})$ possuem os pesos $5$ e $4$, respectivamente. Note que $w^{\min}_c(({+0},{-1})) \le w_c(({+0},{-1})) \le w^{\max}_c(({+0},{-1}))$ e $w^{\min}_c(({+2},{-3})) \le w_c(({+2},{-3})) \le w^{\max}_c(({+2},{-3}))$.

\input{examples/XQPNLJUW}

\begin{lemma}\label{lemma:PSGXFVHD}
Seja $\mathcal{I'} = ((\pi',\breve\pi'),(\iota',\breve\iota'))$ uma instância intergênica rígida balanceada tal que $\mathcal{I'} = \mathcal{F}_{c}^{''}(\mathcal{I})$, onde $\mathcal{I} = ((\pi,\breve\pi),(\iota,\breve\iota^{\min},\breve\iota^{\max}))$ é uma instância intergênica flexível balanceada, então temos que $c(G(\mathcal{I})) = c(G(\mathcal{I}'))$ e $c_d(G(\mathcal{I})) = c_b(G(\mathcal{I}'))$.
\end{lemma}
\begin{proof}
Diretamente pela construção da função $\mathcal{F}_{c}^{''}$.
\end{proof}

\begin{lemma}\label{lemma:WQOEFBXP}
Seja $\mathcal{I'} = ((\pi',\breve\pi'),(\iota',\breve\iota'))$ uma instância intergênica rígida balanceada tal que $\mathcal{I'} = \mathcal{F}_{c}^{'}(\mathcal{I})$, onde $\mathcal{I} = ((\pi,\breve\pi),(\iota,\breve\iota^{\min},\breve\iota^{\max}))$ é uma instância intergênica flexível balanceada, e seja $S'$ uma sequência de eventos de rearranjo tal que $(\pi',\breve\pi') \cdot S' = (\iota',\breve\iota')$, então $S'$ é uma sequência que faz com que o genoma alvo em $\mathcal{I}$ seja atingido.
\end{lemma}
\begin{proof}
A prova é similar a descrita no Lema~\ref{lemma:TQUNQUGX}.
\end{proof}

% ------------------------------------------------------------------ %
\subsection{Instâncias Intergênicas Flexíveis sem Sinais}
% ------------------------------------------------------------------ %

Nesta seção, apresentaremos algoritmos de aproximação para a variação sem sinais dos problemas investigados neste capítulo com base nas funções de redução apresentadas previamente.

% ------------------------------------------------------------------ %
\subsubsection{Reversão}
% ------------------------------------------------------------------ %

Nesta seção, apresentaremos um algoritmo de aproximação com fator $4$ para a variação sem sinais do problema \SbFIR{}. A seguir apresentamos o Algoritmo~\ref{algorithm:BSOTINLZ}.

\begin{algorithm}[!tbh]
  \caption{Um algoritmo de aproximação para o problema \SbFIR{}.\label{algorithm:BSOTINLZ}}
  \Entrada{Uma instância intergênica flexível balanceada sem sinais $\mathcal{I} = ((\pi,\breve\pi),(\iota,\breve\iota^{\min},\breve\iota^{\max}))$}
  \Saida{Uma sequência de eventos de reversão $S$, tal que $(\pi,\breve\pi) \cdot S$ atinge o genoma alvo de $\mathcal{I}$} 
  $\mathcal{I}' = \mathcal{F}_{ir}^{''}(\mathcal{I})$ \\
  Seja $S'$ uma sequência de eventos de reversão fornecida pelo Algoritmo~\ref{algorithm:AKKUXQNR} para a instância $\mathcal{I}'$ \\
  \Retorna{$S'$}
\end{algorithm}

\begin{theorem}\label{theorem:WKATVVBS}
Dada uma instância intergênica flexível balanceada sem sinais $\mathcal{I}$, o Algoritmo~\ref{algorithm:BSOTINLZ} é uma $4$-aproximação para o problema \SbFIR{}.
\end{theorem}
\begin{proof}
Pelo Lema~\ref{lemma:KIVEWTOR}, temos que a sequência fornecida pelo Algoritmo~\ref{algorithm:AKKUXQNR} para a instância intergênica rígida balanceada sem sinais $\mathcal{I'}$, se aplicada no genoma de origem $(\pi,\breve\pi)$ da instância intergênica flexível balanceada sem sinais $\mathcal{I}$, faz com que o genoma alvo seja alcançado. Além disso, note que os problemas \SbIR{} e \SbFIR{} compartilham o mesmo modelo de rearranjo. Logo, a sequência $S'$ utiliza apenas eventos permitidos pelo modelo de rearranjo do problema \SbFIR{}. Pelo Lema~\ref{lemma:RBHACFIP}, temos que $|S'| \le 2ib_1(\mathcal{I'})$. Entretanto, pelo Lema~\ref{lemma:KPGCUTDM}, temos que $ir_i(\mathcal{I}) + ir_a(\mathcal{I}) = ib_1(\mathcal{I'})$. Logo, $|S'| \le 2(ir_i(\mathcal{I}) + ir_a(\mathcal{I}))$. Pelo Teorema~\ref{theorem:KKKUCDHN}, temos o seguinte limitante inferior $df_{\SbFIR}(\mathcal{I}) \ge \frac{ir_i(\mathcal{I}) + ir_a(\mathcal{I})}{2}$, e o teorema segue.
\end{proof}

% ------------------------------------------------------------------ %
\subsubsection{Reversão e Indel}
% ------------------------------------------------------------------ %

Nesta seção, apresentaremos um algoritmo de aproximação com fator $4$ para a variação sem sinais do problema \SbFIRI{}. A seguir apresentamos o Algoritmo~\ref{algorithm:ODSKKWNP}.

\begin{algorithm}[!tbh]
  \caption{Um algoritmo de aproximação para o problema \SbFIRI{}.\label{algorithm:ODSKKWNP}}
  \Entrada{Uma instância intergênica flexível sem sinais $\mathcal{I} = ((\pi,\breve\pi),(\iota,\breve\iota^{\min},\breve\iota^{\max}))$}
  \Saida{Uma sequência de eventos de reversão e indel $S$, tal que $(\pi,\breve\pi) \cdot S$ atinge o genoma alvo de $\mathcal{I}$} 
  $\mathcal{I}' = \mathcal{F}_{ir}^{'}(\mathcal{I})$ \\
  Seja $S'$ uma sequência de eventos de reversão e indel fornecida pelo Algoritmo~\ref{algorithm:LHOPSFVN} para a instância $\mathcal{I}'$ \\
  \Retorna{$S'$}
\end{algorithm}

\begin{theorem}\label{theorem:LXPAWAPW}
Dada uma instância intergênica flexível sem sinais $\mathcal{I}$, o Algoritmo~\ref{algorithm:ODSKKWNP} é uma $4$-aproximação para o problema \SbFIRI{}.
\end{theorem}
\begin{proof}
Pelo Lema~\ref{lemma:SVKOAOXA}, temos que a sequência fornecida pelo Algoritmo~\ref{algorithm:LHOPSFVN} para a instância intergênica rígida sem sinais $\mathcal{I'}$, se aplicada no genoma de origem $(\pi,\breve\pi)$ da instância intergênica flexível sem sinais $\mathcal{I}$, faz com que o genoma alvo seja alcançado. Além disso, note que os problemas \SbIRI{} e \SbFIRI{} compartilham o mesmo modelo de rearranjo. Logo, a sequência $S'$ utiliza apenas eventos permitidos pelo modelo de rearranjo do problema \SbFIRI{}. Pelo Lema~\ref{lemma:XUDIVWPC}, temos que $|S'| \le 2ib_1(\mathcal{I'})$. Entretanto, pelo Lema~\ref{lemma:UFTVNRSX}, temos que $ir_i(\mathcal{I}) = ib_1(\mathcal{I'})$. Logo, $|S'| \le 2ir_i(\mathcal{I})$. Pelo Teorema~\ref{theorem:BOTBXFZQ}, temos o seguinte limitante inferior $df_{\SbFIRI}(\mathcal{I}) \ge \frac{ir_i(\mathcal{I})}{2}$, e o teorema segue.
\end{proof}

% ------------------------------------------------------------------ %
\subsubsection{Reversão e Move}
% ------------------------------------------------------------------ %

Nesta seção, apresentaremos um algoritmo de aproximação com fator $4$ para a variação sem sinais do problema \SbFIRM{}. A seguir apresentamos o Algoritmo~\ref{algorithm:DYDJWEUH}.

\input{algorithms/DYDJWEUH}

\begin{theorem}\label{theorem:MALFMHVQ}
Dada uma instância intergênica flexível balanceada sem sinais $\mathcal{I}$, o Algoritmo~\ref{algorithm:DYDJWEUH} é uma $4$-aproximação para o problema \SbFIRM{}.
\end{theorem}
\begin{proof}
Pelo Lema~\ref{lemma:KIVEWTOR}, temos que a sequência fornecida pelo Algoritmo~\ref{algorithm:OLSRUEFZ} para a instância intergênica rígida balanceada sem sinais $\mathcal{I'}$, se aplicada no genoma de origem $(\pi,\breve\pi)$ da instância intergênica flexível balanceada sem sinais $\mathcal{I}$, faz com que o genoma alvo seja alcançado. Além disso, note que os problemas \SbIRM{} e \SbFIRM{} compartilham o mesmo modelo de rearranjo. Logo, a sequência $S'$ utiliza apenas eventos permitidos pelo modelo de rearranjo do problema \SbFIRM{}. Pelo Lema~\ref{lemma:TZYVWBRT}, temos que $|S'| \le 2ib_1(\mathcal{I'})$. Entretanto, pelo Lema~\ref{lemma:KPGCUTDM}, temos que $ir_i(\mathcal{I}) + ir_a(\mathcal{I}) = ib_1(\mathcal{I'})$. Logo, $|S'| \le 2(ir_i(\mathcal{I}) + ir_a(\mathcal{I}))$. Pelo Teorema~\ref{theorem:KKKUCDHN}, temos o seguinte limitante inferior $df_{\SbFIRM}(\mathcal{I}) \ge \frac{ir_i(\mathcal{I}) + ir_a(\mathcal{I})}{2}$, e o teorema segue.
\end{proof}

% ------------------------------------------------------------------ %
\subsubsection{Reversão, Move e Indel}
% ------------------------------------------------------------------ %

Nesta seção, apresentaremos um algoritmo de aproximação com fator $4$ para a variação sem sinais do problema \SbFIRMI{}. A seguir apresentamos o Algoritmo~\ref{algorithm:MODRXVSQ}.

\begin{algorithm}[!tbh]
  \caption{Um algoritmo de aproximação para o problema \SbFIRI{}.\label{algorithm:MODRXVSQ}}
  \Entrada{Uma instância intergênica flexível sem sinais $\mathcal{I} = ((\pi,\breve\pi),(\iota,\breve\iota^{\min},\breve\iota^{\max}))$}
  \Saida{Uma sequência de eventos de reversão, move e indel $S$, tal que $(\pi,\breve\pi) \cdot S$ atinge o genoma alvo de $\mathcal{I}$} 
  $\mathcal{I}' = \mathcal{F}_{1}^{'}(\mathcal{I})$ \\
  Seja $S'$ uma sequência de eventos de reversão, move e indel fornecida pelo Algoritmo~\ref{algorithm:JAJGNYWD} para a instância $\mathcal{I}'$ \\
  \Retorna{$S'$}
\end{algorithm}

\begin{theorem}\label{theorem:BSLEJJVB}
Dada uma instância intergênica flexível sem sinais $\mathcal{I}$, o Algoritmo~\ref{algorithm:MODRXVSQ} é uma $4$-aproximação para o problema \SbFIRMI{}.
\end{theorem}
\begin{proof}
Pelo Lema~\ref{lemma:SVKOAOXA}, temos que a sequência fornecida pelo Algoritmo~\ref{algorithm:JAJGNYWD} para a instância intergênica rígida sem sinais $\mathcal{I'}$, se aplicada no genoma de origem $(\pi,\breve\pi)$ da instância intergênica flexível sem sinais $\mathcal{I}$, faz com que o genoma alvo seja alcançado. Além disso, note que os problemas \SbIRMI{} e \SbFIRMI{} compartilham o mesmo modelo de rearranjo. Logo, a sequência $S'$ utiliza apenas eventos permitidos pelo modelo de rearranjo do problema \SbFIRMI{}. Pelo Lema~\ref{lemma:SINGKSVU}, temos que $|S'| \le 2ib_1(\mathcal{I'})$. Entretanto, pelo Lema~\ref{lemma:UFTVNRSX}, temos que $ir_i(\mathcal{I}) = ib_1(\mathcal{I'})$. Logo, $|S'| \le 2ir_i(\mathcal{I})$. Pelo Teorema~\ref{theorem:BOTBXFZQ}, temos o seguinte limitante inferior $df_{\SbFIRMI}(\mathcal{I}) \ge \frac{ir_i(\mathcal{I})}{2}$, e o teorema segue.
\end{proof}

% ------------------------------------------------------------------ %
\subsubsection{Reversão e Transposição}
% ------------------------------------------------------------------ %

Nesta seção, apresentaremos um algoritmo de aproximação com fator $4$ para a variação sem sinais do problema \SbFIRT{}. A seguir apresentamos o Algoritmo~\ref{algorithm:KZFLZWRM}.

\input{algorithms/KZFLZWRM}

\begin{theorem}\label{theorem:DSCDQRUP}
Dada uma instância intergênica flexível balanceada sem sinais $\mathcal{I}$, o Algoritmo~\ref{algorithm:KZFLZWRM} é uma $4$-aproximação para o problema \SbFIRT{}.
\end{theorem}
\begin{proof}
Pelo Lema~\ref{lemma:KIVEWTOR}, temos que a sequência fornecida pelo Algoritmo~\ref{algorithm:LCPCUFNZ} para a instância intergênica rígida balanceada sem sinais $\mathcal{I'}$, se aplicada no genoma de origem $(\pi,\breve\pi)$ da instância intergênica flexível balanceada sem sinais $\mathcal{I}$, faz com que o genoma alvo seja alcançado. Além disso, note que os problemas \SbIRT{} e \SbFIRT{} compartilham o mesmo modelo de rearranjo. Logo, a sequência $S'$ utiliza apenas eventos permitidos pelo modelo de rearranjo do problema \SbFIRT{}. Pelo Lema~\ref{lemma:HIIRAXUH}, temos que $|S'| \le \frac{4ib_1(\mathcal{I}')}{3}$. Entretanto, pelo Lema~\ref{lemma:KPGCUTDM}, temos que $ir_i(\mathcal{I}) + ir_a(\mathcal{I}) = ib_1(\mathcal{I'})$. Logo, $|S'| \le \frac{4(ir_i(\mathcal{I}) + ir_a(\mathcal{I}))}{3}$. Pelo Teorema~\ref{theorem:KKKUCDHN}, temos o seguinte limitante inferior $df_{\SbFIRT}(\mathcal{I}) \ge \frac{ir_i(\mathcal{I}) + ir_a(\mathcal{I})}{3}$, e o teorema segue.
\end{proof}

% ------------------------------------------------------------------ %
\subsubsection{Reversão, Transposição e Indel}
% ------------------------------------------------------------------ %

Nesta seção, apresentaremos um algoritmo de aproximação assintótica para a variação sem sinais do problema \SbFIRTI{}. A seguir apresentamos o Algoritmo~\ref{algorithm:JSNLHIVA}.

\begin{algorithm}[!tbh]
  \caption{Um algoritmo de aproximação para o problema \SbFIRTI{}.\label{algorithm:JSNLHIVA}}
  \Entrada{Uma instância intergênica flexível sem sinais $\mathcal{I} = ((\pi,\breve\pi),(\iota,\breve\iota^{\min},\breve\iota^{\max}))$}
  \Saida{Uma sequência de eventos de reversão, transposição e indel $S$, tal que $(\pi,\breve\pi) \cdot S$ atinge o genoma alvo de $\mathcal{I}$} 
  $\mathcal{I}' = \mathcal{F}_{ir}^{'}(\mathcal{I})$ \\
  Seja $S'$ uma sequência de eventos de reversão, transposição e indel fornecida pelo Algoritmo~\ref{algorithm:YIZYUGZZ} para a instância $\mathcal{I}'$ \\
  \Retorna{$S'$}
\end{algorithm}

\begin{theorem}\label{theorem:BBTWMULM}
Dada uma instância intergênica flexível sem sinais $\mathcal{I}$, o Algoritmo~\ref{algorithm:JSNLHIVA} é uma $4$-aproximação assintótica para o problema \SbFIRTI{}.
\end{theorem}
\begin{proof}
Pelo Lema~\ref{lemma:SVKOAOXA}, temos que a sequência fornecida pelo Algoritmo~\ref{algorithm:YIZYUGZZ} para a instância intergênica rígida sem sinais $\mathcal{I'}$, se aplicada no genoma de origem $(\pi,\breve\pi)$ da instância intergênica flexível sem sinais $\mathcal{I}$, faz com que o genoma alvo seja alcançado. Além disso, note que os problemas \SbIRTI{} e \SbFIRTI{} compartilham o mesmo modelo de rearranjo. Logo, a sequência $S'$ utiliza apenas eventos permitidos pelo modelo de rearranjo do problema \SbFIRTI{}. Pelo Lema~\ref{lemma:MUTXDAUG}, temos que $|S'| \le \frac{4ib_1(\mathcal{I}')}{3} + 1$. Entretanto, pelo Lema~\ref{lemma:UFTVNRSX}, temos que $ir_i(\mathcal{I}) = ib_1(\mathcal{I'})$. Logo, $|S'| \le \frac{4ir_i(\mathcal{I})}{3} + 1$. Pelo Teorema~\ref{theorem:BOTBXFZQ}, temos o seguinte limitante inferior $df_{\SbFIRTI}(\mathcal{I}) \ge \frac{ir_i(\mathcal{I})}{3}$. Com isso, temos que, no pior caso, o fator de aproximação do Algoritmo~\ref{algorithm:JSNLHIVA} é de $4df_{\SbFIRTI}(\mathcal{I}) + 1$, e o teorema segue.
\end{proof}

% ------------------------------------------------------------------ %
\subsubsection{Reversão, Transposição e Move}
% ------------------------------------------------------------------ %

Nesta seção, apresentaremos um algoritmo de aproximação com fator $3$ para a variação sem sinais do problema \SbFIRTM{}. A seguir apresentamos o Algoritmo~\ref{algorithm:JRHFSYXO}.

\begin{algorithm}[!tbh]
  \caption{Um algoritmo de aproximação para o problema \SbFIRTM{}.\label{algorithm:JRHFSYXO}}
  \Entrada{Uma instância intergênica flexível balanceada sem sinais $\mathcal{I} = ((\pi,\breve\pi),(\iota,\breve\iota^{\min},\breve\iota^{\max}))$}
  \Saida{Uma sequência de eventos de reversão, transposição e move $S$, tal que $(\pi,\breve\pi) \cdot S$ atinge o genoma alvo de $\mathcal{I}$} 
  $\mathcal{I}' = \mathcal{F}_{1}^{''}(\mathcal{I})$ \\
  Seja $S'$ uma sequência de eventos de reversão, transposição e move fornecida pelo Algoritmo~\ref{algorithm:UZWADMNZ} para a instância $\mathcal{I}'$ \\
  \Retorna{$S'$}
\end{algorithm}

\begin{theorem}\label{theorem:YTAKVOTU}
Dada uma instância intergênica flexível balanceada sem sinais $\mathcal{I}$, o Algoritmo~\ref{algorithm:JRHFSYXO} é uma $3$-aproximação para o problema \SbFIRTM{}.
\end{theorem}
\begin{proof}
Pelo Lema~\ref{lemma:KIVEWTOR}, temos que a sequência fornecida pelo Algoritmo~\ref{algorithm:UZWADMNZ} para a instância intergênica rígida balanceada sem sinais $\mathcal{I'}$, se aplicada no genoma de origem $(\pi,\breve\pi)$ da instância intergênica flexível balanceada sem sinais $\mathcal{I}$, faz com que o genoma alvo seja alcançado. Além disso, note que os problemas \SbIRTM{} e \SbFIRTM{} compartilham o mesmo modelo de rearranjo. Logo, a sequência $S'$ utiliza apenas eventos permitidos pelo modelo de rearranjo do problema \SbFIRTM{}. Pelo Lema~\ref{lemma:UUWLBHHA}, temos que $|S'| \le ib_1(\mathcal{I}')$. Entretanto, pelo Lema~\ref{lemma:KPGCUTDM}, temos que $ir_i(\mathcal{I}) + ir_a(\mathcal{I}) = ib_1(\mathcal{I'})$. Logo, $|S'| \le ir_i(\mathcal{I}) + ir_a(\mathcal{I})$. Pelo Teorema~\ref{theorem:KKKUCDHN}, temos o seguinte limitante inferior $df_{\SbFIRTM}(\mathcal{I}) \ge \frac{ir_i(\mathcal{I}) + ir_a(\mathcal{I})}{3}$, e o teorema segue.
\end{proof}

% ------------------------------------------------------------------ %
\subsubsection{Reversão, Transposição, Move e Indel}
% ------------------------------------------------------------------ %

Nesta seção, apresentaremos um algoritmo de aproximação com fator $3$ para a variação sem sinais do problema \SbFIRTMI{}. A seguir apresentamos o Algoritmo~\ref{algorithm:PJTWIANQ}.

\input{algorithms/PJTWIANQ}

\begin{theorem}\label{theorem:TYVMEDAI}
Dada uma instância intergênica flexível sem sinais $\mathcal{I}$, o Algoritmo~\ref{algorithm:PJTWIANQ} é uma $3$-aproximação para o problema \SbFIRTMI{}.
\end{theorem}
\begin{proof}
Pelo Lema~\ref{lemma:SVKOAOXA}, temos que a sequência fornecida pelo Algoritmo~\ref{algorithm:FMDPGQTJ} para a instância intergênica rígida sem sinais $\mathcal{I'}$, se aplicada no genoma de origem $(\pi,\breve\pi)$ da instância intergênica flexível sem sinais $\mathcal{I}$, faz com que o genoma alvo seja alcançado. Além disso, note que os problemas \SbIRTMI{} e \SbFIRTMI{} compartilham o mesmo modelo de rearranjo. Logo, a sequência $S'$ utiliza apenas eventos permitidos pelo modelo de rearranjo do problema \SbFIRTMI{}. Pelo Lema~\ref{lemma:GCEWGEBP}, temos que $|S'| \le ib_1(\mathcal{I}')$. Entretanto, pelo Lema~\ref{lemma:UFTVNRSX}, temos que $ir_i(\mathcal{I}) = ib_1(\mathcal{I'})$. Logo, $|S'| \le ir_i(\mathcal{I})$. Pelo Teorema~\ref{theorem:BOTBXFZQ}, temos o seguinte limitante inferior $df_{\SbFIRTMI}(\mathcal{I}) \ge \frac{ir_i(\mathcal{I})}{3}$, e o teorema segue.
\end{proof}

% ------------------------------------------------------------------ %
\subsubsection{Transposição}
% ------------------------------------------------------------------ %

Nesta seção, apresentaremos um algoritmo de aproximação com fator $3.5$ para a variação sem sinais do problema \SbFIT{}. 

Note que a versão rígida do problema \SbFIT{}, chamado de Ordenação de Permutações por Transposições Intergênicas (\SbIT), possui um algoritmo de aproximação com um fator de $3.5$, que chamaremos de $3.5$-\SbIT{}. Além disso, temos o seguinte lema.

\begin{lemma}\label{lemma:EIGSYNDP}
Seja $\mathcal{I}' = ((\pi',\breve\pi'),(\iota',\breve\iota'))$ uma instância intergênica rígida balanceada sem sinais, o algoritmo $3.5$-\SbIT{} transforma $(\pi',\breve\pi')$ em $(\iota',\breve\iota')$ utilizando uma sequência de eventos de transposição $S'$, tal que $|S'| \le \frac{7({n+1} - c_b(G(\mathcal{I}')))}{4}$.
\end{lemma}
\begin{proof}
Diretamente pelo Lema 5.1 de Oliveira \textit{et al.}~\cite{2021a-oliveira-etal}.
\end{proof}

A seguir apresentamos o Algoritmo~\ref{algorithm:UMNIXZHY}.

\begin{algorithm}[!tbh]
  \caption{Um algoritmo de aproximação para o problema \SbFIT{}.\label{algorithm:UMNIXZHY}}
  \Entrada{Uma instância intergênica flexível balanceada sem sinais $\mathcal{I} = ((\pi,\breve\pi),(\iota,\breve\iota^{\min},\breve\iota^{\max}))$}
  \Saida{Uma sequência de eventos de transposição $S$, tal que $(\pi,\breve\pi) \cdot S$ atinge o genoma alvo de $\mathcal{I}$} 
  $\mathcal{I}' = \mathcal{F}_{c}^{''}(\mathcal{I})$ \\
  Seja $S'$ uma sequência de eventos de transposição fornecida pelo algoritmo $3.5$-\SbIT{} para a instância $\mathcal{I}'$ \\
  \Retorna{$S'$}
\end{algorithm}

\begin{theorem}\label{theorem:PSELGHNY}
Dada uma instância intergênica flexível balanceada sem sinais $\mathcal{I}$, o Algoritmo~\ref{algorithm:UMNIXZHY} é uma $3.5$-aproximação para o problema \SbFIT{}.
\end{theorem}
\begin{proof}
Pelo Lema~\ref{lemma:WQOEFBXP}, temos que a sequência fornecida pelo algoritmo $3.5$-\SbIT{} para a instância intergênica rígida balanceada sem sinais $\mathcal{I'}$, se aplicada no genoma de origem $(\pi,\breve\pi)$ da instância intergênica flexível balanceada sem sinais $\mathcal{I}$, faz com que o genoma alvo seja alcançado. Além disso, note que os problemas \SbIT{} e \SbFIT{} compartilham o mesmo modelo de rearranjo. Logo, a sequência $S'$ utiliza apenas eventos permitidos pelo modelo de rearranjo do problema \SbFIT{}. Pelo Lema~\ref{lemma:EIGSYNDP}, temos que $|S'| \le \frac{7({n+1} - c_b(G(\mathcal{I}')))}{4}$. Entretanto, pelo Lema~\ref{lemma:PSGXFVHD}, temos que $c_d(G(\mathcal{I})) = c_b(G(\mathcal{I}'))$. Logo, $|S'| \le \frac{7({n+1} - c_d(G(\mathcal{I})))}{4}$. Pelo Teorema~\ref{theorem:PQQUYBMS}, temos o seguinte limitante inferior $df_{\SbFIT}(\mathcal{I}) \ge \frac{{n+1} - c_d(G(\mathcal{I}))}{2}$, e o teorema segue.
\end{proof}

% ------------------------------------------------------------------ %
\subsubsection{Transposição e Move}
% ------------------------------------------------------------------ %

Nesta seção, apresentaremos um algoritmo de aproximação com fator $2.5$ para a variação sem sinais do problema \SbFITM{}. 

Note que a versão rígida do problema \SbFITM{}, chamado de Ordenação de Permutações por Operações Intergênicas de Transposição e Move (\SbITM), possui um algoritmo de aproximação com um fator de $2.5$, que chamaremos de $2.5$-\SbITM{}. Além disso, temos o seguinte lema.

\begin{lemma}\label{lemma:PDDYJXYT}
Seja $\mathcal{I}' = ((\pi',\breve\pi'),(\iota',\breve\iota'))$ uma instância intergênica rígida balanceada sem sinais, o algoritmo $2.5$-\SbITM{} transforma $(\pi',\breve\pi')$ em $(\iota',\breve\iota')$ utilizando uma sequência de eventos de transposição e move $S'$, tal que $|S'| \le \frac{5({n+1} - c_b(G(\mathcal{I})))}{4}$.
\end{lemma}
\begin{proof}
Diretamente pelo Lema 7.10 de Oliveira \textit{et al.}~\cite{2021a-oliveira-etal}.
\end{proof}

A seguir apresentamos o Algoritmo~\ref{algorithm:UEBBPCAK}.

\begin{algorithm}[!tbh]
  \caption{Um algoritmo de aproximação para o problema \SbFITM{}.\label{algorithm:UEBBPCAK}}
  \Entrada{Uma instância intergênica flexível balanceada sem sinais $\mathcal{I} = ((\pi,\breve\pi),(\iota,\breve\iota^{\min},\breve\iota^{\max}))$}
  \Saida{Uma sequência de eventos de transposição e move $S$, tal que $(\pi,\breve\pi) \cdot S$ atinge o genoma alvo de $\mathcal{I}$} 
  $\mathcal{I}' = \mathcal{F}_{c}^{''}(\mathcal{I})$ \\
  Seja $S'$ uma sequência de eventos de transposição e move fornecida pelo algoritmo $2.5$-\SbITM{} para a instância $\mathcal{I}'$ \\
  \Retorna{$S'$}
\end{algorithm}

\begin{theorem}\label{theorem:DWYTBIPX}
Dada uma instância intergênica flexível balanceada sem sinais $\mathcal{I}$, o Algoritmo~\ref{algorithm:UEBBPCAK} é uma $2.5$-aproximação para o problema \SbFIT{}.
\end{theorem}
\begin{proof}
Pelo Lema~\ref{lemma:WQOEFBXP}, temos que a sequência fornecida pelo algoritmo $2.5$-\SbIT{} para a instância intergênica rígida balanceada sem sinais $\mathcal{I'}$, se aplicada no genoma de origem $(\pi,\breve\pi)$ da instância intergênica flexível balanceada sem sinais $\mathcal{I}$, faz com que o genoma alvo seja alcançado. Além disso, note que os problemas \SbITM{} e \SbFITM{} compartilham o mesmo modelo de rearranjo. Logo, a sequência $S'$ utiliza apenas eventos permitidos pelo modelo de rearranjo do problema \SbFITM{}. Pelo Lema~\ref{lemma:PDDYJXYT}, temos que $|S'| \le \frac{5({n+1} - c_b(G(\mathcal{I}')))}{4}$. Entretanto, pelo Lema~\ref{lemma:PSGXFVHD}, temos que $c_d(G(\mathcal{I})) = c_b(G(\mathcal{I}'))$. Logo, $|S'| \le \frac{5({n+1} - c_d(G(\mathcal{I})))}{4}$. Pelo Teorema~\ref{theorem:PQQUYBMS}, temos o seguinte limitante inferior $df_{\SbFITM}(\mathcal{I}) \ge \frac{{n+1} - c_d(G(\mathcal{I}))}{2}$, e o teorema segue.
\end{proof}

% ------------------------------------------------------------------ %
\subsubsection{Resultados Práticos}\label{subsubsection:PWLZZAVH}
% ------------------------------------------------------------------ %

Nesta seção, apresentamos os resultados práticos dos algoritmos apresentados para a variação sem sinas dos problemas \SbFIR{}, \SbFIRI{}, \SbFIRM{}, \SbFIRMI{}, \SbFIRT{}, \SbFIRTI{}, \SbFIRTM{}, \SbFIRTMI{}, \SbFIT{} e \SbFITM{}.

Nós criamos uma base de dados para cada problema e utilizamos os identificadores $U_\SbFIR{}$, $U_\SbFIRI{}$, $U_\SbFIRM{}$, $U_\SbFIRMI{}$, $U_\SbFIRT{}$, $U_\SbFIRTI{}$, $U_\SbFIRTM{}$, $U_\SbFIRTMI{}$, $U_\SbFIT{}$ e $U_\SbFITM{}$ para a base de dados dos problemas \SbFIR{}, \SbFIRI{}, \SbFIRM{}, \SbFIRMI{}, \SbFIRT{}, \SbFIRTI{}, \SbFIRTM{}, \SbFIRTMI{}, \SbFIT{} e \SbFITM{}, respectivamente. Cada base de dados é dividida em cinco grupos. Cada grupo possui 1000 instâncias do tamanho 100, sendo que o tamanho de uma instância é a quantidade de genes do genoma de origem e alvo. Além disso, cada grupo é identificado pelo grau de flexibilização das instâncias contidas nele. Os identificadores do grupos de cada base de dados são 10\%, 20\%, 30\%, 40\% e 50\%. Cada instância é gerada da seguinte forma: Seja $\mathcal{S} = (\pi =(1~2~\dots~100),\breve\pi)$ uma representação intergênica rígida sem sinais de um genoma de origem, de forma que o tamanho de cada região intergênica $\breve\pi_i$ foi escolhido de maneira aleatória no intervalo $[0..100]$. Em seguida, criamos uma representação intergênica flexível sem sinais do genona alvo $\mathcal{T} = (\iota, \breve\iota^{\min},\breve\iota^{\max})$ da seguinte forma: (i) $\iota =(1~2~\dots~100)$; (ii) Seja $f$ o grau de flexibilização adotado no grupo, temos que $\forall \breve\iota^{\min}_i \in \breve\iota^{\min}: \breve\iota^{\min}_i = \lfloor\breve\pi_i - f\rfloor$ e $\forall \breve\iota^{\max}_i \in \breve\iota^{\max} : \breve\iota^{\max}_i = \lceil\breve\pi_i + f\rceil$. Com base na disponibilidade de operações de reversão, transposição, move e indel determinada para cada base de dados e grupo, uma operação $\sigma$ é escolhida de maneira aleatória e aplicada em $\mathcal{S}$ ($\mathcal{S} = \mathcal{S} \cdot \sigma$). Os parâmetros de cada operação também são escolhidos de forma aleatória dentro do limite de valores válidos. O valor do parâmetro $x$ de um indel $\delta^{(i)}_{(x)}$ aplicado em uma região intergênica $\breve\pi_{i}$ é escolhido dentro do intervalo $[-\breve\pi_{i}..\breve\pi_{i}]$, também de maneira aleatória. Quando não houverem operações disponíveis para serem aplicadas, então temos a instância intergênica flexível sem sinais $\mathcal{I}$, que é composta pela dupla $(\mathcal{S},\mathcal{T})$. Esse processo repete-se até que cada grupo possua um total de 1000 instâncias. 

A quantidade de operações disponíveis para gerar cada instância difere entre as bases de dados. A Tabela~\ref{table:YUYKVZOZ} mostra, para cada base de dados, a quantidade de operações utilizada para criar cada instância.

\begin{table}[!htb]
  \caption{Quantidade de operações aplicadas para gerar cada instância intergênica flexível sem sinais.}
  \label{table:YUYKVZOZ}
  \centering
  \begin{tabular}{|p{3cm}|r|r|r|r|}
    \hline
    Base de Dados           & Revesões   & Transposições   & Moves   & Indels    \\ \hline
    $U_\SbFIR{}$            & 50         & 0               &  0      &  0       \\ \hline
    $U_\SbFIRI{}$           & 40         & 0               &  0      & 10       \\ \hline
    $U_\SbFIRM{}$           & 40         & 0               & 10      &  0       \\ \hline
    $U_\SbFIRMI{}$          & 40         & 0               &  5      &  5       \\ \hline
    $U_\SbFIRT{}$           & 25         & 25              &  0      &  0       \\ \hline
    $U_\SbFIRTI{}$          & 20         & 20              &  0      & 10       \\ \hline
    $U_\SbFIRTM{}$          & 20         & 20              & 10      &  0       \\ \hline
    $U_\SbFIRTMI{}$         & 20         & 20              &  5      &  5       \\ \hline
    $U_\SbFIT{}$            & 0          & 50              &  0      &  0       \\ \hline
    $U_\SbFITM{}$           & 0          & 40              & 10      &  0       \\ \hline
  \end{tabular}
\end{table}

Utilizando o conceito de regiões intergênicas e considerando todos os grupos das bases de dados $U_\SbFIR{}$, $U_\SbFIRM{}$, $U_\SbFIRT{}$ e $U_\SbFIRTM{}$, foi observado que 100\% das instâncias pertencem ao cenário de equilíbrio. As instâncias foram geradas com o objetivo de que uma quantidade considerável de operações fosse necessária para fazer com que o genoma de origem atinja o genoma alvo, então poucas regiões intergênicas estáveis tendem a ser mantidas. Isso pode explicar o fato de 100\% das instâncias pertencerem ao cenário de equilíbrio.

Utilizando a estrutura de grafo de ciclos ponderado flexível e considerando todos os grupos das bases de dados $U_\SbFIT{}$ e $U_\SbFITM{}$, foi observado que 55.8\%, 25.8\% e 18.4\% das instâncias pertencem ao cenário de equilíbrio, sourvedouro e fonte, respectivamente.

Para garantir uma proporcionalidade entre os possíveis cenários nos problemas que utilizam um modelo de rearranjo composto exclusivamente por eventos conservativos nós criamos as bases de dados $U_{IR}$ e $U_{C}$. Ambas as bases de dados possuem cinco grupos, sendo que cada grupo possui 3000 instâncias intergênicas flexíveis balanceadas sem sinais de tamanho 100 e é identificado pelo grau de flexibilização máxima das instâncias contidas nele. Os identificadores do grupos são 10\%, 20\%, 30\%, 40\% e 50\%. Utilizando o conceito de regiões intergênicas, cada grupo da base de dados $U_{IR}$ possui 1000 instâncias no cenário de equilíbrio, 1000 instâncias no cenário fonte e 1000 instâncias no cenário sorvedouro. Já na base de dados $U_{C}$, utilizando a estrutura de grafo de ciclos ponderado flexível, cada grupo possui 1000 instâncias no cenário de equilíbrio, 1000 instâncias no cenário fonte e 1000 instâncias no cenário sorvedouro.

A geração de uma instância na base dados $U_{IR}$ é dada da seguinte forma: Seja $\mathcal{S} = (\pi =(1~2~\dots~100),\breve\pi)$ uma representação intergênica rígida sem sinais de um genoma de origem, de forma que o tamanho de cada região intergênica $\breve\pi_i$ foi escolhido de maneira aleatória no intervalo $[0..100]$. Em seguida, criamos uma representação intergênica flexível sem sinais do genona alvo $\mathcal{T} = (\iota, \breve\iota^{\min},\breve\iota^{\max})$ da seguinte forma: (i) $\iota =(1~2~\dots~100)$; (ii) Seja $f$ o grau de flexibilização máxima adotado no grupo. Para cada valor de $i \in \{1,2,\dots,101\}$, temos que $l$ e $u$ são porcentagens escolhidas de maneira aleatória no conjunto $\{0\%,1\%,\dots,f\}$ e os valores de $\breve\iota^{\min}_i$ e $\breve\iota^{\max}_i$ são atualizados para $\lfloor\breve\pi_i - l\rfloor$ e $\lceil\breve\pi_i + u\rceil$, respectivamente. Em seguida, 15 trocas são aplicadas em $\pi$. Uma troca muda a posição de dois elementos de $\pi$, sendo que ambas as posições são escolhidas do forma aleatória. Por fim, com base na disponibilidade de cenários para serem adicionados ao grupo, um cenário é escolhido de maneira aleatória e o seguinte processamento é realizado:
\begin{itemize}
  \item Equilíbrio - Caso a instância intergênica flexível sem sinais $\mathcal{I} = (\mathcal{S},\mathcal{T})$ pertença ao cenário de equilíbrio com base no conceito de regiões intergênicas, então $\mathcal{I}$ é adicionada ao grupo.
  \item Fonte - Neste caso $\lfloor\frac{\sum_{i = 1}^{101}\breve\pi_i - \sum_{i = 1}^{101}\breve\iota^{\min}_i}{2}\rfloor$ nucleotídeos são removidos das regiões intergênicas $\breve\pi$ de forma aleatória. Caso a instância intergênica flexível sem sinais $\mathcal{I} = (\mathcal{S},\mathcal{T})$ resultante pertença ao cenário fonte com base no conceito de regiões intergênicas, então $\mathcal{I}$ é adicionada ao grupo.
  \item Sorvedouro - Neste caso $\lfloor\frac{\sum_{i = 1}^{101}\breve\iota^{\max}_i - \sum_{i = 1}^{101}\breve\pi_i}{2}\rfloor$ nucleotídeos são adicionados nas regiões intergênicas $\breve\pi$ de forma aleatória. Caso a instância intergênica flexível sem sinais $\mathcal{I} = (\mathcal{S},\mathcal{T})$ resultante pertença ao cenário sorvedouro com base no conceito de regiões intergênicas, então $\mathcal{I}$ é adicionada ao grupo.
\end{itemize}
Este processo repete-se até que cada grupo possua 3000 instâncias.

A criação de uma instância na base dados $U_{C}$ é similar ao processo descrito na base de dados $U_{IR}$ direfenciando-se pelo fato de utilizar a estrutura de grafo de ciclos ponderado flexível para determinar o cenário de cada instância que é gerada. Além disso, caso o cenário fonte seja escolhido durante o processo de geração de uma instância, o peso $\lfloor\frac{\sum_{i = 1}^{101}\breve\pi_i - \sum_{i = 1}^{101}\breve\iota^{\min}_i}{2}\rfloor$ é removido das arestas pretas do grafo, também de forma aleatória. Caso o cenário sorvedouro seja escolhido durante o processo de geração de uma instância, o peso $\lfloor\frac{\sum_{i = 1}^{101}\breve\iota^{\max}_i - \sum_{i = 1}^{101}\breve\pi_i}{2}\rfloor$ é adicionado nas arestas pretas do grafo, também de forma aleatória. 

A base de dados $U_{IR}$ foi criada para ser utilizada pelos algoritmos da variação sem sinais dos problemas $\SbFIR{}$, $\SbFIRM{}$, $\SbFIRT{}$ e $\SbFIRTM{}$. Similarmente, a base de dados $U_{C}$ foi criada para ser utilizada pelos algoritmos da variação sem sinais dos problemas $\SbFIT{}$ e $\SbFITM{}$.

A seguir apresentamos os resultados obtidos pelos algoritmos apresentado para a variação sem sinais dos problemas investigados neste capítulo. Nas tabelas que serão utilizadas a seguir temos a informação por grupo do grau de flexibilização adotado e as métricas de distância e aproximação, sendo que para ambas as métricas temos a informação sobre o menor e maior valor registrado e a média obtida.

A Tabela~\ref{table:APQPJYRX} apresenta os resultados do Algoritmo~\ref{algorithm:BSOTINLZ} utilizando as bases de dados $U_\SbFIR{}$ e $U_{\text{IR}}$. A razão de aproximação obtida pelo algoritmo para cada instância foi computada utilizando o limitante inferior apresentado no Teorema~\ref{theorem:KKKUCDHN}.

\begin{table}[!htb]
  \caption{Resultados do Algoritmo~\ref{algorithm:BSOTINLZ} utilizando as bases de dados $U_\SbFIR{}$ e $U_{\text{IR}}$.}
  \label{table:APQPJYRX}
  \centering
  \begin{tabular}{|c|r|r|r|r|r|r|}
    \hline
    \multicolumn{7}{|c|}{$U_\SbFIR{}$}                                                                       \\ \hline
      -            & \multicolumn{3}{c|}{Distância}             & \multicolumn{3}{c|}{Aproximação}           \\ \hline
    Flexibilização & Mínimo       & Média        & Máximo       & Mínimo       & Média        & Máximo       \\ \hline  
    10\%           & 69           & 89.14        & 109          & 2.29         & 2.78         & 3.23         \\ \hline
    20\%           & 72           & 90.52        & 110          & 2.33         & 2.83         & 3.27         \\ \hline
    30\%           & 75           & 92.27        & 119          & 2.44         & 2.89         & 3.31         \\ \hline
    40\%           & 76           & 93.80        & 113          & 2.55         & 2.94         & 3.31         \\ \hline
    50\%           & 74           & 95.03        & 116          & 2.39         & 2.97         & 3.34         \\ \hline    
  \end{tabular}

  \vspace{5mm}

  \begin{tabular}{|c|r|r|r|r|r|r|}
    \hline
    \multicolumn{7}{|c|}{$U_{\text{IR}}$}                                                                    \\ \hline
      -            & \multicolumn{3}{c|}{Distância}             & \multicolumn{3}{c|}{Aproximação}           \\ \hline
    Flexibilização & Mínimo       & Média        & Máximo       & Mínimo       & Média        & Máximo       \\ \hline  
    10\%           & 44           & 65.97        & 85           & 2.20         & 2.71         & 3.16         \\ \hline
    20\%           & 44           & 67.31        & 86           & 2.27         & 2.74         & 3.16         \\ \hline
    30\%           & 39           & 68.68        & 89           & 2.19         & 2.77         & 3.25         \\ \hline
    40\%           & 42           & 69.86        & 93           & 2.29         & 2.79         & 3.23         \\ \hline
    50\%           & 40           & 70.65        & 92           & 2.26         & 2.81         & 3.32         \\ \hline    
  \end{tabular}
\end{table}


% \begin{table}[!htb]
%   \caption{Resultados do Algoritmo~\ref{algorithm:BSOTINLZ} utilizando a base de dados $U_{\text{IR}}$.}
%   \label{table:OJKJSYXE}
%   \centering
%   \begin{tabular}{|c|r|r|r|r|r|r|}
%     \hline
%       -            & \multicolumn{3}{c|}{Distância}             & \multicolumn{3}{c|}{Aproximação}           \\ \hline
%     Flexibilização & Mínimo       & Média        & Máximo       & Mínimo       & Média        & Máximo       \\ \hline  
%     10\%           & 41           & 75.23        & 100          & 2.14         & 2.81         & 3.24         \\ \hline
%     20\%           & 43           & 75.67        & 104          & 2.25         & 2.84         & 3.30         \\ \hline
%     30\%           & 39           & 76.07        & 105          & 2.26         & 2.86         & 3.30         \\ \hline
%     40\%           & 37           & 76.59        & 101          & 2.30         & 2.89         & 3.42         \\ \hline
%     50\%           & 44           & 77.41        & 102          & 2.41         & 2.91         & 3.57         \\ \hline    
%   \end{tabular}
% \end{table}

% \begin{table}[!htb]
  \caption{Resultados do Algoritmo~\ref{algorithm:BSOTINLZ} utilizando a base de dados $U_{\text{cases}}$.}
  \label{table:OJKJSYXE}
  \centering
  \begin{tabular}{|c|r|r|r|r|r|r|}
    \hline
      -      & \multicolumn{3}{c|}{Distância}             & \multicolumn{3}{c|}{Aproximação}           \\ \hline
    Grupo    & Mínimo       & Média        & Máximo       & Mínimo       & Média        & Máximo       \\ \hline  
    100      &  42          &  91.56       & 120          & 2.22         & 2.93         & 3.36         \\ \hline
    200      &  92          & 182.34       & 233          & 2.34         & 2.95         & 3.28         \\ \hline
    300      & 148          & 273.22       & 345          & 2.42         & 2.95         & 3.29         \\ \hline
    400      & 196          & 364.08       & 458          & 2.39         & 2.95         & 3.23         \\ \hline
    500      & 255          & 454.25       & 562          & 2.47         & 2.95         & 3.25         \\ \hline    
  \end{tabular}
\end{table}

Pela Tabela~\ref{table:APQPJYRX} é possível perceber que o algoritmo~\ref{algorithm:BSOTINLZ} na base dados $U_\SbFIR{}$ acabou aplicando mais operações de reversão a medida que o grau de flexibilização aumenta. Este fato pode ser constatado ao verificar as métricas de distância média e aproximação média. Este comportamento também pode ser observado na base de dados $U_{\text{IR}}$, mas vale ressaltar que esta base de dados não foi construída com base em eventos de rearranjo e o grau de flexibilização para o tamanho mínimo e máximo de cada região intergênica no genoma alvo não é fixo. Considerando ambas as bases de dados a aproximação máxima e mínima resgistrada foi de $3.34$ e $2.19$, respectivamente.

A Tabela~\ref{table:EZSBDOGH} apresenta os resultados do Algoritmo~\ref{algorithm:ODSKKWNP} utilizando a base de dados $U_\SbFIRI{}$. A razão de aproximação obtida pelo algoritmo para cada instância foi computada utilizando o limitante inferior apresentado no Teorema~\ref{theorem:BOTBXFZQ}.

\begin{table}[!htb]
  \caption{Resultados do Algoritmo~\ref{algorithm:ODSKKWNP} utilizando a base de dados $U_\SbFIRI{}$.}
  \label{table:EZSBDOGH}
  \centering
  \begin{tabular}{|c|r|r|r|r|r|r|r|}
    \hline
      -      &  -   & \multicolumn{3}{c|}{Distância}             & \multicolumn{3}{c|}{Aproximação}           \\ \hline
    Grupo    & OP   & Mínimo       & Média        & Máximo       & Mínimo       & Média        & Máximo       \\ \hline  
    100      & 50   & 64           &  83.18       & 104          & 2.37         & 2.78         & 3.32         \\ \hline
    200      & 100  & 139          & 164.71       & 189          & 2.50         & 2.78         & 3.07         \\ \hline
    300      & 150  & 206          & 245.43       & 276          & 2.51         & 2.77         & 3.01         \\ \hline
    400      & 200  & 294          & 326.74       & 364          & 2.58         & 2.77         & 3.04         \\ \hline
    500      & 250  & 370          & 407.42       & 453          & 2.61         & 2.76         & 2.99         \\ \hline    
  \end{tabular}
\end{table}

Na Tabela~\ref{table:EZSBDOGH} é possível perceber que o Algoritmo~\ref{algorithm:ODSKKWNP} tende a utilizar menos operações na média a medida que o grau de flexibilização aumenta. Além disso, a aproximação média também apresentou uma tendência de queda a medida que o grau de flexibilização aumenta. Podemos notar também que a distância mínima para cada grupo ficou dentro do intervalo $[61..67]$, sendo que cada instância da base de dados foi gerada a partir de 50 operações de reversão ou indel. Entretanto, para todos os grupos a aproximação média foi menor que $2.80$.

A Tabela~\ref{table:IEBGYPHS} apresenta os resultados do Algoritmo~\ref{algorithm:DYDJWEUH} utilizando as bases de dados $U_\SbFIRM{}$ e $U_{\text{IR}}$. A razão de aproximação obtida pelo algoritmo para cada instância foi computada utilizando o limitante inferior apresentado no Teorema~\ref{theorem:KKKUCDHN}.

\begin{table}[!htb]
  \caption{Resultados do Algoritmo~\ref{algorithm:DYDJWEUH} utilizando as bases de dados $U_\SbFIRM{}$ e $U_{\text{IR}}$.}
  \label{table:IEBGYPHS}
  \centering
  \begin{tabular}{|c|r|r|r|r|r|r|}
    \hline
    \multicolumn{7}{|c|}{$U_\SbFIRM{}$}                                                                      \\ \hline
      -            & \multicolumn{3}{c|}{Distância}             & \multicolumn{3}{c|}{Aproximação}           \\ \hline
    Flexibilização & Mínimo       & Média        & Máximo       & Mínimo       & Média        & Máximo       \\ \hline  
    10\%           & 70           & 88.01        & 109          & 2.22         & 2.82         & 3.22         \\ \hline
    20\%           & 63           & 87.95        & 111          & 2.24         & 2.85         & 3.36         \\ \hline
    30\%           & 66           & 87.37        & 110          & 2.39         & 2.87         & 3.29         \\ \hline
    40\%           & 69           & 87.34        & 108          & 2.26         & 2.90         & 3.37         \\ \hline
    50\%           & 67           & 86.64        & 110          & 2.43         & 2.91         & 3.35         \\ \hline    
  \end{tabular}

  \vspace{5mm}

  \begin{tabular}{|c|r|r|r|r|r|r|}
    \hline
    \multicolumn{7}{|c|}{$U_{\text{IR}}$}                                                                    \\ \hline
      -            & \multicolumn{3}{c|}{Distância}             & \multicolumn{3}{c|}{Aproximação}           \\ \hline
    Flexibilização & Mínimo       & Média        & Máximo       & Mínimo       & Média        & Máximo       \\ \hline  
    10\%           & 34           & 64.30        & 97           & 1.77         & 2.41         & 3.11         \\ \hline
    20\%           & 36           & 64.42        & 96           & 1.84         & 2.42         & 3.18         \\ \hline
    30\%           & 35           & 64.40        & 96           & 1.82         & 2.43         & 3.24         \\ \hline
    40\%           & 32           & 64.81        & 96           & 1.83         & 2.45         & 3.19         \\ \hline
    50\%           & 36           & 64.95        & 94           & 1.88         & 2.45         & 3.21         \\ \hline    
  \end{tabular}
\end{table}


% \begin{table}[!htb]
%   \caption{Resultados do Algoritmo~\ref{algorithm:DYDJWEUH} utilizando a base de dados $U_{\text{IR}}$.}
%   \label{table:RITAXFPQ}
%   \centering
%   \begin{tabular}{|c|r|r|r|r|r|r|}
%     \hline
%       -            & \multicolumn{3}{c|}{Distância}             & \multicolumn{3}{c|}{Aproximação}           \\ \hline
%     Flexibilização & Mínimo       & Média        & Máximo       & Mínimo       & Média        & Máximo       \\ \hline  
%     10\%           & 34           & 64.30        & 97           & 1.77         & 2.41         & 3.11         \\ \hline
%     20\%           & 36           & 64.42        & 96           & 1.84         & 2.42         & 3.18         \\ \hline
%     30\%           & 35           & 64.40        & 96           & 1.82         & 2.43         & 3.24         \\ \hline
%     40\%           & 32           & 64.81        & 96           & 1.83         & 2.45         & 3.19         \\ \hline
%     50\%           & 36           & 64.95        & 94           & 1.88         & 2.45         & 3.21         \\ \hline    
%   \end{tabular}
% \end{table}

% \begin{table}[!htb]
  \caption{Resultados do Algoritmo~\ref{algorithm:DYDJWEUH} utilizando a base de dados $U_{\text{IR}}$.}
  \label{table:RITAXFPQ}
  \centering
  \begin{tabular}{|c|r|r|r|r|r|r|}
    \hline
      -            & \multicolumn{3}{c|}{Distância}             & \multicolumn{3}{c|}{Aproximação}           \\ \hline
    Flexibilização & Mínimo       & Média        & Máximo       & Mínimo       & Média        & Máximo       \\ \hline  
    10\%           & 34           & 64.30        & 97           & 1.77         & 2.41         & 3.11         \\ \hline
    20\%           & 36           & 64.42        & 96           & 1.84         & 2.42         & 3.18         \\ \hline
    30\%           & 35           & 64.40        & 96           & 1.82         & 2.43         & 3.24         \\ \hline
    40\%           & 32           & 64.81        & 96           & 1.83         & 2.45         & 3.19         \\ \hline
    50\%           & 36           & 64.95        & 94           & 1.88         & 2.45         & 3.21         \\ \hline    
  \end{tabular}
\end{table}

Pela Tabela~\ref{table:IEBGYPHS} podemos notar que na base de dados $U_\SbFIRM{}$ a distância média de cada grupo obtida através do Algoritmo~\ref{algorithm:DYDJWEUH} tende a diminuir a medida que o grau de flexibilidade aumenta. Entretanto a aproximação média apresentou um compotamento oposto. A aproximação máxima registrada foi de $3.37$ e ocorreu no grupo com um grau de flexibilização de 40\%. Já na base de dados $U_{\text{IR}}$ tanto a distância média quanto a aproximação média por grupo tende a aumentar a medida que o grau de flexibilidade máxima também aumenta. A aproximação máxima registrada foi de $3.24$ e ocorreu no grupo com um grau de flexibilização máxima de 50\%.

A Tabela~\ref{table:OBVONNLP} apresenta os resultados do Algoritmo~\ref{algorithm:MODRXVSQ} utilizando a base de dados $U_\SbFIRMI{}$. A razão de aproximação obtida pelo algoritmo para cada instância foi computada utilizando o limitante inferior apresentado no Teorema~\ref{theorem:BOTBXFZQ}.

\begin{table}[!htb]
  \caption{Resultados do Algoritmo~\ref{algorithm:MODRXVSQ} utilizando a base de dados $U_\SbFIRMI{}$.}
  \label{table:OBVONNLP}
  \centering
  \begin{tabular}{|c|r|r|r|r|r|r|r|}
    \hline
      -      &  -   & \multicolumn{3}{c|}{Distância}             & \multicolumn{3}{c|}{Aproximação}           \\ \hline
    Grupo    & OP   & Mínimo       & Média        & Máximo       & Mínimo       & Média        & Máximo       \\ \hline  
    100      & 50   & 63           &  84.11       & 104          & 2.25         & 2.75         & 3.20         \\ \hline
    200      & 100  & 145          & 168.19       & 194          & 2.50         & 2.77         & 3.07         \\ \hline
    300      & 150  & 224          & 252.31       & 282          & 2.57         & 2.77         & 3.03         \\ \hline
    400      & 200  & 300          & 335.74       & 376          & 2.56         & 2.77         & 2.99         \\ \hline
    500      & 250  & 376          & 419.40       & 462          & 2.59         & 2.77         & 2.98         \\ \hline    
  \end{tabular}
\end{table}

Na Tabela~\ref{table:OBVONNLP} é possível observar que tanto a distância média como a aproximação média diminui a medida que o grau de flexibilização aumenta. Além disso, a distância mínima para os grupos com 40\% e 50\% de flexibilização foi menor que 60, aproximando-se da quantidade de 50 operações utilizadas para criar cada instância. A aproximação máxima registrada ocorreu foi de $3.22$ e ocorreu no grupo com 10\% de flexibilização.

A Tabela~\ref{table:DRUHLZFM} apresenta os resultados do Algoritmo~\ref{algorithm:KZFLZWRM} utilizando as bases de dados $U_\SbFIRT{}$ e $U_{\text{IR}}$. A razão de aproximação obtida pelo algoritmo para cada instância foi computada utilizando o limitante inferior apresentado no Teorema~\ref{theorem:KKKUCDHN}.

\begin{table}[!htb]
  \caption{Resultados do Algoritmo~\ref{algorithm:KZFLZWRM} utilizando a base de dados $U_\SbFIRT{}$.}
  \label{table:DRUHLZFM}
  \centering
  \begin{tabular}{|c|r|r|r|r|r|r|r|}
    \hline
      -      &  -   & \multicolumn{3}{c|}{Distância}             & \multicolumn{3}{c|}{Aproximação}           \\ \hline
    Grupo    & OP   & Mínimo       & Média        & Máximo       & Mínimo       & Média        & Máximo       \\ \hline  
    100      & 50   & 61           &  71.36       &  81          & 2.82         & 2.93         & 3.04         \\ \hline
    200      & 100  & 129          & 142.89       & 158          & 2.91         & 2.96         & 3.02         \\ \hline
    300      & 150  & 197          & 214.19       & 230          & 2.93         & 2.97         & 3.01         \\ \hline
    400      & 200  & 266          & 285.60       & 306          & 2.94         & 2.98         & 3.00         \\ \hline
    500      & 250  & 331          & 356.60       & 378          & 2.95         & 2.98         & 3.00         \\ \hline    
  \end{tabular}
\end{table}

% \begin{table}[!htb]
  \caption{Resultados do Algoritmo~\ref{algorithm:KZFLZWRM} utilizando a base de dados $U_{\text{IR}}$.}
  \label{table:PZXCAILB}
  \centering
  \begin{tabular}{|c|r|r|r|r|r|r|}
    \hline
      -            & \multicolumn{3}{c|}{Distância}             & \multicolumn{3}{c|}{Aproximação}           \\ \hline
    Flexibilização & Mínimo       & Média        & Máximo       & Mínimo       & Média        & Máximo       \\ \hline  
    10\%           & 32           & 51.97        & 64           & 2.60         & 2.90         & 3.00         \\ \hline
    20\%           & 33           & 51.86        & 64           & 2.64         & 2.90         & 3.00         \\ \hline
    30\%           & 30           & 51.71        & 64           & 2.64         & 2.91         & 3.06         \\ \hline
    40\%           & 28           & 51.67        & 64           & 2.71         & 2.91         & 3.06         \\ \hline
    50\%           & 33           & 51.90        & 64           & 2.69         & 2.91         & 3.12         \\ \hline    
  \end{tabular}
\end{table}

Pela Tabela~\ref{table:DRUHLZFM} podemos perceber que, em ambas as bases de dados, o Algoritmo~\ref{algorithm:KZFLZWRM} apresentou uma razão de aproximação máxima de $3.00$. Além disso, considerando a variação entre a menor aproximação mínima e a maior aproximação máxima entre todos os grupos, obtemos os valores de $0.18$ e $0.36$ para as bases de dados $U_\SbFIRT{}$ e $U_{\text{IR}}$, respectivamente. A distância mínima registrada na base de dados $U_\SbFIRT{}$ considerando todos os grupos foi de $59$, nove a mais do que o número de operações utilizadas para gerar cada instância, enquanto a distância máxima registrada foi de $83$. Também é possível observar uma estabilidade na distância média considerando todos os grupos da base $U_\SbFIRT{}$, onde os valores ficaram entre $71.17$ e $71.63$.

A Tabela~\ref{table:OTZHWXVI} apresenta os resultados do Algoritmo~\ref{algorithm:JSNLHIVA} utilizando a base de dados $U_\SbFIRTI{}$. A razão de aproximação obtida pelo algoritmo para cada instância foi computada utilizando o limitante inferior apresentado no Teorema~\ref{theorem:BOTBXFZQ}.

\begin{table}[!htb]
  \caption{Resultados do Algoritmo~\ref{algorithm:JSNLHIVA} utilizando a base de dados $U_\SbFIRTI{}$.}
  \label{table:OTZHWXVI}
  \centering
  \begin{tabular}{|c|r|r|r|r|r|r|r|}
    \hline
      -      &  -   & \multicolumn{3}{c|}{Distância}             & \multicolumn{3}{c|}{Aproximação}           \\ \hline
    Grupo    & OP   & Mínimo       & Média        & Máximo       & Mínimo       & Média        & Máximo       \\ \hline  
    100      & 50   & 58           &  67.26       &  76          & 2.77         & 2.89         & 2.96         \\ \hline
    200      & 100  & 120          & 133.56       & 148          & 2.88         & 2.94         & 2.97         \\ \hline
    300      & 150  & 181          & 200.15       & 220          & 2.92         & 2.96         & 2.98         \\ \hline
    400      & 200  & 242          & 265.93       & 290          & 2.94         & 2.97         & 2.98         \\ \hline
    500      & 250  & 311          & 331.80       & 355          & 2.95         & 2.97         & 2.99         \\ \hline    
  \end{tabular}
\end{table}

Na Tabela~\ref{table:OTZHWXVI} podemos observar que a razão de aproximação máxima obtida pelo Algoritmo~\ref{algorithm:JSNLHIVA} em todos os grupos foi de $2.96$, sendo um valor próximo ao limite teórico ($3.0$). É possível notar que o Algoritmo~\ref{algorithm:JSNLHIVA} apresentou uma variação pequena em relação a razão de aproximação, este fato pode ser constatado observando a variação entre a aproximação mínima e máxima de cada grupo. Por fim, a distância média fornecida pelo algoritmo apresentou um leve tendência de queda a medida que o grau de flexibilização aumenta.

A Tabela~\ref{table:NKDEXOVQ} apresenta os resultados do Algoritmo~\ref{algorithm:JRHFSYXO} utilizando as bases de dados $U_\SbFIRTM{}$ e $U_{\text{IR}}$. A razão de aproximação obtida pelo algoritmo para cada instância foi computada utilizando o limitante inferior apresentado no Teorema~\ref{theorem:KKKUCDHN}.

\begin{table}[!htb]
  \caption{Resultados do Algoritmo~\ref{algorithm:JRHFSYXO} utilizando as bases de dados $U_\SbFIRTM{}$ e $U_{\text{IR}}$.}
  \label{table:NKDEXOVQ}
  \centering
  \begin{tabular}{|c|r|r|r|r|r|r|}
    \hline
    \multicolumn{7}{|c|}{$U_\SbFIRTM{}$}                                                                     \\ \hline
      -            & \multicolumn{3}{c|}{Distância}             & \multicolumn{3}{c|}{Aproximação}           \\ \hline
    Flexibilização & Mínimo       & Média        & Máximo       & Mínimo       & Média        & Máximo       \\ \hline  
    10\%           & 59           & 68.12        & 77           & 2.79         & 2.91         & 2.96         \\ \hline
    20\%           & 57           & 67.59        & 76           & 2.82         & 2.91         & 2.96         \\ \hline
    30\%           & 58           & 66.92        & 77           & 2.82         & 2.91         & 2.96         \\ \hline
    40\%           & 54           & 66.16        & 77           & 2.81         & 2.91         & 2.96         \\ \hline
    50\%           & 55           & 65.57        & 75           & 2.82         & 2.91         & 2.96         \\ \hline    
  \end{tabular}

  \vspace{5mm}

  \begin{tabular}{|c|r|r|r|r|r|r|}
    \hline
    \multicolumn{7}{|c|}{$U_{\text{IR}}$}                                                                    \\ \hline
      -            & \multicolumn{3}{c|}{Distância}             & \multicolumn{3}{c|}{Aproximação}           \\ \hline
    Flexibilização & Mínimo       & Média        & Máximo       & Mínimo       & Média        & Máximo       \\ \hline  
    10\%           & 33           & 46.86        & 57           & 2.67         & 2.87         & 2.95         \\ \hline
    20\%           & 34           & 47.42        & 58           & 2.62         & 2.87         & 2.95         \\ \hline
    30\%           & 33           & 47.97        & 58           & 2.69         & 2.88         & 2.95         \\ \hline
    40\%           & 34           & 48.40        & 59           & 2.71         & 2.88         & 2.95         \\ \hline
    50\%           & 32           & 48.60        & 60           & 2.71         & 2.88         & 2.95         \\ \hline    
  \end{tabular}
\end{table}

% \begin{table}[!htb]
%   \caption{Resultados do Algoritmo~\ref{algorithm:JRHFSYXO} utilizando a base de dados $U_{\text{IR}}$.}
%   \label{table:JCSOIDPK}
%   \centering
%   \begin{tabular}{|c|r|r|r|r|r|r|}
%     \hline
%       -            & \multicolumn{3}{c|}{Distância}             & \multicolumn{3}{c|}{Aproximação}           \\ \hline
%     Flexibilização & Mínimo       & Média        & Máximo       & Mínimo       & Média        & Máximo       \\ \hline  
%     10\%           & 31           & 51.56        & 63           & 2.60         & 2.88         & 2.95         \\ \hline
%     20\%           & 33           & 51.46        & 64           & 2.69         & 2.88         & 2.95         \\ \hline
%     30\%           & 30           & 51.31        & 63           & 2.64         & 2.88         & 2.95         \\ \hline
%     40\%           & 28           & 51.24        & 63           & 2.71         & 2.88         & 2.95         \\ \hline
%     50\%           & 33           & 51.43        & 63           & 2.69         & 2.89         & 2.95         \\ \hline    
%   \end{tabular}
% \end{table}

% \begin{table}[!htb]
  \caption{Resultados do Algoritmo~\ref{algorithm:JRHFSYXO} utilizando a base de dados $U_{\text{cases}}$.}
  \label{table:JCSOIDPK}
  \centering
  \begin{tabular}{|c|r|r|r|r|r|r|}
    \hline
      -      & \multicolumn{3}{c|}{Distância}             & \multicolumn{3}{c|}{Aproximação}           \\ \hline
    Grupo    & Mínimo       & Média        & Máximo       & Mínimo       & Média        & Máximo       \\ \hline  
    100      &  34          &  59.77       &  74          & 2.58         & 2.89         & 2.96         \\ \hline
    200      &  74          & 120.08       & 146          & 2.74         & 2.94         & 2.97         \\ \hline
    300      & 116          & 180.54       & 214          & 2.82         & 2.95         & 2.98         \\ \hline
    400      & 157          & 240.84       & 286          & 2.81         & 2.96         & 2.98         \\ \hline
    500      & 200          & 300.93       & 355          & 2.88         & 2.97         & 2.99         \\ \hline    
  \end{tabular}
\end{table}

Pela Tabela~\ref{table:NKDEXOVQ} é possível observar que considerando os grupos de cada base de dados a aproximação máxima do Algoritmo~\ref{algorithm:JRHFSYXO} foi de $2.96$ e $2.95$ nas bases de dados $U_\SbFIRTM{}$ e $U_{\text{IR}}$, respectivamente. Uma característica interessante que foi a variação zero considerando a aproximação média entre os grupos da base de dados $U_\SbFIRTM{}$ e de $0.01$ considerando os grupos da base de dados $U_{\text{IR}}$. Por fim, o Algoritmo~\ref{algorithm:JRHFSYXO} forneceu uma distância mínima menor que $60$ para todos os grupos da base de dados $U_\SbFIRTM{}$ (50 operações foram utilizadas para gerar cada instância).

A Tabela~\ref{table:SKOTKEOE} apresenta os resultados do Algoritmo~\ref{algorithm:PJTWIANQ} utilizando a base de dados $U_\SbFIRTMI{}$. A razão de aproximação obtida pelo algoritmo para cada instância foi computada utilizando o limitante inferior apresentado no Teorema~\ref{theorem:BOTBXFZQ}.

\begin{table}[!htb]
  \caption{Resultados do Algoritmo~\ref{algorithm:PJTWIANQ} utilizando a base de dados $U_\SbFIRTMI{}$.}
  \label{table:SKOTKEOE}
  \centering
  \begin{tabular}{|c|r|r|r|r|r|r|}
    \hline
      -            & \multicolumn{3}{c|}{Distância}             & \multicolumn{3}{c|}{Aproximação}           \\ \hline
    Flexibilização & Mínimo       & Média        & Máximo       & Mínimo       & Média        & Máximo       \\ \hline  
    10\%           & 57           & 67.92        & 78           & 2.83         & 2.95         & 3.00         \\ \hline
    20\%           & 58           & 67.55        & 78           & 2.86         & 2.96         & 3.00         \\ \hline
    30\%           & 56           & 67.05        & 77           & 2.86         & 2.96         & 3.00         \\ \hline
    40\%           & 56           & 66.50        & 77           & 2.86         & 2.96         & 3.00         \\ \hline
    50\%           & 55           & 66.04        & 76           & 2.86         & 2.95         & 3.00         \\ \hline    
  \end{tabular}
\end{table}

Na Tabela~\ref{table:SKOTKEOE} podemos ver que a razão de aproximação máxima em todos os grupos pelo Algoritmo~\ref{algorithm:PJTWIANQ} atingiu o limite teórico garantido pelo algoritmo, que é de $3.0$. Vale ressaltar que isso não significa que a aproximação do algoritmo é justa, uma vez que a razão de aproximação computada para cada instância foi realizada utilizando o limitante inferior. O Algoritmo~\ref{algorithm:PJTWIANQ} foi o único que, dentre os algoritmos para as variações sem sinais dos problemas investigados neste capítulo, nos experimentos práticos apresentou uma razão de aproximação (utilizando o limitante inferior) que atingiu o limite teórico de aproximação. Por fim, é possível notar que o algoritmo apresentou pouca variação em relação as métricas de distância e aproximação considerando todos os grupos.

A Tabela~\ref{table:ZQDEMKFX} apresenta os resultados do Algoritmo~\ref{algorithm:UMNIXZHY} utilizando as bases de dados $U_\SbFIT{}$ e $U_{\text{C}}$. A razão de aproximação obtida pelo algoritmo para cada instância foi computada utilizando o limitante inferior apresentado no Teorema~\ref{theorem:PQQUYBMS}.

\begin{table}[!htb]
  \caption{Resultados do Algoritmo~\ref{algorithm:UMNIXZHY} utilizando as bases de dados $U_\SbFIT{}$ e $U_{\text{C}}$.}
  \label{table:ZQDEMKFX}
  \centering
  \begin{tabular}{|c|r|r|r|r|r|r|}
    \hline
    \multicolumn{7}{|c|}{$U_\SbFIT{}$}                                                                       \\ \hline
      -            & \multicolumn{3}{c|}{Distância}             & \multicolumn{3}{c|}{Aproximação}           \\ \hline
    Flexibilização & Mínimo       & Média        & Máximo       & Mínimo       & Média        & Máximo       \\ \hline  
    10\%           & 41           & 54.29        & 79           & 1.10         & 1.42         & 1.88         \\ \hline
    20\%           & 42           & 53.74        & 75           & 1.15         & 1.42         & 1.89         \\ \hline
    30\%           & 40           & 54.45        & 72           & 1.13         & 1.45         & 1.95         \\ \hline
    40\%           & 42           & 55.20        & 71           & 1.18         & 1.48         & 1.89         \\ \hline
    50\%           & 40           & 56.47        & 77           & 1.19         & 1.52         & 1.95         \\ \hline    
  \end{tabular}

  \vspace{5mm}

  \begin{tabular}{|c|r|r|r|r|r|r|}
    \hline
    \multicolumn{7}{|c|}{$U_{\text{C}}$}                                                                     \\ \hline
      -            & \multicolumn{3}{c|}{Distância}             & \multicolumn{3}{c|}{Aproximação}           \\ \hline
    Flexibilização & Mínimo       & Média        & Máximo       & Mínimo       & Média        & Máximo       \\ \hline  
    10\%           & 16           & 35.09        & 52           & 1.13         & 1.70         & 2.27         \\ \hline
    20\%           & 16           & 36.30        & 57           & 1.20         & 1.72         & 2.53         \\ \hline
    30\%           & 18           & 37.60        & 56           & 1.20         & 1.75         & 2.40         \\ \hline
    40\%           & 17           & 38.55        & 58           & 1.20         & 1.76         & 2.40         \\ \hline
    50\%           & 17           & 39.28        & 60           & 1.20         & 1.78         & 2.36         \\ \hline    
  \end{tabular}
\end{table}

% \begin{table}[!htb]
%   \caption{Resultados do Algoritmo~\ref{algorithm:UMNIXZHY} utilizando a base de dados $U_{\text{C}}$.}
%   \label{table:CMSNUCXZ}
%   \centering
%   \begin{tabular}{|c|r|r|r|r|r|r|}
%     \hline
%       -            & \multicolumn{3}{c|}{Distância}             & \multicolumn{3}{c|}{Aproximação}           \\ \hline
%     Flexibilização & Mínimo       & Média        & Máximo       & Mínimo       & Média        & Máximo       \\ \hline  
%     10\%           & 16           & 36.44        & 54           & 1.13         & 1.63         & 2.00         \\ \hline
%     20\%           & 15           & 37.87        & 58           & 1.14         & 1.66         & 2.11         \\ \hline
%     30\%           & 16           & 38.76        & 59           & 1.13         & 1.69         & 2.11         \\ \hline
%     40\%           & 17           & 39.17        & 63           & 1.13         & 1.70         & 2.25         \\ \hline
%     50\%           & 16           & 39.23        & 59           & 1.14         & 1.72         & 2.20         \\ \hline    
%   \end{tabular}
% \end{table}

% \begin{table}[!htb]
  \caption{Resultados do Algoritmo~\ref{algorithm:UMNIXZHY} utilizando a base de dados $U_{\text{C}}$.}
  \label{table:CMSNUCXZ}
  \centering
  \begin{tabular}{|c|r|r|r|r|r|r|}
    \hline
      -            & \multicolumn{3}{c|}{Distância}             & \multicolumn{3}{c|}{Aproximação}           \\ \hline
    Flexibilização & Mínimo       & Média        & Máximo       & Mínimo       & Média        & Máximo       \\ \hline  
    10\%           & 16           & 36.44        & 54           & 1.13         & 1.63         & 2.00         \\ \hline
    20\%           & 15           & 37.87        & 58           & 1.14         & 1.66         & 2.11         \\ \hline
    30\%           & 16           & 38.76        & 59           & 1.13         & 1.69         & 2.11         \\ \hline
    40\%           & 17           & 39.17        & 63           & 1.13         & 1.70         & 2.25         \\ \hline
    50\%           & 16           & 39.23        & 59           & 1.14         & 1.72         & 2.20         \\ \hline    
  \end{tabular}
\end{table}

Na Tabela~\ref{table:ZQDEMKFX} é possível notar que o Algoritmo~\ref{algorithm:UMNIXZHY} apresentou um ótimo resultado prático em comparação com o limite teórico que garantido ($3.5$-aproximação). Na base de dados $U_\SbFIT{}$ a maior aproximação máxima foi de $1.95$, registrada nos grupos 30\% e 50\%, já na base de dados a maior aproximação máxima foi de $2.53$, no grupo 20\%. É importante notar também que em todos os grupos da base de dados $U_\SbFIT{}$ o Algoritmo~\ref{algorithm:UMNIXZHY} forneceu uma distância mínima menor que $43$. Vale ressaltar que em todas as instâncias da base de dados foram utilizadas $50$ operações para gerar cada uma delas. Entretanto, algumas operações podem desfazer operações aplicadas previamente durante o processo de criação. Por esse motivo, é possível que em algumas instâncias seja necessário menos que $50$ operações para atingir o genoma alvo.

A Tabela~\ref{table:OAYXLAOR} apresenta os resultados do Algoritmo~\ref{algorithm:UEBBPCAK} utilizando as bases de dados $U_\SbFITM{}$ e $U_{\text{C}}$. A razão de aproximação obtida pelo algoritmo para cada instância foi computada utilizando o limitante inferior apresentado no Teorema~\ref{theorem:PQQUYBMS}.

\begin{table}[!htb]
  \caption{Resultados do Algoritmo~\ref{algorithm:UEBBPCAK} utilizando a base de dados $U_\SbFITM{}$.}
  \label{table:OAYXLAOR}
  \centering
  \begin{tabular}{|c|r|r|r|r|r|r|}
    \hline
      -            & \multicolumn{3}{c|}{Distância}             & \multicolumn{3}{c|}{Aproximação}           \\ \hline
    Flexibilização & Mínimo       & Média        & Máximo       & Mínimo       & Média        & Máximo       \\ \hline  
    10\%           & 38           & 51.13        & 68           & 1.13         & 1.41         & 1.78         \\ \hline
    20\%           & 37           & 49.85        & 64           & 1.11         & 1.40         & 1.83         \\ \hline
    30\%           & 37           & 50.07        & 67           & 1.09         & 1.43         & 1.88         \\ \hline
    40\%           & 37           & 50.59        & 67           & 1.16         & 1.45         & 1.89         \\ \hline
    50\%           & 37           & 51.57        & 72           & 1.14         & 1.49         & 1.97         \\ \hline    
  \end{tabular}
\end{table}

% \begin{table}[!htb]
  \caption{Resultados do Algoritmo~\ref{algorithm:UEBBPCAK} utilizando a base de dados $U_{\text{C}}$.}
  \label{table:BYRVTGIR}
  \centering
  \begin{tabular}{|c|r|r|r|r|r|r|}
    \hline
      -            & \multicolumn{3}{c|}{Distância}             & \multicolumn{3}{c|}{Aproximação}           \\ \hline
    Flexibilização & Mínimo       & Média        & Máximo       & Mínimo       & Média        & Máximo       \\ \hline  
    10\%           & 16           & 35.88        & 53           & 1.13         & 1.61         & 1.96         \\ \hline
    20\%           & 15           & 37.29        & 57           & 1.14         & 1.64         & 2.00         \\ \hline
    30\%           & 16           & 38.14        & 57           & 1.13         & 1.66         & 2.04         \\ \hline
    40\%           & 17           & 38.48        & 59           & 1.13         & 1.67         & 2.11         \\ \hline
    50\%           & 16           & 38.41        & 59           & 1.14         & 1.68         & 2.07         \\ \hline    
  \end{tabular}
\end{table}

Na Tabela~\ref{table:OAYXLAOR} podemos notar que a aproximação máxima registrada do Algoritmo~\ref{algorithm:UEBBPCAK} em todos os grupos da base de dados $U_\SbFITM{}$ foi menor que $2.0$. Já na base de dados $U_{\text{C}}$ a maior aproximação máxima foi de $2.20$, registrada no grupo 40\%. Utilizando a base de dados $U_\SbFITM{}$ o Algoritmo~\ref{algorithm:UEBBPCAK} também apresentou para todos os grupos uma distância mínima menor do que o número de operações utilizada para gerar cada instância, em todos os grupos esse valor foi menor que $39$.

De maneira geral todos os algoritmo apresentaram um bom desempenho na prática, sendo que o Algoritmo~\ref{algorithm:PJTWIANQ} foi o único em que a razão de aproximação máxima atingiu o limite teórico que é garantido pelo mesmo. 

% ------------------------------------------------------------------ %
\subsection{Instâncias Intergênicas Flexíveis com Sinais}
% ------------------------------------------------------------------ %

Nesta seção, apresentaremos algoritmos de aproximação para a variação com sinais dos problemas investigados neste capítulo com base nas funções de redução apresentadas previamente.

% ------------------------------------------------------------------ %
\subsubsection{Reversão}
% ------------------------------------------------------------------ %

Nesta seção, apresentaremos um algoritmo de aproximação com fator $2$ para a variação com sinais do problema \SbFIR{}. 

Note que a variação com sinais do problema \SbIR{} possui um algoritmo de aproximação com um fator de $2$~\cite{2021a-oliveira-etal}, que chamaremos de $2$-\SbIR{}. Além disso, temos o seguinte lema.

\begin{lemma}\label{lemma:XMVOFRZM}
Seja $\mathcal{I}' = ((\pi',\breve\pi'),(\iota',\breve\iota'))$ uma instância intergênica rígida balanceada com sinais, o algoritmo $2$-\SbIR{} transforma $(\pi',\breve\pi')$ em $(\iota',\breve\iota')$ utilizando uma sequência de eventos de reversão $S'$, tal que $|S'| \le 2({n+1} - c_b(G(\mathcal{I})))$.
\end{lemma}
\begin{proof}
Diretamente pelo Algoritmo 1 de Oliveira \textit{et al.}~\cite{2021b-oliveira-etal}.
\end{proof}

A seguir apresentamos o Algoritmo~\ref{algorithm:HSDBEFII}.

\begin{algorithm}[!tbh]
  \caption{Um algoritmo de aproximação para o problema \SbFIR{}.\label{algorithm:HSDBEFII}}
  \Entrada{Uma instância intergênica flexível balanceada com sinais $\mathcal{I} = ((\pi,\breve\pi),(\iota,\breve\iota^{\min},\breve\iota^{\max}))$}
  \Saida{Uma sequência de eventos de reversão $S$, tal que $(\pi,\breve\pi) \cdot S$ atinge o genoma alvo de $\mathcal{I}$} 
  $\mathcal{I}' = \mathcal{F}_{c}^{''}(\mathcal{I})$ \\
  Seja $S'$ uma sequência de eventos de reversão fornecida pelo algoritmo $2$-\SbIR{} para a instância $\mathcal{I}'$ \\
  \Retorna{$S'$}
\end{algorithm}

\begin{theorem}\label{theorem:GTWKCOJR}
Dada uma instância intergênica flexível balanceada com sinais $\mathcal{I}$, o Algoritmo~\ref{algorithm:HSDBEFII} é uma $2$-aproximação para o problema \SbFIR{}.
\end{theorem}
\begin{proof}
Pelo Lema~\ref{lemma:WQOEFBXP}, temos que a sequência fornecida pelo algoritmo $2$-\SbIR{} para a instância intergênica rígida balanceada com sinais $\mathcal{I'}$, se aplicada no genoma de origem $(\pi,\breve\pi)$ da instância intergênica flexível balanceada com sinais $\mathcal{I}$, faz com que o genoma alvo seja alcançado. Além disso, note que os problemas \SbIR{} e \SbFIR{} compartilham o mesmo modelo de rearranjo. Logo, a sequência $S'$ utiliza apenas eventos permitidos pelo modelo de rearranjo do problema \SbFIR{}. Pelo Lema~\ref{lemma:XMVOFRZM}, temos que $|S'| \le 2({n+1} - c_b(G(\mathcal{I})))$. Entretanto, pelo Lema~\ref{lemma:PSGXFVHD}, temos que $c_d(G(\mathcal{I})) = c_b(G(\mathcal{I}'))$. Logo, $|S'| \le 2({n+1} - c_d(G(\mathcal{I})))$. Pelo Teorema~\ref{theorem:EUNBEQEX}, temos o seguinte limitante inferior $df_{\SbFIR}(\mathcal{I}) \ge {n+1} - c_d(G(\mathcal{I}))$, e o teorema segue.
\end{proof}

% ------------------------------------------------------------------ %
\subsubsection{Reversão e Indel}
% ------------------------------------------------------------------ %

Nesta seção, apresentaremos um algoritmo de aproximação com fator $2$ para a variação com sinais do problema \SbFIRI{}. 

Note que a variação com sinais do problema \SbIRI{} possui um algoritmo de aproximação com um fator de $2$~\cite{2021b-oliveira-etal}, que chamaremos de $2$-\SbIRI{}. Além disso, temos o seguinte lema.

\begin{lemma}\label{lemma:FZWPBXFK}
Seja $\mathcal{I}' = ((\pi',\breve\pi'),(\iota',\breve\iota'))$ uma instância intergênica rígida com sinais, o algoritmo $2$-\SbIRI{} transforma $(\pi',\breve\pi')$ em $(\iota',\breve\iota')$ utilizando uma sequência de eventos de reversão e indel $S'$, tal que $|S'| \le 2({n+1} - c_b(G(\mathcal{I})))$.
\end{lemma}
\begin{proof}
Diretamente pelo Algoritmo 2 de Oliveira \textit{et al.}~\cite{2021b-oliveira-etal}.
\end{proof}

A seguir apresentamos o Algoritmo~\ref{algorithm:ZCBCGAUW}.

\input{algorithms/ZCBCGAUW}

\begin{theorem}\label{theorem:UEOFTCVZ}
Dada uma instância intergênica flexível com sinais $\mathcal{I}$, o Algoritmo~\ref{algorithm:ZCBCGAUW} é uma $2$-aproximação para o problema \SbFIRI{}.
\end{theorem}
\begin{proof}
Pelo Lema~\ref{lemma:TQUNQUGX}, temos que a sequência fornecida pelo algoritmo $2$-\SbIRI{} para a instância intergênica rígida com sinais $\mathcal{I'}$, se aplicada no genoma de origem $(\pi,\breve\pi)$ da instância intergênica flexível com sinais $\mathcal{I}$, faz com que o genoma alvo seja alcançado. Além disso, note que os problemas \SbIRI{} e \SbFIRI{} compartilham o mesmo modelo de rearranjo. Logo, a sequência $S'$ utiliza apenas eventos permitidos pelo modelo de rearranjo do problema \SbFIRI{}. Pelo Lema~\ref{lemma:FZWPBXFK}, temos que $|S'| \le 2({n+1} - c_b(G(\mathcal{I})))$. Entretanto, pelo Lema~\ref{lemma:AOKHMVAY}, temos que $c_e(G(\mathcal{I})) = c_b(G(\mathcal{I}'))$. Logo, $|S'| \le 2({n+1} - c_e(G(\mathcal{I})))$. Pelo Teorema~\ref{theorem:SZNBDWOM}, temos o seguinte limitante inferior $df_{\SbFIRI}(\mathcal{I}) \ge {n+1} - c_e(G(\mathcal{I}))$, e o teorema segue.
\end{proof}

% ------------------------------------------------------------------ %
\subsubsection{Reversão e Move}
% ------------------------------------------------------------------ %

Nesta seção, apresentaremos um algoritmo de aproximação com fator $2$ para a variação com sinais do problema \SbFIRM{}. A seguir apresentamos o Algoritmo~\ref{algorithm:VOHUBSMM}.

\begin{algorithm}[!tbh]
  \caption{Um algoritmo de aproximação para o problema \SbFIRM{}.\label{algorithm:VOHUBSMM}}
  \Entrada{Uma instância intergênica flexível balanceada com sinais $\mathcal{I} = ((\pi,\breve\pi),(\iota,\breve\iota^{\min},\breve\iota^{\max}))$}
  \Saida{Uma sequência de eventos de reversão e move $S$, tal que $(\pi,\breve\pi) \cdot S$ atinge o genoma alvo de $\mathcal{I}$} 
  $\mathcal{I}' = \mathcal{F}_{c}^{''}(\mathcal{I})$ \\
  Seja $S'$ uma sequência de eventos de reversão e move fornecida pelo Algoritmo~\ref{algorithm:EHDLZXJA} para a instância $\mathcal{I}'$ \\
  \Retorna{$S'$}
\end{algorithm}

\begin{theorem}\label{theorem:NBFXUXJG}
Dada uma instância intergênica flexível balanceada com sinais $\mathcal{I}$, o Algoritmo~\ref{algorithm:VOHUBSMM} é uma $2$-aproximação para o problema \SbFIRM{}.
\end{theorem}
\begin{proof}
Pelo Lema~\ref{lemma:WQOEFBXP}, temos que a sequência fornecida pelo Algoritmo~\ref{algorithm:EHDLZXJA} para a instância intergênica rígida balanceada com sinais $\mathcal{I'}$, se aplicada no genoma de origem $(\pi,\breve\pi)$ da instância intergênica flexível balanceada com sinais $\mathcal{I}$, faz com que o genoma alvo seja alcançado. Além disso, note que os problemas \SbIRM{} e \SbFIRM{} compartilham o mesmo modelo de rearranjo. Logo, a sequência $S'$ utiliza apenas eventos permitidos pelo modelo de rearranjo do problema \SbFIRM{}. Pelo Lema~\ref{lemma:APHTXLZC}, temos que $|S'| \le 2(n + 1) - (c(G(\mathcal{I}')) + c_b(G(\mathcal{I}')))$. Entretanto, pelo Lema~\ref{lemma:PSGXFVHD}, temos que $c(G(\mathcal{I})) = c(G(\mathcal{I}'))$ e $c_d(G(\mathcal{I})) = c_b(G(\mathcal{I}'))$. Logo, $|S'| \le 2(n + 1) - (c(G(\mathcal{I})) + c_d(G(\mathcal{I})))$. Pelo Teorema~\ref{theorem:CNMFNKPK}, temos o seguinte limitante inferior $df_{\SbFIRM}(\mathcal{I}) \ge {n+1} - \frac{(c(G(\mathcal{I})) + c_d(G(\mathcal{I})))}{2}$, e o teorema segue.
\end{proof}

% ------------------------------------------------------------------ %
\subsubsection{Reversão, Move e Indel}
% ------------------------------------------------------------------ %

Nesta seção, apresentaremos um algoritmo de aproximação com fator $2$ para a variação com sinais do problema \SbFIRMI{}. A seguir apresentamos o Algoritmo~\ref{algorithm:TAJJYPTG}.

\input{algorithms/TAJJYPTG}

\begin{theorem}\label{theorem:PQWPQJDG}
Dada uma instância intergênica flexível com sinais $\mathcal{I}$, o Algoritmo~\ref{algorithm:TAJJYPTG} é uma $2$-aproximação para o problema \SbFIRMI{}.
\end{theorem}
\begin{proof}
Pelo Lema~\ref{lemma:TQUNQUGX}, temos que a sequência fornecida pelo Algoritmo~\ref{algorithm:RYNLKYUJ} para a instância intergênica rígida com sinais $\mathcal{I'}$, se aplicada no genoma de origem $(\pi,\breve\pi)$ da instância intergênica flexível com sinais $\mathcal{I}$, faz com que o genoma alvo seja alcançado. Além disso, note que os problemas \SbIRMI{} e \SbFIRMI{} compartilham o mesmo modelo de rearranjo. Logo, a sequência $S'$ utiliza apenas eventos permitidos pelo modelo de rearranjo do problema \SbFIRMI{}. Pelo Lema~\ref{lemma:PBDEKMXG}, temos que $|S'| \le 2(n + 1) - (c(G(\mathcal{I}')) + c_b(G(\mathcal{I}')))$. Entretanto, pelo Lema~\ref{lemma:AOKHMVAY}, temos que $c(G(\mathcal{I})) = c(G(\mathcal{I}'))$ e $c_e(G(\mathcal{I})) = c_b(G(\mathcal{I}'))$. Logo, $|S'| \le 2(n + 1) - (c(G(\mathcal{I})) + c_e(G(\mathcal{I})))$. Pelo Teorema~\ref{theorem:XQPRYMFX}, temos o seguinte limitante inferior $df_{\SbFIRMI}(\mathcal{I}) \ge {n+1} - \frac{c(G(\mathcal{I})) + c_e(G(\mathcal{I}))}{2}$, e o teorema segue.
\end{proof}

% ------------------------------------------------------------------ %
\subsubsection{Reversão e Transposição}
% ------------------------------------------------------------------ %

Nesta seção, apresentaremos um algoritmo de aproximação com fator $3$ para a variação com sinais do problema \SbFIRT{}. 

Note que a variação com sinais do problema \SbIRT{} possui um algoritmo de aproximação com um fator de $3$~\cite{2021a-oliveira-etal}, que chamaremos de $3$-\SbIRT{}. Além disso, temos o seguinte lema.

\begin{lemma}\label{lemma:MNQTVIRT}
Seja $\mathcal{I}' = ((\pi',\breve\pi'),(\iota',\breve\iota'))$ uma instância intergênica rígida balanceada com sinais, o algoritmo $3$-\SbIRT{} transforma $(\pi',\breve\pi')$ em $(\iota',\breve\iota')$ utilizando uma sequência de eventos de reversão e transposição $S'$, tal que $|S'| \le \frac{3({n+1} - c_b(G(\mathcal{I}')))}{2}$.
\end{lemma}
\begin{proof}
Diretamente pelo Lema 6.3 de Oliveira \textit{et al.}~\cite{2021a-oliveira-etal}.
\end{proof}

A seguir apresentamos o Algoritmo~\ref{algorithm:EMLPACHB}.

\begin{algorithm}[!tbh]
  \caption{Um algoritmo de aproximação para o problema \SbFIRT{}.\label{algorithm:EMLPACHB}}
  \Entrada{Uma instância intergênica flexível balanceada com sinais $\mathcal{I} = ((\pi,\breve\pi),(\iota,\breve\iota^{\min},\breve\iota^{\max}))$}
  \Saida{Uma sequência de eventos de reversão e transposição $S$, tal que $(\pi,\breve\pi) \cdot S$ atinge o genoma alvo de $\mathcal{I}$} 
  $\mathcal{I}' = \mathcal{F}_{c}^{''}(\mathcal{I})$ \\
  Seja $S'$ uma sequência de eventos de reversão e transposição fornecida pelo algoritmo $3$-\SbIRT{} para a instância $\mathcal{I}'$ \\
  \Retorna{$S'$}
\end{algorithm}

\begin{theorem}\label{theorem:QISZKAHW}
Dada uma instância intergênica flexível balanceada com sinais $\mathcal{I}$, o Algoritmo~\ref{algorithm:EMLPACHB} é uma $3$-aproximação para o problema \SbFIRT{}.
\end{theorem}
\begin{proof}
Pelo Lema~\ref{lemma:WQOEFBXP}, temos que a sequência fornecida pelo algoritmo $3$-\SbIRT{} para a instância intergênica rígida balanceada com sinais $\mathcal{I'}$, se aplicada no genoma de origem $(\pi,\breve\pi)$ da instância intergênica flexível balanceada com sinais $\mathcal{I}$, faz com que o genoma alvo seja alcançado. Além disso, note que os problemas \SbIRT{} e \SbFIRT{} compartilham o mesmo modelo de rearranjo. Logo, a sequência $S'$ utiliza apenas eventos permitidos pelo modelo de rearranjo do problema \SbFIRT{}. Pelo Lema~\ref{lemma:MNQTVIRT}, temos que $|S'| \le \frac{3({n+1} - c_b(G(\mathcal{I}')))}{2}$. Entretanto, pelo Lema~\ref{lemma:PSGXFVHD}, temos que $c_d(G(\mathcal{I})) = c_b(G(\mathcal{I}'))$. Logo, $|S'| \le \frac{3({n+1} - c_d(G(\mathcal{I})))}{2}$. Pelo Teorema~\ref{theorem:HELIIGVZ}, temos o seguinte limitante inferior $df_{\SbFIRT}(\mathcal{I}) \ge \frac{{n+1} - c_d(G(\mathcal{I}))}{2}$, e o teorema segue.
\end{proof}

% ------------------------------------------------------------------ %
\subsubsection{Reversão, Transposição e Indel}
% ------------------------------------------------------------------ %

Nesta seção, apresentaremos um algoritmo de aproximação com fator $3$ para a variação com sinais do problema \SbFIRTI{}. A seguir apresentamos o Algoritmo~\ref{algorithm:WWDUHPBG}.

\input{algorithms/WWDUHPBG}

\begin{theorem}\label{theorem:IRGQGPKZ}
Dada uma instância intergênica flexível com sinais $\mathcal{I}$, o Algoritmo~\ref{algorithm:WWDUHPBG} é uma $3$-aproximação para o problema \SbFIRTI{}.
\end{theorem}
\begin{proof}
Pelo Lema~\ref{lemma:TQUNQUGX}, temos que a sequência fornecida pelo Algoritmo~\ref{algorithm:YMHYMYQC} para a instância intergênica rígida com sinais $\mathcal{I'}$, se aplicada no genoma de origem $(\pi,\breve\pi)$ da instância intergênica flexível com sinais $\mathcal{I}$, faz com que o genoma alvo seja alcançado. Além disso, note que os problemas \SbIRTI{} e \SbFIRTI{} compartilham o mesmo modelo de rearranjo. Logo, a sequência $S'$ utiliza apenas eventos permitidos pelo modelo de rearranjo do problema \SbFIRTI{}. Pelo Lema~\ref{lemma:TKRHFREQ}, temos que $|S'| \le \frac{3(n+1 - c_b(G(\mathcal{I})))}{2}$. Entretanto, pelo Lema~\ref{lemma:AOKHMVAY}, temos que $c(G(\mathcal{I})) = c(G(\mathcal{I}'))$ e $c_e(G(\mathcal{I})) = c_b(G(\mathcal{I}'))$. Logo, $|S'| \le \frac{3(n+1 - c_e(G(\mathcal{I})))}{2}$. Pelo Teorema~\ref{theorem:SZNBDWOM}, temos o seguinte limitante inferior $df_{\SbFIRTI}(\mathcal{I}) \ge \frac{n+1 - c_e(G(\mathcal{I}))}{2}$, e o teorema segue.
\end{proof}

% ------------------------------------------------------------------ %
\subsubsection{Reversão, Transposição e Move}
% ------------------------------------------------------------------ %

Nesta seção, apresentaremos um algoritmo de aproximação com fator $2.5$ para a variação com sinais do problema \SbFIRTM{}. 

Note que a variação com sinais do problema \SbIRTM{} possui um algoritmo de aproximação com um fator de $2.5$~\cite{2021a-oliveira-etal}, que chamaremos de $2.5$-\SbIRTM{}. Além disso, temos o seguinte lema.

\begin{lemma}\label{lemma:TPROVWMO}
Seja $\mathcal{I}' = ((\pi',\breve\pi'),(\iota',\breve\iota'))$ uma instância intergênica rígida balanceada com sinais, o algoritmo $2.5$-\SbIRTM{} transforma $(\pi',\breve\pi')$ em $(\iota',\breve\iota')$ utilizando uma sequência de eventos de reversão, transposição e move $S'$, tal que $|S'| \le \frac{5({n+1} - c_b(G(\mathcal{I}')))}{4}$.
\end{lemma}
\begin{proof}
Diretamente pelo Lema 7.11 de Oliveira \textit{et al.}~\cite{2021a-oliveira-etal}.
\end{proof}

A seguir apresentamos o Algoritmo~\ref{algorithm:XXIGKPAV}.

\begin{algorithm}[!tbh]
  \caption{Um algoritmo de aproximação para o problema \SbFIRTM{}.\label{algorithm:XXIGKPAV}}
  \Entrada{Uma instância intergênica flexível balanceada com sinais $\mathcal{I} = ((\pi,\breve\pi),(\iota,\breve\iota^{\min},\breve\iota^{\max}))$}
  \Saida{Uma sequência de eventos de reversão, transposição e move $S$, tal que $(\pi,\breve\pi) \cdot S$ atinge o genoma alvo de $\mathcal{I}$} 
  $\mathcal{I}' = \mathcal{F}_{c}^{''}(\mathcal{I})$ \\
  Seja $S'$ uma sequência de eventos de reversão, transposição e move fornecida pelo algoritmo $2.5$-\SbIRTM{} para a instância $\mathcal{I}'$ \\
  \Retorna{$S'$}
\end{algorithm}

\begin{theorem}\label{theorem:BZSXXPYW}
Dada uma instância intergênica flexível balanceada com sinais $\mathcal{I}$, o Algoritmo~\ref{algorithm:XXIGKPAV} é uma $2.5$-aproximação para o problema \SbFIRTM{}.
\end{theorem}
\begin{proof}
Pelo Lema~\ref{lemma:WQOEFBXP}, temos que a sequência fornecida pelo algoritmo $2.5$-\SbIRT{} para a instância intergênica rígida balanceada com sinais $\mathcal{I'}$, se aplicada no genoma de origem $(\pi,\breve\pi)$ da instância intergênica flexível balanceada com sinais $\mathcal{I}$, faz com que o genoma alvo seja alcançado. Além disso, note que os problemas \SbIRTM{} e \SbFIRTM{} compartilham o mesmo modelo de rearranjo. Logo, a sequência $S'$ utiliza apenas eventos permitidos pelo modelo de rearranjo do problema \SbFIRTM{}. Pelo Lema~\ref{lemma:TPROVWMO}, temos que $|S'| \le \frac{5({n+1} - c_b(G(\mathcal{I}')))}{4}$. Entretanto, pelo Lema~\ref{lemma:PSGXFVHD}, temos que $c_d(G(\mathcal{I})) = c_b(G(\mathcal{I}'))$. Logo, $|S'| \le \frac{5({n+1} - c_d(G(\mathcal{I})))}{4}$. Pelo Teorema~\ref{theorem:HELIIGVZ}, temos o seguinte limitante inferior $df_{\SbFIRTM}(\mathcal{I}) \ge \frac{{n+1} - c_d(G(\mathcal{I}))}{2}$, e o teorema segue.
\end{proof}

% ------------------------------------------------------------------ %
\subsubsection{Reversão, Transposição, Move e Indel}
% ------------------------------------------------------------------ %

Nesta seção, apresentaremos um algoritmo de aproximação com fator $3$ para a variação com sinais do problema \SbFIRTMI{}. A seguir apresentamos o Algoritmo~\ref{algorithm:JBNSEPGG}.

\begin{algorithm}[!tbh]
  \caption{Um algoritmo de aproximação para o problema \SbFIRTMI{}.\label{algorithm:JBNSEPGG}}
  \Entrada{Uma instância intergênica flexível com sinais $\mathcal{I} = ((\pi,\breve\pi),(\iota,\breve\iota^{\min},\breve\iota^{\max}))$}
  \Saida{Uma sequência de eventos de reversão, transposição, move e indel $S$, tal que $(\pi,\breve\pi) \cdot S$ atinge o genoma alvo de $\mathcal{I}$} 
  $\mathcal{I}' = \mathcal{F}_{c}^{'}(\mathcal{I})$ \\
  Seja $S'$ uma sequência de eventos de reversão, transposição, move e indel fornecida pelo Algoritmo~\ref{algorithm:EIFZNOAH} para a instância $\mathcal{I}'$ \\
  \Retorna{$S'$}
\end{algorithm}

\begin{theorem}\label{theorem:AKZNNSGT}
Dada uma instância intergênica flexível com sinais $\mathcal{I}$, o Algoritmo~\ref{algorithm:JBNSEPGG} é uma $3$-aproximação para o problema \SbFIRTMI{}.
\end{theorem}
\begin{proof}
Pelo Lema~\ref{lemma:TQUNQUGX}, temos que a sequência fornecida pelo Algoritmo~\ref{algorithm:EIFZNOAH} para a instância intergênica rígida com sinais $\mathcal{I'}$, se aplicada no genoma de origem $(\pi,\breve\pi)$ da instância intergênica flexível com sinais $\mathcal{I}$, faz com que o genoma alvo seja alcançado. Além disso, note que os problemas \SbIRTMI{} e \SbFIRTMI{} compartilham o mesmo modelo de rearranjo. Logo, a sequência $S'$ utiliza apenas eventos permitidos pelo modelo de rearranjo do problema \SbFIRTMI{}. Pelo Lema~\ref{lemma:TEVTTPGB}, temos que $|S'| \le \frac{3(n+1 - c_b(G(\mathcal{I})))}{2}$. Entretanto, pelo Lema~\ref{lemma:AOKHMVAY}, temos que $c(G(\mathcal{I})) = c(G(\mathcal{I}'))$ e $c_e(G(\mathcal{I})) = c_b(G(\mathcal{I}'))$. Logo, $|S'| \le \frac{3(n+1 - c_e(G(\mathcal{I})))}{2}$. Pelo Teorema~\ref{theorem:SZNBDWOM}, temos o seguinte limitante inferior $df_{\SbFIRTMI}(\mathcal{I}) \ge \frac{n+1 - c_e(G(\mathcal{I}))}{2}$, e o teorema segue.
\end{proof}

% ------------------------------------------------------------------ %
\subsubsection{Resultados Práticos}
% ------------------------------------------------------------------ %

Nesta seção, apresentaremos os resultados práticos dos algoritmos apresentados para a variação com sinas dos problemas \SbFIR{}, \SbFIRI{}, \SbFIRM{}, \SbFIRMI{}, \SbFIRT{}, \SbFIRTI{}, \SbFIRTM{} e \SbFIRTMI{}.

Nós também criamos uma base de dados para cada problema e utilizamos os identificadores $S_\SbFIR{}$, $S_\SbFIRI{}$, $S_\SbFIRM{}$, $S_\SbFIRMI{}$, $S_\SbFIRT{}$, $S_\SbFIRTI{}$, $S_\SbFIRTM{}$ e $S_\SbFIRTMI{}$ para a base de dados dos problemas \SbFIR{}, \SbFIRI{}, \SbFIRM{}, \SbFIRMI{}, \SbFIRT{}, \SbFIRTI{}, \SbFIRTM{} e \SbFIRTMI{}, respectivamente. As bases de dados foram criadas de forma similar ao processo descrito na Seção~\ref{subsubsection:PWLZZAVH}, diferindo apenas que cada instância foi criada a partir das representações intergênicas rígida e flexível com sinais. Logo, ao aplicar um evento de reversão os genes no segmento afetado também acabam tendo a orientação invertida.

Utilizando a estrutura de grafo de ciclos ponderado flexível e considerando todos os grupos das bases de dados $S_\SbFIR{}$, $S_\SbFIRM{}$, $S_\SbFIRT{}$ e $S_\SbFIRTM{}$, foi observado que 79.55\%,  8.95\% e 11.50\% das instâncias pertencem ao cenário de equilíbrio, fonte e sorvedouro, respectivamente. 

Para garantir uma proporcionalidade entre os possíveis cenários nos problemas que utilizam um modelo de rearranjo composto exclusivamente por eventos conservativos nós criamos as bases de dados $S_{C}$. A base de dados possui cinco grupos, sendo que cada grupo possui 3000 instâncias intergênicas flexíveis balanceadas com sinais de tamanho 100 e é identificado pelo grau de flexibilização máxima das instâncias contidas nele. Os identificadores do grupos são 10\%, 20\%, 30\%, 40\% e 50\%. Utilizando a estrutura de grafo de ciclos ponderado flexível, cada grupo possui 1000 instâncias no cenário de equilíbrio, 1000 instâncias no cenário fonte e 1000 instâncias no cenário sorvedouro. O processo de criação de uma instância da base de dados $S_{C}$ é semelhante a utilizada pela base de dados $U_{C}$, descrito na Seção~\ref{subsubsection:PWLZZAVH}, diferenciando apenas pelo fato de que cada instância foi criada a partir das representações intergênicas rígida e flexível com sinais e que a operação de troca também altera a orientação dos elementos afetados. A base de dados $S_{C}$ foi criada para ser utilizada pelos algoritmos da variação com sinais dos problemas $\SbFIR{}$, $\SbFIRM{}$, $\SbFIRT{}$ e $\SbFIRTM{}$.

A seguir apresentamos os resultados obtidos pelos algoritmos apresentado para a variação com sinais dos problemas investigados neste capítulo. Nas tabelas que serão utilizadas a seguir temos a informação por grupo do grau de flexibilização adotado e as métricas de distância e aproximação, sendo que para ambas as métricas temos a informação sobre o menor e maior valor registrado e a média obtida.

\begin{table}[!htb]
  \caption{Resultados do Algoritmo~\ref{algorithm:HSDBEFII} utilizando a base de dados $S_\SbFIR{}$.}
  \label{table:MGYFELVA}
  \centering
  \begin{tabular}{|c|r|r|r|r|r|r|}
    \hline
      -            & \multicolumn{3}{c|}{Distância}             & \multicolumn{3}{c|}{Aproximação}           \\ \hline
    Flexibilização & Mínimo       & Média        & Máximo       & Mínimo       & Média        & Máximo       \\ \hline  
    10\%           & 45           & 49.98        & 55           & 1.00         & 1.01         & 1.10         \\ \hline
    20\%           & 44           & 49.81        & 54           & 1.00         & 1.01         & 1.09         \\ \hline
    30\%           & 43           & 49.73        & 56           & 1.00         & 1.01         & 1.12         \\ \hline
    40\%           & 44           & 49.80        & 55           & 1.00         & 1.01         & 1.10         \\ \hline
    50\%           & 45           & 49.74        & 54           & 1.00         & 1.01         & 1.09         \\ \hline    
  \end{tabular}
\end{table}

% \begin{table}[!htb]
  \caption{Resultados do Algoritmo~\ref{algorithm:HSDBEFII} utilizando a base de dados $S_{\text{C}}$.}
  \label{table:SZTGMABR}
  \centering
  \begin{tabular}{|c|r|r|r|r|r|r|}
    \hline
      -            & \multicolumn{3}{c|}{Distância}             & \multicolumn{3}{c|}{Aproximação}           \\ \hline
    Flexibilização & Mínimo       & Média        & Máximo       & Mínimo       & Média        & Máximo       \\ \hline  
    10\%           & 24           & 55.64        & 80           & 1.00         & 1.23         & 1.50         \\ \hline
    20\%           & 23           & 57.54        & 84           & 1.00         & 1.25         & 1.52         \\ \hline
    30\%           & 24           & 58.97        & 87           & 1.00         & 1.27         & 1.54         \\ \hline
    40\%           & 24           & 59.39        & 87           & 1.00         & 1.27         & 1.54         \\ \hline
    50\%           & 22           & 59.38        & 90           & 1.00         & 1.28         & 1.56         \\ \hline    
  \end{tabular}
\end{table}

\begin{table}[!htb]
  \caption{Resultados do Algoritmo~\ref{algorithm:ZCBCGAUW} utilizando a base de dados $S_\SbFIRI{}$.}
  \label{table:CGVMJXDZ}
  \centering
  \begin{tabular}{|c|r|r|r|r|r|r|}
    \hline
      -            & \multicolumn{3}{c|}{Distância}             & \multicolumn{3}{c|}{Aproximação}           \\ \hline
    Flexibilização & Mínimo       & Média        & Máximo       & Mínimo       & Média        & Máximo       \\ \hline  
    10\%           & 40           & 45.97        & 52           & 1.00         & 1.01         & 1.09         \\ \hline
    20\%           & 40           & 44.63        & 50           & 1.00         & 1.01         & 1.11         \\ \hline
    30\%           & 39           & 43.84        & 50           & 1.00         & 1.01         & 1.14         \\ \hline
    40\%           & 38           & 42.98        & 50           & 1.00         & 1.01         & 1.14         \\ \hline
    50\%           & 37           & 42.34        & 50           & 1.00         & 1.01         & 1.11         \\ \hline    
  \end{tabular}
\end{table}

\begin{table}[!htb]
  \caption{Resultados do Algoritmo~\ref{algorithm:VOHUBSMM} utilizando as bases de dados $S_\SbFIRM{}$ e $S_{\text{C}}$.}
  \label{table:ZTVCHZCR}
  \centering
  \begin{tabular}{|c|r|r|r|r|r|r|}
    \hline
    \multicolumn{7}{|c|}{$S_\SbFIRM{}$}                                                                      \\ \hline
      -            & \multicolumn{3}{c|}{Distância}             & \multicolumn{3}{c|}{Aproximação}           \\ \hline
    Flexibilização & Mínimo       & Média        & Máximo       & Mínimo       & Média        & Máximo       \\ \hline  
    10\%           & 41           & 49.50        & 57           & 1.02         & 1.10         & 1.23         \\ \hline
    20\%           & 40           & 47.48        & 55           & 1.00         & 1.08         & 1.19         \\ \hline
    30\%           & 39           & 45.74        & 53           & 1.00         & 1.07         & 1.22         \\ \hline
    40\%           & 38           & 44.68        & 53           & 1.00         & 1.05         & 1.18         \\ \hline
    50\%           & 37           & 43.66        & 50           & 1.00         & 1.04         & 1.19         \\ \hline    
  \end{tabular}

  \vspace{5mm}

  \begin{tabular}{|c|r|r|r|r|r|r|}
    \hline
    \multicolumn{7}{|c|}{$S_{\text{C}}$}                                                                     \\ \hline
      -            & \multicolumn{3}{c|}{Distância}             & \multicolumn{3}{c|}{Aproximação}           \\ \hline
    Flexibilização & Mínimo       & Média        & Máximo       & Mínimo       & Média        & Máximo       \\ \hline  
    10\%           & 23           & 40.14        & 53           & 1.00         & 1.13         & 1.29         \\ \hline
    20\%           & 24           & 41.01        & 56           & 1.00         & 1.14         & 1.35         \\ \hline
    30\%           & 24           & 41.82        & 57           & 1.00         & 1.15         & 1.32         \\ \hline
    40\%           & 24           & 42.46        & 57           & 1.00         & 1.16         & 1.33         \\ \hline
    50\%           & 25           & 42.94        & 58           & 1.00         & 1.16         & 1.32         \\ \hline    
  \end{tabular}
\end{table}

% \begin{table}[!htb]
%   \caption{Resultados do Algoritmo~\ref{algorithm:VOHUBSMM} utilizando a base de dados $S_{\text{C}}$.}
%   \label{table:DEMGNKRC}
%   \centering
%   \begin{tabular}{|c|r|r|r|r|r|r|}
%     \hline
%       -            & \multicolumn{3}{c|}{Distância}             & \multicolumn{3}{c|}{Aproximação}           \\ \hline
%     Flexibilização & Mínimo       & Média        & Máximo       & Mínimo       & Média        & Máximo       \\ \hline  
%     10\%           & 24           & 43.17        & 56           & 1.00         & 1.16         & 1.36         \\ \hline
%     20\%           & 23           & 44.00        & 57           & 1.00         & 1.17         & 1.35         \\ \hline
%     30\%           & 24           & 44.35        & 60           & 1.00         & 1.18         & 1.39         \\ \hline
%     40\%           & 24           & 44.37        & 59           & 1.00         & 1.18         & 1.36         \\ \hline
%     50\%           & 22           & 44.22        & 60           & 1.00         & 1.18         & 1.38         \\ \hline    
%   \end{tabular}
% \end{table}

% \begin{table}[!htb]
  \caption{Resultados do Algoritmo~\ref{algorithm:VOHUBSMM} utilizando a base de dados $S_{\text{C}}$.}
  \label{table:DEMGNKRC}
  \centering
  \begin{tabular}{|c|r|r|r|r|r|r|}
    \hline
      -            & \multicolumn{3}{c|}{Distância}             & \multicolumn{3}{c|}{Aproximação}           \\ \hline
    Flexibilização & Mínimo       & Média        & Máximo       & Mínimo       & Média        & Máximo       \\ \hline  
    10\%           & 24           & 43.17        & 56           & 1.00         & 1.16         & 1.36         \\ \hline
    20\%           & 23           & 44.00        & 57           & 1.00         & 1.17         & 1.35         \\ \hline
    30\%           & 24           & 44.35        & 60           & 1.00         & 1.18         & 1.39         \\ \hline
    40\%           & 24           & 44.37        & 59           & 1.00         & 1.18         & 1.36         \\ \hline
    50\%           & 22           & 44.22        & 60           & 1.00         & 1.18         & 1.38         \\ \hline    
  \end{tabular}
\end{table}

\begin{table}[!htb]
  \caption{Resultados do Algoritmo~\ref{algorithm:TAJJYPTG} utilizando a base de dados $S_\SbFIRMI{}$.}
  \label{table:NNPRKQHC}
  \centering
  \begin{tabular}{|c|r|r|r|r|r|r|}
    \hline
      -            & \multicolumn{3}{c|}{Distância}             & \multicolumn{3}{c|}{Aproximação}           \\ \hline
    Flexibilização & Mínimo       & Média        & Máximo       & Mínimo       & Média        & Máximo       \\ \hline  
    10\%           & 39           & 48.31        & 55           & 1.02         & 1.11         & 1.21         \\ \hline
    20\%           & 40           & 46.50        & 55           & 1.01         & 1.09         & 1.21         \\ \hline
    30\%           & 39           & 45.19        & 53           & 1.00         & 1.07         & 1.20         \\ \hline
    40\%           & 38           & 44.21        & 51           & 1.00         & 1.06         & 1.19         \\ \hline
    50\%           & 37           & 43.50        & 50           & 1.00         & 1.05         & 1.15         \\ \hline    
  \end{tabular}
\end{table}

\begin{table}[!htb]
  \caption{Resultados do Algoritmo~\ref{algorithm:EMLPACHB} utilizando a base de dados $S_\SbFIRT{}$.}
  \label{table:HQMTSFGN}
  \centering
  \begin{tabular}{|c|r|r|r|r|r|r|}
    \hline
      -            & \multicolumn{3}{c|}{Distância}             & \multicolumn{3}{c|}{Aproximação}           \\ \hline
    Flexibilização & Mínimo       & Média        & Máximo       & Mínimo       & Média        & Máximo       \\ \hline  
    10\%           & 53           & 62.70        & 74           & 1.62         & 1.87         & 2.09         \\ \hline
    20\%           & 52           & 62.86        & 76           & 1.64         & 1.88         & 2.12         \\ \hline
    30\%           & 52           & 63.06        & 75           & 1.68         & 1.90         & 2.15         \\ \hline
    40\%           & 53           & 63.58        & 74           & 1.68         & 1.92         & 2.16         \\ \hline
    50\%           & 51           & 64.36        & 76           & 1.65         & 1.94         & 2.21         \\ \hline    
  \end{tabular}
\end{table}

% \begin{table}[!htb]
  \caption{Resultados do Algoritmo~\ref{algorithm:EMLPACHB} utilizando a base de dados $S_{\text{C}}$.}
  \label{table:XQLOOFHX}
  \centering
  \begin{tabular}{|c|r|r|r|r|r|r|}
    \hline
      -            & \multicolumn{3}{c|}{Distância}             & \multicolumn{3}{c|}{Aproximação}           \\ \hline
    Flexibilização & Mínimo       & Média        & Máximo       & Mínimo       & Média        & Máximo       \\ \hline  
    10\%           & 22           & 41.93        & 56           & 1.47         & 1.91         & 2.08         \\ \hline
    20\%           & 19           & 42.96        & 58           & 1.53         & 1.92         & 2.15         \\ \hline
    30\%           & 21           & 43.39        & 60           & 1.47         & 1.92         & 2.14         \\ \hline
    40\%           & 20           & 43.47        & 59           & 1.40         & 1.92         & 2.15         \\ \hline
    50\%           & 19           & 43.37        & 60           & 1.43         & 1.93         & 2.15         \\ \hline    
  \end{tabular}
\end{table}

\begin{table}[!htb]
  \caption{Resultados do Algoritmo~\ref{algorithm:WWDUHPBG} utilizando a base de dados $S_\SbFIRTI{}$.}
  \label{table:SMSSPGPF}
  \centering
  \begin{tabular}{|c|r|r|r|r|r|r|}
    \hline
      -            & \multicolumn{3}{c|}{Distância}             & \multicolumn{3}{c|}{Aproximação}           \\ \hline
    Flexibilização & Mínimo       & Média        & Máximo       & Mínimo       & Média        & Máximo       \\ \hline  
    10\%           & 53           & 60.90        & 70           & 1.93         & 2.01         & 2.17         \\ \hline
    20\%           & 52           & 59.96        & 68           & 1.93         & 2.02         & 2.14         \\ \hline
    30\%           & 49           & 59.22        & 67           & 1.93         & 2.02         & 2.15         \\ \hline
    40\%           & 49           & 58.73        & 66           & 1.93         & 2.02         & 2.13         \\ \hline
    50\%           & 49           & 58.30        & 67           & 1.93         & 2.02         & 2.20         \\ \hline    
  \end{tabular}
\end{table}

\begin{table}[!htb]
  \caption{Resultados do Algoritmo~\ref{algorithm:XXIGKPAV} utilizando as bases de dados $S_\SbFIRTM{}$ e $S_{\text{C}}$.}
  \label{table:EXRPXQLG}
  \centering
  \begin{tabular}{|c|r|r|r|r|r|r|}
    \hline
    \multicolumn{7}{|c|}{$S_\SbFIRTM{}$}                                                                     \\ \hline
      -            & \multicolumn{3}{c|}{Distância}             & \multicolumn{3}{c|}{Aproximação}           \\ \hline
    Flexibilização & Mínimo       & Média        & Máximo       & Mínimo       & Média        & Máximo       \\ \hline  
    10\%           & 52           & 62.60        & 71           & 1.83         & 1.97         & 2.03         \\ \hline
    20\%           & 51           & 61.30        & 68           & 1.90         & 1.97         & 2.04         \\ \hline
    30\%           & 52           & 60.21        & 69           & 1.89         & 1.97         & 2.09         \\ \hline
    40\%           & 50           & 59.35        & 66           & 1.90         & 1.97         & 2.07         \\ \hline
    50\%           & 49           & 58.87        & 66           & 1.89         & 1.97         & 2.08         \\ \hline    
  \end{tabular}

  \vspace{5mm}

  \begin{tabular}{|c|r|r|r|r|r|r|}
    \hline
    \multicolumn{7}{|c|}{$S_{\text{C}}$}                                                                     \\ \hline
      -            & \multicolumn{3}{c|}{Distância}             & \multicolumn{3}{c|}{Aproximação}           \\ \hline
    Flexibilização & Mínimo       & Média        & Máximo       & Mínimo       & Média        & Máximo       \\ \hline  
    10\%           & 23           & 40.02        & 52           & 1.75         & 1.96         & 2.08         \\ \hline
    20\%           & 24           & 40.89        & 55           & 1.81         & 1.96         & 2.07         \\ \hline
    30\%           & 24           & 41.70        & 57           & 1.84         & 1.96         & 2.08         \\ \hline
    40\%           & 24           & 42.32        & 57           & 1.85         & 1.97         & 2.07         \\ \hline
    50\%           & 25           & 42.84        & 59           & 1.85         & 1.97         & 2.07         \\ \hline    
  \end{tabular}
\end{table}

% \begin{table}[!htb]
%   \caption{Resultados do Algoritmo~\ref{algorithm:XXIGKPAV} utilizando a base de dados $S_{\text{C}}$.}
%   \label{table:JXNBJWFA}
%   \centering
%   \begin{tabular}{|c|r|r|r|r|r|r|}
%     \hline
%       -            & \multicolumn{3}{c|}{Distância}             & \multicolumn{3}{c|}{Aproximação}           \\ \hline
%     Flexibilização & Mínimo       & Média        & Máximo       & Mínimo       & Média        & Máximo       \\ \hline  
%     10\%           & 24           & 43.02        & 56           & 1.84         & 1.97         & 2.00         \\ \hline
%     20\%           & 23           & 43.84        & 57           & 1.84         & 1.97         & 2.07         \\ \hline
%     30\%           & 24           & 44.20        & 58           & 1.88         & 1.97         & 2.07         \\ \hline
%     40\%           & 24           & 44.22        & 58           & 1.84         & 1.97         & 2.07         \\ \hline
%     50\%           & 22           & 44.07        & 60           & 1.81         & 1.97         & 2.14         \\ \hline    
%   \end{tabular}
% \end{table}

% \begin{table}[!htb]
  \caption{Resultados do Algoritmo~\ref{algorithm:XXIGKPAV} utilizando a base de dados $S_{\text{C}}$.}
  \label{table:JXNBJWFA}
  \centering
  \begin{tabular}{|c|r|r|r|r|r|r|}
    \hline
      -            & \multicolumn{3}{c|}{Distância}             & \multicolumn{3}{c|}{Aproximação}           \\ \hline
    Flexibilização & Mínimo       & Média        & Máximo       & Mínimo       & Média        & Máximo       \\ \hline  
    10\%           & 24           & 43.02        & 56           & 1.84         & 1.97         & 2.00         \\ \hline
    20\%           & 23           & 43.84        & 57           & 1.84         & 1.97         & 2.07         \\ \hline
    30\%           & 24           & 44.20        & 58           & 1.88         & 1.97         & 2.07         \\ \hline
    40\%           & 24           & 44.22        & 58           & 1.84         & 1.97         & 2.07         \\ \hline
    50\%           & 22           & 44.07        & 60           & 1.81         & 1.97         & 2.14         \\ \hline    
  \end{tabular}
\end{table}

\begin{table}[!htb]
  \caption{Resultados do Algoritmo~\ref{algorithm:JBNSEPGG} utilizando a base de dados $S_\SbFIRTMI{}$.}
  \label{table:XHZJQXXJ}
  \centering
  \begin{tabular}{|c|r|r|r|r|r|r|}
    \hline
      -            & \multicolumn{3}{c|}{Distância}             & \multicolumn{3}{c|}{Aproximação}           \\ \hline
    Flexibilização & Mínimo       & Média        & Máximo       & Mínimo       & Média        & Máximo       \\ \hline  
    10\%           & 54           & 62.25        & 70           & 1.94         & 2.01         & 2.13         \\ \hline
    20\%           & 53           & 61.21        & 69           & 1.90         & 2.01         & 2.17         \\ \hline
    30\%           & 52           & 60.28        & 68           & 1.93         & 2.01         & 2.13         \\ \hline
    40\%           & 51           & 59.56        & 70           & 1.93         & 2.01         & 2.15         \\ \hline
    50\%           & 50           & 59.05        & 66           & 1.96         & 2.01         & 2.13         \\ \hline    
  \end{tabular}
\end{table}

% ------------------------------------------------------------------ %
\section{Conclusões}
% ------------------------------------------------------------------ %

Neste capítulo, estudamos uma generalização dos problemas que consideram tanto a ordem dos genes como o tamanho estrito das regiões intergênicas. Nessa versão generalizada adicionamos um grau de flexibilidade em relação ao tamanho das regiões regiões intergênicas desejadas no genoma alvo. Para isso, nos modelos que propomos, chamados de modelos intergênicos flexíveis, é possível especificar um intervalo de valores permitidos para o tamanho de cada região intergênica no genoma alvo. 

Esse grau de flexibilidade tenta trazer uma importância maior para a ordem e orientação dos genes em comparação com o tamanho das regiões intergênicas, uma vez que considerando um intervalo de valores permitidos para o tamanho de cada região intergênica no genoma alvo acabos ampliando as possibilidades de configuração para os tamanhos das regiões intergênicas de modo que todas as restrições de um modelo sejam atendidas.

Para todas as variações dos problemas investigados neste capítulo, nós apresentamos algoritmos de aproximação com um fator constante com base em um processo de redução que permite o uso de resultados que foram apresentados para as respectivas versões rígidas dos problemas. Por fim, realizamos testes experimentais para verificar o desempenho prático dos algoritmos que foram apresentados.

Nesta tese apresentamos os resultados obtidos durante o período do doutorado. De maneira geral, os resultados aqui apresentados incorporam novas características aos problemas de rearranjo de genomas, que vão desde a adição de uma nova restrição de proporção entre a quantidade de um tipo de evento em relação ao tamanho da sequência de eventos de rearranjo em uma solução, até a inclusão de novas estruturas genéticas na representação computacional adotada nos modelos. Além disso, foram investigados problemas considerando um grau de flexibilidade nas características de um genoma alvo desejado. Para a grande maioria dos problemas que investigamos, apresentamos a prova de NP-dificuldade e desenvolvemos algoritmos de aproximação. Além disso, sempre que possível, criamos mecanismos com o intuito de melhorar os resultados práticos dos algoritmos propostos. Os resultados apresentados nessa tese, geraram os seguintes artigos:

\begin{itemize}
  \item ``\textit{A New Approach for the Reversal Distance with Indels and Moves in Intergenic Regions}'', em coautoria com Andre Rodrigues Oliveira, Alexsandro Oliveira Alexandrino, Ulisses Dias e Zanoni Dias, foi apresentado no 19th Annual Satellite Conference of RECOMB on Comparative Genomics (RECOMB-CG), realizada em La Jolla, USA, no mês de Maio de 2022~\cite{2022b-brito-etal}.

  \item ``\textit{Genome Rearrangement Distance with a Flexible Intergenic Regions Aspect}'', aceito para publicação no mês de Abril de 2022, na revista IEEE-ACM Transactions on Computational Biology and Bioinformatics, e em coautoria com Alexsandro Oliveira Alexandrino, Andre Rodrigues Oliveira, Ulisses Dias e Zanoni Dias~\cite{2022a-brito-etal}. Uma versão preliminar deste artigo foi apresentada na 8th International Conference on Algorithms for Computational Biology (AlCoB), realizada de forma virtual, no mês de Novembro de 2021~\cite{2021c-brito-etal}. Uma versão abordando instâncias sem sinais foi apresentada no XI Latin and American Algorithms, Graphs and Optimization Symposium (LAGOS), realizado de forma virtual, no mês de Maio de 2021~\cite{2021d-brito-etal}.

  \item ``\textit{An Improved Approximation Algorithm for the Reversal and Transposition Distance Considering Gene order and Intergenic Sizes}'', publicado no mês de Dezembro de 2021, na revista Algorithms for Molecular Biology, e em coautoria com Andre Rodrigues Oliveira, Alexsandro Oliveira Alexandrino, Ulisses Dias e Zanoni Dias~\cite{2021b-brito-etal}.

  \item ``\textit{Reversals and Transpositions Distance with Proportion Restriction}'', publicado no mês de Agosto de 2021, na revista Journal of Bioinformatics and Computational Biology, e em coautoria com Alexsandro Oliveira Alexandrino, Andre Rodrigues Oliveira, Ulisses Dias e Zanoni Dias~\cite{2021a-brito-etal}. Uma versão preliminar deste artigo foi apresentada no 13th Brazilian Symposium on Bioinformatics (BSB), realizado de forma virtual, no mês de Novembro de 2020~\cite{2020c-brito-etal}.

  \item ``\textit{Sorting by Genome Rearrangements on Both Gene Order and Intergenic Sizes}'', publicado no mês de Fevereiro de 2020, na revista Journal of Computational Biology, e em coautoria com Géraldine Jean, Guillaume Fertin, Andre Rodrigues Oliveira, Ulisses Dias e Zanoni Dias~\cite{2020a-brito-etal}. Uma versão preliminar deste artigo foi apresentada no 15th International Symposium on Bioinformatics Research and Applications (ISBRA), realizado em Barcelona, Espanha, no mês de Junho de 2019~\cite{2019-brito-etal}.
\end{itemize}

Além dos artigos listados acima, que estão diretamente relacionados com essa tese, durante o doutorado contribuímos para a obtenção de  outros resultados, apresentados nos seguintes artigos:

\begin{itemize}
  \item ``\textit{A 1.375-Approximation Algorithm for Sorting by Transpositions with Faster Running Time}'', em coautoria com Alexsandro Oliveira Alexandrino, Andre Rodrigues Oliveira, Ulisses Dias e Zanoni Dias, apresentado no 15th Brazilian Symposium on Bioinformatic (BSB), realizado em Búzios, Brasil, no mês de Setembro de 2022~\cite{2022bsb-alexandrino-etal}.

  \item ``\textit{Reversal and Indel Distance with Intergenic Region Information}'', aceito para publicação no mês de Outubro de 2022, na revista IEEE-ACM Transactions on Computational Biology and Bioinformatics, e em coautoria com Alexsandro Oliveira Alexandrino, Andre Rodrigues Oliveira, Ulisses Dias e Zanoni Dias~\cite{2022b-alexandrino-etal}. Uma versão preliminar deste artigo foi apresentada na 7th-8th International Conference on Algorithms for Computational Biology (AlCoB), realizada de forma virtual, no mês de Novembro de 2021~\cite{2021b-alexandrino-etal}.

  \item ``\textit{Sorting Permutations by Intergenic Operations}'', publicado no mês de Novembro de 2021, na revista IEEE-ACM Transactions on Computational Biology and Bioinformatics, e em coautoria com Andre Rodrigues Oliveira, Géraldine Jean, Guillaume Fertin, Ulisses Dias e Zanoni Dias~\cite{2021a-oliveira-etal}. Uma versão preliminar deste artigo foi apresentada na 7th-8th International Conference on Algorithms for Computational Biology (AlCoB), realizada de forma virtual, no mês de Novembro de 2021~\cite{2020-oliveira-etal}.

  \item ``\textit{Sorting Signed Permutations by Intergenic Reversals}'', publicado no mês de Novembro de 2021, na revista IEEE-ACM Transactions on Computational Biology and Bioinformatics, e em coautoria com Andre Rodrigues Oliveira, Géraldine Jean, Guillaume Fertin, Laurent Bulteau, Ulisses Dias e Zanoni Dias~\cite{2021b-oliveira-etal}.

  \item ``\textit{Heuristics for Genome Rearrangement Distance with Replicated Genes}'', publicado no mês de Novembro de 2021, na revista IEEE-ACM Transactions on Computational Biology and Bioinformatics, e em coautoria com Gabriel Siqueira, Ulisses Dias e Zanoni Dias~\cite{2021a-siqueira-etal}. Uma versão preliminar deste artigo foi apresentada na 7th-8th International Conference on Algorithms for Computational Biology (AlCoB), realizada de forma virtual, no mês de Novembro de 2021~\cite{2020-siqueira-etal}.

  \item ``\textit{On the Complexity of Sorting by Reversals and Transpositions Problem}'', publicado no mês de Novembro de 2019, na revista Journal of Computational Biology, e em coautoria com Andre Rodrigues Oliveira, Ulisses Dias e Zanoni Dias~\cite{2019b-oliveira-etal}.

  \item ``\textit{Block-Interchange Distance Considering Intergenic Regions}'', em coautoria com Ulisses Dias, Andre Rodrigues Oliveira e Zanoni Dias, apresentado no 12th Brazilian Symposium on Bioinformatics (BSB), realizado em Fortaleza, Brasil, no mês de Outubro de 2019~\cite{2019-dias-etal}.
\end{itemize}

A partir das contribuições apresentadas, é possível mencionar algumas possibilidades de trabalhos futuros: (i) considerando problemas com restrição de proporção entre operações, é possível incorporar uma representação intergênica aos modelos. Além disso, é possível investigar outras restrições de proporção considerando mais eventos de rearranjo; (ii) para os modelos que consideram a informação referente aos genes e ao tamanho das regiões intergênicas, é possível adicionar eventos de rearranjo não conservativos, que afetam tanto os genes como as regiões intergênicas; (iii) outra opção é considerar uma representação intergênica que permita múltiplas cópias de um gene; (iv) por fim, uma possível linha de investigação seria o estudo de novos algoritmos visando obter melhores resultados práticos ou teóricos para os problemas aqui investigados.  Essas são algumas das possibilidades de investigação de trabalhos futuros que podem permitir uma aplicação dos resultados de maneira mais direta em genomas reais. 



% As referências:
\bibliographystyle{plain}
\bibliography{bibfile}


% Os anexos, se houver, vêm depois das referências:
% \appendix
% \chapter{Anexo 1}
% \chapter{Anexo 2}

\end{document}
